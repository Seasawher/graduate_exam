\section{平成20年度 数学I}

\subsubsection{}%1
\barquo{
$V$と$W$を複素数体上の有限次元ベクトル空間とし、$f \colon V \to V$, $g \colon W \to W$をそれぞれ線形変換とする。さらに、$f$と$g$は同じ固有値を持たないとする。線形写像$\vp \colon V \to W$が、条件$\vp \circ f = g \circ \vp$を満たしているとき、$\vp = 0$となることを示せ。
}
\begin{proof}
  ハイリホーによる。$\Im \vp \neq 0$と仮定しよう。$f$の$V$上での最小多項式を$P$, $g$の$W$上での最小多項式を$Q$とする。$0 = \vp \circ P(f) = P(g) \circ \vp$により$P(g)$は$\Im \vp$上でゼロ。よって$P$は$g$の$\Im \vp$上での最小多項式$Q_{\vp}$で割り切れる。
  一方で$Q_{\vp}$は$Q$も割り切るので、$P$と$Q$に共通因子があることになり矛盾。よって$\Im \vp=0$である。
\end{proof}

\begin{com}
  広義固有分解を使った別解がある。広義固有空間を$E$で表すことにしよう。つまり$\grl \in \C$に対して
  \begin{align*}
    E(f,\grl) & = \setmid{ v \in V}{\exists m \st (f - \grl \id_V)^m v = 0 } \\
    E(g,\grl) & = \setmid{ w \in W}{\exists m \st (g - \grl \id_W)^m w = 0 }
  \end{align*}
  と定める。$f$の異なる固有値を$\grl_1, \cdots , \grl_k$とおく。任意に$v \in V$が与えられたと仮定する。
  \[
  V = \bigoplus_{i=1}^k E(f, \grl_i)
  \]
  なので$v \in E(f, \grl_j)$なる$j$がある。すると仮定から十分大きい$m$に対して
  \begin{align*}
    0 = \vp \circ (f - \grl_j \id_V)^m v = (g - \grl_j \id_W)^m \circ \vp (v)
  \end{align*}
  が成り立つ。つまり$\vp(v) \in E(g, \grl_j)$である。ところが$g$と$f$に共通の固有値はないと仮定していたので$E(g, \grl_j)= 0$である。よって$\vp(v)=0$がわかる。$v \in V$は任意だったから$\vp = 0$が結論される。
\end{com}

\newpage

\subsubsection{}%2
\barquo{
$f$と$g$を$\R$上定義された一様連続な実数値関数とする。このとき、次の問に答えよ。
\begin{description}
  \item[(1)] 関数
  \[
  \f{f(x)}{1 + \abs{x}}
  \]
  は$\R$上有界であることを示せ。
  \item[(2)] 関数
  \[
  \f{f(x)g(x)}{1 + \abs{x}}
  \]
  は$\R$上一様連続であることを示せ。
\end{description}
}
\begin{proof} ${}$
  \begin{description}
    \item[(1)] 次の補題を準備する。

    \lem{
    $f$は$\R$上の一様連続関数とする。このときある$\grd > 0$が存在して、すべての自然数$n \geq 1$に対して
    \[
    \abs{x} \leq \grd n \to \abs{f(x) - f(0)} \leq n
    \]
    が成り立つ。
    }
    \begin{proof}
      $f$は一様連続と仮定したので
      \[
      \abs{x - y } \leq \grd \to \abs{f(x) - f(y)} \leq 1 \quad \quad  (*)
      \]
      となるような$\grd > 0$が存在する。この$\grd$が条件を満たすことを、帰納法により示す。$n=1$で成立することは(*)に$y=0$を代入してみればあきらか。$n$が$k$以下のとき成立すると仮定する。このとき(*)に$y = k \grd$を代入すると
      \[
      k \grd  \leq x \leq (k+1) \grd \to \abs{f(x) - f(0)} \leq \abs{f(x) - f(\grd k)} + \abs{f(\grd k) - f(0)} \leq k+1
      \]
      が成り立つ。さらに(*)に$y = - k \grd$を代入して
      \[
        -(k+1) \grd  \leq x \leq -k \grd \to \abs{f(x) - f(0)}  \leq k+1
      \]
      も得る。よって$n=k+1$のときにも成立する。ゆえに帰納法が回り、示すべきことがいえた。
    \end{proof}
  \end{description}
\end{proof}

\newpage

\subsubsection{}%3
\barquo{
$p,l$を素数とする。次数が$l$の$\F_p$上のモニックな一変数既約多項式の数を求めよ。ただし、$\F_p$は$p$個の元からなる体である。
}

\newpage
\subsubsection{}%4
\barquo{
$S^n = \setmid{(x_0, \cdots , x_n) \in \R^{n+1}}{ x_0^2 + \cdots +  x_n^2 = 1 }$を$n$次元球面とする。次の問に答えよ。
\begin{description}
  \item[(1)] $n \geq 1$とし、$f \colon S^n \to \R$を連続写像とする。このとき、$f(x) = f(-x)$を満たす$x \in S^n$が存在することを示せ。
  \item[(2)] $n \geq 2$とし、$f \colon S^n \to S^1$を連続写像とする。このとき、$f(x) = f(-x)$を満たす$x \in S^n$が存在することを示せ。
\end{description}
}

\newpage

\subsubsection{}%5
\barquo{
$f(z)$は領域$D = \setmid{z \in \C}{0 < \abs{z} < 1}$で定義された正則関数で、
\[
\int_D \abs{f(x + iy)}^2 \ dx dy < \infty
\]
を満たすとする。このとき、$z = 0$は$f(z)$の除去可能特異点であることを示せ。
}
