\section{平成20年度 数学I}

\subsubsection{}%1
\barquo{
$V$と$W$を複素数体上の有限次元ベクトル空間とし、$f \colon V \to V$, $g \colon W \to W$をそれぞれ線形変換とする。さらに、$f$と$g$は同じ固有値を持たないとする。線形写像$\vp \colon V \to W$が、条件$\vp \circ f = g \circ \vp$を満たしているとき、$\vp = 0$となることを示せ。
}

\newpage

\subsubsection{}%2
\barquo{
$f$と$g$を$\R$上定義された一様連続な実数値関数とする。このとき、次の問に答えよ。
\begin{description}
  \item[(1)] 関数
  \[
  \f{f(x)}{1 + \abs{x}}
  \]
  は$\R$上有界であることを示せ。
  \item[(2)] 関数
  \[
  \f{f(x)g(x)}{1 + \abs{x}}
  \]
  は$\R$上一様連続であることを示せ。
\end{description}
}

\newpage

\subsubsection{}%3
\barquo{
$p,l$を素数とする。次数が$l$の$\F_p$上のモニックな一変数既約多項式の数を求めよ。ただし、$\F_p$は$p$個の元からなる体である。
}

\newpage
\subsubsection{}%4
\barquo{
$S^n = \setmid{(x_0, \cdots , x_n) \in \R^{n+1}}{ x_0^2 + \cdots +  x_n^2 = 1 }$を$n$次元球面とする。次の問に答えよ。
\begin{description}
  \item[(1)] $n \geq 1$とし、$f \colon S^n \to \R$を連続写像とする。このとき、$f(x) = f(-x)$を満たす$x \in S^n$が存在することを示せ。
  \item[(2)] $n \geq 2$とし、$f \colon S^n \to S^1$を連続写像とする。このとき、$f(x) = f(-x)$を満たす$x \in S^n$が存在することを示せ。
\end{description}
}

\newpage

\subsubsection{}%5
\barquo{
$f(z)$は領域$D = \setmid{z \in \C}{0 < \abs{z} < 1}$で定義された正則関数で、
\[
\int_D \abs{f(x + iy)}^2 \ dx dy < \infty
\]
を満たすとする。このとき、$z = 0$は$f(z)$の除去可能特異点であることを示せ。
}
