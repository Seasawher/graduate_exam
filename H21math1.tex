\section{平成21年度 数学I}

\subsubsection{}%1
\barquo{
標数$0$の体$K$上の有限次元ベクトル空間$V$の一次変換$T$が
\[
T^3 = 1_V, \quad \Tr(T) = 0
\]
を満たすとする。このとき、$V$の次元は$3$の倍数であることを示せ。
}
\begin{sol}
  $\gro$を$1$の原始$3$乗根とする。$T^3 = 1_V$より、$T$の固有値は$1, \gro, \gro^2$のどれか。固有値$1, \gro, \gro^2$の重複度をそれぞれ$m_1, m_2,m_3$とすると仮定により
  \begin{align*}
    m_1 + m_2 +m_3 &= \dim V \\
    m_1 + m_2 \gro +m_3 \gro^2 &= 0
  \end{align*}
  である。$2$つめの式の複素共役をとると$m_i \in \R$なので$m_1 + m_2 \gro^2 +m_3 \gro = 0$を得る。これらを辺々足して$3m_1 = \dim V$であることが従う。よって$\dim V$は$3$の倍数。
\end{sol}

\newpage


\subsubsection{}%2
\barquo{
$f,g$を区間$(0,\infty)$上で定義された連続かつ広義可積分な非負値関数とし
\[
\lim_{x \to 0} f(x) = 0, \quad \lim_{x \to \infty} xg(x) = 0
\]
を満たすとする。このとき
\[
\lim_{n \to \infty} n \int_0^{\infty} f(x) g(nx) \ dx = 0
\]
を示せ。
}
\begin{sol}
  任意に正数$\ve > 0$が与えられたとする。$\lim_{x \to \infty} xg(x) = 0$という仮定より
  \[
  \forall x \geq R \quad x g(x) < \ve
  \]
  を満たすような$R > 0$がとれる。$R \geq 1$としてよい。このとき
  \begin{align*}
    n \int_0^{\infty} f(x) g(nx) \ dx &= n \int_0^{R} f(x) g(nx) \ dx + n \int_R^{\infty} f(x) g(nx) \ dx \\
    &\leq \int_0^{nR} f \left( \f{y}{n} \right) g(y) \ dy +  n \int_R^{\infty} f(x) \f{ \ve }{ nx } \ dx \\
    &\leq \int_0^{nR} f \left( \f{y}{n} \right) g(y) \ dy + \ve  \int_R^{\infty}  \f{ f(x)  }{ x } \ dx \\
    &\leq \int_0^{nR} f \left( \f{y}{n} \right) g(y) \ dy + \f{ \ve}{R}  \int_R^{\infty}  f(x)  \ dx \\
  \end{align*}
  と評価できる。$f$は広義可積分という仮定より$N := \int_0^{\infty} f(x) \ dx$とおくことができて
  \[
  n \int_0^{\infty} f(x) g(nx) \ dx \leq \int_0^{nR} f \left( \f{y}{n} \right) g(y) \ dy + N \ve
  \]
  と評価できる。実はここで
  \[
  \lim_{n \to \infty} \int_0^{nR} f \left( \f{y}{n} \right) g(y) \ dy = 0
  \]
  である。なぜならば!$f$は連続関数なので$M := \max_{ 0 \leq x \leq R} \abs{ f(x) }$が存在する。ここで$n$によらず一様に
  \[
  \abs{ f(y/n) g(y) \chi_{[0,nR]}(y) } \leq M g(y)
  \]
  であり、仮定により$Mg$は$(0, \infty)$上の可積分関数である。よってLebesgueの収束定理が適用できて$\lim_{x \to 0} f(x) = 0$という仮定から
  \[
\lim_{n \to \infty} \int_0^{nR} f \left( \f{y}{n} \right) g(y) \ dy = 0
  \]
であることがわかる。ゆえにかくのごとし。したがって
\[
\limsup_{n \to \infty} n \int_0^{\infty} f(x) g(nx) \ dx \leq N \ve
\]
である。$\ve > 0$は任意だったから
\[
\lim_{n \to \infty} n \int_0^{\infty} f(x) g(nx) \ dx = 0
\]
でなくてはならない。
\end{sol}


\newpage


\subsubsection{}%3
\barquo{
$p$は素数とする。方程式
\[
x^2 + y^2 = 1
\]
の解$(x,y) \in \zyu{p} \tm \zyu{p}$について、以下の問に答えよ。
\begin{description}
  \item[(1)] $p \equiv 1 \mod 4$のときの解の個数を求めよ。
  \item[(2)] $p \equiv 3 \mod 4$のときの解の個数は$p+1$であることを示せ。
\end{description}
}
\begin{sol} ${}$
  \begin{description}
    \item[(1)] $p \equiv 1 \mod 4$という仮定より巡回群$\F_p^{\tm}$の位数は$4$の倍数であり、したがって$\gra^2 = -1 $なる$\gra \in \F_p $が存在する。
    \begin{align*}
      A &= \setmid{ (x,y) \in \F_p^2 }{ x^2 + y^2 = 1 } \\
      B &= \setmid{ (s,t) \in \F_p^2 }{ st = 1 }
    \end{align*}
    とおく。写像$\psi \colon A \to B$を$\vp(x,y) = (x + \gra y, x - \gra y)$として定めると、これは$p$が奇素数であることから全単射になる。したがって$\# A = \# B = p-1$がわかる。
    \item[(2)] $p \equiv 3 \mod 4$という仮定より$x^2 + 1 \in \F_p[x]$は既約多項式なので$F := \F_p[\gra]/(\gra^2 + 1)$は体である。そこで$\F_p$上のノルムが誘導する群準同型$N \colon F^{\tm} \to \F_p^{\tm}$を考える。$N(x + \gra y) = x^2 + y^2$であるので
    $\# \Ker N = p+1$を示せば十分である。

    準同型定理より$F^{\tm} / \Ker N \cong \Im N$である。よって$\# F^{\tm} / \Ker N \leq p-1$であるから$\# F^{\tm} = p^2-1$より
    \[
    \# \Ker N \geq p + 1
    \]
    がわかる。

    逆を示そう。$\Ker N \sm \F_p$の元に対して、その$\F_p$上の最小多項式を与える写像
    \[
    \vp \colon \Ker N \sm \F_p \to \setmid{ f \in \F_p[x] }{  \text{$f$は$f(x)=x^2 + ax + 1$という形の既約多項式} }
    \]
    を考える。右辺の集合を$C$とおく。ここで$\# (\Ker N \sm \F_p) \leq 2 \# C$という評価が成り立つ。なぜならば!$f \in C$が与えられたとする。$f$の$\ol{\F_p}$における根のひとつを$\beta$とすると$f$は$\beta$の$\F_p$上の最小多項式である。一方で有限体の一意性より$F \cong \F_p[\beta]$であるため、$\beta$に対応する$F$の元$\beta'$が存在する。
    このとき$\vp(\beta') = f$だから$\# \vp^{-1}(f) \geq 1$である。かつ$F / \F_p$は分離拡大なので$\# \vp^{-1}(f) > 1$である。$\# \vp^{-1}(f) \leq 2$はあきらかだから$\# \vp^{-1}(f) =2$である。$f$は任意だったから
    $\# (\Ker N \sm \F_p) \leq 2 \# C$という結論がでる。ゆえにかくのごとし。

    以上により$\# C$の評価に帰着されるわけだが、これは単純な数え上げで遂行できる。$C$は$x^2 + ax + 1 \; ( a \in \F_p)$という形の多項式のうち$(x - b)(x - b^{-1}) \; ( b \in \F_p^{\tm})$という形には表せないもの全体である。よって$b= \pm 1$でない限り$b \neq b^{-1}$であることに注意すれば
    \[
    \# C = p - \left( \f{p-3}{2} + 2 \right) = \f{p-1}{2}
    \]
    である。よって$\# ( \Ker N \sm \F_p) \leq p-1$であり、$\Ker N \cap \F_p = \{ \pm 1 \}$より$\# \Ker N \leq p+1$である。以上により$\# \Ker N = p+1$であることが示せた。
  \end{description}
\end{sol}

\newpage


\subsubsection{}%4
\barquo{
$f$を$n$次元球面$S^n$から$n$次元ユークリッド空間$\R^n$への$C^{\infty}$級写像とする。ただし$n \geq 1$であるとしておく。このとき$S^n$上の点で、その点における$f$の微分の階数が$n-1$以下になるものが存在することを示せ。
}
\begin{sol}
  ハイリホーによる。$f$の微分$df_p$がいたるところ正則だとする。このとき逆関数定理により$f \colon S^n \to \R^n$は局所微分同相である。よってとくに$f(S^n)$は$\R^n$の開集合。一方$S^n$はコンパクトなので$f(S^n)$もコンパクトで、$\R^n$はHausdorffだから$f(S^n)$は$\R^n$の閉集合でもある。したがって$\R^n$の連結性により$f(S^n) = \R^n$であるが、$\R^n$はコンパクトではないから矛盾。よって微分の階数が落ちる点があることがいえた。
\end{sol}

\newpage

\subsubsection{}%5
\barquo{
複素平面$\C$上で零点を持たない整関数の列$f_n(z) \; (n=1,2, \cdots )$がある。もし$f_n(z)$が$\C$上で多項式$p(z)$に広義一様収束するならば、$p(z)$は定数であることを示せ。
}
\begin{sol}
  $p$が多項式としてゼロなら示すことはないので、$p$はゼロではないとしてよい。代数学の基本定理により$p$がもし定数でなければ$p$は$\C$にゼロ点を持つはずであるので、$p$がいたるところゼロでないことを示せば十分である。

  $z_0 \in \C$が与えられたとする。$p$はゼロでない多項式なので$p$の零点は孤立している。ゆえに実数$r > 0$をうまく選べば、中心が抜けた円盤
  \[
  D = \setmid{ z \in \C }{0 < \abs{z - z_0} \leq r }
  \]
  上に$p$がゼロ点を持たないようにできる。$f_n$は$\C$上広義一様に$p$に収束しているので、$f_n'$は$\C$上広義一様に$p'$に収束している。よって$C := \del D$とおくと$\C$上で$f_n$は$p$に、$f_n'$は$p'$にそれぞれ一様収束する。

  ここで$C$はコンパクトなので$1/f_n$は$1/p$に$C$上一様収束する。なぜならば!$\ve > 0$が与えられたとする。$m := \inf_{z \in C} \abs{p(z)}$とおく。$C$のコンパクト性から$m > 0$であることに気を付けてほしい。$\ve < m/2$としてよい。$f_n$が$p$に$C$上一様収束していることから
  \[
  n \geq N \to \norm{f_n - p}_C < \ve
  \]
  なる正整数$N$がとれる。このとき$n \geq N$ならば
  \begin{align*}
    \norm{ \f{1}{f_n} - \f{1}{p} }_C &= \max_{z \in C} \abs{ \f{ p'(z) - f_n(z) }{  f_n(z) p(z) }  } \\
    &\leq \max_{z \in C}  \f{ \norm{ p' - f_n}_C  }{  (\abs{p(z) } - \ve) \abs{ p(z) } }   \\
    &\leq \f{2 \ve }{m^2}
  \end{align*}
  である。$\ve > 0$は任意だったから$C$上一様に$1/f_n \to 1/p$であることがいえた。ゆえにかくのごとし。

  さらに、再び$C$のコンパクト性から$C$上一様に$f_n'/f_n \to p'/p$であることも示せる。なぜならば!$\ve > 0$が与えられたとする。$\ve \leq 1$としてよい。仮定から
  \[
  n \geq N \to \norm{f_n' - p'}_C < \ve , \quad \norm{1/f_n - 1/p}_C < \ve
  \]
  なる正整数$N$がとれる。$M := \sup_{z \in C} \abs{p'(z)}$とおく。$C$はコンパクトなので$M < \infty$である。このとき$n \geq N$なら
  \begin{align*}
    \norm{ \f{f_n'}{f_n} - \f{p'}{p} }_C &= \max_{z \in C} \abs{ f_n'(z) \left( \f{1}{f_n(z)} - \f{1}{p(z)} \right)   + \f{1}{p(z)} ( f_n'(z) - p'(z) )  } \\
    &\leq \max_{z \in C} (\abs{p(z) } + \ve )  \norm{ \f{1}{f_n} - \f{1}{p} }_C + \f{1}{m} \norm{f_n' - p'}_C \\
    &\leq (M + \ve)\ve + \f{  \ve }{ m} \\
    &\leq (M + 1 + 1/m)\ve
  \end{align*}
  という評価ができる。$M$も$m$も$\ve$によらないので、これで$C$上一様に$f_n'/f_n$が$p'/p$に収束することがわかる。ゆえにかくのごとし。したがって積分と極限の交換ができることになり
  \[
  \lim_{n \to \infty} \f{1}{2 \pi i} \int_C \f{f_n'(z)}{ f(z)} \ dz = \f{1}{2 \pi i} \int_C \f{p'(z)}{ p(z)} \ dz
  \]
  がわかる。左辺は$f_n$が零点を持たないという仮定から、偏角の原理によりゼロである。よって右辺もゼロなので、$p$は閉曲線$C$の内部にゼロ点を持たない。よって$p(z_0 ) \neq 0$である。$z_0 \in \C$は任意だったから、これで示すべきことがいえた。
\end{sol}
