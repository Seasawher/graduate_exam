\documentclass[12pt]{jsarticle}%文字サイズが12ptのjsarticle

%%%%%%%%%%%%%%%%%%%%%%%%%%%%%%%%%%%%%%%%%%%%%%%%%%%%%%%
%%  パッケージ                                        %%
%%%%%%%%%%%%%%%%%%%%%%%%%%%%%%%%%%%%%%%%%%%%%%%%%%%%%%%
%使用しないときはコメントアウトしてください
\usepackage{amsthm}%定理環境
\usepackage{framed}%文章を箱で囲う
\usepackage{amsmath,amssymb}%数式全般
\usepackage[dvipdfmx]{graphicx}%図の挿入
%\usepackage{tikz}%描画
\usepackage{titlesec}%見出しの見た目を編集できる
\usepackage[dvipdfmx, usenames]{color}%色をつける
%\usepackage{tikz-cd}%可換図式
%\usepackage{mathtools}%数式関連
\usepackage{amsfonts}%数式のフォント
\usepackage[all]{xy}%可換図式
\usepackage{mathrsfs}%花文字
\usepackage{comment}%コメント環境
\usepackage{picture}%お絵かき
\usepackage{url}%URLを出力
\usepackage[dvipdfmx]{hyperref}
\usepackage{pxjahyper}%日本語しおりの文字化けを防ぐ

%%%%%%%%%%%%%%%%%%%%%%%%%%%%%%%%%%%%%%%%%%%%%%%%%%%%%%%
%%  表紙                                             %%
%%%%%%%%%%%%%%%%%%%%%%%%%%%%%%%%%%%%%%%%%%%%%%%%%%%%%%%
\makeatletter
\def\thickhrulefill{\leavevmode \leaders \hrule height 1pt\hfill \kern \z@}
\renewcommand{\maketitle}{\begin{titlepage}%
    \let\footnotesize\small
    \let\footnoterule\relax
    \parindent \z@
    \reset@font
    \null\vfil
    \begin{flushleft}
      \huge \@title
    \end{flushleft}
    \par
    \hrule height 4pt
    \par
    \begin{flushright}
      \LARGE \@author \par
    \end{flushright}
    \vskip 60\p@
    \vfil\null
    \begin{flushright}
        {\small \@date}%
    \end{flushright}
  \end{titlepage}%
  \setcounter{footnote}{0}%
}
\makeatother

%%%%%%%%%%%%%%%%%%%%%%%%%%%%%%%%%%%%%%%%%%%%%%%%%%%%%%%
%%  sectionの修飾                                     %%
%%%%%%%%%%%%%%%%%%%%%%%%%%%%%%%%%%%%%%%%%%%%%%%%%%%%%%%
\titleformat{\section}[block]
{}{}{0pt}
{
  \colorbox{black}{\begin{picture}(0,10)\end{picture}}
  \hspace{0pt}
  \normalfont \Large\bfseries
  \hspace{-4pt}
}
[
\begin{picture}(100,0)
  \put(3,18){\color{black}\line(1,0){300}}
\end{picture}
\\
\vspace{-30pt}
]

%%%%%%%%%%%%%%%%%%%%%%%%%%%%%%%%%%%%%%%%%%%%%%%%%%%%%%%
%%  太字section                                     %%
%%%%%%%%%%%%%%%%%%%%%%%%%%%%%%%%%%%%%%%%%%%%%%%%%%%%%%%
\newcommand{\bfsubsection}[1]{\subsection*{\textbf{#1}}}
\newcommand{\bfsection}[1]{\section*{\textbf{#1}}}

%%%%%%%%%%%%%%%%%%%%%%%%%%%%%%%%%%%%%%%%%%%%%%%%%%%%%%%
%%  番号付き定理環境                                  %%
%%%%%%%%%%%%%%%%%%%%%%%%%%%%%%%%%%%%%%%%%%%%%%%%%%%%%%%
%注:defというコマンドはもうある
\theoremstyle{definition}%定理環境のアルファベットを斜体にしない
\renewcommand{\proofname}{\textgt{証明}}%proof環境の修正

%%%%%%%%%%%%%%%%%%%%%%%%%%%%%%%%%%%%%%%%%%%%%%%%%%%%%%%
%%  番号なし定理環境                                  %%
%%%%%%%%%%%%%%%%%%%%%%%%%%%%%%%%%%%%%%%%%%%%%%%%%%%%%%%
%一部箱付き
\newtheorem*{lemma}{補題}
\newtheorem*{proposition}{命題}
\newtheorem*{definition}{定義}
\newcommand{\lem}[1]{\begin{oframed} \begin{lemma} #1 \end{lemma} \end{oframed}}%箱付きほだい
\newcommand{\prop}[1]{\begin{oframed} \begin{proposition} #1 \end{proposition} \end{oframed}}%箱付きめいだい

\newtheorem*{claim}{主張}
\newtheorem*{sol}{解答}
\newtheorem*{prob}{問題}
\newtheorem*{quo}{引用}
\newtheorem*{rem}{注意}



%%%%%%%%%%%%%%%%%%%%%%%%%%%%%%%%%%%%%%%%%%%%%%%%%%%%%%%
%%  左側に線を引く                                  %%
%%%%%%%%%%%%%%%%%%%%%%%%%%%%%%%%%%%%%%%%%%%%%%%%%%%%%%%
%leftbar環境の定義
\makeatletter
\renewenvironment{leftbar}{%
%  \def\FrameCommand{\vrule width 3pt \hspace{10pt}}%  デフォルトの線の太さは3pt
  \renewcommand\FrameCommand{\vrule width 1pt \hspace{10pt}}%
  \MakeFramed {\advance\hsize-\width \FrameRestore}}%
 {\endMakeFramed}
\newcommand{\exbf}[2]{ \begin{leftbar} \textbf{#1} #2 \end{leftbar} }%左線つき太字
%\newcommand{\barquo}[1]{\begin{leftbar} \begin{quo} #1 \end{quo} \end{leftbar}}%左線つき引用%びふぉあ
\newcommand{\barquo}[1]{\begin{leftbar} \noindent #1  \end{leftbar}}%左線つき引用%あふたー
\newcommand{\lbar}[1]{\begin{leftbar} #1 \end{leftbar}}



%%%%%%%%%%%%%%%%%%%%%%%%%%%%%%%%%%%%%%%%%%%%%%%%%%%%%%%
%%  色をつける                                      %%
%%%%%%%%%%%%%%%%%%%%%%%%%%%%%%%%%%%%%%%%%%%%%%%%%%%%%%%
\newcommand{\textblue}[1]{\textcolor{blue}{\textbf{#1}}}

%%%%%%%%%%%%%%%%%%%%%%%%%%%%%%%%%%%%%%%%%%%%%%%%%%%%%%%
%%  よく使う記号の略記                                 %%
%%%%%%%%%%%%%%%%%%%%%%%%%%%%%%%%%%%%%%%%%%%%%%%%%%%%%%%
\newcommand{\setmid}[2]{\left\{ #1 \mathrel{} \middle| \mathrel{} #2 \right\}}%集合の内包記法
\newcommand{\sm}{\setminus}%集合差
\newcommand{\abs}[1]{\left \lvert #1 \right \rvert}%絶対値
\newcommand{\norm}[1]{\left \lVert #1 \right \rVert}%ノルム
\newcommand{\transpose}[1]{\, {\vphantom{#1}}^t\!{#1}}%行列の転置
\newcommand{\pmat}[1]{ \begin{pmatrix} #1 \end{pmatrix} }%まるかっこ行列
\newcommand{\f}[2]{\frac{#1}{#2}}%分数
\newcommand{\kakko}[1]{ \langle #1  \rangle}%鋭角かっこ%\angleはもうある
\newcommand{\I}{\sqrt{-1}}%虚数単位。\iは既にある。
\newcommand{\single}{\{ 0 \}}%0のシングルトン
\newcommand{\clsub}{\subset_{\text{closed}}}%閉部分集合
\newcommand{\opsub}{\subset_{\text{open}}}%開部分集合
\newcommand{\clirr}{\subset_{\text{closed irr}}}%閉既約部分集合
\newcommand{\loc}{\subset_{\text{loc. closed}}}%局所閉部分集合
\newcommand{\wt}[1]{\widetilde{#1}}%わいどちるだあ
\newcommand{\ol}[1]{\overline{#1}}%オーバーライン
\newcommand{\wh}[1]{\widehat{#1}}%ワイドハット
\newcommand{\To}{\Rightarrow}%ならば%自然変換
\newcommand{\xto}[1]{\xrightarrow}%上側文字付き右向き矢印
\newcommand{\st}{\; \; \text{s.t.} \; \;}%空白付きsuch that
\newcommand{\ts}{\otimes}%テンソル積
\newcommand{\tm}{\times}%直積
\newcommand{\vartm}{\times^{\text{Var}}}%多様体の圏における直積。集合の直積と区別するとき用。
\newcommand{\la}{\overleftarrow}%上付き左矢印
\newcommand{\ra}{\overrightarrow}%上付き右矢印
\newcommand{\del}{\partial}%偏微分の記号



%%%%%%%%%%%%%%%%%%%%%%%%%%%%%%%%%%%%%%%%%%%%%%%%%%%%%%%
%%       演算子                                       %%
%%%%%%%%%%%%%%%%%%%%%%%%%%%%%%%%%%%%%%%%%%%%%%%%%%%%%%%
%log型
\DeclareMathOperator{\rank}{rank}%行列の階数
\DeclareMathOperator{\rk}{rk}%行列の階数
\DeclareMathOperator{\corank}{corank}%行列の核の次元
\renewcommand{\Re}{\operatorname{Re}}%実部
\DeclareMathOperator{\Res}{Res}%留数
\DeclareMathOperator{\Gal}{Gal}%Galois群
\DeclareMathOperator{\Hom}{Hom}%射の集合
\DeclareMathOperator{\ind}{Ind}
\DeclareMathOperator{\tr}{Trace}%トレース
\DeclareMathOperator{\Aut}{Aut}%自己同型群
\DeclareMathOperator{\trdeg}{tr\text{.}deg}%超越次数
\DeclareMathOperator{\Frac}{Frac}%商体をとる操作
\renewcommand{\Im}{\operatorname{Im}}%写像の像。Abel圏の像対象。虚部が出力できなくなった。
\DeclareMathOperator{\Ker}{Ker}%写像の核。Abel圏の核対象。
\DeclareMathOperator{\im}{im}%写像の像
\DeclareMathOperator{\coker}{coker}%余核%対象のほう
\DeclareMathOperator{\Coker}{Coker}%余核%射のほう
\DeclareMathOperator{\Spec}{Spec}%スペクトル
\DeclareMathOperator{\Sing}{Sing}%Singular point.特異点の集合。歌ってるわけではないぞ
\DeclareMathOperator{\Supp}{Supp}%台
\DeclareMathOperator{\ann}{ann}%アナイアレーター
\DeclareMathOperator{\Ass}{Ass}%素因子
\DeclareMathOperator{\ord}{ord}%おーだー
\DeclareMathOperator{\height}{ht}%素イデアルの高度。\htはもうある
\DeclareMathOperator{\coht}{coht}%素イデアルの余高度
\DeclareMathOperator{\Lan}{Lan}%左Kan拡張
\DeclareMathOperator{\Ran}{Ran}%右Kan拡張



%limit型
\DeclareMathOperator*{\llim}{\varprojlim}%極限。逆極限。射影極限。
\DeclareMathOperator*{\rlim}{\varinjlim}%余極限。順極限。入射極限。

%%%%%%%%%%%%%%%%%%%%%%%%%%%%%%%%%%%%%%%%%%%%%%%%%%%%%%%
%%  黒板太字(blackboard bold)                         %%
%%%%%%%%%%%%%%%%%%%%%%%%%%%%%%%%%%%%%%%%%%%%%%%%%%%%%%%
\newcommand{\bba}{{\mathbb A}}
\newcommand{\bbb}{{\mathbb B}}
\newcommand{\bbc}{{\mathbb C}}
\newcommand{\bbd}{{\mathbb D}}
\newcommand{\bbe}{{\mathbb E}}
\newcommand{\bbf}{{\mathbb F}}
\newcommand{\bbg}{{\mathbb G}}
\newcommand{\bbh}{{\mathbb H}}
\newcommand{\bbi}{{\mathbb I}}
\newcommand{\bbj}{{\mathbb J}}
\newcommand{\bbk}{{\mathbb K}}
\newcommand{\bbl}{{\mathbb L}}
\newcommand{\bbm}{{\mathbb M}}
\newcommand{\bbn}{{\mathbb N}}
\newcommand{\bbo}{{\mathbb O}}
\newcommand{\bbp}{{\mathbb P}}
\newcommand{\bbq}{{\mathbb Q}}
\newcommand{\bbr}{{\mathbb R}}
\newcommand{\bbs}{{\mathbb S}}
\newcommand{\bbt}{{\mathbb T}}
\newcommand{\bbu}{{\mathbb U}}
\newcommand{\bbv}{{\mathbb V}}
\newcommand{\bbw}{{\mathbb W}}
\newcommand{\bbx}{{\mathbb X}}
\newcommand{\bby}{{\mathbb Y}}
\newcommand{\bbz}{{\mathbb Z}}

%%%%%%%%%%%%%%%%%%%%%%%%%%%%%%%%%%%%%%%%%%%%%%%%%%%%%%%
%%  よく使う黒板太字                                  %%
%%%%%%%%%%%%%%%%%%%%%%%%%%%%%%%%%%%%%%%%%%%%%%%%%%%%%%%
\newcommand{\Z}{\bbz}
\newcommand{\A}{\bba}
\newcommand{\Q}{\bbq}
\newcommand{\R}{\bbr}
\newcommand{\C}{\bbc}
\newcommand{\F}{\bbf}
\newcommand{\N}{\bbn}
\renewcommand{\P}{\bbp}%パラグラフ記号が出力できなくなった

%%%%%%%%%%%%%%%%%%%%%%%%%%%%%%%%%%%%%%%%%%%%%%%%%%%%%%%
%%  カリグラフィー                                %%
%%%%%%%%%%%%%%%%%%%%%%%%%%%%%%%%%%%%%%%%%%%%%%%%%%%%%%%
%大文字しかどうせ使わない
\newcommand{\cala}{\mathcal{A}}
\newcommand{\calb}{\mathcal{B}}
\newcommand{\calc}{\mathcal{C}}
\newcommand{\cald}{\mathcal{D}}
\newcommand{\calf}{\mathcal{F}}
\newcommand{\calo}{\mathcal{O}}

%%%%%%%%%%%%%%%%%%%%%%%%%%%%%%%%%%%%%%%%%%%%%%%%%%%%%%%
%%  ギリシャ文字(Greek letters)小文字                 %%
%%%%%%%%%%%%%%%%%%%%%%%%%%%%%%%%%%%%%%%%%%%%%%%%%%%%%%%
%コマンドが5字以上のもの
\newcommand{\gra}{{\alpha}}
\newcommand{\grg}{{\gamma}}
\newcommand{\grd}{{\delta}}
\newcommand{\gre}{{\epsilon}}
\newcommand{\grt}{{\theta}}
\newcommand{\grk}{{\kappa}}
\newcommand{\grl}{{\lambda}}
\newcommand{\grs}{{\sigma}}
\newcommand{\gru}{{\upsilon}}
\newcommand{\gro}{{\omega}}

\newcommand{\ve}{{\varepsilon}}
\newcommand{\vp}{{\varphi}}

%%%%%%%%%%%%%%%%%%%%%%%%%%%%%%%%%%%%%%%%%%%%%%%%%%%%%%%
%%  ギリシャ文字(Greek letters)大文字                 %%
%%%%%%%%%%%%%%%%%%%%%%%%%%%%%%%%%%%%%%%%%%%%%%%%%%%%%%%
%コマンドが5字以上のもの
\newcommand{\grG}{{\Gamma}}
\newcommand{\grD}{{\Delta}}
\newcommand{\grT}{{\Theta}}
\newcommand{\grL}{{\Lambda}}
\newcommand{\grS}{{\Sigma}}
\newcommand{\grU}{{\Upsilon}}
\newcommand{\grO}{{\Omega}}

%%%%%%%%%%%%%%%%%%%%%%%%%%%%%%%%%%%%%%%%%%%%%%%%%%%%%%%
%%  フラクトゥール                                  %%
%%%%%%%%%%%%%%%%%%%%%%%%%%%%%%%%%%%%%%%%%%%%%%%%%%%%%%%
\newcommand{\fraka}{\mathfrak{a}}
\newcommand{\frakb}{\mathfrak{b}}
\newcommand{\frakm}{\mathfrak{m}}
\newcommand{\frakp}{\mathfrak{p}}

\newcommand{\frakA}{\mathfrak{A}}
\newcommand{\frakB}{\mathfrak{B}}
\newcommand{\frakT}{\mathfrak{T}}

\newcommand{\Top}{\mathfrak{Top}}%開部分集合全体のなす有向集合
\newcommand{\Ab}{\mathfrak{Ab}}%Abel群のなす圏

%%%%%%%%%%%%%%%%%%%%%%%%%%%%%%%%%%%%%%%%%%%%%%%%%%%%%%%
%%  花文字                                          %%
%%%%%%%%%%%%%%%%%%%%%%%%%%%%%%%%%%%%%%%%%%%%%%%%%%%%%%%
%大文字しかどうせ使わない
\newcommand{\scra}{\mathscr{A}}
\newcommand{\scrf}{\mathscr{F}}
\newcommand{\scrg}{\mathscr{G}}
\newcommand{\scrh}{\mathscr{H}}
\newcommand{\scrs}{\mathscr{S}}

%%%%%%%%%%%%%%%%%%%%%%%%%%%%%%%%%%%%%%%%%%%%%%%%%%%%%%%
%%  太字                                            %%
%%%%%%%%%%%%%%%%%%%%%%%%%%%%%%%%%%%%%%%%%%%%%%%%%%%%%%%
\newcommand{\Sh}{\textbf{Sh}}%層の圏
\newcommand{\PSh}{\textbf{PSh}}%前層の圏
\newcommand{\bfzero}{\textbf{0}}%太字のゼロ

%小文字
\newcommand{\bfb}{\textbf{b}}
\newcommand{\bfv}{\textbf{v}}
\newcommand{\bfx}{\textbf{x}}
\newcommand{\bfy}{\textbf{y}}


%大文字
\newcommand{\bfC}{\textbf{C}}
\newcommand{\bfD}{\textbf{D}}
\newcommand{\bfE}{\textbf{E}}


\begin{document}

\title{半直積とGalois群}
\author{\url{https://seasawher.hatenablog.com/} \\ @seasawher}
\date{\today}
\maketitle


\prop{
(直積の内部特徴づけ)\\
群$G$とその部分群$N$, $H$があるとする。このとき次は同値。
\begin{description}
  \item[(1)] $G$と直積$N \tm H$は自然に同型である。つまり積をとる写像$\vp \colon N \tm H \to G$は群の準同型であって、かつ同型になる。次の図式
\[
\xymatrix{
{} & N \tm H \ar[d]^-{\vp} & {} \\
N \ar[ru] \ar[r] & G &  H \ar[l] \ar[lu]
}
\]
を可換にするような同型$\vp$があるといってもよい。
\item[(2)] $N \lhd G$かつ$H \lhd G$であり、かつ$N \cap H = 1$で$NH = G$である。
\end{description}
}
\begin{proof} ${}$
  \begin{description}
    \item[(1)$\To$(2)] $q \in N$, $g \in G$が与えられたとする。($n \in N$としないのは、$n$と$h$の形が似ていて間違えやすいため) $\vp$の逆写像$\psi$をとっておく。すると、$\psi(g) = (g_N, g_H)$と表せる。ゆえに
    \begin{align*}
      \psi(gqg^{-1}) &= \psi(g) (q,1) \psi(g)^{-1} \\
      &= (g_N,g_H) (q,1) (g_N^{-1}, g_H^{-1}) \\
      &= ( g_N q g_N^{-1} , 1)
    \end{align*}
    である。したがって$gqg^{-1} = \vp(g_N q g_N^{-1} , 1) \in N$であるから、$N \lhd G$である。同様にして$H \lhd G$もいえる。また、$x \in N \cap H$とすると、$\psi(x) \in N \tm H$は$(1,1)$でなくてはならない。したがって、$x \in \Ker \psi$である。$\psi$は同型だから$x=1$であって、$N \cap H =1$がいえた。
    さらに、$G = NH$であることはあきらかであろう。
    \item[(2)$\To$(1)] $N$, $H$は$G$の部分群なので、積をとる写像$\vp \colon N \tm H \to G$が定義できる。$N \lhd G$, $H \lhd G$なので交換子$[N,H]$は$N \cap H$の部分群であるが、$N \cap H = 1$なので$[N,H] = 1$である。よって$N$の元と$H$の元は可換であり、$\vp$は群準同型になる。
    $N \cap H = 1$より$\vp$は単射であり、$NH=G$より$\vp$は全射である。
  \end{description}
\end{proof}


\lem{
(半直積の基本的な性質) \\
群$N$, $H$と群作用$\Phi \colon  H \to \Aut N$があって、半直積$N \rtimes_{\Phi} H$を考えているとする。$q \in N$, $h \in H$とする。このとき次が成り立つ。
\begin{description}
  \item[(1)] 作用成分への射影$N \rtimes_{\Phi} H \to H \st (q,h) \mapsto h$は準同型である。
  \item[(2)] 正規成分への入射$N \to N \rtimes_{\Phi} H \st q \mapsto (q,1)$は準同型である。
  \item[(3)] 作用成分への入射$H \to N \rtimes_{\Phi} H \st h \mapsto (1,h)$は準同型である。
  \item[(4)] $h \in \Ker \Phi$ならば$(q,h) = (1,h)(q,1)$である。
  \item[(5)] 常に$(q,h) = (q,1)(1,h)$が成り立つ。
  \item[(6)] 自然な入射と射影は、分裂する短完全列
  \[
  \xymatrix{
  1 \ar[r] & N \ar[r] & N \rtimes_{\Phi} H \ar[r] & H \ar[r] & 1 \\
  {} & {} & H \ar[u] \ar[ru]_-{1} & {} & {}
  }
  \]
  をなす。
\end{description}
}
\begin{proof}
  あきらか。
\end{proof}




\prop{
(半直積の内部特徴づけ)\\
群$G$の部分群$N$, $H$が与えられているとする。このとき次は同値。
\begin{description}
  \item[(1)] ある群作用$\Phi \colon H \to \Aut N$が存在して、$G$は半直積$N \rtimes_{\Phi} H$と自然に同型である。つまり積をとる写像$\vp \colon N \rtimes_{\Phi} H \to G \st (q,h) \mapsto qh$は群準同型で、かつ同型である。次の図式
\[
\xymatrix{
{} & N \rtimes_{\Phi} H \ar[d]^-{\vp} & {} \\
N \ar[ru] \ar[r] & G &  H \ar[l] \ar[lu]
}
\]
を可換にするような同型$\vp$があるといってもよい。
  \item[(2)] $N \lhd G$かつ$NH=G$かつ$N \cap H =1$が成り立つ。
\end{description}
}
\begin{proof} ${}$
  \begin{description}
    \item[(1)$\To$(2)] $NH=G$はあきらか。$x \in N \cap H$とすると$(x,x^{-1}) \in \Ker \vp$だから$x =1$でなくてはならない。よって$N \cap H = 1$である。
    $N \lhd G$を示そう。$g \in G$と$q \in N$が与えられたとする。$p \colon N \rtimes_{\Phi} H \to H$を射影とし、$\psi$を$\vp$の逆写像とする。
    このとき$\psi(g)= (g_N, g_H)$と表せる。ゆえに
    \begin{align*}
      p \circ \psi (gqg^{-1}) &= p( (g_N,g_H) (q,1) (g_N^{-1}, g_H^{-1}) ) \\
      &= 1
    \end{align*}
    である。したがって$gqg^{-1} \in \vp(\Ker p)=N$である。よって$N \lhd G$がわかった。
    \item[(2)$\To$(1)] $N \lhd G$より、群作用$\Phi \colon H \to \Aut N$を$\Phi_h(q) = hqh^{-1}$により定めることができる。(順序を逆にして$\Phi_h(q) = h^{-1}qh$とするとうまくいかないことに注意) このとき$q_1,q_2 \in N$と$h_1, h_2 \in H$が与えられたとすれば
    \begin{align*}
      \vp((q_1,h_1) (q_2,h_2) ) &= \vp( q_1 \Phi_{h_1}(q_2), h_1h_2 ) \\
      &= \vp(q_1 h_1 q_2 h_1^{-1} , h_1 h_2) \\
      &= q_1 h_1 q_2 h_2 \\
      &= \vp(q_1,h_1) \vp(q_2,h_2)
    \end{align*}
    だから$\vp$は群準同型になる。$\vp$が単射であることは$N \cap H =1$より従い、全射であることは$NH = G$より従う。
  \end{description}
\end{proof}



\prop{
(半直積の関手性 その1) \\
$N_1, N_2,H$が群で群作用$\Phi \colon H \to \Aut N_1$が与えられていたとする。このとき同型$g \colon N_1 \to N_2$に対して${}_g\Phi \colon H \to \Aut N_2$を${}_g\Phi(h) = g \circ \Phi_h \circ g^{-1}$で定めると、
写像$g_* \colon N_1 \rtimes_{\Phi} H \to N_2 \rtimes_{{}_g\Phi} H \st g_*(q, h ) = (g(q), h)$は群の準同型である。
}
\begin{proof}
  計算すればわかる。実際に行ってみると
  \begin{align*}
    g_*((q, h_1)(q',h_2) ) &= g_*(q \Phi_{h_1}(q'), h_1h_2) \\
    &= (g(q) g(\Phi_{h_1}(q') ) , h_1h_2) \\
    (g(q), h_1) (g(q'), h_2) &= ( g(q) {}_g\Phi_{h_1} (g(q')) ,h_1h_2   ) \\
    &= (g(q) g(\Phi_{h_1}(q') ) , h_1h_2)
  \end{align*}
  であるから一致する。
\end{proof}



\prop{
(半直積の関手性 その2) \\
$N$, $H_1$, $H_2$が群で群作用$\Phi \colon H_2 \to \Aut N$が与えられていたとする。このとき群準同型$f \colon H_1 \to H_2$に対して$\Phi_f \colon H_1 \to \Aut N$を$(\Phi_f)_h = \Phi_{f(h)}$により定める。
そうすると写像$f_* \colon N \rtimes_{\Phi_f} H_1 \to  N \rtimes_{\Phi} H_2 \st f_*(q,h) = (q,f(h))$は群の準同型である。
}
\begin{proof}
  計算すればわかる。実際に行ってみると
  \begin{align*}
    f_* ( (q_1,h)(q_2,h') ) &= f_*( q_1 \Phi_{f(h)}  (q_2), h h') \\
    &=( q_1 \Phi_{f(h)}  (q_2), f(h) f(h') ) \\
    &= f_*(q_1, h) f_*(q_2, h')
  \end{align*}
  であるから一致。
\end{proof}



\prop{
(分裂する完全列からの半直積の構成) \\
群$G$, $H$, $N$と準同型$i$, $j$, $p$からなる分裂する短完全列
\[
\xymatrix{
1 \ar[r] & N \ar[r]^-j & G \ar[r]^-p & H \ar[r] & 1 \\
{} & {} & H \ar[u]^-i \ar[ru]_-{1} & {} & {}
}
\]
が与えられたとする。このとき、ある群作用$\Psi \colon H \to \Aut N$が存在して、自然な同型$G \cong N \rtimes_{\Psi} H$がある。すなわち、ある同型$\psi$が存在して次の図式
\[
\xymatrix{
{} & N \rtimes_{\Psi} H \ar[d]^-{\psi} & {} \\
N \ar[r]^-j \ar[ru] & G & H \ar[l]_-i \ar[lu]
}
\]
が可換になる。
}
\begin{proof}
  $N' = j(N)$, $H'=i(H)$とおく。このとき$N' = \Ker p$より$N' \lhd G$である。$x \in N' \cap H'$とすると$x = j(q) = i(h)$なる$q \in N, h \in H$があるが、$p(x) = 1 = h$より$x = 1$でなくてはならない。よって$N' \cap H' = 1$である。また$g \in G$とすると$g (i \circ p)(g^{-1}) \in \Ker p$なので$g (i \circ p)(g^{-1}) = j(q)$なる
  $q \in N$がある。したがって$g = j(q) (i \circ p)(g) \in N'H'$だから$G = N'H'$が成り立つ。よって、ある同型$\vp$と群作用$\Phi \colon H' \to \Aut N'$であって、次の図式
  \[
  \xymatrix{
  {} & N' \rtimes_{\Phi} H' \ar[d]^-{\vp} & {} \\
  N' \ar[ru] \ar[r] & G &  H' \ar[l] \ar[lu]
  }
  \]
  を可換にするようなものがある。ここで$i,j$は単射であるので、同型$I \colon H \to H'$と$K \colon N' \to N$が存在して、次の図式
  \[
  \xymatrix{
  N \ar[r] \ar[d]^-{K^{-1}} & N \rtimes_{ {}_K \Phi_I } H  \ar[d]^-{K_*^{-1}} & H \ar[l] \ar@{=}[d] \\
  N' \ar[r] \ar@{=}[d] & N' \rtimes_{\Phi_I } H  \ar[d]^-{I_*} & H \ar[l] \ar[d]^-{I} \\
  N' \ar[r] \ar@{=}[d] & N' \rtimes_{\Phi } H'  \ar[d]^-{\vp} & H' \ar[l] \ar@{=}[d] \\
  N' \ar[r] \ar[d]^-K & G  \ar@{=}[d] & H' \ar[l] \ar[d]^{I^{-1}} \\
  N \ar[r]^-j  & G   & H \ar[l]_-i
   }
  \]
  は可換になる。これで示すべきことがいえた。
\end{proof}


\prop{
(有限巡回群の半直積の表示) \\
群$N$, $H$は有限巡回群であり群作用$\Phi \colon H \to \Aut N$が存在して半直積$N \rtimes_{\Phi} H$を考えているとする。$N$, $H$の生成元$q,h$をそれぞれとって固定し$\Phi_h(q) = q^t$となる$t \in \Z$をとることができる。このとき
\[
N \rtimes_{\Phi} H \cong \setmid{q,h}{q^{\# N} = h^{\# H} = 1, hqh^{-1} = q^t}
\]
が成り立つ。
}
\begin{proof}
  右辺の群を$G$とおく。自由群の普遍性により、自由群$F_2$から$N \rtimes_{\Phi} H$への準同型$\vp$であって$\vp(q) = (q,1)$かつ$\vp(h)=(1,h)$なるものがある。なお、ここで$q \in F_2$と$q \in N$は本来別の記号で書くべきだが、かえって煩雑になるので同じ記号とした。$\vp$は全射である。このとき$q^{\# N}, h^{\# H} \in \Ker \vp$はあきらか。また
  \begin{align*}
    \vp(hqh^{-1}) &= (1,h)(q,1)(1,h^{-1}) \\
    &= (\Phi_h(q),1) \\
    &= (q^t,1) \\
    &= \vp(q)^t
  \end{align*}
  だから$hqh^{-1}q^{-t} \in \Ker \vp$である。したがって、全射$\psi \colon G \to N \rtimes_{\Phi} H$が誘導される。ここで$N \rtimes_{\Phi} H$の位数は$\# (N \tm H)$であるので$\# G \geq \# (N \tm H)$である。
  一方で$\# G \leq \# (N \tm H)$はあきらかなので結局$\# G = \# (N \tm H)$であり、$\psi$は同型でなくてはならない。
\end{proof}


\prop{
(半直積とGalois群) \\
有限次Galois拡大$L/K$があり、その中間体$M,N$があって$L = M \cdot N$かつ$K = M \cap N$を満たすとする。
\[
\xymatrix{
{} & L & {} \\
M \ar[ru]^{Gal} & { } & N \ar[lu]_{Gal} \\
{} & K \ar[lu]_{Gal} \ar[ru] & {}
}
\]
さらに$M/K$はGalois拡大であるとする。このとき
\[
\Gal(L/K) \cong \Gal(L/M) \rtimes \Gal(L/N)
\]
が成り立つ。
}
\begin{proof}
  $M/K$はGalois拡大なので$\Gal(L/M) \lhd \Gal(L/K)$である。$L$は$M$と$N$の合成なので$\Gal(L/M) \cap \Gal(L/N) = 1$である。またGalois拡大の推進定理(雪江\cite{雪江2} 定理4.6.1)により$\Gal(L/N) \cong \Gal(M/K)$なのでとくに$[L:N] = [M:K]$であり、したがって$[L:N][L:M] = [L:K]$である。
  ゆえに、$\Gal(L/K) \cong \Gal(L/M) \rtimes \Gal(L/N)$がわかる。
\end{proof}

\begin{thebibliography}{1}%参考文献の リスト
  \bibitem{雪江2} 雪江明彦『代数学2 環と体とガロア理論』(日本評論社, 2010)
  %\bibitem{松坂} 松坂和夫『集合・位相入門』(岩波書店, 1968)
  %\bibitem{齋藤} 齋藤正彦『線型代数学』(東京図書, 2014)
  %\bibitem{内田} 内田伏一『集合と位相』(裳華房, 1986)
  %\bibitem{松村} 松村英之『可換環論』(共立出版, 1980)
  %\bibitem{雪江3} 雪江明彦『代数学3 代数学のひろがり』(日本評論社, 2011)
  %\bibitem{Harris} Joe Harris『Algebraic Geometry』(Springer, 1992)
  %\bibitem{GW} Ulrich G\"{o}rtz, Torsten Wedhorn『Algebraic Geometry I : Schemes  with Examples and Exercises』(Springer, 2010)
  %\bibitem{Liu} Qing Liu『Algebraic Geometry and Arithmetic Curves』(Oxford University Press, 2002)
  %\bibitem{Wedhorn} Torsten Wedhorn『Manifolds, Sheaves, and Cohomology』(Springer, 2016)
  %\bibitem{Rotman} Joseph J.Rotman『An Introduction to Homological Algebra』(Springer, 2009)
  %\bibitem{MacLane} Saunders Mac Lane『Categories for the Working Mathematician』(Springer, 1971)
  %\bibitem{Riehl} Emily Riehl『Category Theory in Context』(Dover, 2016)
\end{thebibliography}

\end{document}
