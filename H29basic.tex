\bfsection{平成29年度 基礎科目}

\bfsubsection{問1}
\barquo{
次の重積分を求めよ。
\[
\iint_D e^{- \max \{ x^2,y^2 \} } \ dx dy
\]
ここで$D=\setmid{(x,y) \in \R^2}{0 \leq x \leq 1, 0 \leq y \leq 1}$とする。
}
\begin{sol}
  $E = \setmid{(x,y) \in D}{x \geq y}$とおく。このとき
  \begin{align*}
    \iint_D e^{- \max \{ x^2,y^2 \} } \ dx dy &= 2 \iint_E e^{- x^2 } \ dx dy \\
    &= 2 \int_0^1 \left(  \int_0^x e^{-x^2} \ dy   \right) \ dx \\
    &= 2 \int_0^1 x e^{-x^2} \ dx \\
    &= \int_0^1 e^{-z} \ dz &(z = x^2 \text{とおいた}) \\
    &= 1 - e^{-1}
  \end{align*}
\end{sol}


\newpage


\bfsubsection{問2}
\barquo{
実行列
\[
A= \pmat{
1 & -2 & -1 & 1 & 0 \\
-2 & 5 & 3 & -2 & 1 \\
1& 1& 2 &0 &-1 \\
5&  0&  5&  3&  2 \\
}
\]
について、以下の問に答えよ。
\begin{description}
  \item[(i)] 連立一次方程式
  \[
  A \pmat{
  x_1 \\
  x_2 \\
  x_3 \\
  x_4 \\
  x_5
  }
  = \pmat{ 0\\ 0\\ 0\\ 0 }
  \]
  の解をすべて求めよ。
  \item[(ii)]
  連立一次方程式
  \[
  A \pmat{
  x_1 \\
  x_2 \\
  x_3 \\
  x_4 \\
  x_5
  }
  = \pmat{0 \\ -1 \\ 1\\  c}
  \]
  が解を持つような実数$c$をすべて求めよ。
\end{description}
}
\begin{sol} ${}$
  \begin{description}
    \item[(i)] 行列$A$に行基本変形を繰り返し行っていく。
    \begin{align*}
      A &= \pmat{
      1 & -2 & -1 & 1 & 0 \\
      -2 & 5 & 3 & -2 & 1 \\
      1& 1& 2 &0 &-1 \\
      5&  0&  5&  3&  2 \\
      } \\
      &\sim \pmat{
      1 & -2 & -1 & 1 & 0 \\
      0 & 1 & 1 & 0 & 1 \\
      0& 3& 3 &-1 &-1 \\
      0&  10&  10&  -2&  2 \\
      } & \pmat{ R_1 \\ R_2+2R_1 \\ R_3-R_1 \\ R_4-5R_1} \\
     &\sim \pmat{
      1 & 0 & 1 & 1 & 2 \\
      0 & 1 & 1 & 0 & 1 \\
      0& 0& 0 &-1 &-4 \\
      0&  0&  0&  -2&  -8 \\
      } & \pmat{ R_1+2R_2 \\ R_2 \\ R_3-3R_2  \\ R_4 -10R_2} \\
      &\sim \pmat{
        1 & 0 & 1 & 0 & -2 \\
        0 & 1 & 1 & 0 & 1 \\
        0& 0& 0 &1 &4 \\
        0&  0&  0&  0&  0 \\
        } & \pmat{R_1+R_3 \\ R_2 \\ -R_3 \\ R_4-2R_3 }
    \end{align*}
    したがって、$A\bfx=\bfzero$の解空間は$x_3,x_5 \in \R$で貼られる$2$次元実ベクトル空間
    \[
    S = x_3 \pmat{-1 \\-1 \\1 \\0 \\0 } + x_5 \pmat{2\\ -1\\ 0\\ -4\\ 1}
    \]
    である。
    \item[(ii)] 次の事実に気を付ける。
    \prop{
    $k$は体、$A$は$k$係数の$(n,m)$行列であり$\bfx \in k^m, \bfb \in k^n$であるとする。このとき$\bfx$についての一次方程式$A\bfx = \bfb$が解を持つことと、$\rank A = \rank (A \; \bfb)$は同値。
    }
    \begin{proof}
      まず次は同値である。
      \[
        \exists \bfx \; A\bfx = \bfb \iff \exists \bfx \;  \pmat{A & \bfb} \pmat{\bfx \\ -1} = \bfzero
      \]
      ここで、$\Ker A \to \Ker (A \; \bfb) \st \bfx \mapsto {}^t(\bfx \; 0)$によって$\Ker A$は$\Ker (A \; \bfb)$の部分空間$\Ker (A \; \bfb) \cap \setmid{\bfy \in k^{m+1}}{y_{m+1}=0}$だと思えることに気を付けると
      \begin{align*}
      \exists \bfx \; A\bfx = \bfb &\iff \dim \Ker (A \; \bfb) > \dim  \Ker A \\
        &\iff 0 \leq \rank (A \; \bfb) - \rank A < 1 \\
        &\iff \rank A = \rank (A \; \bfb)
      \end{align*}
      であることがわかる。
    \end{proof}
    (ii)の解答に戻る。$\bfb = {}^t(0 \ -1 \;  1 \; c)$とおく。拡大係数行列$(A \; \bfb)$は行基本変形により
    \[
    (A \; \bfb) \sim \pmat{1& 0& 1&  0& -2& 2 \\ 0 &1& 1& 0& 1& -1 \\ 0& 0& 0& 1& 4& -4 \\ 0& 0& 0& 0& 0& c+2}
    \]
    と変形できる。したがって求める$c$の値は$c=-2$である。
  \end{description}
\end{sol}





\newpage





\bfsubsection{問3}
\barquo{
$m,n$を正の整数とし、$A$を複素$(n,m)$行列、$B$を複素$(m,n)$行列とする。複素数$\grl \neq 0$について、以下の問に答えよ。
\begin{description}

  \item[(i)] $\grl$が$BA$の固有値ならば、$\grl$は$AB$の固有値でもあることを示せ。
  \item[(ii)] $\C^m$, $\C^n$の部分空間$V,W$をそれぞれ
\begin{align*}
  V &= \setmid{\bfx \in \C^m}{ \text{ある正の整数$k$に対して} (BA - \grl I_m)^k \bfx = \bfzero \text{が成り立つ} } \\
  W &= \setmid{\bfy \in \C^n}{ \text{ある正の整数$l$に対して} (AB - \grl I_n)^l \bfy = \bfzero \text{が成り立つ} }
\end{align*}
で定める。ただし、$I_m,I_n$は単位行列、$\bfzero$は零ベクトルを表す。このとき、$\dim V = \dim W$であることを示せ。
\end{description}
}
\begin{sol} ${}$
  \begin{description}
    \item[(i)] $BA \bfv = \grl \bfv$なる$\bfv \neq \bfzero$があったとする。このとき
    \begin{align*}
      AB(A \bfv) &= A(BA \bfv) \\
      &= A (\grl \bfv) \\
      &= \grl A \bfv
    \end{align*}
    である。もしも$A \bfv = \bfzero$ならば$\grl \bfv = \bfzero$となり矛盾。したがって$A \bfv \in \C^n$は$AB$の固有ベクトルである。
    \item[(ii)] $M = AB, N = BA$とする。$MA = AN$である。いま$\bfx \in V$とする。ある$k$が存在して$(N - \grl I_m)^k \bfx = \bfzero$である。このとき
    \[
(M - \grl I_n)^k (A \bfx) = A  (N - \grl I_m)^k \bfx = \bfzero
    \]
    であるから$A \bfx \in W$である。したがって行列$A$は線形写像$A \colon V \to W$であるとみなせる。このとき$A$は$V$の定義および$\grl \neq 0$により単射だから、$\dim V \leq \dim W$である。同様にして逆が言えるので$\dim V = \dim W$が従う。
  \end{description}
\end{sol}

\newpage


\bfsubsection{問4}
\barquo{
$f$を$I=\setmid{x \in \R}{x \geq 0}$上の実数値連続関数とする。正の整数$n$に対し、$I$上の関数$f_n$を
\[
f_n(x)=f(x+n)
\]
で定める。関数列$\{  f_n \}_{n=1}^{\infty}$が$I$上で一様収束するとき、以下の問に答えよ。
\begin{description}
\item[(i)] $I$上の関数$g$を
\[
g(x) = \lim_{n \to \infty} f_n(x)
\]
で定める。このとき$g$は$I$上で一様連続であることを示せ。
\item[(ii)] $f$は$I$上で一様連続であることを示せ。
\end{description}
}
\begin{proof} 以下$I$上の連続関数$h$に対してその一様ノルムを$\norm{h} = \sup_{x \in I} \abs{h(x)}$とかく。
  \begin{description}
    \item[(i)] 連続関数$f_n$の一様極限なので$g$は連続である。さらに定義より$g(x+1)=g(x)$だから、$g$はコンパクト集合$\R / \Z$上の連続関数であるとみなせ、したがって一様連続である。
    \item[(ii)] $\ve > 0$が与えられたとする。$g$の一様連続性から
    \[
    \forall x,y \in I \; \abs{x-y}< \grd_0 \to \abs{g(x) - g(y) } < \ve
    \]
    なる$\grd_0 > 0$がある。$f_n$は$g$に一様収束するので
    \[
    n \geq N \to \norm{f_n - g} < \ve
    \]
    なる$N \in \Z$がある。このとき
    \[
    \forall x,y \in [N,\infty) \; \abs{x - y } < \grd_0 \to \abs{f(x) - f(y)} \leq 3\ve
    \]
    が成り立つ。なぜなら
    \[
    \abs{f(x) - f(y)}  \leq \abs{f_N(x-N) - g(x-N)} + \abs{g(x)-g(y)} + \abs{g(y-N)-f_N(y-N)}
    \]
    であるから。また$f$は連続なので、コンパクト集合$[0,N]$上ではとくに一様連続である。したがって
    \[
    \forall x,y \in [0,N] \; \abs{x - y } < \grd_1 \to \abs{f(x) - f(y)} \leq \ve
    \]
    なる$\grd_1 > 0$がある。したがって$\grd= \min_i{\grd_i}$とすると
    \[
    \forall x,y \in I \; \abs{x - y } < \grd \to \abs{f(x) - f(y)} \leq 4\ve
    \]
    であり、これで$f$が$I$上一様連続であることがいえた。
  \end{description}
\end{proof}






\newpage



\bfsubsection{問5}
\barquo{
$p$を正の実数とし、$f(t)$を$\R$上の実数値連続関数で
\[
\int_0^{\infty} \abs{f(t)} \ dt < \infty
\]
を満たすものとする。このとき$\R$上の常微分方程式
\[
\f{dx}{dt} = -px + f(t)
\]
の任意の解$x(t)$に対し$\lim_{t \to \infty} x(t) = 0$が成り立つことを示せ。
}
\begin{proof}
  任意に$\ve > 0$が与えられたとする。仮定により
  \[
  \int_R^{\infty}  \abs{f(t)} \ dt < \ve
  \]
  となるような$R \geq 0$がある。$x = y e^{-pt}$と置いて変数変換をすると
  \[
  \f{dy}{dt} = e^{pt}f(t)
  \]
  となる。よってある定数$C$により
  \[
  y(t) = \int_0^t f(s)e^{ps} \ ds + C
  \]
と表せる。$C$の値は$t \to \infty$での$x$の振る舞いに関与しないので、はじめから$C=0$と仮定してよい。これにより
\[
x(t) =   \int_0^t f(s)e^{p(s-t)} \ ds
\]
であることがわかる。そこで$M =  \int_0^R \abs{f(s)} \ ds $とおき、$ t > \max \{ R, R+ \f{1}{p} \log \f{M}{\ve} \}$とする。このとき
\begin{align*}
  \abs{x(t)} &\leq \int_0^R \abs{f(s)}e^{p(s-t)} \ ds  + \int_R^t \abs{f(s)}e^{p(s-t)} \ ds  \\
  &\leq e^{p(R-t)} M + \int_R^t \abs{f(s)} \ ds  \\
  &\leq \ve  + \int_R^{\infty} \abs{f(s)}  ds \\
  &\leq 2\ve
\end{align*}
である。よって$\lim_{t \to \infty} x(t) = 0$である。
\end{proof}



\newpage





\bfsubsection{問6}
\barquo{
$X,Y$を位相空間とし、直積集合$X \tm Y$を積位相によって位相空間とみなす。写像$f \colon X \tm Y \to Y$を$f(x,y)=y$で定める。$X$がコンパクトならば、$X \tm Y$の任意の閉集合$Z$に対し、$f(Z)$は$Y$の閉集合であることを示せ。
}
\begin{rem}
  $X$がコンパクトという仮定は必要である。例えば、$X=Y=\R$かつ$Z=\setmid{(x,y) \in \R^2}{xy=1}$としてみればわかる。
\end{rem}
\begin{proof}
$Y \sm f(Z)$の元$y$が任意に与えられたとする。このとき$f^{-1}(y) \subset X \tm Y \sm Z$である。ここで$Z$が$X\tm Y$の閉集合という仮定から、$X \tm Y \sm Z \opsub X \tm Y$である。したがって積位相の定義により、ある開集合の族$U_i \subset X$と$V_i \subset Y$であって$X \tm Y \sm Z = \bigcup_{i \in I} U_i \tm V_i$なるものがある。
$f^{-1}(y) = X \tm \{ y \} \cong X$はコンパクトであると仮定したので、ある有限集合$J \subset I$が存在して$X \tm \{y\} = f^{-1}(y) \subset \bigcup_{i \in I} U_i \tm V_i$が成り立つ。

ここで$V = \bigcap_{i \in J} V_i$とおく。$J$は有限集合なので$V$は$Y$の開集合であり、かつ$y$を含む。また$X = \bigcup_{i \in J} U_i$であることより$Z \cap f^{-1}(V) = Z \cap (X \tm V) = Z \cap \bigcup_{i \in J} (U_i \tm V) \subset Z \cap (X \tm Y \sm Z) = \emptyset$となる。
これは$V \cap f(Z) = \emptyset$を意味し、$y$は内点であったことがわかった。よって$f(Z)$は$Y$の閉集合。
\end{proof}


\newpage

\bfsubsection{問7}
\barquo{
$n$を正の整数とし、$\R^n$の2点$x=(x_1, \cdots , x_n)$, $y= (y_1, \cdots , y_n)$の距離$d(x,y)$を
\[
d(x,y) = \sqrt{(x_1 - y_1)^2 + \cdots +  (x_n - y_n)^2}
\]
と定める。$\R^n$の空でない部分集合$A$に対し、関数$f \colon \R^n \to \R$を
\[
f(x) = \inf_{z \in A} d(x,z)
\]
で定めるとき、$\R^n$の任意の2点$x,y$に対して$\abs{f(x)-f(y)} \leq d(x,y)$が成り立つことを示せ。
}
\begin{proof}
  $d(x,0) = \abs{x}$と書くことにする。任意に$\ve > 0$が与えられたとしよう。このとき$f$の定義から、$f(y) + \ve > \abs{y-w} \geq f(y)$なる$w \in A$が存在する。このとき$f(x) \leq \abs{x-w}$が成り立つので、
  \begin{align*}
    f(x) - f(y) - \ve &\leq f(x) - \abs{y-w} \\
    &\leq \abs{x-w} - \abs{y-w} \\
    &\leq \abs{x-y}
  \end{align*}
  である。$\ve > 0$は任意だったので、$f(x) - f(y) \leq \abs{x-y}$でなくてはならない。同様にして$f(x) - f(y) \leq \abs{x-y}$がいえるので、示すべきことがいえた。
\end{proof}
