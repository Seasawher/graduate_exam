\section{平成25年度 基礎数学}

\subsubsection{}%1
\barquo{
$\R^4$に標準的な内積を入れる。$V$を
\[
\pmat{1 \\ -1  \\ -1 \\ 1}, \quad \pmat{1 \\ -1  \\ 1 \\ -1}
\]
で生成される$\R^4$の部分ベクトル空間とする。このとき$V$の$\R^4$における直交補空間$W$の基底を$1$組求めよ。
}
\begin{sol}
  計算すると
  \[
  W = \Ker \pmat{ 1 & -1 & -1 & 1 \\ 1 & -1 & 1 & -1  }
  \]
  の基底としてたとえば
  \[
  \pmat{1 \\ 1  \\ 0 \\ 0}, \quad \pmat{0 \\ 0  \\ 1 \\ 1}
  \]
  がとれることがわかる。
\end{sol}

\newpage

\subsubsection{}%2
\barquo{
$3$次の複素正方行列
\[
A = \pmat{ -4 & -1 &-1 \\ 1& -2&  1 \\ 0& 0& -3}, \quad B = \pmat{ -2 & 1 &0 \\ -1& -4&  1 \\ 0& 0& -3}
\]
を考える。行列$A$と$B$は相似かどうか理由を答えよ。ただし、行列$A$と$B$が相似とは、複素正方行列$P$で$A=P^{-1}BP$をみたすものが存在することをいう。
}
\begin{sol}
  固有多項式は$A$も$B$も$(t+3)^3$になるが、
  \begin{align*}
\rank ((-3)E-A) &= 1 \\
\rank ((-3)E-B) &= 2
  \end{align*}
なので固有空間の次元が異なる。よって$A$と$B$は相似ではない。
\end{sol}

\newpage

\subsubsection{}%3
\barquo{
$\R^2$上の関数$f(x,y ) = (3xy+1)e^{-(x^2 + y^2)}$の最大値が存在することを示し、その最大値を求めよ。
}
\begin{sol}
  $x = r \cos \grt$, $y = r \sin \grt$と変数変換して$g(r, \grt ) = f(x,y)$とおく。このとき$\grt$に関係なく一様に
  \[
  \lim_{r \to \infty} \abs{g(r,\grt)} \leq \lim_{r \to \infty} (\f{3}{2}r^2 + 1)e^{-r^2} = 0
  \]
  だから、ある$R > 0$があって、$r \geq R$のとき$ \abs{g(r,\grt)} \leq 1 = f(0,0)$である。$g$は連続なので$[0,R] \tm [0,2\pi]$上で最大値を持っており、それが$g$および$f$の最大値となる。よって最大値の存在がいえた。

  $g$の停留点をすべて求めよう。方程式
\begin{align*}
\PD{g}{r} &= (3(1-r^2) \sin 2\grt - 2)re^{-r^2} = 0 \\
\PD{g}{\grt} &= (3r^2 \cos 2\grt) e^{-r^2} = 0
\end{align*}
を考える。これを解いて次の解を得る。
\begin{description}
  \item[(1)] $r=0$
  \item[(2)] $r= 1/ \sqrt{3}$, $\sin 2 \grt = 1$
  \item[(3)] $r = \sqrt{5/3}$, $\sin 2 \grt = -1$
\end{description}
それぞれの場合に$g$の値を求めると(1)のとき$g=1$, (2)のとき$g= 3/2 e^{- 1/3}$で、(3)のとき$g= - 3/2 e^{- 5/3}$である。最大値は停留値のなかにあるので、このうち最大のもの、つまり$3/2 e^{- 1/3}$が$g$そして$f$の最大値である。
\end{sol}

\newpage

\subsubsection{}%4
\barquo{
$\gra, \beta$を実数とする。広義積分
\[
\int_1^{\infty} \f{x^{\gra} \log x}{(1+x)^{\beta}} \ dx
\]
が収束するような$\gra, \beta$の範囲を求めよ。
}
\begin{sol}
  $F(x) = x^{\gra} (1+x)^{- \beta} \log x $, $G = x^{\gra - \beta } \log x$とおいたとき
  \[
  \lim_{x \to \infty} \f{F}{G} = \left( \f{x}{1+x} \right)^{\beta} = 1
  \]
  なので$\int_1^{\infty} F \ dx$と$\int_1^{\infty} G \ dx$の収束は同値。そこで$\grg = \gra - \beta$とおいて
  \[
  I = \int_1^{\infty} x^{\grg} \log x \ dx
  \]
  の収束を考えればよい。いま$\grg \geq -1$とすると
  \begin{align*}
    I &\geq \int_e^{\infty} x^{\grg} \log x \ dx \\
    &\geq \int_e^{\infty} x^{\grg}  \ dx
  \end{align*}
  より$I$は発散する。逆に$\grg < -1$としよう。このとき
  \begin{align*}
I &= \f{1}{\grg + 1} \int_e^{\infty} (x^{\grg + 1})' \log x \ dx \\
&= - \f{1}{\grg + 1} \int_e^{\infty} x^{\grg + 1} \ dx
  \end{align*}
  より$I$は収束する。まとめると、$\grg = \gra - \beta$としたとき、$\grg \geq -1$なら発散で$\grg < -1$なら収束。
\end{sol}
