\section{平成24年度 数学II}

\subsubsection{}%1
\barquo{
$1$以上の整数$a$に対して$K = \Q( \s{3 + a \s{5}})$が$\Q$のGalois拡大体となるものを求め、そのような$a$に対してGalois群$\Gal(K/\Q)$を求めよ。
}
\begin{sol}
  以下、この解答では$[\Q(\s{2},\s{5}): \Q] = 4$は認めて使う。

  $\beta = \s{3 + a \s{5}}$とおく。$\beta$は$f(X) = X^4 - 6X^2 + 9 -5a^2 \in \Z[X]$の根である。$\beta$の共役元を調べたいので、$f$の既約性をいいたい。そのために$[K:\Q]=4$を示そう。$M := \Q(\s{5})$とおく。$[K:M]=2$を示せば十分である。

  ハイリホーで$[K:M]=2$を示そう。仮にそうでないとする。このとき$\beta \in M$である。$\beta^2 = 3 + a\s{5}$より、
  \[
  \Norm_{M/\Q} (\beta)^2 = 9 - 5a^2
  \]
  である。$\beta \in M$は$\Z$上整なので、$\Norm_{M/\Q} (\beta) \in \Z$であり、したがって$9 - 5a^2 \in \Z$は平方数である。これは$a=1$でなくてはならないことを意味する。このとき、そもそも
  \[
  \beta = \s{3 + \s{5}} = \f{ 1 + \s{5}}{ \s{2}}
  \]
  だから$\s{2} \in M$ということになる。これは矛盾。したがって$[K:M]=2$であり、とくに$f \in \Q[X]$は既約多項式である。

  よって$\grg := \s{3 - a\s{5}}$としたとき$\beta$の$\Q$上の共役元は$\{ \pm \beta, \pm \grg \}$である。ゆえに$K$の$\Q$上のGalois閉包を$L$とおくと$L = \Q(\beta, \grg) = K(\s{9-5a^2})$である。これにより$a \geq 2$のとき$K \neq L$なので$K/\Q$はGalois拡大ではない。
  よって$K/\Q$がGalois拡大になるのは$a=1$のときである。

  $a=1$とすると先述のように$\beta = (1+\s{5})/\s{2}$なので$\Q(\s{2},\s{5}) \subset K$である。$\Q$上の拡大次数が同じなので$\Q(\s{2},\s{5}) = K$であり、Galois拡大の推進定理から$\Gal(K/\Q) = \zyu{2} \tm \zyu{2}$であることが判る。
\end{sol}
