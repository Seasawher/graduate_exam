\section{平成24年度 基礎数学}


\subsubsection{}%1
\barquo{
\begin{align*}
  v_1 = \pmat{1 \\ 1\\ 1 \\ 1}  \quad v_2 = \pmat{1  \\1\\ 0\\ 0 }   \quad v_3= \pmat{ 1\\ 0\\ 1\\ 0} \\
  w_1 = \pmat{1 \\ -1\\ -1 \\ 1}   \quad w_2 = \pmat{1  \\ 0 \\ -1 \\ 0 }   \quad w_3= \pmat{ 1\\ -1 \\ 0 \\ 0}
\end{align*}
とおく。$V$を$v_1, v_2 , v_3$で生成される$\R^4$の部分ベクトル空間とし、$W$を$w_1, w_2, w_3$で生成される$\R^4$の部分ベクトル空間とする。このとき、$V \cap W$の基底をひとつ求めよ。
}
\begin{sol}

\end{sol}

\newpage


\subsubsection{}%2
\barquo{
複素数体$\C$の元を成分とする$n$次正方行列全体のなす集合を$M_n(\C)$とする。
\begin{description}
  \item[(1)] $M_n(\C)$の元$N$が、ある自然数$k$に対して$N^k$が零行列になるとする。このとき、$N$の固有値がすべて$0$であることを示せ。
  \item[(2)] $M_n(\C)$の元$A$をひとつ決めて、写像$f_A \colon M_n(\C) \to M_n(\C)$を$f_A(X) = XA - AX$によって定義する。$M_n(\C)$は複素数体$\C$上の$n^2$次元のベクトル空間であり、$f_A$は$M_n(\C)$の線形変換である。このとき、ある自然数$m$に対して$A^m$が零行列になるとすると、線形変換$f_A$の固有値がすべて$0$となることを示せ。
\end{description}
}
\begin{sol}

\end{sol}


\newpage



\subsubsection{}%3
\barquo{
$x > 0$で定義された次の関数項級数は各点収束するが$(0,\infty)$上で一葉収束しないことを示せ。
\[
\sum_{n=0}^{\infty} \f{x^2}{n^2 x + 1}
\]
}
\begin{sol}

\end{sol}

\newpage


\subsubsection{}%4
\barquo{
次の広義積分が収束するような実数$s$の範囲を求めよ。またそのときの積分値を計算せよ。
\[
\iint_{\R^2} \f{ dx dy }{  (x^2 - xy + y^2 + 1)^s }
\]
}
\begin{sol}

\end{sol}
