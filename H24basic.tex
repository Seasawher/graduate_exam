\section{平成24年度 基礎数学}


\subsubsection{}%1
\barquo{
\begin{align*}
  v_1 = \pmat{1 \\ 1\\ 1 \\ 1}  \quad v_2 = \pmat{1  \\1\\ 0\\ 0 }   \quad v_3= \pmat{ 1\\ 0\\ 1\\ 0} \\
  w_1 = \pmat{1 \\ -1\\ -1 \\ 1}   \quad w_2 = \pmat{1  \\ 0 \\ -1 \\ 0 }   \quad w_3= \pmat{ 1\\ -1 \\ 0 \\ 0}
\end{align*}
とおく。$V$を$v_1, v_2 , v_3$で生成される$\R^4$の部分ベクトル空間とし、$W$を$w_1, w_2, w_3$で生成される$\R^4$の部分ベクトル空間とする。このとき、$V \cap W$の基底をひとつ求めよ。
}
\begin{sol}
$V \cap W = (V^{\bot} + W^{\bot})^{\bot}$を用いて求める。計算すると
\begin{align*}
  V^{\bot} &= \setmid{ x \pmat{1 \\ -1 \\ -1 \\ 1} }{ x \in \R} \\
  W^{\bot} &= \setmid{ x \pmat{1 \\ 1 \\ 1 \\ 1} }{ x \in \R}
\end{align*}
であるから
\[
V \cap W = \setmid{ x \pmat{1 \\ 0 \\ 0 \\ -1 }  + y \pmat{0 \\ 1 \\ -1 \\ 0}}{ x,y \in \R}
\]
であることが判る。
\end{sol}

\newpage


\subsubsection{}%2
\barquo{
複素数体$\C$の元を成分とする$n$次正方行列全体のなす集合を$M_n(\C)$とする。
\begin{description}
  \item[(1)] $M_n(\C)$の元$N$が、ある自然数$k$に対して$N^k$が零行列になるとする。このとき、$N$の固有値がすべて$0$であることを示せ。
  \item[(2)] $M_n(\C)$の元$A$をひとつ決めて、写像$f_A \colon M_n(\C) \to M_n(\C)$を$f_A(X) = XA - AX$によって定義する。$M_n(\C)$は複素数体$\C$上の$n^2$次元のベクトル空間であり、$f_A$は$M_n(\C)$の線形変換である。このとき、ある自然数$m$に対して$A^m$が零行列になるとすると、線形変換$f_A$の固有値がすべて$0$となることを示せ。
\end{description}
}
\begin{sol} $E(\grl,T)$で$T$の$\grl$に属する固有空間を表すことにする。
\begin{description}
  \item[(1)] $\grl$を$N$の固有値とし、$v \in E(\grl,N) \sm \{ 0 \}$とする。すると$N^k v = \grl^k v$である。$N^k=0$であるから、$\grl=0$でなくてはならない。
  \item[(2)] $f_A$の$k$回合成を$f_A^k$と書くことにする。このときあきらかに
  \[
  f_A^k(X) = \sum_{i+j=k} a_{i,j}(k) A^i X A^j
  \]
が成り立つような$a_{i,j}(k) \in \C$が存在する。よって$f_A^{2m}=0$なので$f_A$はべき零であり、(1)よりとくに固有値はすべて$0$である。
\end{description}
\end{sol}

\begin{com}
  読者は(2)の解答をどう思われたであろうか。Lie環のブラケットのごとき豊富な構造をもつものを題材としながら、かくも自明であろうとは無礼千万、到底許すべからざることだ……との感想を持たれたとしたら、ご意見全く同感である。実際$A$がべき零であるという仮定を外して、(2)の主張を次のように一般化することができる。
\end{com}

\prop{
$A \in M_n(\C)$に対して$f_A \colon M_n(\C) \to M_n(\C)$を$f_A(X) = [X,A] = XA - AX$によって定義する。このとき$A$の固有値の重複度を込めた全体、つまり$A$のスペクトラムを$\{ \mu_1 , \cdots , \mu_n \}$とすると$f_A$のスペクトラムは
$\setmid{ \mu_i - \mu_j }{1 \leq i,j \leq n }$である。
}
\begin{proof}
  $A$が対角化可能であるときにまず示す。双線形形式
  \begin{align*}
    \C^n \tm \C^n &\to M_n(\C) \\
    (u,v) &\to uv^T
  \end{align*}
  が誘導する線形写像$s \colon \C^n \ts \C^n \to M_n(\C)$を考える。$\C^n$の標準基底を$\{ e_i \}$としたとき$\{ e_i \ts e_j \}_{i,j}$は$\C^n \ts \C^n$の基底になるが、$s$はこれをどう写すだろうか?ある$c_{ij} \in \C$があり
  \[
  \sum_{i,j} c_{ij} s(e_i \ts e_j) = 0
  \]
  だったとする。このとき任意の$k$について
  \begin{align*}
    0 &= \left( \sum_{i,j} c_{ij} e_i e_j^T  \right) e_k \\
    &= \sum_{i,j} c_{ij} \grd_{jk}  e_i \\
    &= \sum_i c_{ik } e_i
  \end{align*}
  であるから$c_{ik}=0$である。$k$は任意だったから、$s$が単射であることがわかる。$\C^n \ts \C^n$の$\C$ベクトル空間としての次元は$n^2$で、$M_n(\C)$と同じなので$s$は同型でもある。

  いま$A$は対角化可能と仮定したので$A^T$も対角化可能である。$A$と$A^T$の特性多項式は同じなのでスペクトラムも同一であり、
  \[
  \C^n \ts \C^n = \bigoplus_{i,j} E(\mu_i,A) \ts E(\mu_j,A^T)
  \]
  と書ける。同型$s$による$E(\mu_i,A) \ts E(\mu_j,A^T)$の像を考えよう。$u \in E(\mu_i,A)$と$v \in E(\mu_j,A^T)$が与えられたとする。このとき
  \begin{align*}
    [uv^T, A] &= uv^T A - A uv^T \\
    &= u (A^T v)^T - (Au) v^T \\
    &= (\mu_j - \mu_i) uv^T
  \end{align*}
  であるから$s(u \ts v) \in E(\mu_j - \mu_i, f_A)$である。$s$は線形なので$s$による$E(\mu_i,A) \ts E(\mu_j,A^T)$の像は$ E(\mu_j - \mu_i, f_A)$に含まれる。$s$は同型なので、これにより$f_A$も対角化可能であって、そのスペクトラムが$\setmid{ \mu_i - \mu_j }{1 \leq i,j \leq n }$となることがいえた。

  対角化可能とは限らない一般の$A$の場合。このとき$A$に収束する対角化可能な行列の列$\{ A_m \} \subset M_n(\C)$がとれる。$A$のスペクトラムを$\{  \mu_i \}$, $A_m$のスペクトラムを$\{ \mu_i^{(m)} \}$とすると、固有値は係数に連続的に依存するので$\lim_{m \to \infty} \mu_i^{(m)} = \mu_i$である。(そうなるように番号を対応させる)したがって$f_A$のスペクトラムは$f_{A_m}$のスペクトラム
  $\setmid{ \mu_i^{(m)} - \mu_j^{(m)}  }{1 \leq i,j \leq n}$の極限であるので、$\setmid{ \mu_i - \mu_j }{1 \leq i,j \leq n }$となる。
\end{proof}


\newpage



\subsubsection{}%3
\barquo{
$x > 0$で定義された次の関数項級数は各点収束するが$(0,\infty)$上で一葉収束しないことを示せ。
\[
\sum_{n=0}^{\infty} \f{x^2}{n^2 x + 1}
\]
}
\begin{sol}
$n \to \infty$の極限での振る舞いを単純化しよう。
\[
f_n(x) = \f{ x^2 }{ n^2 x + 1 }, \quad g_n(x) = \f{x}{n^2}
\]
として、$E=[1, \infty)$とおく。$E$上では$x$によらずに
\[
\abs{ \f{f_n(x)}{g_n(x)} - 1 } \leq \f{1}{n^2 x + 1} \leq \f{1}{n^2}
\]
だから、$E$上で$f_n/g_n$は$1$に一様収束する。したがって
\[
n \geq R \to \forall x \in E \quad \f{1}{2} g_n(x) \leq f_n(x) \leq \f{3}{2} g_n(x)
\]
なる$R>0$がある。いま任意に自然数$L > 0$が与えられたとする。このとき$L'=\max \{ L+1,R  \}$とすると$E$上において
\begin{align*}
  \sum_{n=0}^{\infty} f_n(x) - \sum_{n=0}^{L} f_n(x) &= \sum_{n=L+1}^{\infty} f_n(x) \\
  &\geq \sum_{n=L'}^{\infty} f_n(x) \\
  &\geq \f{1}{2}  \sum_{n=L'}^{\infty} g_n(x) \\
  &\geq \f{x}{2}  \sum_{n=L'}^{\infty} \f{1}{n^2}
\end{align*}
と評価できる。ここで
\[
\norm{ \f{x}{2}  \sum_{n=L'}^{\infty} \f{1}{n^2} }_E \geq \f{(L')^2}{2}  \sum_{n=L'}^{\infty} \f{1}{n^2} \geq 1
\]
であることより
\[
\norm{ \sum_{n=0}^{\infty} f_n(x) - \sum_{n=0}^{L} f_n(x) }_{(0, \infty)} \geq 1
\]
となる。$L$は任意にとっていたはずだから、これで一様収束しないことがいえた。
\end{sol}

\newpage


\subsubsection{}%4
\barquo{
次の広義積分が収束するような実数$s$の範囲を求めよ。またそのときの積分値を計算せよ。
\[
\iint_{\R^2} \f{ dx dy }{  (x^2 - xy + y^2 + 1)^s }
\]
}
\begin{sol}
$x^2 - xy + y^2 = (x - y/2)^2 + (\s{3}y/2)^2$であるから
\begin{align*}
  x - \f{y}{2} &= \zeta = r \cos \grt \\
  \f{\s{3}}{2} y &= \eta = r \sin \grt
\end{align*}
とおくと
\[
d\zeta d \eta = \abs{ \det \pmat{1 & -1/2 \\ 0 & \s{3}/2 } } dx dy = \f{\s{3}}{2} dx dy
\]
より$dx dy = 2/\s{3} d\zeta d\eta = 2r/\s{3} dr d\grt$である。よって
\begin{align*}
  \iint_{\R^2} \f{ dx dy }{  (x^2 - xy + y^2 + 1)^s } &= \f{2}{\s{3}} \int_0^{2\pi} \int_0^{\infty} \f{r}{(r^2 + 1)^s} \ dr \\
  &= \f{1}{\s{3}} \int_0^{2\pi} \int_0^{\infty} \f{1}{(r + 1)^s} \ dr
\end{align*}
である。ゆえにこの積分は$s > 1$のとき、かつそのときに限って収束する。そして$s > 1$であるとき積分値は
\[
\f{2 \pi}{\s{3} (s-1)}
\]
である。
\end{sol}
