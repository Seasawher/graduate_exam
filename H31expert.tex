\bfsection{平成31年度 専門科目}

\bfsubsection{問1}
\barquo{
$\R[X,Y]$を変数$X,Y$に関する実数係数の$2$変数多項式環とする。$I$を$X^2 + Y^2$で生成された$\R[X,Y]$のイデアルとする。$A = \R[X,Y]/I $とおく。このとき、以下の問に答えよ。
\begin{description}
  \item[(i)] $A$は整域であることを示せ。
  \item[(ii)] $A$の商体を$K$とおき、$A$の$K$における整閉包を$B$とおく。$A$加群としての$B$の生成系を一組与えよ。
\end{description}
}
\begin{sol} ${}$
  \begin{description}
    \item[(i)] $\R[X,Y]$はUFDなので、$X^2 + Y^2$が既約元であることを示せばよい。可約であると仮定する。そうするとある実数$a,b,c,d $が存在して$X^2 + Y^2 = (aX + bY)(cX + dY)$が成り立つことになるが、そうすると$ac - 1 = ad + bc = bd - 1 = 0$でなくてはならない。これは$a,b,c,d$が実数であったことに矛盾。よって$X^2 + Y^2$は既約元であり、$I \subset \R[X,Y]$は素イデアル。
    \item[(ii)] $a = Y/X$とする。$a^2 + 1 = 0$なので$a \in B$である。$B = A[a]$を示そう。それには、$A[a]$が整閉であることを示せば十分である。$\R$代数の準同形$\vp \colon \R[X,\I] \to A[a]$を$\vp(\I)=a, \vp(X)=X$で定める。これはwell-definedであり、あきらかに全射。$f \in \Ker \vp$とする。
    \[
    f = \sum_{i=0}^n (a_i+\I b_i) X^i
    \]
    と表せる。そうすると
    \[
    0 = \sum_{i=0}^n a_i X^i + Y \sum_{i=1}^n b_i X^{i-1}  + b_0 \f{Y}{X}
    \]
    である。ここから$f=0$が導かれる。よって$\vp$は同型であり、$A[a] \cong \R[X,\I] \cong \C[X]$である。とくに$A[a]$は整閉だから$B = A[a]$が示された。よって、$B$の$A$加群としての生成系としては$\{ 1, a \}$がとれる。
  \end{description}
\end{sol}


\newpage

\bfsubsection{問2}
\barquo{
有限群$G$に対して、次の条件$(*)$を考える。
\begin{description}
  \item[$(*)$] 任意の正整数$n$に対して、$G$の部分群のうち、位数が$n$のものの個数は$1$以下である。
\end{description}
以下の問に答えよ。
\begin{description}
  \item[(i)] $G$は有限Abel群で$(*)$を満たすとする。このとき、$G$は巡回群であることを示せ。
  \item[(ii)] $G$は有限群で$(*)$を満たすとする。$H$を$G$の正規部分群とする。このとき、$G/H$も$(*)$を満たすことを示せ。
  \item[(iii)] $G$は有限群で$(*)$を満たすとする。このとき、$G$は巡回群であることを示せ。
\end{description}
}
\begin{proof}
  ほげほげ
\end{proof}

\newpage


\bfsubsection{問3}
\barquo{
多項式$f(X) = X^4 + 6X^2 + 2 \in \Q[X]$の$\Q$上の最小分解体を$K$とおく。$K$を$\C$の部分体とみなし、$F=K \cap \R$とおく。このとき、次の問に答えよ。
\begin{description}
  \item[(i)] 拡大次数$[F : \Q]$を求めよ。
  \item[(ii)] $F/\Q$はGalois拡大であることを示せ。
\end{description}
}
\begin{proof} ${}$
  \begin{description}
\item[(i)] $X^4 + 6X^2 + 2$は複2次式なので因数分解ができる。
 \begin{align*}
  X^4 + 6X^2 + 2 &= (X^2 + 3)^2 - 7 \\
  &= (X^2 + 3 + \sqrt{7} )(X^2 + 3 - \sqrt{7} )
  \end{align*}
  なので、この多項式の根は
  $
  \pm \sqrt{ 3 \pm \sqrt{7}  } i
  $
  である。$\gra = \sqrt{ 3 + \sqrt{7}  } i $, $\beta =  \sqrt{ 3 - \sqrt{7}  } i$とおく。$K = \Q(\gra, \beta) = \Q(\gra, \sqrt{2})$である。
  ゆえに$F = \Q(\sqrt{7}, \sqrt{2})$であり、$\sqrt{7} \not\in \Q(\sqrt{2})$から$[F:\Q] = 4$である。
  \item[(ii)] $\Q$は標数$0$なので完全体であり、したがって$F/\Q$は分離拡大。また$F$は$\Q$上$\sqrt{7}$と$\sqrt{2}$で生成されている。これらの共役はすべて$F$に含まれているので、$F/\Q$は正規拡大。よって$F/\Q$はGalois拡大である。
  \end{description}
\end{proof}



\newpage


\bfsubsection{問4}
\barquo{
$n \geq 2$に対して、
\[
S^{n-1} = \setmid{(x_1, \cdots , x_n) \in \R^n }{x_1^2 + \cdots + x_n^2 = 1 } \quad \bbs^1 = \setmid{z \in \C}{\abs{z} = 1}
\]
とし、写像$\Phi \colon S^{n-1} \tm \bbs^1 \to \C^n$を
\[
\Phi(x_1, \cdots , x_n) = (x_1z, \cdots , x_nz)
\]
と定める。
\begin{description}
  \item[(1)] $\Phi$の像$M$が$\C^n$の実$n$次元部分多様体であることを示せ。
  \item[(2)] $n$が偶数のとき、$M$が向き付け可能であることを示せ。
\end{description}
}
\begin{proof} ${}$
  \begin{description}
    \item[(1)] $\Phi(x,z) = \Phi(y,w)$とする。すると$\forall i \; x_i z = y_i w$である。$S^{n-1}$の定義により$x_i \neq 0$なる$i$がある。よって$z/w = y_i / x_i \in \R$であるので、$z=w$または$z = -w$である。したがってず$w \in M$に対して$\# \Phi^{-1}(w) =2$であることが分かった。

    $N = S^{n-1} \tm \bbs^1$とおく。$N$に$(x,z) \sim (-x, -z)$で生成される同値関係$\sim$を定義する。このとき$\Phi(x,z) = \Phi(y,w)$と$(x,z) \sim (y,w)$は同値である。ゆえに次の図式
    \[
    \xymatrix{
    N \ar[r]^-{\Phi} \ar[d]_-{P}  & M \\
    N/ {\sim} \ar@{.>}[ru]_-{ \wt{\Phi} }
    }
    \]
    を可換にするような全単射連続写像$\wt{\Phi}$がある。$N/{\sim}$はコンパクトで、$M$はHausdorffなので$\wt{\Phi}$は同相でなければならない。したがって$M$の代わりに$N/{\sim}$が$n$次元位相多様体であることをいえばよいが、$P$が被覆写像であるためこれはあきらか。
    \item[(2)] $n$は偶数と仮定されているので$n=2k$とおける。接ベクトル束$TM$の切断$s$であって、至る所ゼロでないものの存在をいえば十分である。$\beta = (x,z) \in N$に対して
    \[
    \wt{z} = (x_2, -x_1, \cdots , x_{2k}, -x_{2k-1}, -y_2, y_1)
    \]
    と定めておき、これによりベクトル場$N \to TN \st z \mapsto (z, \wt{z})$を定める。このベクトル場は$N/{\sim}$上のベクトル場を誘導し、あきらかに至る所ゼロでない。よって示せた。
  \end{description}
\end{proof}



\newpage

\bfsubsection{問5}
\barquo{
$\C$の部分空間
\[
X = \setmid{1- e^{i\grt} \in \C }{0 \leq \grt < 2\pi} \cup \setmid{-1 + e^{i\grt} \in \C}{0 \leq \grt < 2\pi }
\]
を考える。整数$p,q$に対して、写像$f \colon X \to X$を
\begin{align*}
  f(1- e^{i\grt}) &= -1 + e^{ip\grt} \\
    f(-1 + e^{i\grt}) &= 1 - e^{iq\grt}
\end{align*}
で定め、$X \tm [0,1]$に
\[
(x,0) \sim (f(x),1)
\]
($x \in X$)で生成される同値関係$\sim$を与える。商空間$Y = (X \tm [0,1])/ {\sim}$の整数係数ホモロジー群を計算せよ。
}
\begin{sol}

セル複体を使ってホモロジーを求めよう。空間$Y$を直接書くことは難しいが、次のようなものを想像することはできる。

\begin{center}
\includegraphics[width=5cm]{H31expert05_01.png}
\end{center}

この対になった円筒は、$X \tm I$および$Y$を表している。上下の円盤に見える部分は円周であり、ちくわを2つくっつけたような形をしている。側面も輪郭しか書かれていないが、面になっている。垂直方向が$I$成分を表しており、上が$t=1$で下が$t=0$であるものとしよう。また右を実軸のプラス方向、奥を虚軸のプラス方向とする。上部にある点は原点を表す。図に$e$と書かれているのはセルである。それぞれ具体的には次のように与えられる。
\begin{align*}
  e^0 &= (0,1) \\
  e^1_a &= \setmid{ (-1+e^{i\grt},1) }{0 < \grt < 2\pi } \\
  e^1_b &= \setmid{ (1-e^{i\grt},1) }{0 < \grt < 2\pi } \\
  e^1_c &= \setmid{ (0,t) }{0 < t < 1 } \\
  e^2_a &= \setmid{ (-1+e^{i\grt},t) }{0 < \grt < 2\pi , 0 < t < 1 } \\
  e^2_b &= \setmid{ (1-e^{i\grt},t) }{0 < \grt < 2\pi , 0 < t < 1 }
\end{align*}
このとき、次に注意する。
\begin{align*}
  e^0 &= (0,0) \\
  pe_a^1 &= \setmid{ (1-e^{i\grt},0) }{0 < \grt < 2\pi } \\
  qe_b^1 &= \setmid{ (-1+e^{i\grt},1) }{0 < \grt < 2\pi }
\end{align*}
さて以上の準備の下セル複体のホモロジーを計算しよう。$Y$の$0$セル、$1$セル、$2$セルの数はそれぞれ$1,3,2$個なので
\[
\xymatrix{
0 \ar[r] & \Z^2 \ar[r]^{\del} & \Z^3 \ar[r]^{\grs} & \Z \ar[r] & 0
}
\]
という図式に表されるような状況になっている。まず$\grs$だが、$0$セルはただひとつしかないのでこれはゼロ写像である。よって$H_0(Y)=\Z$がわかる。
次に$\del$を計算する。次の図

\begin{center}
\includegraphics[width=5cm]{H31expert05_02.png}
\end{center}

のような状況になっているので
\[
\del(e_a^2) = e_a^1 - qe_b^1 \quad \del(e_b^2) = e_b^1 - pe_a^1
\]
である。したがって$\del$は次の行列
\[
\del = \pmat{
1 & -p \\
-q & 1 \\
0 & 0
}
\]
で表される写像である。この行列の階数は$pq =1$のとき$1$でそうでないとき$2$である。よって$pq = 1$のとき
\begin{align*}
  H_1(Y) &= \Z^3 / \Im \del \\
  &= \Z^2 \\
  H_2(Y) &= \Ker \del \\
  &= \Z
\end{align*}
である。$pq \neq 1$ならば
\begin{align*}
  H_1(Y) &= \Z^3 / \Im \del \\
  &= (a \Z \oplus  b \Z \oplus c \Z)  / (a - qb, b - pa) \\
  &= (a \Z \oplus  b \Z )  / ( (1-pq)a, b - pa) \oplus c \Z \\
  &= \Z / (1-pq)\Z \oplus \Z \\
  H_2(Y) &= \ker \del \\
  &= 0
\end{align*}
である。以上により求めるホモロジーは、$pq = 1$のとき
\[
H_i(Y) = \begin{cases}
\Z &(i=0,2) \\
\Z^2 &(i=1) \\
0 &(\text{otherwise})
\end{cases}
\]
であり、$pq \neq 1$のとき
\[
H_i(Y) = \begin{cases}
\Z &(i=0) \\
\Z / (pq-1)\Z \oplus \Z &(i=1) \\
0 &(\text{otherwise})
\end{cases}
\]
\end{sol}
