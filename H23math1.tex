\section{平成23年度 数学I}

\subsubsection{}%1
\barquo{
$V$を$\C$上の有限次元ベクトル空間とし、$f \colon V \to V$を一次変換とする。$W_1, W_2$を$V$の部分空間で$V = W_1 + W_2$, $f(W_1 ) \subset W_1$, $f(W_2 ) \subset W_2$を満たすとする。$f$の$W_1$への制限を$f|_{W_1} \colon W_1 \to W_1$とおき、$f$の$W_2$への制限を
$f|_{W_2} \colon W_2 \to W_2$とおく。
\begin{description}
  \item[(1)] $f|_{W_1}$の最小多項式を$P_1(x)$, $f|_{W_2}$の最小多項式を$P_2(x)$とおく。$f$の最小多項式は$P_1(x), P_2(x)$の最小公倍元であることを示せ。
  \item[(2)] $f|_{W_1}, f|_{W_2}$が対角化可能であるとき、$f$も対角化可能であることを示せ。
\end{description}
ただし、$P(x), Q(x) \in \C[x]$に対し、$P(x)$が$Q(x)$で割り切れるとき、$P(x)$は$Q(x)$の倍元であるという。また$P(x), Q(x)$の最小公倍元とは$P(x),Q(x)$の倍元のうち次数が最小のモニック多項式のことをいう。
}


\newpage

\subsubsection{}%2
\barquo{
区間$[0,1]$上の実数値連続関数$f(x)$は$f(0)=0$, $f(1)=1$をみたしている。このとき、極限値
\[
\lim_{n \to \infty} n \int_0^1 f(x) x^{2n} \ dx
\]
を求めよ。
}

\newpage

\subsubsection{}%3
\barquo{
$L$を階数$2$の自由アーベル群$\Z^2$の部分群で$(a,b), (c,d) \in \Z^2$により生成されるものとする。このとき、以下の問に答えよ。
\begin{description}
  \item[(1)] 商群$\Z^2 / L$の位数が有限になるための必要十分条件を$a,b,c,d$を用いて表せ。
  \item[(2)] $abcd \neq 0$をみたし、かつ$\Z^2 / L$の位数が有限となるもののうちから、
  \begin{description}
    \item[(i)] $\Z^2 / L = \{0\}$となる例
    \item[(ii)] $\Z^2 / L$が非自明な巡回群になる例
    \item[(iii)] $\Z^2 / L$が巡回群にならない例
  \end{description}
  を各々$1$つずつ与えよ。
\end{description}
}

\newpage

\subsubsection{}%4
\barquo{
$n$を$2$以上の自然数とする。
\begin{description}
  \item[(1)] $n$次元実射影空間$\R P^n$の基本群$\pi_1(\R P^n)$を計算せよ。ただし、$n$次元単位球面$S^n$が単連結であることを用いてよい。
  \item[(2)] $\R P^n$から単位円$S^1$への連続写像は、常に定置写像とホモトピックであることを示せ。
\end{description}
}

\newpage

\subsubsection{}%5
\barquo{
函数
\[
f(z) = \f{ e^{(1+i)z} }{(e^z+1)^2} \quad (z \in \C)
\]
に関する以下の問に答えよ。
\begin{description}
  \item[(1)] $L>0$とし、複素平面上の$4$点$L, L + 2\pi i, -L + 2\pi i , -L$を頂点とする長方形の内部にある$f(z) \ dz$の極と留数を求めよ。
  \item[(2)] 広義積分$\int_{- \infty}^{\infty} f(x) \ dx$を計算せよ。
\end{description}
}
