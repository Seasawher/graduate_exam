\section{平成23年度 数学I}

\subsubsection{}%1
\barquo{
$V$を$\C$上の有限次元ベクトル空間とし、$f \colon V \to V$を一次変換とする。$W_1, W_2$を$V$の部分空間で$V = W_1 + W_2$, $f(W_1 ) \subset W_1$, $f(W_2 ) \subset W_2$を満たすとする。$f$の$W_1$への制限を$f|_{W_1} \colon W_1 \to W_1$とおき、$f$の$W_2$への制限を
$f|_{W_2} \colon W_2 \to W_2$とおく。
\begin{description}
  \item[(1)] $f|_{W_1}$の最小多項式を$P_1(x)$, $f|_{W_2}$の最小多項式を$P_2(x)$とおく。$f$の最小多項式は$P_1(x), P_2(x)$の最小公倍元であることを示せ。
  \item[(2)] $f|_{W_1}, f|_{W_2}$が対角化可能であるとき、$f$も対角化可能であることを示せ。
\end{description}
ただし、$P(x), Q(x) \in \C[x]$に対し、$P(x)$が$Q(x)$で割り切れるとき、$P(x)$は$Q(x)$の倍元であるという。また$P(x), Q(x)$の最小公倍元とは$P(x),Q(x)$の倍元のうち次数が最小のモニック多項式のことをいう。
}
\begin{sol} ${}$
  \begin{description}
    \item[(1)] $f$の最小多項式を$P \in \C[x]$とする。$P(f) \colon V \to V$は零写像なので、とくに$W_i$上でも零。よって$P$は$P_1$でも$P_2$でも割り切れる。逆にある$Q \in \C[x]$があって$Q$が$P_1, P_2$の倍元だとすると$Q(f) \colon V \to V$は零。よって$Q$は$P$で割り切れる。
    以上により$P$は$P_1, P_2$の最小公倍元である。
    \item[(2)] $f|_{W_1}, f|_{W_2}$が対角化可能なら、$P_1$と$P_2$は重根を持たない。よって(1)より$P$も重根を持たない。ゆえに$f$は対角化可能。
  \end{description}
\end{sol}

\newpage

\subsubsection{}%2
\barquo{
区間$[0,1]$上の実数値連続関数$f(x)$は$f(0)=0$, $f(1)=1$をみたしている。このとき、極限値
\[
\lim_{n \to \infty} n \int_0^1 f(x) x^{2n} \ dx
\]
を求めよ。
}
\begin{rem}
  $f(0)=0$という仮定は不要である。
\end{rem}
\begin{sol}
$\ve > 0$が任意に与えられたとする。$f$は連続なので
\[
\abs{1-x } < \grd \to \abs{ 1 - f(x)} < \ve
\]
なる$\grd > 0$が存在する。したがって
\begin{align*}
  \abs{ n \int_0^1 f(x) x^{2n} \ dx - n \int_0^1 x^{2n} \ dx } &\leq n \int_0^1 \abs{ f(x)- 1} x^{2n} \ dx \\
  &\leq n \int_{1- \grd}^1 \abs{ f(x)- 1} x^{2n} \ dx + n \int_0^{1-\grd} \abs{ f(x)- 1} x^{2n} \ dx  \\
  &\leq n \ve \int_{1- \grd}^1  x^{2n} \ dx + n (1 - \grd)^{2n} \int_0^{1-\grd} \abs{ f(x)- 1}  \ dx  \\
  &\leq \f{  n (1-(1-\grd)^{2n+1}) }{ 2n+1 } \ve + n (1 - \grd)^{2n} \int_0^{1-\grd} \abs{ f(x)- 1}  \ dx
\end{align*}
という評価ができる。$n \to \infty$としたとき$(1 - \grd)^n \to 0$なので
\[
\limsup_{n \to \infty } \abs{ n \int_0^1 f(x) x^{2n} \ dx - n \int_0^1 x^{2n} \ dx } \leq \ve /2
\]
であることがわかる。$\ve > 0$は任意だったから
\[
\lim_{n \to \infty} n \int_0^1 f(x) x^{2n} \ dx = \lim_{n \to \infty} n \int_0^1  x^{2n} \ dx =  \f{1}{2}
\]
であることがいえた。
\end{sol}

\begin{com}
  Lebesgueの収束定理を用いた別解がある。多項式近似定理により、連続関数$f$は$[0,1]$上$C^{\infty}$級関数で一様近似できる。$n \int_0^1  x^{2n} \ dx$は定数で抑えられるので、$f \in C^1[0,1]$として良い。そうすると部分積分が使えて
  \[
  n \int_0^1 f(x) x^{2n} \ dx = \f{n}{2n+1} -  \int_0^1 f'(x) \left(   \f{ nx^{2n+1} }{ 2n+1 } \right) \ dx
  \]
  である。ここで$[0,1]$上では$n$によらず一様に
  \[
  \abs{  f'(x) \left(   \f{ nx^{2n+1} }{ 2n+1 } \right) } \leq \abs{ f'(x) }
  \]
  であって$f'$は可積分関数なので、Lebesgueの収束定理により
  \[
   \lim_{n \to \infty}  \int_0^1 f'(x) \left(   \f{ nx^{2n+1} }{ 2n+1 } \right) \ dx = 0
   \]
  であることが言える。よって求める積分の値は$1/2$である。
\end{com}

\newpage

\subsubsection{}%3
\barquo{
$L$を階数$2$の自由アーベル群$\Z^2$の部分群で$(a,b), (c,d) \in \Z^2$により生成されるものとする。このとき、以下の問に答えよ。
\begin{description}
  \item[(1)] 商群$\Z^2 / L$の位数が有限になるための必要十分条件を$a,b,c,d$を用いて表せ。
  \item[(2)] $abcd \neq 0$をみたし、かつ$\Z^2 / L$の位数が有限となるもののうちから、
  \begin{description}
    \item[(i)] $\Z^2 / L = \{0\}$となる例
    \item[(ii)] $\Z^2 / L$が非自明な巡回群になる例
    \item[(iii)] $\Z^2 / L$が巡回群にならない例
  \end{description}
  を各々$1$つずつ与えよ。
\end{description}
}
\begin{sol} ${}$
  \begin{description}
    \item[(1)] $A \in M_2(\Z)$を
    \[
    A = \pmat{ a& b \\ c &d}
    \]
    により定めると、$L$は$A$の像$A(\Z^2)$と一致することに注意する。有限生成Abel群の基本定理により$\# (\Z^2 / L ) < \infty$と$\Z^2 / L$が位数$\infty$の元を持たないことは同値である。よって、$\Z^2 / L$のすべての元の位数が有限になるための条件を求めれば十分である。

    さて$\Z^2 / L$のすべての元の位数が有限と仮定しよう。このとき任意に与えられた$p \in \Z^2$の$\Z^2 / L $における像の位数は有限であり、よって$np \in L$なる$0$でない整数$n$がある。ゆえに$n p \in A(\Z^2)$であるから、とくに$p \in A(\Q^2)$である。$p \in \Z^2$は任意にとっていたから、$\Z^2 \subset A(\Q^2)$ということだが、
    $\Q^2$の元は適当に$0$でない整数を乗ずれば$\Z^2$の元にすることができるので$\Q^2 = A(\Q^2)$である。
    つまり$A \in GL_2(\Q)$であるから、$ad - bc \neq 0$ということになる。

    逆に$ad - bc \neq 0$とする。このとき任意の$\Z^2 / L $の元の位数は有限となることがわかる。なぜならば!任意に$p \in \Z^2$が与えられたとしよう。$A \in GL_2(\Q)$なので、$p = Ax$なる$x \in \Q^2$が存在する。$m x \in \Z^2$なる$0$でない整数$m$をとると、$mp = A(mx) \in L$なので$p$の$\Z^2 / L$における像の位数は有限であることがわかる。$p$は任意だったから、示すべきことがいえた。
    \item[(2)]
    \begin{description}
      \item[(i)] $A \in GL_2(\Z)$であって、どの成分も$0$でなければなんでもいい。たとえば
      \[
      A = \pmat{ 2 &1 \\ 1& 1 }
      \]
      など。
      \item[(ii)] たとえば
      \[
      A = \pmat{ 1& 1 \\ -1& 1}
      \]
      とおくと$L=\setmid{(x,y)^T }{x - y \in 2\Z}$である。よって$\Z^2 / L$の各元は$0$か$(1,0)^T$で代表されるので$\# (\Z^2 / L)=2$であり、$\Z^2 / L \cong \zyu{2}$であるということになる。
      \item[(iii)] たとえば(i)で与えた行列の各成分を$2$倍して
      \[
      A = \pmat{ 4 &2 \\ 2& 2 }
      \]
      とすれば
      \[
      L = \pmat{2 \\ 0} \Z \oplus  \pmat{0 \\ 2} \Z
      \]
      なので$\Z^2 / L \cong \zyu{2} \oplus \zyu{2}$であることが判る。
    \end{description}
  \end{description}
\end{sol}

\newpage

\subsubsection{}%4
\barquo{
$n$を$2$以上の自然数とする。
\begin{description}
  \item[(1)] $n$次元実射影空間$\R P^n$の基本群$\pi_1(\R P^n)$を計算せよ。ただし、$n$次元単位球面$S^n$が単連結であることを用いてよい。
  \item[(2)] $\R P^n$から単位円$S^1$への連続写像は、常に定置写像とホモトピックであることを示せ。
\end{description}
}

\newpage

\subsubsection{}%5
\barquo{
函数
\[
f(z) = \f{ e^{(1+i)z} }{(e^z+1)^2} \quad (z \in \C)
\]
に関する以下の問に答えよ。
\begin{description}
  \item[(1)] $L>0$とし、複素平面上の$4$点$L, L + 2\pi i, -L + 2\pi i , -L$を頂点とする長方形の内部にある$f(z) \ dz$の極と留数を求めよ。
  \item[(2)] 広義積分$\int_{- \infty}^{\infty} f(x) \ dx$を計算せよ。
\end{description}
}
