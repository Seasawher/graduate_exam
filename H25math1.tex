\section{平成25年度 数学I}

\subsubsection{}%1
\barquo{
複素数を成分とする$2$次正方行列全体の集合を$M_2(\C)$で表す。$A \in M_2(\C)$は単位行列のスカラー倍ではないとし、$S = \setmid{B \in M_2(\C)}{AB = BA}$とおく。このとき、$X,Y \in S$なら$XY = YX$であることを示せ。
}
\begin{sol}
  $A$のJordan標準形を考えることにより
  \[
  PAP^{-1} = \pmat{ a & b  \\ 0 & c }
  \]
  となるような正則行列$P$の存在がいえる。ただし$b=0$または$a=c$である。スカラー行列でないという仮定から、$a=c$のとき$b \neq 0$である。
\end{sol}

\newpage


\subsubsection{}%2
\barquo{
$b>a>0$を実数、$f \colon [0,\infty) \to \R$を連続関数とする。このとき以下を示せ。
\begin{description}
  \item[(i)]
  \[
  \lim_{\ve \to + 0} \int_{a \ve}^{a \ve} \f{f(x) }{x} \ dx = f(0) \log \f{b}{a}
  \]
  \item[(ii)] 広義積分$\int_{1}^{\infty} \f{f(x) }{x} \ dx$が収束するなら
  \[
  \int_{1}^{\infty} \f{f(bx) - f(ax) }{x} \ dx = f(0) \log \f{a}{b}
  \]
  が成り立つ。
\end{description}
}
\begin{sol}

\end{sol}

\newpage


\subsubsection{}%3
\barquo{
$p$を素数とする。アーベル群$A$は位数$p^4$であり、位数$p$の部分群$N$で$A/N \cong \zyu{p^3}$となるものをもつとする。このような$A$を同型を除いてすべて求めよ。
}
\begin{sol}

\end{sol}


\newpage

\subsubsection{}%4
\barquo{
写像$F \colon \R^4 \to \R^4$を$F(x, y , z, w) = (xy, y, z, w)$と定め、写像$f \colon S^3 \to \R^4$を$F$の$3$次元球面
\[
S^3 = \setmid{ (x,y,z,w) \in \R^4 }{ x^2 + y^2 + z^2 + w^2 = 1 }
\]
への制限とする。$S^3$の各点$p$における$f$の微分$df_p$の階数を求めよ。
}
\begin{sol}

\end{sol}

\newpage

\subsubsection{}%5
\barquo{
$a,b>0$を実数、$n \geq 2$を整数とするとき、次の広義積分を求めよ。
\[
I_n = \int_{- \infty }^{\infty} \f{ \exp(ia (x-ib) )  }{ (x-ib)^n } \ dx
\]
}
\begin{sol}

\end{sol}
