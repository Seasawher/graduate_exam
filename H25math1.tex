\section{平成25年度 数学I}

\subsubsection{}%1
\barquo{
複素数を成分とする$2$次正方行列全体の集合を$M_2(\C)$で表す。$A \in M_2(\C)$は単位行列のスカラー倍ではないとし、$S = \setmid{B \in M_2(\C)}{AB = BA}$とおく。このとき、$X,Y \in S$なら$XY = YX$であることを示せ。
}
\begin{sol}
この解答では、単に環といったとき可換とは限らないものとする。$X \in M_n(\C)$に対して部分環$Z(X)$を
\[
Z(X) = \setmid{ B \in M_2(\C) }{ XB = BX }
\]
により定める。$Y \in M_2(\C)$と$X$が共役で、$Y = PXP^{-1}$なる$P \in GL_2(\C)$が存在するとき、$B \in Z(X)$に対して
\begin{align*}
  (PBP^{-1}) Y &= PBXP^{-1} \\
  &= PXBP^{-1} \\
  &= Y (PBP^{-1})
\end{align*}
であるから$PBP^{-1} \in Z(Y)$である。つまり写像
\begin{align*}
  Z(X) &\to Z(Y) \\
  B &\mapsto PBP^{-1}
\end{align*}
が存在する。これは全単射であり、環としての同型である。

  $A$のJordan標準形を考えることにより
  \[
  \grL := PAP^{-1} = \pmat{ \beta & \grg \\ 0 & \grd}
  \]
  となるような正則行列$P$の存在がいえる。ただし$\grg=0$または$\beta = \grd$であるものとする。スカラー行列でないという仮定から、$\grg=0$のとき$\beta \neq \grd$であり$\beta = \grd$のときでも$\grg \neq 0$である。$Z(\grL)$を特定しよう。$B \in M_2(\C)$が与えられたとし
  \[
  B = \pmat{a& b \\ c &d  } \quad \grL = \pmat{ \beta & \grg \\ 0 & \grd}
  \]
  と表されていたとする。このとき計算すると
  \begin{align*}
    B \grL - \grL B = \begin{cases}
    \grg \pmat{-c & a-d \\ 0 & c} &(\beta=\grd, \grg \neq 0) \\
    (\beta - \grg) \pmat{0 & b \\ c & 0} &(\beta \neq \grd, \grg =0)
  \end{cases}
  \end{align*}
  だから$Z(\grL)$は次のように求まる。

  \begin{description}
    \item[(1)] $\beta=\grd, \grg \neq 0$のとき
    \[
    Z(\grL) = \setmid{ \pmat{a & b \\ 0 & a} }{a,b \in \C }
    \]
    \item[(2)] $\beta \neq \grd, \grg =0$のとき
    \[
    Z(\grL) = \setmid{ \pmat{a & 0 \\ 0 & d} }{a,d \in \C}
    \]
  \end{description}

  したがっていずれにせよ$Z(\grL)$は可換環である。よってそれと同型な$Z(A)$も可換環。
\end{sol}

\newpage


\subsubsection{}%2
\barquo{
$b>a>0$を実数、$f \colon [0,\infty) \to \R$を連続関数とする。このとき以下を示せ。
\begin{description}
  \item[(i)]
  \[
  \lim_{\ve \to + 0} \int_{a \ve}^{a \ve} \f{f(x) }{x} \ dx = f(0) \log \f{b}{a}
  \]
  \item[(ii)] 広義積分$\int_{1}^{\infty} \f{f(x) }{x} \ dx$が収束するなら
  \[
  \int_{1}^{\infty} \f{f(bx) - f(ax) }{x} \ dx = f(0) \log \f{a}{b}
  \]
  が成り立つ。
\end{description}
}
\begin{sol}

\end{sol}

\newpage


\subsubsection{}%3
\barquo{
$p$を素数とする。アーベル群$A$は位数$p^4$であり、位数$p$の部分群$N$で$A/N \cong \zyu{p^3}$となるものをもつとする。このような$A$を同型を除いてすべて求めよ。
}
\begin{sol}

\end{sol}


\newpage

\subsubsection{}%4
\barquo{
写像$F \colon \R^4 \to \R^4$を$F(x, y , z, w) = (xy, y, z, w)$と定め、写像$f \colon S^3 \to \R^4$を$F$の$3$次元球面
\[
S^3 = \setmid{ (x,y,z,w) \in \R^4 }{ x^2 + y^2 + z^2 + w^2 = 1 }
\]
への制限とする。$S^3$の各点$p$における$f$の微分$df_p$の階数を求めよ。
}
\begin{sol}

\end{sol}

\newpage

\subsubsection{}%5
\barquo{
$a,b>0$を実数、$n \geq 2$を整数とするとき、次の広義積分を求めよ。
\[
I_n = \int_{- \infty }^{\infty} \f{ \exp(ia (x-ib) )  }{ (x-ib)^n } \ dx
\]
}
\begin{sol}

\end{sol}
