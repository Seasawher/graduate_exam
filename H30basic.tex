\section{平成30年度 基礎科目}

\subsubsection{} %{問1}
\lbar{
広義積分
\[
\iiint_V \f{1}{(1+x^2 + y^2)z^{ \f{3}{2} }} \; dx dy dz
\]
を計算せよ。ただし、$V=\setmid{(x,y,z) \in \R^3}{x^2+y^2 \leq z}$とする。
}

\begin{sol}
  極座標変換$(x,y,z) \mapsto (r,\grt, z)$を考える。このとき$dx dy dz = r dr d\grt dz$であり、
  \begin{align*}
    \iiint_V \f{1}{(1+x^2 + y^2)z^{ \f{3}{2} }} \; dx dy dz &= \int_{0}^{2\pi} \; d\grt \int_{0}^{\infty} \f{r}{1+r^2} \left( \int_{r^2}^{\infty} z^{ -\f{3}{2}  } \; dz \right) \; dr \\
    &= 2\pi \int_{0}^{\infty} \f{r}{1+r^2} \left[ (-2)z^{-\f{1}{2}} \right]^{\infty}_{r^2} \; dr \\
    &= 4\pi \int_{0}^{\infty} \f{r}{1+r^2} \f{1}{r} \; dr \\
    &= 4\pi \int_{0}^{\infty} \f{1}{1+r^2} \; dr \\
    &= 4\pi \cdot \f{\pi}{2} \\
    &= 2\pi^2
  \end{align*}
  と計算できる。
\end{sol}

\newpage

\subsubsection{} %{問2}
\lbar{
$a,b$を実数とする。実行列
\[
A = \begin{pmatrix}
1 & 1 & a & b \\
0 & 1 & 2 & 0 \\
2 & 0 & 1 & 4
\end{pmatrix}
\]
について、以下の問に答えよ。
\begin{description}
\item[(1)] 行列$A$の階数を求めよ。
\item[(2)] 連立1次方程式
\[
A \begin{pmatrix}
x_1 \\
x_2 \\
x_3 \\
x_4
\end{pmatrix} = \begin{pmatrix}
1 \\ 1 \\ 1
\end{pmatrix}
\]
が解を持つような実数$a,b$をすべて求めよ。
\end{description}
}

\begin{sol} ${}$
  \begin{description}
    \item[(1)] ある行に別の行の定数倍を足す操作を繰り返すと
    \begin{align*}
      A &\sim \begin{pmatrix}
      1 & 1 & a & b \\
      0 & 1 & 2 & 0 \\
      0 & -2 & 1-2a & 4-2b
      \end{pmatrix} \\
      &\sim \begin{pmatrix}
      1 & 1 & a & b \\
      0 & 1 & 2 & 0 \\
      0 & 0 & 5-2a & 4-2b
      \end{pmatrix}
    \end{align*}
    と変形できる。したがって$\rank A \geq 2$であり、$(a,b) = (\f{5}{2}, 2)$のときは$\rank A =2$で、$(a,b) \neq  (\f{5}{2}, 2)$のときは$\rank A =3$である。
    \item[(2)] $(a,b) \neq  (\f{5}{2}, 2)$ならば、$A \colon \R^4 \to \R^3$は全射なので、解がある。$(a,b) = (\f{5}{2}, 2)$のとき、拡大係数行列を考えると
    \begin{align*}
      \begin{pmatrix}
      1 & 1 & \f{5}{2} & 2 & 1 \\
      0 & 1 & 2 & 0  & 1 \\
      2 & 0 & 1 & 4 & 1
      \end{pmatrix} \sim
      \begin{pmatrix}
      1 & 1 & \f{5}{2} & 2 & 1 \\
      0 & 1 & 2 & 0  & 1 \\
      0 & -2 & -4 & 0 & -1
      \end{pmatrix}
      \sim \begin{pmatrix}
      1 & 1 & \f{5}{2} & 2 & 1 \\
      0 & 1 & 2 & 0  & 1 \\
      0 & 0 & 0 & 0 &  1
      \end{pmatrix}
    \end{align*}
    となるので、解はない。
  \end{description}
\end{sol}


\newpage

\subsubsection{} %{問 3}
\barquo{
広義積分
\[
\int_{-\infty}^{\infty} \f{\cos(\pi x)}{1 + x^2 + x^4} \; dx
\]
を求めよ。
}
\begin{sol}
$f$, $F$を
    \[
    f(x) = \f{\cos(\pi x)}{1 + x^2 + x^4}, \quad F(z) = \f{e^{\I \pi z}}{1 + z^2 + z^4}
    \]
により定める。$x \in \R$なら$f(x) = \Re F(x)$である。

ここで分母の$1 + z^2 + z^4$を因数分解しておく。
$\zeta = \exp(\I \pi / 3) = (1 + \sqrt{-3})/ 2$とする。$1+z + z^2$の根は$1$の原始$3$乗根であることから
\begin{align*}
  z^4 + z^2 + 1 &= (z^2 - \zeta^2)(z^2 - \zeta^4) \\
  &= (z - \zeta )(z + \zeta) (z - \zeta^2)(z + \zeta^2)
\end{align*}
である。

上反平面に含まれる半径$R$の半円を$C_R$とする。留数定理により、任意の$R > 1$について
\[
2 \pi \I ( \Res_{z=\zeta} F + \Res_{z=\zeta^2} F ) = \int_{-R}^R F(x) \; dx + \int_{C_R} f(z) \; dz
\]
が成り立つ。

ここで、
\begin{align*}
  \abs{\int_{C_R} f(z) \; dz } &\leq \int_0^{\pi} \abs{ \f{R \exp(\I R \pi e^{\I \grt}) }{ 1 + R^2e^{2 \I \grt} +  R^4e^{4 \I \grt}} } \; d\grt \\
  &\leq  \int_0^{\pi} \f{R e^{- R \pi \sin \grt} }{R^4 - R^2 - 1}   \; d\grt \\
  &\leq \f{R  }{R^4 - R^2 - 1}  \int_0^{\pi} \; d\grt \\
  &\leq  \f{R \pi }{R^4 - R^2 - 1}
\end{align*}
だから、$R \to \infty$のとき$\int_{C_R} f(z) \; dz \to 0$である。したがって
\[
 \int_{-\infty}^{\infty} f(x) \; dx =  \Re( 2 \pi \I ( \Res_{z=\zeta} F + \Res_{z=\zeta^2} F ) )
\]
であることがわかる。

実際に留数を計算しよう。詳細は省略するが、堅実な計算により
\begin{align*}
  \Res_{z=\zeta} F &= \f{ \exp(\I \pi \f{1 + \sqrt{-3}}{2} ) }{(2\zeta) (\zeta - \zeta^2) (\zeta + \zeta^2)  } \\
  &= \f{ - \I \exp(- \f{\sqrt{3}\pi}{2} ) }{2(1 - \zeta^2)} \\
  \Res_{z=\zeta^2} F &= \f{ \exp(\I \pi \f{-1 + \sqrt{-3}}{2} ) }{(\zeta^2 - \zeta) (\zeta^2 + \zeta) (\zeta^2 + \zeta^2)  } \\
  &=  \f{ - \I \exp(- \f{\sqrt{3}\pi}{2} ) }{2(1 + \zeta)}
\end{align*}
がわかる。$\gra = \exp(- \f{\sqrt{3}\pi}{2} )$とおこう。すると
\begin{align*}
  2 \pi \I ( \Res_{z=\zeta} F + \Res_{z=\zeta^2} F ) &= \gra \pi \left( \f{1}{1 - \zeta^2} + \f{1}{1+ \zeta} \right) \\
  &= \gra \pi \left( \f{2 - \zeta}{ 1 - \zeta^2} \right) \\
  &= \gra \pi
\end{align*}
である。$\gra \in \R$だから、
\[
 \int_{-\infty}^{\infty} f(x) \; dx = e^{- \f{\sqrt{3}\pi}{2} } \pi
\]
が結論される。
\end{sol}
\newpage

\subsubsection{} %{問 4}
\lbar{
閉区間$[0,1]$上の実数値関数列$\{ f_n \}^{\infty}_{n=1}$について、各$f_n$は広義単調増加であるものとする。つまり、$0 \leq x < y \leq 1$なら、$f_n(x) \leq f_n(y)$である。この関数列$\{ f_n \}^{\infty}_{n=1} $が$n \to \infty$で関数$f$に各点収束したとする。
\begin{description}
  \item[(1)] 任意の$0 \leq x < y \leq 1$に対し、不等式
  \[
  \sup_{x \in [x,y]} \abs{f_n(z) - f(z)} \leq \max\{ \abs{f_n(x)- f(y)}, \abs{f_n(y)-f(x)}  \}
  \]
  を示せ。
  \item[(2)] 関数$f$が連続であるとき、関数列$\{ f_n \}^{\infty}_{n=1}$は$f$に$[0,1]$上で一様収束することを示せ。
\end{description}
}
\begin{sol} ${}$
\begin{description}
  \item[(1)] まず$f$が広義単調増加であることを示す。$0 \leq x < y \leq 1$とする。$\ve > 0$が与えられたとする。$f_n$が$f$に各点収束することにより
  \begin{align*}
    n \geq N(x) \to \abs{f(x) - f_n(x)} < \ve \\
      n \geq N(y) \to \abs{f(y) - f_n(y)} < \ve
  \end{align*}
  なる$N(x), N(y)$の存在がわかる。したがって$n \geq \max\{ N(x), N(y) \}$のとき
  \begin{align*}
    f(y) - f(x) + 2\ve &= (f(y) + \ve ) - f(x) + \ve \\
    &\geq f_n(y) - f(x) + \ve &(-\ve < f(y)-f_n(y) < \ve \text{より}) \\
    &\geq f_n(y) - f_n(x) &(-\ve < f(x)-f_n(x) < \ve \text{より}) \\
    &\geq 0
  \end{align*}
  がわかる。$\ve > 0$は任意だったから、$f(y) \geq f(x)$がわかる。つまり$f$は広義単調増加である。

  したがって任意の$z \in [x,y]$に対して
\begin{align*}
f_n(z) - f(z) \leq f_n(y) - f(x) \\
f(z) - f_n(z) \leq f(y) - f_n(x)
\end{align*}
が成り立つので、
\[
\abs{f_n(z) - f(z)} \leq \max\{ \abs{f_n(x)- f(y)}, \abs{f_n(y)-f(x)}  \}
\]
である。右辺は$z$の取り方によらないので、
\[
\sup_{x \in [x,y]} \abs{f_n(z) - f(z)} \leq \max\{ \abs{f_n(x)- f(y)}, \abs{f_n(y)-f(x)}  \}
\]
がいえた。
\item[(2)] $\ve > 0$が与えられたとする。$I = [0,1]$はコンパクトなので、$f$は一様連続であることまでいえる。そこで
\[
\abs{x -y} < \grd \to \abs{f(x) - f(y)} < \ve
\]
なる$\grd > 0$がある。この$\grd$を固定し、$B(z) = [z - \grd/3, z + \grd/3] \cap I$とする。$\grd > 0$なので、$I = \bigcup_{i=1}^m B(z_i)$なる有限個の$z_i \in I$をとることができる。$B(z_i ) = [x_i, y_i]$と表すことにする。

$f_n$は$f$に各点収束しているので、
\begin{align*}
  n \geq N(x_i) &\to \abs{f(x_i) - f_n(x_i)} < \ve \\
  n \geq N(y_i) &\to \abs{f(y_i) - f_n(y_i)} < \ve
\end{align*}
なる$N(x_i), N(y_i)$がある。そこで
\[
n \geq \max\{ N(x_1), \cdots , N(x_m) , N(y_1) , \cdots , N(y_m)  \}
\]
とする。このとき
\begin{align*}
  \abs{f_n(x_i)- f(y_i) } &\leq \abs{ f_n(x_i) - f(x_i) } + \abs{ f(x_i) - f(y_i) } \\
  &\leq 2\ve \\
\abs{f_n(y_i)- f(x_i) }  &\leq \abs{ f_n(y_i) - f(y_i) } + \abs{ f(y_i) - f(x_i) } \\
&\leq 2\ve
\end{align*}
が成り立つ。したがって(1)により、不等式評価を端点に押しつけることができて
\begin{align*}
\sup_{z \in I } \abs{f_n(z) - f(z)} &\leq \max_{1 \leq i \leq m} \sup_{x \in [x_i,y_i]} \abs{f_n(z) - f(z)} \\
&\leq \max_{1 \leq i \leq m} \max\{ \abs{f_n(x_i)- f(y_i)}, \abs{f_n(y_i)-f(x_i)}  \} \\
&\leq 2\ve
\end{align*}
である。これで一様収束がいえた。
\end{description}
\end{sol}

\newpage

\subsubsection{} %{問 5}
\barquo{
$p$を素数とし、$\F_p = \Z / p \Z$を位数$p$の有限体とする。行列の乗法による群$G$を
\[
G = \setmid{ \pmat{1 & a & b \\ 0 & 1 & c \\ 0 & 0 & 1 } }{a,b,c \in \F_p}
\]
で定める。このとき、$G$から乗法群$\C^{\tm} = \C \sm \{ 0 \}$への準同形写像の個数を求めよ。
}
\begin{sol}
 集合$\Hom(G, \C^{\tm})$と$\Hom(G/[G,G], \C^{\tm})$の間には全単射がある。したがって$G/[G,G]$の構造を決定すればよい。そのためにまず$[G,G]$を決定する。
    \[
    A = \pmat{1 & a & b \\ 0 & 1 & c \\ 0 & 0 & 1},  \quad B = \pmat{1 & \grd & \beta \\ 0 & 1 & \grg \\ 0 & 0 & 1}
    \]
    とおく。($\gra$は$a$と間違えやすいので、$\grd$を使った。)計算すれば、このとき
    \begin{align*}
    ABA^{-1}B^{-1} = \pmat{1 & 0  & a \grg - c \grd \\ 0 & 1 & 0 \\ 0 & 0 & 1 }
  \end{align*}
  であることが判る。$a \grg - c \grd$は$\F_p$全体をわたるので、
  \[
  [G,G] = \setmid{ \pmat{1 & 0 & d \\ 0 & 1 & 0 \\ 0 & 0 & 1} }{d \in \F_p }
  \]
  が結論できる。

  次に$G/[G,G]$の構造を決定したい。
  \[
  E_1 = \pmat{1 & 1 & 0 \\ 0 & 1 & 0 \\ 0 & 0 & 1}, \quad E_2 =  \pmat{1 & 0 & 0 \\ 0 & 1 & 1 \\ 0 & 0 & 1}
  \]
  とし、$E_1, E_2 \in G/ [G,G]$と見なす。
  \[
  E_1^n = \pmat{1 & n & 0 \\ 0 & 1 & 0 \\ 0 & 0 & 1}, \quad E_2^m =  \pmat{1 & 0 & 0 \\ 0 & 1 & m \\ 0 & 0 & 1}
  \]
  なので、$E_1, E_2$は位数がちょうど$p$である。また、$C = E_1^n = E_2^m$とするとき
  \[
  1 = E_1^n E_2^{-m} = \pmat{ 1 & n & -nm \\ 0 & 1 & -m \\ 0 & 0 & 1}
  \]
  だから$n=m=0$が従う。つまり$\kakko{E_1} \cap \kakko{E_2} = 1$である。$G/[G,G]$はAbel群なので積による準同形
  $
\kakko{E_1} \tm \kakko{E_2} \to G/[G,G]
  $
  がある。これは、$\kakko{E_1} \cap \kakko{E_2} = 1$により単射である。位数$p^2$の有限群の間の単射なので、とくに同型である。よって$G/[G,G] \cong \F_p^2$がわかった。

あとは$\# \Hom(\F_p^2, \C^{\tm})$を求めよう。これは$\# \Hom(\F_p,\C^{\tm})$の$2$乗である。$\# \Hom(\F_p,\C^{\tm}) = p$より求める答えは$p^2$である。

\end{sol}

\newpage

\subsubsection{} %{問6}
\barquo{
$\R^4$の部分空間$M$を
\[
M = \setmid{(x,y,z,w) \in \R^4}{x^2 + y^2 + z^2 + w^2 = 1, \; xy + zw = 0}
\]
で定める。
\begin{description}
  \item[(1)] $M$が$2$次元微分可能多様体になることを示せ。
  \item[(2)] $M$上の関数$f$を
  \[
  f(x,y,z,w) = x
  \]
  で定めるとき、$f$の臨界点をすべて求めよ。ただし、$p \in M$が$f$の臨界点であるとは、$p$における$M$の局所座標$(u,v)$に関して
  \[
  \f{\del f}{\del u}(p) =   \f{\del f}{\del v}(p) = 0
  \]
  となることである。
\end{description}
}
\begin{sol} ${}$
  \begin{description}
    \item[(1)] $F \colon \R^4 \to \R^2$を
      \[
      F(x,y,z,w) = \pmat{x^2 + y^2 + z^2 + w^2 - 1 \\ xy+zw}
      \]
      により定める。$M = F^{-1}(O)$である。$p=(x,y,z,w) \in M$としよう。$p$におけるヤコビアンを計算すると
      \[
      JF_p = \pmat{2x & 2y & 2z & 2w \\ y & x & w & z }
      \]
      である。ここで$p \neq O$より$\rank JF_p \geq 1$である。仮に$\rank JF_p = 1$ならば、$JF_p$の2つの行は1次従属である。よって、$p \neq O$により$(y , x, w,z) = c(x,y,z,w)$なる定数$c \in \R$がある。このとき$xy + zw = c(x^2 + z^2) = 0$となり、$p \neq O$に矛盾。よって$\rank JF_p = 2$である。ゆえに$p$は$F$の正則点であり、$M$は$\R^4$の$2$次元部分多様体。
      $F$は$\bfC^{\infty}$級なので、$M$は微分可能になる。
      \item[(2)] $f \colon M \to \R$の$\R^4$への自然な拡張を$\wt{f}$とする。このとき$p \in M$に対して$T_p M \subset \R^4$と見なせば、$T_p M = \Ker JF_p$であるから、
      \begin{align*}
        \text{$p$が$f$の臨界点} &\iff \rank (df_p \colon T_p M \to \R) < 1 \\
        &\iff \dim \Ker df_p = 2 \\
        &\iff \dim \Ker \pmat{JF_p \\ J\wt{f}_p} = 2 \\
        &\iff \rank \pmat{2x & 2y & 2z & 2w \\ y & x & w & z \\ 1  & 0 & 0 & 0} = 2 \\
        &\iff \rank \pmat{0 & 2y & 2z & 2w \\ 0 & x & w & z \\ 1  & 0 & 0 & 0} = 2
      \end{align*}
      である。いま$p=(x,y,z,w) \in M$が臨界点であったと仮定する。このとき$(x,w,z)$と$(y,z,w)$は1次従属である。よって$(y,w,z) = 0$かまたは、ある$c \in \R$が存在して$(x,w,z)=  c(y,z,w )  $である。$(y,w,z) = 0$なら$p =(\pm 1, 0, 0 ,0)$である。$(x,w,z)=  c(y,z,w )  $なら、$c(y^2 + z^2)=0$より$p=(0, \pm 1, 0, 0)$である。

      逆に$p=(\pm 1, 0, 0 ,0), (0, \pm 1, 0, 0)$ならば$p \in M$であり、$f$の臨界点であることはあきらかなので、臨界点はこれですべて求まったことになる。
  \end{description}

\end{sol}


\newpage

\subsubsection{} %{問7}
\barquo{
$A$を実正方行列、$k$を正の整数とし、$\rk (A^{k+1}) = \rk (A^k)$が成り立つとする。このとき、任意の整数$m \geq k$に対し、$\rk (A^m) = \rk (A^k)$であることを証明せよ。ここで行列$X$に対し、$\rk(X)$は$X$の階数を表す。
}
\begin{proof}
  仮定から、$\Ker A^{k+1}$の次元と$\Ker A^k$の次元は等しい。包含関係があって次元が等しいので、$\Ker A^{k+1} = \Ker A^k$である。ここで$m \geq k +2$に対して$x \in \Ker A^m$と仮定する。そうすると$A^m x = A^{m-k-1} A^{k+1} x$だから$A^{m-k-1} x \in \Ker A^{k+1} = \Ker A^k$である。よって$A^k A^{m-k-1} x = A^{m-1} x = 0$であり、$x \in \Ker A^{m-1}$がわかる。これを帰納的に繰り返して、
  $\Ker A^m \subset \Ker A^{m-1} \subset \cdots \subset \Ker A^k$を得る。逆はあきらかなので$\Ker A^m = \Ker A^k$である。よってとくに階数も等しい。
\end{proof}
