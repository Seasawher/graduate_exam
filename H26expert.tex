\section{平成26年度 専門科目}

\subsubsection{}%1
\barquo{
$\C[X,Y]$を複素数係数の$2$変数多項式環、$A=\C[X,Y]/(X^2 + Y^3 - 1)$とし、$X,Y \in \C[X,Y]$の$A$での類をそれぞれ$x,y $とおく。
\begin{description}
  \item[(i)] $A$が整域であり、$A$の商体$L$が$\C(y)$の$2$次拡大であることを証明せよ。
  \item[(ii)] $A$が$\C[y]$の$L$における整閉包であることを証明せよ。
  \item[(iii)] $y$が$A$の既約元であることを証明せよ。
  \item[(iv)] $A$がUFD(一意分解整域)であるかどうか理由をつけて決定せよ。
\end{description}
}
\begin{sol} ${}$
  \begin{description}
    \item[(i)] $B:= \C[Y]$, $K:= \C(Y)$とし、$A$は$B$代数として$A= B[X]/(X^2 + Y^3-1)$ととらえなおす。$B$はPIDなので$B[X]$はUFDである。このとき$X^2 + Y^3 - 1$は$\frakp = (Y-1)$に関するEisenstein多項式であるため$K[X]$の元として既約かつ素元である。ゆえに$A$は整域。

    $A$を$B$代数として捉えたのと同様$L$も$K$代数として$L=K[X]/(X^2 + Y^3 -1)$と捉えなおす。
    \[
    \xymatrix{
    K \ar[r] & L \\
    B \ar[u] \ar[r] & A \ar[u]
    }
    \]
    $X \in K[X]$の$L$での像を$x$と書くことにする。$L$は$K$上$x$で生成されていて$x$の$K$上の最小多項式は$T^2 + Y^3 - 1 \in K[T]$なので$[L:K]=2$であることが判る。
    \item[(ii)] $x \in L$は$B$上整なので$A/B$は整拡大。逆に$z \in L$が$B$上整だったと仮定する。$[L:K] =2$なので$z = ax + b$なる$a,b \in K$がある。ここで$L/K$は分離拡大なので、$G := \Hom_K^{\text{al}}(L,\ol{K}) = \{ \grs_1, \grs_2 \}$とするとトレース
    $\Tr_{L/K} \colon L \to K$とノルム$\Norm_{L/K} \colon L \to K$は
    \begin{align*}
      \Tr_{L/K}(s) &= \sum_{\grs \in G} \grs(s) \\
      \Norm_{L/K}(s) &= \prod_{\grs \in G} \grs(s)
    \end{align*}
    と計算できる。$B$はPIDなのでとくに整閉であり、$s \in L$が$B$上整ならば$s$のトレースとノルムも$B$の元となる。したがって
    \begin{align*}
      \Tr_{L/K}(z) &= 2b \in B \\
      \Norm_{L/K}(z-b) &= a^2(Y^3 - 1) \in B
    \end{align*}
    が得られる。$Y^3 - 1$は平方因子を持たないので$a \in B$でなくてはならない。よって$z \in A$である。$z$は任意だったから、これで$A$が$L$における$B$の整閉包であることがいえた。
\item[(iii)] ハイリホーによる。$Y \in A$が可約だったとする。このとき非単元$\gra , \beta \in A$があって$Y = \gra \beta$を満たす。ノルムをとって$Y^2 = \Norm_{L/K}(\gra) \Norm_{L/K}(\beta)$を得る。$B$はUFDなので素元$Y \in B$によるオーダーを考えることができる。(ii)により$A$は$L$における$B$の整閉包であったので、各$\grs \in G$は$\grs(A) \subset A$を満たす。
$\gra, \beta$は$A$の単元ではないので、よって$\Norm_{L/K}(\gra), \Norm_{L/K}(\beta)$も$B$の単元ではない。以上により$\Norm_{L/K}(\gra), \Norm_{L/K}(\beta)$の素元$Y \in B$に関するオーダーは$1$である。よって$\Norm_{L/K}(\gra) = uY$なる単元
$u \in B^{\tm} = \C^{\tm}$がある。$\gra = c x + d \; (c,d \in B)$とおくと
\[
u^{-1} Y = c^2 (Y^3-1) + d^2
\]
が得られる。ここで$\deg(c^2(Y^3-1)) = 2 \deg c + 3$は奇数で$\deg d^2 = 2 \deg d$は偶数なので
\begin{align*}
  1 &= \deg (u^{-1}Y) \\
  &= \deg( c^2 (Y^3-1) + d^2 ) \\
  &= \max\{ 2 \deg c + 3,  2 \deg d \} \\
  &\geq 3
\end{align*}
となって矛盾。よって$Y \in A$は既約元である。
\item[(iv)] 計算すると
\begin{align*}
  A/(Y) &\cong B[X]/(X^2 + Y^3 -1, Y) \\
  &\cong \C[X,Y]/(X^2-1,Y) \\
  &\cong \C[X]/(X-1)(X+1) \\
  &\cong \C^2
\end{align*}
なので$Y \in A$は素元ではない。もしも$A$がUFDなら既約元はすべて素元であるはずなので、$A$はUFDではない。
  \end{description}
\end{sol}

\newpage


\subsubsection{}%2
\barquo{
$K \subset \C$を部分体、$p$を素数とする。$\C$に含まれる任意の有限次拡大$L/K$に対し、$L=K$でなければ$[L:K]$は$p$で割り切れると仮定する。このとき、$\C$に含まれる任意の有限次拡大$L/K$に対し、$[L:K]$は$p$のべき($1$を含む)であることを証明せよ。
}
\begin{sol}
  $K$の有限次拡大$L \subset \C$が与えられたとする。$L$の$K$上のGalois閉包を$\wt{L}$とする。$G:= \Gal(\wt{L}/K)$とおく。$G$のSylow-$p$部分群を$H$とし、$H$の不変体
  \[
  \wt{L}^H = \setmid{x \in \wt{L}}{\forall \grs \in H \quad \grs(x) = x }
  \]
  を考える。Galoisの基本定理により$[\wt{L}:\wt{L}^H] = \# H$だから、$[\wt{L}^H : K] = \# (G / H)$であり$[\wt{L}^H : K]$は$p$で割り切れない。よって仮定により$\wt{L}^H =K$であるから$H=G$であり、とくに$[\wt{L}:K]$は$p$のベキである。$[L:K]$は$[\wt{L}:K]$を割り切るので、
  $[L:K]$も$p$ベキ($1$を含む)である。
\end{sol}


\newpage


\subsubsection{}%3
\barquo{
$\zeta$を$1$の原始$7$乗根$e^{2\pi \I / 7}$とし、$\C$の部分集合
\[
A = \setmid{ a_1 \zeta +  a_2 \zeta^2 +  a_3 \zeta^3 + a_4 \zeta^4 + a_5 \zeta^5  + a_6 \zeta^6 }{a_1, a_2, a_3, a_4, a_5, a_6 \in \{0,1 \}  }
\]
を考える。このとき$\Q(\gra) = \Q(\zeta)$となる$a \in A$となる$a \in A$の個数を求めよ。
}
\begin{sol}
  $\Q(\zeta)$は$\Q$のGalois拡大であり、円分体論により$G:= \Gal(\Q(\zeta)/ \Q)$は$(\zyu{7})^{\tm}$と同型である。ここで次が成り立つ。

\lem{
次は同値。
\begin{description}
  \item[(i)] $\Q(\gra) = \Q(\zeta)$
  \item[(ii)] $\forall \grs \in G \sm \{ 1 \} \quad \grs(\gra) \neq \gra$
\end{description}
}
\begin{proof} ${}$
  \begin{description}
    \item[(i)$\To$(ii)] 対偶を示す。ある$\grs \in G \sm \{ 1 \}$に対し$\grs(\gra)= \gra$だとする。このとき$\Q(\gra)$は$\kakko{\grs}$の不変体に含まれ、$\Q(\zeta)$より真に小さい。
    \item[(i)$\To$(ii)] 仮定より単射$G \to \Hom_{\Q}(\Q(\gra), \ol{\Q})$があるので
\[
[\Q(\gra): \Q] = \# \Hom_{\Q}(\Q(\gra), \ol{\Q}) \geq \# G = [\Q(\zeta):\Q]
\]
    が得られる。
    よって$\Q(\gra) = \Q(\zeta)$である。
  \end{description}
\end{proof}
したがって$\# \setmid{ \gra \in A }{ \forall \grs \in G \sm \{ 1 \} \quad \grs(\gra) \neq \gra }$を求めればよい。$(\zyu{7})^{\tm}$は$3$を生成元とする巡回群である。対応する$G$の生成元を$\tau$とする。$I := \{ 0,1\}$とおく。$a_i$の番号を付け替えて
\[
A = \setmid{  a_1 \zeta +  a_2 \zeta^3 +  a_3 \zeta^2 + a_4 \zeta^6 + a_5 \zeta^4  + a_6 \zeta^5 }{ (a_1, \cdots , a_6) \in I^6 }
\]
とみなす。このとき$A$の元に対する$\tau$の作用は$I^6$の元に対する$s := (123456) \in \frakS_6$の作用と解釈できる。ただし
\begin{align*}
\kakko{s} \tm I^6 &\to I^6 \\
(\grs, (a_i)_i ) &\mapsto (a_{\grs(i)})_i
\end{align*}
として作用を定めるものとする。したがって求めるべきものは、集合
\[
 B :=  \setmid{ a \in I^6 }{ \forall \grs \in \kakko{s} \sm \{ 1 \} \quad \grs(a) \neq a   }
\]
の位数である。ここで条件$\forall \grs \in \kakko{s} \sm \{ 1 \} \quad \grs(a) \neq a$は$\# \Orbit (a) = 6$つまり$\# \Stab(a) = 1$を意味する。したがって
\begin{align*}
  \# B &=  \setmid{ a \in I^6 }{ \# \Stab(a) = 1  } \\
  &= 64 - \setmid{ a \in I^6 }{ \# \Stab(a) \geq 2  } \\
  &= 64-  \setmid{ a \in I^6 }{ \Stab(a) = \kakko{s^2}  } - \setmid{ a \in I^6 }{ \Stab(a) = \kakko{s^3}  } - \setmid{ a \in I^6 }{ \Stab(a) = \kakko{s}  }
\end{align*}
である。いま$\Stab(a) = \kakko{s}$となる$a \in I^6$は
\[
(0,0,0,0,0,0) \quad (1,1,1,1,1,1)
\]
の$2$個。$\Stab(a) = \kakko{s^2}$となる$a \in I^6$は
\[
(0,1,0,1,0,1) \quad (1,0,1,0,1,0)
\]
の$2$個。$\Stab(a) = \kakko{s}$となる$a \in I^6$は
\begin{align*}
  &(0,0,1,0,0,1) \quad (0,1,0,0,1,0) \\
&(0,1,1,0,1,1) \quad (1,0,0,1,0,0) \\
&(1,0,1,1,0,1) \quad (1,1,0,1,1,0)
\end{align*}
の$6$個。ゆえに$\# B = 64 - (2+2+6) = 54$が求める答えである。
\end{sol}
