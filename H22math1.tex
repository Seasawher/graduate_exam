\section{平成22年度 数学I}

\subsubsection{}%1
\barquo{
$A^5 = 2E_n$を満たす$n$次正方行列で成分がすべて有理数のものが存在するための、$n$に関する必要十分条件を求めよ。(ただし、$E_n$は$n$次単位行列である)
}
\begin{sol}
  $A^5 = 2E_n$なる$A \in M_n(\Q)$が存在したとする。このとき$(\det A)^5 = 2^n$より$\det A \in \Q$は$\Z$上整である。よって$\Z$が整閉であることから$\det A \in \Z$であり、$n$は$5$の倍数である。

  逆に$n$が$5$の倍数だと仮定する。$A^5 = 2E_n$となるような$A$を構成したいが、ブロック分けすることにより$n=5$の場合に帰着できる。だから$n=5$としよう。固有多項式を$\Phi$で表すことにする。このとき$A^5 = 2E_5$であることと$\Phi_A(t) = t^5 -2$であることは同値である。なぜならば!$\Phi_A(t) = t^5 -2$ならばCaly-Hamiltonの定理より$A^5 = 2E_5$であることが直ちにわかる。逆に$A^5 = 2E_5$とする。このとき$A$の$\Q$上の最小多項式は$t^5 -2$を割り切るはずだが、
  $t^5 -2 \in \Q[t]$は既約多項式なので、$t^5 -2$が最小多項式である。ゆえに再びCaly-Hamiltonの定理より$t^5 - 2$は$\Phi_A(t)$を割り切ることがわかるが、どちらも$5$次モニック多項式なので両者は一致していなくてはならない。よって$\Phi_A(t)=t^5 -2$が従う。ゆえにかくのごとし。そこで$\Phi_A(t) = t^5 -2$となるような$A$を構成すればいいことになるが、それは
  \[
  t^5 - 2 = \det \pmat{ t& -1& 0& 0& 0 \\ 0& t& -1& 0& 0 \\ 0& 0& t& -1& 0  \\ 0& 0& 0& t& -1 \\ -2 &0& 0& 0& t}
  \]
  なので
  \[
  A =  \pmat{ 0& 1& 0& 0& 0 \\ 0& 0& 1& 0& 0 \\ 0& 0& 0& 1& 0  \\ 0& 0& 0& 0& 1 \\ 2 &0& 0& 0& 0}
  \]
  とおけば$\Phi_A(t)=t^5 - 2$となる。以上により求める必要十分条件は、$n$が$5$の倍数であることである。
\end{sol}

\newpage

\subsubsection{}%2
\barquo{
$f$を、点$0$を含む開区間で$C^1$級の函数とするとき、極限
\[
\lim_{h \to + 0} \f{1}{h^2} \left\{ \int_0^h f(x) \ dx - h f(0) \right\}
\]
を求めよ。
}

\newpage

\subsubsection{}%3
\barquo{
$(\zyu{525})^{\tm}$の元で位数が$4$であるものの個数を求めよ。ただし$(\zyu{525})^{\tm}$は可換環$\zyu{525}$の可逆な元全体の作る群である。
}

\newpage

\subsubsection{}%4
\barquo{
$X$を位相空間、$e \in X$とし
\[
\grO = \setmid{\gra \colon [0,1] \to X}{\text{$\gra$は連続写像、$\gra(0)=\gra(1)=e$}}
\]
とする。$\grO$の同値関係$\simeq$を次で定義する。

$\gra , \beta \in \grO$に対して、連続写像$F \colon [0,1] \tm [0,1] \to X$で
\begin{align*}
  F(t,0) &= \gra(t) \quad (0 \leq t \leq 1) \\
  F(t,1) &= \beta(t) \quad (0 \leq t \leq 1) \\
  F(0,s) &= F(1,s)= e \quad (0 \leq s \leq 1)
\end{align*}
を満たすものが存在するとき、$\gra \simeq \beta$と定める。また、位相空間$X$は
\[
\mu(x,e) = \mu(e,x) = x \quad (x \in X)
\]
を満たす連続写像$\mu \colon X \to X \to X$をもつものとする。$\gra , \beta \in \grO$に対して$\grO$の元$\gra * \beta$と$\gra \sh \beta$を次で定義する。
\begin{align*}
  (\gra * \beta)(t) &= \begin{cases}
  \gra(2t) &(0 \leq t \leq 1/2) \\
  \beta(2t-1) &(1/2 \leq t \leq 1)
\end{cases} \\
(\gra \sh \beta)(t) &= \mu(\gra(t), \beta(t)) \quad (0 \leq t \leq 1)
\end{align*}
このとき、次の命題(1),(2),(3)を示せ。
\begin{description}
  \item[(1)] $\gra , \beta, \gra',\beta' \in \grO$に対して$\gra \simeq \beta$, $\gra' \simeq \beta'$ならば$\gra \sh \gra' \simeq \beta \sh \beta'$である。
  \item[(2)] $\gra , \beta \in \grO$に対して$\gra \sh \beta \simeq \gra * \beta$である。
  \item[(3)] $\gra , \beta \in \grO$に対して$\gra \sh \beta \simeq \beta * \gra$である。
\end{description}
}

\newpage


\subsubsection{}%5
\barquo{
次の積分を求めよ。
\[
\int_{- \infty}^{\infty} \f{ e^{ix} - 1 }{x(x^2 +1)} dx
\]
}
