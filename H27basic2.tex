\bfsection{平成27年度 基礎科目 II}

\bfsubsection{問4}
\barquo{
$1$以上$3500$以下の整数$x$のうち、$x^3 + 3x$が$3500$で割り切れるものの個数を求めよ。
}
\begin{sol}
$3500 = 2^2 \tm 5^3 \tm 7$なので、中国式剰余定理より$\Z / 3500 \Z \cong \Z / 4\Z \tm \Z / 125 \Z \tm \Z / 7\Z$である。
いま$\Z / 4\Z$で$x$に値を代入することにより調べると$x^3 + 3x = 0 \in \Z / 4 \Z \iff x = 0, \pm 1$がわかる。

$\Z / 7 \Z$で考えると$x^3 + 3x = x(x^2 + 3) = x(x^2 - 4)= x(x+2)(x-2)$であり、$\Z / 7 \Z$は整域だから$x^3 + 3x = 0 \in \Z / 7 \Z \iff x = 0, \pm 2$が判る。

$\Z / 125 \Z$で考える。$x^3 + 3x = x(x^2 + 3)$であるが、$x^2 + 3$は決して$5$の倍数にならないので$\Z / 125 \Z$において常に単元である。よって$x^3 + 3x = 0 \in \Z / 125 \Z \iff x = 0$が判る。

以上の議論により求める$x$の個数は$3 \tm 3 \tm 1 = 9$個である。
\end{sol}

\newpage

\bfsubsection{問7}
\barquo{
$A$を実正方行列とする。このとき、ある正の整数$k$が存在して$\tr (A^k) \geq 0$となることを示せ。ただし$\tr$は行列のトレースを表す。
}
