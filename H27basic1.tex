\bfsection{平成27年度 基礎科目I}

\bfsubsection{問1}
\barquo{
次の広義積分を求めよ。
\[
\iint_D \f{y^2 e^{-xy} }{ x^2 + y^2 } \ dx dy
\]
ここで、$D=\setmid{(x,y) \in \R^2 }{0 < y \leq x}$とする。
}
\begin{sol}
  $x=r \cos \grt$, $y = r\sin \grt$とおくと$dx dy = r dr d\grt$であって
  \begin{align*}
    \iint_D \f{y^2 e^{-xy} }{ x^2 + y^2 } \ dx dy &= \int_0^{\pi/4} \ d\grt \int_0^{\infty} r \sin^2 \grt e^{- r^2 \sin \grt \cos \grt} \ dr \\
    &= \f{1}{2} \int_0^{\pi/4} \sin^2 \grt \left( \int_0^{\infty}   e^{- s \sin \grt \cos \grt} \ ds \right) \ d\grt \\
    &= \f{1}{2} \int_0^{\pi/4} \f{ \sin \grt }{\cos \grt } \ d\grt \\
    &= - \f{1}{2} \int_1^{1/ \sqrt{2} } \f{dt}{t} \\
    &= \f{\log 2}{4}
  \end{align*}
\end{sol}

\newpage

\bfsubsection{問2}
\barquo{
$\R^2$で定義された関数
\[
f(x,y) = \f{4x^2 + (y+2)^2}{x^2 + y^2 +1}
\]
のとりうる値の範囲を求めよ。
}
\begin{sol}
  
\end{sol}
