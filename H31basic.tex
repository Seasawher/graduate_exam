\section{平成31年度 基礎科目}


\subsubsection{}

\barquo{
$\gra$は$0 < \gra < \f{\pi}{2}$を満たす定数とする。このとき広義積分
\[
\iint_D e^{-(x^2 + 2xy \cos \gra + y^2)} \ dx dy
\]
を計算せよ。ただし、$D=\setmid{(x,y) \in \R^2 }{ x \geq 0, y \geq 0}$とする。
}
\begin{sol}
$x = r \cos \grt$, $y = r \sin \grt$と変数変換する。領域$D$は、$\setmid{(r,\grt)}{r \geq 0, 0 \leq \grt \leq \f{\pi}{2} }$へ移る。すると$dx dy = r dr d\grt$であって
\begin{align*}
  \iint_D e^{-(x^2 + 2xy \cos \gra + y^2)} \ dx dy &= \int_0^{\f{\pi}{2}} \ d\grt \int_{0}^{\infty} e^{-r^2(1 + \sin 2 \grt \cos \gra)} r \ dr \\
  &= \f{1}{2} \int_0^{\f{\pi}{2}} \ d\grt \int_{0}^{\infty} e^{-r(1 + \sin 2 \grt \cos \gra)}  \ dr \\
  &= \f{1}{2} \int_0^{\f{\pi}{2}} \f{d \grt}{1 + \sin 2\grt \cos \gra} \\
  &= \f{1}{4} \int_0^{\pi} \f{d \grt}{1 + \sin \grt \cos \gra}
\end{align*}
と計算できる。さらに$t = \tan \f{\grt}{2}$として変数変換を行う。$d\grt = 2(1+t^2)^{-1} dt$で、$\sin \grt = 2t / (1+t^2)$だから
\begin{align*}
    \iint_D e^{-(x^2 + 2xy \cos \gra + y^2)} \ dx dy &= \f{1}{4} \int_0^{\infty} \f{2(1+t^2)^{-1} dt}{1 + 2t(1+t^2)^{-1}  \cos \gra} \\
    &= \f{1}{2} \int_0^{\infty} \f{dt}{(t+ \cos \gra)^2 + \sin^2 \gra } \\
    &= \f{1}{2} \int_{\cos \gra}^{\infty} \f{dt}{t^2 + \sin^2 \gra } \\
    &= \f{1}{2 \sin \gra} \int_{1 / \tan \gra}^{\infty} \f{dt}{t^2 + 1} \\
    &= \f{1}{2 \sin \gra} \left( \f{\pi}{2} - \arctan \left( \f{1}{\tan \gra} \right) \right)
\end{align*}
である。ここで、$\tan(\f{\pi}{2} - \gra ) = \f{1}{\tan \gra}$であることから、結論として次を得る。
\[
\iint_D e^{-(x^2 + 2xy \cos \gra + y^2)} \ dx dy = \f{\gra}{2 \sin \gra}
\]
\end{sol}


\newpage


\subsubsection{}
\barquo{
複素数$\gra$に対し、3次複素正方行列$A(\gra)$を次のように定める。
\[
A(\gra) = \pmat{ \gra -4 & \gra +4 & -2 \gra +1 \\ -2 & 2 \gra +1 & -2 \gra +2 \\ -1 & \gra & - \gra +2}
\]
\begin{description}
  \item[(1)] $A(\gra)$の行列式を求めよ。
  \item[(2)] $A(\gra)$の階数を求めよ。
\end{description}
}
\begin{sol} ${}$
\begin{description}
  \item[(1)] ある行に別の行の定数倍を足す操作を繰り返し行っていくと
  \begin{align*}
    A(\gra) &\sim \pmat{ \gra -3 & 4 & - \gra -1 \\ 0 & 1  & -2 \\ -1 & \gra & - \gra +2} \\
    &\sim \pmat{ \gra -3 & 0 & - \gra +7 \\ 0 & 1  & -2 \\ -1 & 0 & \gra +2} \\
    &\sim \pmat{ 0 & 0 & (\gra - 1)^2 \\ 0 & 1  & -2 \\ -1 & 0 & \gra +2} \\
  \end{align*}
  と変形できる。よって$\det A(\gra) = (\gra - 1)^2$である。
  \item[(2)] $\gra=1$のときは階数2である。それ以外のときは正則で、階数は3である。
\end{description}

\end{sol}


\newpage



\subsubsection{}
\barquo{
$(x_0, y_0) \in \R^2 \sm \{(0,0)\}$に対して、$\R$上の連立常微分方程式
\[
\begin{cases}
  \f{dx}{dt} = -x^2 y - y^3 \\
    \f{dy}{dt} = x^3 +  xy^2
\end{cases}
\quad
\begin{cases}
x(0) = x_0 \\
y(0) = y_0
\end{cases}
\]
の解$(x(t),y(t))$は周期を持つことを示し、最小の周期を求めよ。ただし正の実数$T$が$(x(t),y(t))$の周期であるとは、任意の$t \in \R$に対して
\[
(x(t + T),y(t + T)) = (x(t),y(t))
\]
が成り立つことである。
}
\begin{sol}
与式より
\begin{align*}
  x \f{dx}{dt} + y \f{dy}{dt} &= 0 \\
  \f{d}{dt}(x^2 + y^2) &= 0
\end{align*}
を得る。したがって$C = x^2 + y^2$は定数であり、$C = x_0^2 + y_0^2$が成り立つ。ゆえに与式は
\[
\begin{cases}
  \f{dx}{dt} = - Cy \\
    \f{dy}{dt} = Cx
\end{cases}
\]
と書き直せる。この連立方程式を一変数にまとめると
\[
\f{d^2 x}{dt^2} = - C^2 x
\]
となるが、この解空間は$\cos (C t)$と$\sin (Ct)$で張られる。したがって、一般解はこの線形結合で書けるのだから
\begin{align*}
  x(t) = x_0 \cos(Ct) - y_0 \sin (Ct) \\
  y(t) = y_0 \cos(Ct) + x_0 \sin (Ct)
\end{align*}
でなくてはならない。常微分方程式の初期値問題の解の一意性より、解はこれだけである。よって求める周期は$2\pi / C$である。
\end{sol}

\newpage


\subsubsection{}
\barquo{
$f$は$\R$上の実数値$C^1$級関数で任意の$x \in \R$に対して$f(x+1)=f(x)$を満たすとする。このとき以下の2条件は同値であることを示せ。
\begin{description}
  \item[(A)] 広義積分
  \[
  \int_1^{\infty} \f{1}{x^{1+f(x)^2}} \ dx
  \]
  が収束する。
  \item[(B)] $f(x) = 0$となる$x \in \R$が存在しない。
\end{description}
}
\begin{sol} ${}$
  \begin{description}
    \item[(B)$\To$(A)] このときある$\ve > 0$が存在して$\forall x \; f(x)^2 > \ve$が成り立つ。よって
    \begin{align*}
            \int_1^{\infty} \f{1}{x^{1+f(x)^2}} \ dx \leq \int_1^{\infty} \f{dx}{x^{1+\ve}}
            \leq \f{1}{\ve}
    \end{align*}
    より積分は有界である。被積分関数は正の値しかとらないので、これで広義積分の収束がいえた。
    \item[(A)$\To$(B)] 対偶を示そう。$f(a)=0$なる$a$があったとする。周期性から$f(a_1)=0$なる$1 \leq a_1 < 2$がとれる。$n \geq 2$に対し$ n \leq a_n < n+1$を$a_n = a_1 + n-1$で定める。$f(x)=f(x+1)$より、$f$はコンパクト空間$\R / \Z$上の$C^1$級関数である。
    とくに$f'$は有界であり、$\forall x \; \abs{f'(x)} \leq M$なる$M > 0$をとることができる。したがって平均値の定理を適用することにより、任意の$n$について
    \[
    \abs{f(x)} = \abs{f(x) - f(a_n)} \leq M \abs{x - a_n}
    \]
    が成り立つことがわかる。ここまでの議論を踏まえると次の補題が示せる。
\lem{
ある$r > 0$が存在して、任意の自然数$n \geq 2$に対して
\[
\int_{2n-2}^{2n} x^{-f(x)^2} \ dx \geq \f{r}{ \sqrt{\log 2n} }
\]
が成り立つ。
}
\begin{proof}
  以下のように計算できる。
  \begin{align*}
    \int_{2n-2}^{2n} x^{-f(x)^2} \ dx  &\geq \int_{2n-2}^{2n} \exp \left\{ - (\log x) f(x)^2 \right\} \ dx \\
    &\geq \int_{2n-2}^{2n} \exp \left\{ - (\log 2n) f(x)^2 \right\} \ dx \\
  &\geq \int_{a_{2n-2}}^{a_{2n-2}+1} \exp \left\{ - (\log 2n) f(x)^2 \right\} \ dx \\
  &\geq \int_{a_{2n-2}}^{a_{2n-2}+1} \exp \left\{ - M^2(\log 2n) (x-a_{2n-2})^2 \right\} \ dx \\
  &\geq \int_{a_{2n-2}}^{a_{2n-2}+1} \exp \left\{ - (M \sqrt{\log 2n} (x-a_{2n-2}) )^2 \right\} \ dx
\end{align*}
変数変換$y=M \sqrt{\log 2n}(x-a_{2n-2})$を行って
\begin{align*}
 \int_{2n-2}^{2n} x^{-f(x)^2} \ dx  &\geq  \f{1}{M \sqrt{\log 2n}} \int_{0}^{M \sqrt{\log 2n} } e^{-y^2} \ dy  \\
  &\geq  \f{1}{M \sqrt{\log 2n}} \int_{0}^{M \sqrt{\log 4} } e^{-y^2} \ dy
  \end{align*}
  したがって
  \[
  r = \f{1}{M} \int_{0}^{M \sqrt{\log 4} } e^{-y^2} \ dy
  \]
  とおけばよい。
\end{proof}
(A)$\To$(B)の証明に戻る。$R \geq 4$に対し、$4 \leq 2N \leq R$を満たす最大の$N \in \Z$を$N_R$とおく。すると
\begin{align*}
  \int_1^R \f{dx}{x^{1+f(x)^2}} &\geq \sum_{n=2}^{N_R} \int_{2n-2}^{2n} \f{dx}{x^{1+f(x)^2}} \\
  &\geq \sum_{n=2}^{N_R} \f{1}{2n} \int_{2n-2}^{2n} \f{dx}{x^{f(x)^2}} \\
  &\geq \sum_{n=2}^{N_R} \f{r}{2n \sqrt{\log 2n}}
\end{align*}
というように評価できる。さらに$1/x\sqrt{\log x}$は単調減少なので
\begin{align*}
\int_1^R \f{dx}{x^{1+f(x)^2}} &\geq r \int_{2}^{N_R+1} \f{dx}{2x \sqrt{\log 2x}}  \\
&\geq \f{r}{2} \int_{4}^{2N_R+2} \f{dy}{y \sqrt{\log y}}  \\
&\geq r (\sqrt{\log(2N_R + 2)} - \sqrt{\log 4}) \\
&\geq r (\sqrt{R} - \sqrt{\log 4})
\end{align*}
である。ゆえに結論が従う。
  \end{description}
\end{sol}


\newpage

\subsubsection{} %\bfsubsection{問5}
\barquo{
$n$を2以上の整数、$A$を$n$次複素正方行列とする。$A^{n-1}$は対角化可能でないが、$A^n$が対角化可能であるとき、$A^n=0$となることを示せ。
}
\begin{sol}
$\C$係数なので、Jordan標準形が存在する。$A$ははじめからJordan標準形であるとしてよい。
\[
A = \bigoplus_{i=1}^r J_{\grl_i}(a_i)
\]
とする。$a_1, \cdots , a_r$は(異なるとは限らない)固有値であり、$\grl_i$はそれぞれのジョルダン細胞のサイズである。
\[
A^n = \bigoplus_{i=1}^r J_{\grl_i}(a_i)^n
\]
は対角化可能なので、各$J_{\grl_i}(a_i)^n$も対角化可能。ここで$J_{\grl_i}(a_i)$のJordan分解
\[
S_i = \pmat{a_i &  & &  \\  &  \ddots & &  \\ & & \ddots & \\ &  & & a_i} \quad N_i = \pmat{0 & 1 & & \\ & \ddots & \ddots & \\ & & \ddots & 1 \\ & & & 0 }
\]
を考える。
\[
J_{\grl_i}(a_i)^n = S_i^n + \sum_{k=1}^n \binom{n}{k} S_i^{n-k} N_i^k
\]
であって、$S_i^n$は対角行列で$\sum_{k=1}^n \binom{n}{k} S_i^{n-k} N_i^k$はべき零行列だから、Jordan分解の一意性より
\[
\sum_{k=1}^n \binom{n}{k} S_i^{n-k} N_i^k = 0
\]
を得る。左辺は具体的に書くことができて、次のような$\grl_i$次行列
\[
\pmat{
0 & \binom{n}{1} a_i^{n-1} & \binom{n}{2} a_i^{n-2} & \cdots & \binom{n}{\grl_i - 1} a_i \\
  & 0 & \binom{n}{1} a_i^{n-1} & \cdots & \binom{n}{\grl_i - 2} a_i^2 \\
  &   &  \ddots &  & \vdots \\
  &   &        &  &  0
}
\]
である。$\grl_i = 1$のときにはこの等式から情報を得ることはできない。しかし$\grl_i \geq 2$ならば$a_i = 0$であることがわかる。つまりサイズが$2$以上のJordan細胞はべき零である。実はサイズが$1$のJordan細胞は存在しない。ハイリホーで示す。仮に存在したとする。$n \geq 2$という仮定より、このときサイズが$2$以上のJordan細胞のサイズは$n-1$以下でなくてはならない。したがって、サイズが$2$以上のJordan細胞はすべて$n-1$乗するとゼロである。よって$A^{n-1}$は対角化可能となるが、これは仮定に反しており矛盾。ゆえにサイズが$1$のJordan細胞は存在しないことが判るので、$A$のJordan細胞はことごとくべき零であり、$A^n=0$であることが導かれる。
\end{sol}

\newpage

\subsubsection{} %\bfsubsection{問6}
\barquo{
$\R^2$上の実数値連続関数$f$についての次の条件($*$)を考える。

($*$) 任意の正の実数$R$に対して、次の集合は有界である。
\[
\setmid{(x,y) \in \R^2}{ \abs{f(x,y)} \leq R }
\]
以下の問に答えよ。
\begin{description}
  \item[(1)] 条件($*$)をみたす連続関数$f$の例を与え、それが($*$)をみたすことを示せ。
  \item[(2)] 連続関数$f$が条件($*$)を満たすとき、次のいずれかが成り立つことを示せ。

  (a) $f$は最大値を持つが、最小値は持たない。

  (b) $f$は最小値を持つが、最大値は持たない。
\end{description}
}
\begin{sol} ${}$
  \begin{description}
    \item[(1)] たとえば$f(x,y) = x^2 + y^2$とすればよい。これが($*$)を満たすことはあきらか。
    \item[(2)] $f$が条件($*$)を満たすとする。$f$の可能性としては、次の4通りが考えられる。
\begin{description}
  \item[(A1)] $f$は上にも下にも有界
  \item[(A2)] $f$は上に有界だが下に有界でない
  \item[(A3)] $f$は下に有界だが上に有界でない
  \item[(A4)] $f$は上にも下にも有界でない
\end{description}
    それぞれの場合について考えていく。まず(A1)の場合、任意の$x$について$\abs{f(x)} \leq M$なる$M > 0$が存在する。よって仮定より、$\R^2$が有界となって矛盾。つまりそんな関数はない。

    次に(A2)の場合。$\sup f(x) = R$とする。仮定から集合
    \[
    V = \setmid{(x,y) \in \R^2}{ \abs{f(x,y)} \leq R }
    \]
    は有界閉集合である。よって$V$はコンパクト。$f(V)$もコンパクトなので、$f(V)$は最大値$M$を持つ。あきらかに$M \leq R$である。
    任意に$0 < \ve \leq R/2$が与えられたとしよう。$\sup f(x) = R$より$R-\ve < f(z)$なる$z$がある。このとき$z \in V$だから$R - M \leq \ve$であり、$0 < \ve \leq R/2$は任意だったから$R \leq M$でなくてはならない。よって$R=M$であり、$f$は最大値を持つが、最小値は持たない関数である。(A3)は(A2)と同様で、このとき$f$は最小値を持つが最大値を持たない。

    残る(A4)について考えよう。
    $
    K = \setmid{(x,y) \in \R^2}{f(x,y)=0 }
    $
    とすると、仮定から$K$は有界閉集合である。$M$を十分に大きな正の実数として、$K$をすっぽり含むような閉円板
    $
    B= \setmid{(x,y) \in \R^2}{ x^2 + y^2 \leq M}
    $
    をとることができる。$\R^2$を全体として補集合をとることにすると、このとき$B^c$は連結開集合である。
    \[
    U = \setmid{(x,y) \in \R^2}{f(x,y) > 0} \quad V = \setmid{(x,y) \in \R^2}{f(x,y) < 0}
    \]
    とおく。このとき$U$と$V$の共通部分は空であり、ともに開集合である。だから、$B^c = (U \cap B^c) \cup (V \cap B^c)$から、$B^c$が連結集合であることに矛盾。よってそのような関数はない。以上により示すべきことがいえた。
  \end{description}
\end{sol}


\newpage




\subsubsection{} %\bfsubsection{問7}
\barquo{
$2$以上の整数$n$に対し、$(i,j)$成分が$\abs{i-j}$となる$n$次正方行列を$A_n$とする。すなわち
\[
A_n = \pmat{
0 & 1 & 2 & \cdots & n-1 \\
1 & 0 & 1 & \cdots & n-2 \\
2 & 1 & 0 & \cdots & n-3 \\
\vdots & \vdots & \vdots & \ddots & \vdots \\
n-1 & n-2 & n-3 & \cdots & 0
}
\]
とする。$A_n$の行列式を求めよ。
}
\begin{sol}
  $n \leq 4$のときに具体的に求めることは省略する。説明の都合上、$n \geq 5$とする。行または列に関する基本変形によって行列式は不変であることを利用しよう。$1$列目に$n$列目を足すと
\[
    \det A_n = \det \pmat{
    n-1 & 1 & 2 & \cdots & n-1 \\
    n-1 & 0 & 1 & \cdots & n-2 \\
    n-1 & 1 & 0 & \cdots & n-3 \\
    \vdots & \vdots & \vdots & \ddots & \vdots \\
    n-1 & n-2 & n-3 & \cdots & 0
    }
  \]
  のように数字が揃えられる。$1$行目を$2$行目以降から引くことにより、ある$n-1$次正方行列$B_n$に関して
  \[
  \det A_n = (n-1)\det \pmat{
  1 & * \\
  0 & B_n
  }
  \]
  という形になる。ここで$B_n$の$(i,j)$成分を$b_{i,j}$とすると
  \[
  b_{i,j} = \abs{i-j} - j = \begin{cases}
-i &(i \leq j, \text{上半分}) \\
  i - 2j &(i \geq j, \text{下半分})
\end{cases}
  \]
  である。つまり、具体的に書けば
  \[
  B_n = \pmat{
  -1 & -1 & -1 & \cdots & -1 \\
  0 & -2 & -2 & \cdots & -2 \\
  1 & -1 & -3 & \cdots & -3 \\
  \vdots & \vdots & \vdots & \ddots & \vdots \\
  n-3 & n-5 & n-7 & \cdots & -(n-1)
  }
  \]
  ということである。$B_n$の$1$行目の$i-2$倍を$i$行目に加えることにより、ある$n-2$次正方行列$C_n$に関して
  \[
  B_n \sim \pmat{
  -1 & * \\
  0 & C_n
  }
  \]
  という形になる。ここで$C_n$の$(i,j)$成分を$c_{i,j}$とすると
  \begin{align*}
    c_{i,j} &= b_{i+1,j+1} - (i-1) \\
    &= \begin{cases}
  -2i &(i \leq j, \text{上半分}) \\
   - 2j &(i \geq j, \text{下半分} )
  \end{cases}
  \end{align*}
  が成り立つ。つまり、具体的に書けば
  \[
  C_n = \pmat{
  -2 & -2 & -2 & \cdots & -2 \\
  -2 & -4 & -4 & \cdots & -4 \\
  -2 & -4 & -6 & \cdots & -6 \\
  \vdots & \vdots & \vdots & \ddots & \vdots \\
 -2 & -4 & -6 & \cdots & -2(n-2)
  }
  \]
  ということである。この行列は行基本変形で対角成分がすべて$-2$であるような上三角行列に変形できる。したがって$\det C_n = (-2)^{n-2}$である。ゆえに
  \[
  \det A_n = (n-1) \det B_n = (n-1)(-1) \det C_n = - (n-1) (-2)^{n-2}
  \]
  である。$n \geq 5$という仮定は$C_n$があまり小さくならないようにするためだけの仮定であり、この式は一般に成り立つ。そのことの確認は読者に任せる。
\end{sol}
