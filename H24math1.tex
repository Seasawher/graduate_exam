\section{平成24年度 数学I}



\subsubsection{}%1
\barquo{
$A,B$は複素数係数の$n$行$m$列行列、$f(X)$は複素数係数の多項式とする。
\[
A f(B) = B
\]
が成り立っているとする。次を証明せよ。
\begin{description}
  \item[(1)] $f(B)$が正則ならば$A$と$B$は可換である。
  \item[(2)] $f(B)$が正則でなければ$f(0)=0$である。
\end{description}
}
\begin{sol} ${}$
\begin{description}
  \item[(1)] $f(B)$は$B$と可換なので
  \[
  (BA - AB)f(B) = B Af(B) - A f(B) B = B^2 - B^2 = 0
  \]
  である。$f(B)$は正則だったから$BA=AB$が従う。
  \item[(2)] $f(B)$が正則でないので、ある$v \in \C^n \sm \{ 0 \}$であって$f(B)v=0$なるものがある。よって$Bv = Af(B) v = 0$だから$f(0)v= 0$である。$v \neq 0$より$f(0)=0$でなくてはならない。
\end{description}
\end{sol}



\newpage



\subsubsection{}%2
\barquo{
$p \geq 3$を奇素数、$n$を自然数とする。行列の乗法を演算とする群
\[
G = \setmid{ \pmat{a & b \\ 0 & d} }{ a,b,d \in \zyu{p^n}, ad=1 }
\]
には位数$p^{2n-1}$の部分群がただ一つ存在することを示せ。
}
\begin{sol}
まず$G$の位数を求めると、$\# G = p^n \tm \#(\zyu{p^n})^{\tm}  = p^{2n-1}(p-1)$である。したがって、$G$のSylow-$p$部分群が正規部分群であることを示せば十分。いま$H \subset G$を$G$のSylow-$p$部分群とする。このとき$H$の共役の数を$s$とすると、$s$は$\# (G/H) = p-1$の約数であってかつ$s \equiv 1 \mod p$を満たす。よって$s = 1$である。つまり$H \lhd G$なので、示すべきことがいえた。
\end{sol}


\newpage



\subsubsection{}%3
\barquo{
$f(x)$は$[0,\infty )$上の非負実数値連続関数で単調非増加であり、かつ$f(x) / \s{x} $は$[0, \infty )$上広義積分をもつと仮定する。このとき、以下の問に答えよ。
\begin{description}
\item[(1)] $\lim_{x \to \infty} \s{x} f(x) = 0$を示せ。
\item[(2)] 任意の$0 < \ve < 1$に対し
\[
\lim_{x \to \infty} \int_{\ve x}^{x} \f{ f(y)  }{ \s{x-y}  } \ dy = 0
\]
を示せ。
  \end{description}
}
\begin{sol} ${}$
\begin{description}
\item[(1)] 計算すると
\begin{align*}
  \int_x^{2x} \f{ f(t) }{ \s{t} } \ dt &\geq f(2x) \int_x^{2x} \f{ dt }{ \s{t} } \\
  &\geq 2 f(2x) ( \s{2x} - \s{x}) \\
  &\geq (2 - \s{2}) \s{2x} f(2x)
\end{align*}
であることより、あきらか。
\item[(2)] 計算すると
\begin{align*}
  \int_{\ve x}^{x} \f{ f(y)  }{ \s{x-y}  } \ dy &\leq f(\ve x) \int_{\ve x}^{x} \f{ dy }{ \s{x-y}  } \\
  &\leq 2 \s{1-\ve} \s{x} f(\ve x) \\
  &\leq 2 \s{(1-\ve)/\ve} \s{\ve x} f(\ve x)
\end{align*}
である。よって(1)よりあきらか。
\end{description}
\end{sol}


\newpage



\subsubsection{}%4
\barquo{
$n$を正の整数とし、$\T^n$を$\C^n$に標準的に埋め込まれた$n$次元トーラス、すなわち
\[
\T^n = \setmid{ (z_1, \cdots , z_n) \in \C^n  }{  \abs{z_1} = \cdots =\abs{z_n} = 1  }
\]
とする。$f \colon \T^n \to \T^n$を、連続写像ですべての$(z_1, \cdots , z_n) \in \T^n$について
\[
f(z_1, \cdots , z_n ) = f(\ol{ z_1}, \cdots , \ol{z_n} )
\]
をみたすものとする。($\ol{z}$は$z \in \C$の複素共役を表す)
\begin{description}
\item[(1)] $S^1$を単位円$\setmid{z \in \C}{ \abs{z} =1 }$とし、写像$\grg \colon S^1 \to \T^n$を$\grg(z) = (z,1 \cdots , 1)$で定める、このとき$f \circ \grg $は定置写像とホモトピックであることを示せ。
\item[(2)] $f$が誘導する基本群の間の写像$f_* \colon \pi_1(\T^n) \to \pi_1(\T^n)$は零写像であることを示せ。
\item[(3)] $f$は定置写像とホモトピックであることを示せ。
\end{description}

(注) 位相空間$X,Y$とその間の連続写像$F \colon X \to Y$について、$F$が定置写像とホモトピックであるとは、連続写像$H \colon X \tm [0,1] \to Y$と$q_* \in Y$で、すべての$p \in X$について$H(p,0) = q_*$と$H(p,1) = F(p)$が成り立つものが存在することをいう。
}
\begin{rem}
  (1)においてホモトピー$H \colon S^1 \tm [0,1] \to \T^n$を
  \[
  H(e^{i \grt},t) = f(e^{i \grt t}, 1 , \cdots , 1)
  \]
  とすればいいというのは誤りである。$\grt$には$2 \pi$の整数倍分のあいまいさがあるので、この$H$はwell-definedにならない。
\end{rem}
\begin{sol} ${}$
\begin{description}
  \item[(1)] $e^{i \grt} = z$という対応により$\grg \colon S^1 \to \T^n$を$\R / 2 \pi \Z$上の写像とみなす。$f$についての仮定により$f \circ \grg$は$P := f \circ \grg(0) = f(1, \cdots , 1)$を出発して$Q := f \circ \grg(\pi) = f(-1, 1, \cdots , 1)$まで行き、その後来た道を引き返して
  $P$に戻っていくようなパスである。つまり
  \[
  f \circ \grg (\grt) = f(e^{i(\pi - \abs{\pi - \grt})}, 1, \cdots , 1   )
  \]
  と表せる。したがって$H \colon \R / 2 \pi \Z \tm [0,1] \to \T^n$を
  \[
  H(\grt, t) = f(e^{it(\pi - \abs{\pi - \grt})}, 1, \cdots , 1 )
  \]
  とおけば、これは$f \circ \grg$と定置写像の間のホモトピーである。
  \item[(2)] $\pi_1(\T^n) = \pi_1(S_1 \tm \cdots \tm S^1) = \pi_1(S^1) \tm \cdots \tm \pi_1(S^1)$だから、$f_* \colon \pi_1(\T^n) \to \pi_1(\T^n)$が零写像であることを示すには、
  $\pi_1(S^1) \cong \Z$の生成元を$a$として$f_*(a,1, \cdots , 1) = 0$を示せばよい。ところが$f_*(a,1, \cdots , 1) =  [f \circ \grg]$なので(1)により、示すべきことがいえた。
  \item[(3)] (2)により$f_* (\pi_1(\T^n))=0$なので、次を可換にするリフト$g$が存在することがわかる。
  \[
  \xymatrix{
  {} & \R^n \ar[d]^p \\
  \T^n \ar[ru]^g \ar[r]^f & \T^n
  }
  \]
  ただし$p \colon \R^n \to \T^n$は普遍被覆である。そこで$H \colon \T^n \tm [0,1] \to \T^n$を
  \[
  H(w,t) =  p( t g(w))
  \]
  とすると、これは$f$と定置写像の間のホモトピーになっている。
\end{description}
\end{sol}


\newpage


\subsubsection{}%5
\barquo{
関数$f$を
\[
f(z) = \f{1}{2\pi i} \int_{- \infty}^{\infty} \f{ e^{- \abs{x} }  }{ x - z } \ dx
\]
と定める。このとき$f(z)$は$z \in \C \sm \R = \setmid{ z \in \C }{ z \not\in \R }$で正則であることを示せ。また、極限
\[
\lim_{\ve \to + 0} ( f(i \ve) - f( - i \ve ))
\]
を求めよ。
}
\begin{sol}
微分と積分が交換するだろうという楽観的な予想が正しければ、
\[
f'(z) = \f{1}{2 \pi i} \int_{- \infty}^{\infty} \f{ e^{- \abs{x} }  }{ (x - z )^2} \ dx
\]
となっているはずである。これが正しいことを確かめればよい。つまり与えられた開集合$U := \C \sm \R$上で
\[
\lim_{w \to 0} \abs{ \f{ f(z + w)-f(z) }{w} - \f{1}{2 \pi i} \int_{- \infty}^{\infty} \f{ e^{- \abs{x} }  }{ (x - z )^2} \ dx } = 0
\]
となることを示そうというのである。そうすれば$f$が$U$上で正則であることがいえる。(ある点$z$での正則性とは、$z$のある開近傍上で微分可能であるということであって、$z$における微分可能性より強い性質だが、$U$が開集合なので問題ない)

$z \in U$とする。計算すると
\[
\abs{ \f{ f(z + w)-f(z) }{w} - \f{1}{2 \pi i} \int_{- \infty}^{\infty} \f{ e^{- \abs{x} }  }{ (x - z )^2} \ dx } \leq \f{\abs{w} }{2 \pi} \int_{- \infty}^{\infty} \f{ e^{- \abs{x} }  }{ \abs{(x - z )^2 (x-z-w) }  } \ dx
\]
であることが判る。$\abs{w}$が十分小さいとき、$\abs{ \Im(z+w)} \geq \abs{ \Im(z)}  /2$としてよい。よって$y := \abs{ \Im(z)}$とおけば$z \in U$より$y > 0$であって
\begin{align*}
  \f{\abs{w} }{2 \pi} \int_{- \infty}^{\infty} \f{ e^{- \abs{x} }  }{ \abs{(x - z )^2 (x-z-w) }  } \ dx &\leq    \f{\abs{w} }{2 \pi}  \left( \f{y}{2} \right)^{-3}  \int_{- \infty}^{\infty} e^{- \abs{x} }  \ dx \\
  &\leq \f{ 8 \abs{w} }{ y^3 \pi }
\end{align*}
である。よって$w \to 0$のときこの積分はゼロになる。これで$f$が$U$上で正則であることがいえた。

また、愚直に計算すると
\[
f(i \ve) - f(- i \ve) = \f{ 2\ve }{ \pi } \int_0^{\infty} \f{e^{-x}}{x^2 + \ve^2} \ dx
\]
である。$x = \ve \tan \grt$とおいて変数変換すると
\[
f(i \ve) - f(- i \ve) = \f{ 2 }{ \pi } \int_0^{\pi /2} e^{ - \ve \tan \grt} \ d\grt
\]
と表せる。ここで$\ve_n \to 0$なる点列$\ve_n > 0$が任意に与えられたとする。$n$によらず一様に
\[
\abs{ e^{ - \ve_n \tan \grt} } \leq 1
\]
であって$1$は区間$[0, \pi/2)$上可積分なので、Lebesgueの収束定理により
\[
\lim_{n \to \infty} (f(i \ve_n) - f(- i \ve_n)) = \f{ 2 }{ \pi } \int_0^{\pi /2}  \ d\grt = 1
\]
である。$\ve_n$は任意だったから、求める極限は$1$である。
\end{sol}
