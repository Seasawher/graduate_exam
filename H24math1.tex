\section{平成24年度 数学I}



\subsubsection{}%1
\barquo{
$A,B$は複素数係数の$n$行$m$列行列、$f(X)$は複素数係数の多項式とする。
\[
A f(B) = B
\]
が成り立っているとする。次を証明せよ。
\begin{description}
  \item[(1)] $f(B)$が正則ならば$A$と$B$は可換である。
  \item[(2)] $f(B)$が正則でなければ$f(0)=0$である。
\end{description}
}
\begin{sol} ${}$
\begin{description}
  \item[(1)] $f(B)$は$B$と可換なので
  \[
  (BA - AB)f(B) = B Af(B) - A f(B) B = B^2 - B^2 = 0
  \]
  である。$f(B)$は正則だったから$BA=AB$が従う。
  \item[(2)] $f(B)$が正則でないので、ある$v \in \C^n \sm \{ 0 \}$であって$f(B)v=0$なるものがある。よって$Bv = Af(B) v = 0$だから$f(0)v= 0$である。$v \neq 0$より$f(0)=0$でなくてはならない。
\end{description}
\end{sol}



\newpage



\subsubsection{}%2
\barquo{
$p \geq 3$を奇素数、$n$を自然数とする。行列の乗法を演算とする群
\[
G = \setmid{ \pmat{a & b \\ 0 & d} }{ a,b,d \in \zyu{p^n}, ad=1 }
\]
には位数$p^{2n-1}$の部分群がただ一つ存在することを示せ。
}
\begin{sol}
まず$G$の位数を求めると、$\# G = p^n \tm \#(\zyu{p^n})^{\tm}  = p^{2n-1}(p-1)$である。したがって、$G$のSylow-$p$部分群が正規部分群であることを示せば十分。いま$H \subset G$を$G$のSylow-$p$部分群とする。このとき$H$の共役の数を$s$とすると、$s$は$\# (G/H) = p-1$の約数であってかつ$p \equiv 1 \mod p$を満たす。よって$s = 1$である。つまり$H \lhd G$なので、示すべきことがいえた。
\end{sol}


\newpage



\subsubsection{}%3
\barquo{
$f(x)$は$[0,\infty )$上の非負実数値連続関数で単調非増加であり、かつ$f(x) / \s{x} $は$[0, \infty )$上広義積分をもつと仮定する。このとき、以下の問に答えよ。
\begin{description}
\item[(1)] $\lim_{x \to \infty} \s{x} f(x) = 0$を示せ。
\item[(2)] 任意の$0 < \ve < 1$に対し
\[
\lim_{x \to \infty} \int_{\ve x}^{x} \f{ f(y)  }{ \s{x-y}  } \ dy = 0
\]
を示せ。
  \end{description}
}
\begin{sol}

\end{sol}


\newpage



\subsubsection{}%4
\barquo{
$n$を正の整数とし、$\T^n$を$\C^n$に標準的に埋め込まれた$n$次元トーラス、すなわち
\[
\T^n = \setmid{ (z_1, \cdots , z_n) \in \C^n  }{  \abs{z_1} = \cdots =\abs{z_n} = 1  }
\]
とする。$f \colon \T^n \to \T^n$を、連続写像ですべての$(z_1, \cdots , z_n) \in \T^n$について
\[
f(z_1, \cdots , z_n ) = f(\ol{ z_1}, \cdots , \ol{z_n} )
\]
をみたすものとする。($\ol{z}$は$z \in \C$の複素共役を表す)
\begin{description}
\item[(1)] $S^1$を単位円$\setmid{z \in \C}{ \abs{z} =1 }$とし、写像$\grg \colon S^1 \to \T^n$を$\grg(z) = (z,1 \cdots , 1)$で定める、このとき$f \circ \grg $は定置写像とホモトピックであることを示せ。
\item[(2)] $f$が誘導する基本群の間の写像$f_* \colon \pi_1(\T^n) \to \pi_1(\T^n)$は零写像であることを示せ。
\item[(3)] $f$は定置写像とホモトピックであることを示せ。
\end{description}

(注) 位相空間$X,Y$とその間の連続写像$F \colon X \to Y$について、$F$が定置写像とホモトピックであるとは、連続写像$H \colon X \tm [0,1] \to Y$と$q_* \in Y$で、すべての$p \in X$について$H(p,0) = q_*$と$H(p,1) = F(p)$が成り立つものが存在することをいう。
}
\begin{sol}

\end{sol}


\newpage


\subsubsection{}%5
\barquo{
関数$f$を
\[
f(z) = \f{1}{2\pi i} \int_{- \infty}^{\infty} \f{ e^{- \abs{x} }  }{ x - z } \ dx
\]
と定める。このとき$f(z)$は$z \in \C \sm \R = \setmid{ z \in \C }{ z \not\in \R }$で正則であることを示せ。また、極限
\[
\lim_{\ve \to + 0} ( f(i \ve) - f( - i \ve ))
\]
を求めよ。
}
\begin{sol}

\end{sol}
