

\section{Appendix 2. DirichletのDiophantus近似定理}

平成27年度基礎科目IIの問7のための補足です。あの問題はDirichletの近似定理を認めてしまえばほぼ当たり前なのですが、しかしDirichletの近似定理ってあんまり本に載ってませんからね。見たことないひとも多いと思います。そこで、Dirichletの近似定理の内容とその証明をここで補うことにしました。鳩の巣論法を使った証明もあります。それについてはDirichletのDiophantus近似定理で調べてください。




\begin{definition}
$\R^n$の部分集合$S$があるとする。このとき$S$が点対称(centrally symmetric)であるとは、任意の$x \in S$に対して$-x \in S$であることをいう。また$S$が凸(convex)であるとは、任意の$x,y \in S$に対し、$x$と$y$を結ぶ線分が$S$に含まれるということである。つまり任意の$0 \leq t \leq 1$に対して$tx + (1-t)y \in S$であることを指す。
\end{definition}

\prop{
(Minkowskiの定理) \\
$\mu$はLebesgue測度とする。$S \subset \R^n$が点対称かつ凸な可測集合で、$\mu(S) > 2^n$ならば$S \cap \Z^n \sm \{0\} \neq \emptyset$である。
}
\begin{proof}
ハイリホーによる。$S \cap \Z^n = \{0\}$と仮定しよう。$I$を$\R^n$の標準基底が張る超立方体とする。このとき
\begin{align*}
\f{1}{2^n} \mu \left( S \right) &= \mu\left(\f{S}{2}\right) \\
&= \mu\left(\f{S}{2} \cap \coprod_{d \in \Z^n} \left(I + d\right) \right) \\
&= \mu\left( \coprod_{d \in \Z^n}  \f{S}{2} \cap \left(I + d\right) \right) \\
&= \sum_{d \in \Z^n} \mu\left(   \f{S}{2} \cap \left(I + d\right) \right) \\
&= \sum_{d \in \Z^n} \mu\left(   \left(\f{S}{2}- d\right) \cap I \right)
\end{align*}
である。ここで$S$が凸かつ点対称という仮定により$d \neq d'$のとき$(S/2 + d) \cap (S/2 + d') = \emptyset$である。したがって
\begin{align*}
\f{1}{2^n} \mu\left(S\right) &= \mu\left(   \coprod_{d \in \Z^n} \left(\f{S}{2}- d\right) \cap I \right) \\
&\leq \mu\left(I\right) \\
&= 1
\end{align*}
となって矛盾。

\end{proof}


\prop{
(Dirichletの近似定理) \\
$d \geq 1$とする。実数$\gra_1, \cdots , \gra_d$と$N \in \N$が与えられたとき次が成り立つ。
\[
\exists q , p_i \in \Z \st 1 \leq q \leq N   \quad \text{and} \quad \forall i \quad  \abs{q \gra_i - p_i} \leq \f{1}{N^{1/d}}
\]
}
\begin{proof}
次のような$\R^{1+d}$の部分集合$S$を考える。
\[
S = \setmid{(x,y_1, \cdots , y_d) \in \R^{1+d}}{   -N-1/2 \leq x \leq N + 1/2, \forall i \quad  \abs{x \gra_i - y_i} \leq \f{1}{N^{1/d}} }
\]
このとき$S$はあきらかに凸かつ点対称な可測集合なので、あとは$\mu(S) > 2^{d+1}$がいえればMinkowskiの定理から主張が従う。計算すると$M = \f{1}{N^{1/d}}$として
\begin{align*}
\mu(S) &= \int_{-N-1/2}^{N+1/2} \ dx \int_{\gra_d x - M}^{\gra_d x + M} \ dy_d \cdots \int_{\gra_1 x - M}^{\gra_1 x + M} \ dy_1 \\
&= (2M)^d (2N+1) \\
&= \f{2^d (2N+1)}{N} \\
&> 2^{d+1}
\end{align*}
である。よって示すべきことがいえた。
\end{proof}
