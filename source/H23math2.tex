\section{平成23年度 数学II}

\subsubsection{}%1
\barquo{
$p \geq 3$を奇素数、$\F_p$を$p$個の元からなる有限体とする。$\F_p$の元を成分とする正則な$2$次元正方行列全体のなす群を$GL_2(\F_p)$とおく。
\begin{description}
  \item[(1)] $GL_2(\F_p)$の元で固有値がすべて$1$となるものの個数を求めよ。
  \item[(2)] $GL_2(\F_p)$の半単純でない元の個数を求めよ。ただし、$A \in GL_2(\F_p)$が半単純とは、$\F_p$の代数的閉包$\ol{\F_p}$の元を成分とする正則な$2$次正方行列$P$が存在して$P^{-1}AP$が対角行列となることである。
\end{description}
}
\begin{sol} 以下$G := GL_2(\F_p)$とおく。単位行列は$E$で表す。
  \begin{description}
    \item[(1)] 平成28年度専門科目問3(ii)で「最小多項式が同じならば共役」ということを示した。その証明と同様にして、固有値がすべて$1$であるような$G$の元は$E$または$2$次のJordan細胞
    \[
    \grL = \pmat{1 &1 \\ 0& 1}
    \]
    と共役であることが言える。そこで、$E$と$\grL$の共役類の位数を求めればよい。$E$の共役類はあきらかに$\{ E \}$だから気にしなくてよい。問題は$\grL$である。

$G$の$G$自身への共役による作用を考える。すると求めるべき値は$1 + \# \Orbit( \grL)$と表される。
\[
\# \Orbit( \grL) = \# G / \# \Stab(\grL)
\]
であることと$\# G = p(p-1)^2 (p+1)$であることより、$\# \Stab(\grL)$が判ればよい。

計算すると
\[
A = \pmat{a &b  \\ c &d }
\]
と表されている元$A \in G$に対して
\[
A \grL - \grL A = \pmat{ -c & a-d \\ 0 &c}
\]
だから
\[
\Stab (\grL) = \setmid{ \pmat{a& b \\ 0& a} }{a,b \in \F_p , a \neq 0 }
\]
と求まる。よって$\# \Stab(\grL) = p(p-1)$なので$\# \Orbit(\grL)= p^2 -1$であって、求める「固有値がすべて$1$の$G$の元の個数」は$1 + (p^2 -1) = p^2$である。
    \item[(2)] $L := GL_2(\ol{\F_p})$とおく。代数閉体上であればJordan標準形が存在することを利用すると、半単純でない$G$の元全体とは、
    \[
    X := \setmid{ A \in G}{ \text{ある$P \in L$が存在して$P^{-1}AP$が$2$次のJordan細胞に等しい} }
    \]
    であることが言える。このとき$A \in X$の固有値は$\F_p$の元である。なぜならば!$\beta \in \ol{\F_p}$に対して
    \[
    \grL_{\beta} = \pmat{\beta & 1 \\ 0 & \beta }
    \]
    とおこう。$A \in X$とし、$P^{-1}AP = \grL_{\beta}$なる$\beta \in \ol{\F_p}$と$P \in L$をとる。特性多項式を考えると、これは$L$上で共役をとっても変化しないので$\Phi_A(t) = (t - \beta)^2 \in \F_p[t]$である。よって$2\beta \in \F_p$ということになるが、$p$は奇素数と仮定していたのだった。ゆえにかくのごとし。

    ということは実は$\grL_{\beta} \in G$だったのである。すると次に気になるのは$2$つの$G$の元が$L$上で共役ならば$G$上でも共役だろうか?ということだが、これはすでに平成28年度専門科目問3(ii)の証明で肯定的に解決済みであった。すなわち$X$の元とは$G$において$2$次のJordan細胞と共役なもの全体である。つまり
    \[
    \# X = \sum_{\beta \in \F_p^{\tm} } \# \Orbit(\grL_{\beta})
    \]
    と計算できることになる。(1)と同様に$\# \Orbit(\grL_{\beta}) = p^2 -1$なので、
    \[
    \# X = (p-1)^2(p+1)
    \]
    である。
  \end{description}
\end{sol}

\newpage


\subsubsection{}%2
\barquo{
複素数体の部分体$K$を$K = \Q( i \s{17 + 4 \s{17}} )$によって定める。
\begin{description}
  \item[(1)] $K$は$\Q$の$4$次のGalois拡大であることを示せ。
  \item[(2)] $K$の$\Q$上のGalois群$\Gal(K/\Q)$を求めよ。
\end{description}
}
\begin{sol} ${}$
  \begin{description}
    \item[(1)] $\gra = i \s{17 + 4 \s{17}} $とおく。計算すると$\gra$は$f(X) = X^4 + 34X^2 + 17 \in \Z[X]$の根である。$f$は$p=17$に関するEisenstein多項式だから$\Q[X]$の元として既約で、よって$[K:\Q] = 4$である。
    \[
    f(X) = (X^2 + 17 - 4\s{17})(X^2 + 17 + 4\s{17})
    \]
    より$\gra$の$\Q$上の共役は$\beta := i \s{ 17 - 4\s{17} }$とおいたとき$\{ \pm \gra, \pm \beta \}$である。ゆえに$K$の$\Q$上のGalois閉包を$\wt{K}$とすると$\wt{K}= \Q(\gra, \beta) = K(\s{17})$である。ここで
    \[
    \s{17} = - \f{ \gra^2 + 17 }{ 4 }
    \]
    だから$\s{17} \in K$であって、ゆえに$K = \wt{K}$と結論できる。つまり$K/\Q$はGalois拡大である。
    \item[(2)] $G := \Gal(K/\Q)$とおく。$K$は既約多項式$f$の最小分解体なので$G$は根の集合$\{ \pm \gra, \pm \beta \}$に推移的に作用する。
    \[
    \grg_1 = \gra , \quad \grg_2 = - \gra, \quad \grg_3 = \beta , \quad \grg_4 = - \beta
    \]
    と添え字付けることにより$G \subset \frakS_4$とみなす。

    $K$は実数でない元を含むので、$G$の元の中には複素共役がある。$\tau \in G$を複素共役とすると、この同一視により$\tau = (12)(34)$である。また推移性により$\grs (\gra) = \beta$なる$\grs \in G$の存在がわかる。このとき
    \[
    \grs(\s{17}) = \grs \left( - \f{ \gra^2 + 17 }{ 4 }  \right) = - \f{ \beta^2 + 17 }{ 4 } = - \s{17}
    \]
    であることから
    \[
    - \gra \beta = \s{17} = - \grs(\s{17}) = \grs( \gra \beta) = \beta \grs( \beta )
    \]
    より$\grs(\beta) = - \gra$であり、$\grs = (1324)$である。とくに$\grs \in G$は位数$4$の元なので$G$は巡回群であって$G \cong \zyu{4}$であることが結論される。
  \end{description}
\end{sol}
