\section{平成20年度 数学II}

\subsubsection{}%1
\barquo{
有理整数環$\Z$の素数$p$が定める$\Z$の素イデアル$(p)$による局所化
\[
\Z_{(p)} = \setmid{ \f{n}{m} }{ m,n \in \Z, m \not\in (p) }
\]
を考える。$\Z_{(p)}$上の一変数多項式環$\Z_{(p)}[x]$のイデアル$(ax+b) \; (a,b \in \Z_{(p)}, a \neq 0)$が極大イデアルであるための$a,b$に関する必要十分条件を求めよ。
}
\begin{rem}
  $R = \Z_{(p)}$とおき、$\frakp = (ax+b) \subset R[x]$とする。$\dim R[x] = 2$であり、$R[x]$がNoether環であることからKrullの標高定理により$\height \frakp = 1$である。だから$\frakp $が極大イデアルになることなどありそうにないと思うかもしれないが、それは違う。一般の環$A$とその素イデアル$\frakq \subset A$について、必ずしも
  \[
  \height \frakq + \coht \frakq = \dim A
  \]
  は成立しないのである。たとえば$A$が体上の有限生成整域なら成立するのだが、いまはそういう状況ではない。
\end{rem}
\begin{proof}
$\frakp \subset R[x]$が極大イデアルであると仮定する。$a \neq 0$という仮定より$p \not\in \frakp$なので、極大性により$(p, ax+b) = R[x]$である。したがって
\[
pf(x) + (ax+b)g(x) = 1
\]
なる$f,g \in R[x]$が存在する。$x=0$を代入して、とくに$bg(0) = 1 - pf(0)$なので、$b \in R^{\tm}$である。一方で$a \not\in R^{\tm}$であることをハイリホーで示す。仮に$a \in R^{\tm}$だとする。このとき$a \beta + b = 0$なる$\beta \in R$が存在する。このとき$p f(\beta) = 1$であるが、これは$p \in R^{\tm}$を意味しており、矛盾。
よって$\ord_p(a) \geq 1$である。

逆に$\ord_p(a) \geq 1$かつ$b \in R^{\tm}$だと仮定する。$a =  p^n c$なる$c \in R^{\tm}$と$n \geq 1$があることになる。このとき
\[
p(p^{n-1}c x) + (ax + b)(-1) = -b
\]
であって$b \in R^{\tm}$だから、$p \in R[x]/\frakp$は可逆元である。

これを使って$R[x]/\frakp$が体であることを示そう。$\pi \colon R[x] \to R[x]/\frakp$を自然な写像とする。ゼロでない元$\pi(f) \in R[x]/\frakp$が与えられたとする。$f$を$ax +b$で割ることにより$f(x) = g(x)(ax+b) + r$なる$r \in \Q$と$g \in \Q[x]$の存在がいえる。$mr \in R$かつ$mg \in R[x]$となるような
$m \in \Z$をとる。$m$は$p$のベキであるとしてよい。$mf(x) = mg(x)(ax+b) + mr$より$\pi(m)\pi(f) = \pi(m r)$である。一方、$p \in R[x]/\frakp$が可逆なことから、$R[x]/\frakp$において$\pi(m)$は可逆で、$\pi(mr)$も(ゼロでないので)可逆である。よって$\pi(f)$も可逆。$\pi(f)$は任意だったので、$R[x]/\frakp$は体であり、
とくに$\frakp \subset R[x]$は極大イデアルである。

したがって求める条件は、$\ord_p(a) \geq 1$かつ$b \in R^{\tm}$である。
\end{proof}

\newpage

\subsubsection{}%2
\barquo{
$t$を不定元とし複素数体上の有利関数体$K = \C(t)$を考える。複素数$p,q \in \C$をとり$L = \C(t^3 + pt + q)$とおくとき、$K/L$は代数拡大であることを示し、さらに$K/L$上のGalois閉包($K$を含む最小の$L$上のGalois拡大体)とその$L$上の拡大次数を求めよ。
}
\begin{proof}
  $s = t^3 + pt + q$とおく。$t$は$X^3 + pX + q -s \in L[X]$の根なので$K/L$は代数拡大である。$f(X) = X^3 + pX + q - s \in L[X]$とおく。$f$の既約性をハイリホーで示そう。$f$がもし可約なら、ある$\gra \in L$が存在して$f(\gra)=0$を満たす。一方で$s $は$\C$上超越的なので$B := \C[s]$は整閉整域であり、したがって$\gra \in B$である。
  ここで$s = \gra^3 + p \gra + q$が得られているわけだが、$s$についての$\gra$の次数$\deg \gra$で場合分けを行う。いま$\deg \gra >1$だとすると
  \[
  3 \deg(\gra) = \deg( s - p \gra - q ) = \deg(\gra)
  \]
  となり矛盾。もしも$\deg \gra = 1$ならば
  \[
  3 = \deg(\gra^3) = \deg(s - p \gra - q) \leq 1
  \]
  となり矛盾。最後に$\deg \gra = 0$ならば、$\gra \in \C$より$s \in \C$となって矛盾である。以上により$f \in L[X]$は既約である。とくに、$[K:L]=3$であることがわかる。

  $M$を$K$の$L$上のGalois閉包とする。$M$は、$t \in K$の$L$上の共役をすべて$K$に添加して得られる体である。$f(X) = (X - t)(X^2 + t X + t^2 + p)$なので、$g(X):= X^2 +tX + t^2 + p \in K[X]$の根を$\gra, \beta$としたとき$t \in K$の$L$上の共役は$\{t, \gra ,\beta \}$である。よって$M = K(\gra)$となる。

  $g \in K[X]$の判別式を$D$とおく。$D = - 3t^2 - 4p$は$p \neq 0$である限り$\C[t]$の平方元ではないので、$p \neq 0$なら$g$は既約であり$[M: K]=2$である。一方で$p=0$ならば
  \[
  g(X) = \left( X - \f{-1 + \s{-3}}{2} t \right) \left( X - \f{-1 - \s{-3}}{2} t \right)
  \]
  だから$M = K$となる。

  以上をまとめると
  \[
  M = \begin{cases}
  K &(p=0) \\
  K[X]/(X^2 + tX + t^2 + p) &(p \neq 0)
\end{cases}
  \]
  \[
  [M:L] = \begin{cases}
  3 &(p=0) \\
  6 &(p \neq 0)
\end{cases}
  \]
  である。
\end{proof}
