
\section{Appendix 1. 半直積とGalois群}


平成30年度の院試専門科目の問3を解いているときに補題として準備したものです。

有限次Galois拡大の合成であるような体拡大があるとき、そのGalois群が直積で表されるという定理がありました。この仮定を少し緩めても、直積を半直積に代えて同様のことが成り立つというのがここでの主張です。半直積の形のままだと積がどうなっているかがやや判りにくく、元の位数や共役類を調べるのに不都合ですが、幸いにして有限巡回拡大の合成であるような場合ならGalois群の表示が容易にわかります。ついでに、半直積についてまとまった事実を紹介している文献がなかなか見つからなかったので主要と思われるものを簡単にまとめておきました。

結果的にこの補題を使っても件の院試の問題はそんなに楽にはならなくてガッカリしましたが、半直積が自然に出てくる状況をひとつ見つけられたことは嬉しいです。




\prop{
(直積の特徴づけ)\\
群$G$とその部分群$N$, $H$があるとする。このとき次は同値。
\begin{description}
  \item[(1)] $G$と直積$N \tm H$は自然に同型である。つまり積をとる写像$\vp \colon N \tm H \to G$は群の準同型であって、かつ同型になる。次の図式
\[
\xymatrix{
{} & N \tm H \ar[d]^-{\vp} & {} \\
N \ar[ru] \ar[r] & G &  H \ar[l] \ar[lu]
}
\]
を可換にするような同型$\vp$があるといってもよい。
\item[(2)] $N \lhd G$かつ$H \lhd G$であり、かつ$N \cap H = 1$で$NH = G$である。
\end{description}
}
\begin{proof} ${}$
  \begin{description}
    \item[(1)$\To$(2)] $q \in N$, $g \in G$が与えられたとする。($n \in N$としないのは、$n$と$h$の形が似ていて間違えやすいため) $\vp$の逆写像$\psi$をとっておく。すると、$\psi(g) = (g_N, g_H)$と表せる。ゆえに
    \begin{align*}
      \psi(gqg^{-1}) &= \psi(g) (q,1) \psi(g)^{-1} \\
      &= (g_N,g_H) (q,1) (g_N^{-1}, g_H^{-1}) \\
      &= ( g_N q g_N^{-1} , 1)
    \end{align*}
    である。したがって$gqg^{-1} = \vp(g_N q g_N^{-1} , 1) \in N$であるから、$N \lhd G$である。同様にして$H \lhd G$もいえる。また、$x \in N \cap H$とすると、$\psi(x) \in N \tm H$は$(1,1)$でなくてはならない。したがって、$x \in \Ker \psi$である。$\psi$は同型だから$x=1$であって、$N \cap H =1$がいえた。
    さらに、$G = NH$であることはあきらかであろう。
    \item[(2)$\To$(1)] $N$, $H$は$G$の部分群なので、積をとる写像$\vp \colon N \tm H \to G$が定義できる。$N \lhd G$, $H \lhd G$なので交換子$[N,H]$は$N \cap H$の部分群であるが、$N \cap H = 1$なので$[N,H] = 1$である。よって$N$の元と$H$の元は可換であり、$\vp$は群準同型になる。
    $N \cap H = 1$より$\vp$は単射であり、$NH=G$より$\vp$は全射である。
  \end{description}
\end{proof}


\lem{
(半直積の基本的な性質) \\
群$N$, $H$と群作用$\Phi \colon  H \to \Aut N$があって、半直積$N \rtimes_{\Phi} H$を考えているとする。$q \in N$, $h \in H$とする。このとき次が成り立つ。
\begin{description}
  \item[(1)] 作用成分への射影$N \rtimes_{\Phi} H \to H \st (q,h) \mapsto h$は準同型である。
  \item[(2)] 正規成分への入射$N \to N \rtimes_{\Phi} H \st q \mapsto (q,1)$は準同型である。
  \item[(3)] 作用成分への入射$H \to N \rtimes_{\Phi} H \st h \mapsto (1,h)$は準同型である。
  \item[(4)] $h \in \Ker \Phi$ならば$(q,h) = (1,h)(q,1)$である。
  \item[(5)] 常に$(q,h) = (q,1)(1,h)$が成り立つ。
  \item[(6)] 自然な入射と射影は、分裂する短完全列
  \[
  \xymatrix{
  1 \ar[r] & N \ar[r] & N \rtimes_{\Phi} H \ar[r] & H \ar[r] & 1 \\
  {} & {} & H \ar[u] \ar[ru]_-{1} & {} & {}
  }
  \]
  をなす。
\end{description}
}
\begin{proof}
  あきらか。
\end{proof}




\prop{
(半直積の特徴づけ)\\
群$G$の部分群$N$, $H$が与えられているとする。このとき次は同値。
\begin{description}
  \item[(1)] ある群作用$\Phi \colon H \to \Aut N$が存在して、$G$は半直積$N \rtimes_{\Phi} H$と自然に同型である。つまり積をとる写像$\vp \colon N \rtimes_{\Phi} H \to G \st (q,h) \mapsto qh$は群準同型で、かつ同型である。次の図式
\[
\xymatrix{
{} & N \rtimes_{\Phi} H \ar[d]^-{\vp} & {} \\
N \ar[ru] \ar[r] & G &  H \ar[l] \ar[lu]
}
\]
を可換にするような同型$\vp$があるといってもよい。
  \item[(2)] $N \lhd G$かつ$NH=G$かつ$N \cap H =1$が成り立つ。
\end{description}
}
\begin{proof} ${}$
  \begin{description}
    \item[(1)$\To$(2)] $NH=G$はあきらか。$x \in N \cap H$とすると$(x,x^{-1}) \in \Ker \vp$だから$x =1$でなくてはならない。よって$N \cap H = 1$である。
    $N \lhd G$を示そう。$g \in G$と$q \in N$が与えられたとする。$p \colon N \rtimes_{\Phi} H \to H$を射影とし、$\psi$を$\vp$の逆写像とする。
    このとき$\psi(g)= (g_N, g_H)$と表せる。ゆえに
    \begin{align*}
      p \circ \psi (gqg^{-1}) &= p( (g_N,g_H) (q,1) (g_N^{-1}, g_H^{-1}) ) \\
      &= 1
    \end{align*}
    である。したがって$gqg^{-1} \in \vp(\Ker p)=N$である。よって$N \lhd G$がわかった。
    \item[(2)$\To$(1)] $N \lhd G$より、群作用$\Phi \colon H \to \Aut N$を$\Phi_h(q) = hqh^{-1}$により定めることができる。(順序を逆にして$\Phi_h(q) = h^{-1}qh$とするとうまくいかないことに注意) このとき$q_1,q_2 \in N$と$h_1, h_2 \in H$が与えられたとすれば
    \begin{align*}
      \vp((q_1,h_1) (q_2,h_2) ) &= \vp( q_1 \Phi_{h_1}(q_2), h_1h_2 ) \\
      &= \vp(q_1 h_1 q_2 h_1^{-1} , h_1 h_2) \\
      &= q_1 h_1 q_2 h_2 \\
      &= \vp(q_1,h_1) \vp(q_2,h_2)
    \end{align*}
    だから$\vp$は群準同型になる。$\vp$が単射であることは$N \cap H =1$より従い、全射であることは$NH = G$より従う。
  \end{description}
\end{proof}



\prop{
(半直積の関手性 その1) \\
$N_1, N_2,H$が群で群作用$\Phi \colon H \to \Aut N_1$が与えられていたとする。このとき同型$g \colon N_1 \to N_2$に対して${}_g\Phi \colon H \to \Aut N_2$を${}_g\Phi(h) = g \circ \Phi_h \circ g^{-1}$で定めると、
写像$g_* \colon N_1 \rtimes_{\Phi} H \to N_2 \rtimes_{{}_g\Phi} H \st g_*(q, h ) = (g(q), h)$は群の準同型である。
}
\begin{proof}
  計算すればわかる。実際に行ってみると
  \begin{align*}
    g_*((q, h_1)(q',h_2) ) &= g_*(q \Phi_{h_1}(q'), h_1h_2) \\
    &= (g(q) g(\Phi_{h_1}(q') ) , h_1h_2) \\
    (g(q), h_1) (g(q'), h_2) &= ( g(q) {}_g\Phi_{h_1} (g(q')) ,h_1h_2   ) \\
    &= (g(q) g(\Phi_{h_1}(q') ) , h_1h_2)
  \end{align*}
  であるから一致する。
\end{proof}



\prop{
(半直積の関手性 その2) \\
$N$, $H_1$, $H_2$が群で群作用$\Phi \colon H_2 \to \Aut N$が与えられていたとする。このとき群準同型$f \colon H_1 \to H_2$に対して$\Phi_f \colon H_1 \to \Aut N$を$(\Phi_f)_h = \Phi_{f(h)}$により定める。
そうすると写像$f_* \colon N \rtimes_{\Phi_f} H_1 \to  N \rtimes_{\Phi} H_2 \st f_*(q,h) = (q,f(h))$は群の準同型である。
}
\begin{proof}
  計算すればわかる。実際に行ってみると
  \begin{align*}
    f_* ( (q_1,h)(q_2,h') ) &= f_*( q_1 \Phi_{f(h)}  (q_2), h h') \\
    &=( q_1 \Phi_{f(h)}  (q_2), f(h) f(h') ) \\
    &= f_*(q_1, h) f_*(q_2, h')
  \end{align*}
  であるから一致。
\end{proof}



\prop{
(分裂する完全列からの半直積の構成) \\
群$G$, $H$, $N$と準同型$i$, $j$, $p$からなる分裂する短完全列
\[
\xymatrix{
1 \ar[r] & N \ar[r]^-j & G \ar[r]^-p & H \ar[r] & 1 \\
{} & {} & H \ar[u]^-i \ar[ru]_-{1} & {} & {}
}
\]
が与えられたとする。このとき、ある群作用$\Psi \colon H \to \Aut N$が存在して、自然な同型$G \cong N \rtimes_{\Psi} H$がある。すなわち、ある同型$\psi$が存在して次の図式
\[
\xymatrix{
{} & N \rtimes_{\Psi} H \ar[d]^-{\psi} & {} \\
N \ar[r]^-j \ar[ru] & G & H \ar[l]_-i \ar[lu]
}
\]
が可換になる。
}
\begin{proof}
  $N' = j(N)$, $H'=i(H)$とおく。このとき$N' = \Ker p$より$N' \lhd G$である。$x \in N' \cap H'$とすると$x = j(q) = i(h)$なる$q \in N, h \in H$があるが、$p(x) = 1 = h$より$x = 1$でなくてはならない。よって$N' \cap H' = 1$である。また$g \in G$とすると$g (i \circ p)(g^{-1}) \in \Ker p$なので$g (i \circ p)(g^{-1}) = j(q)$なる
  $q \in N$がある。したがって$g = j(q) (i \circ p)(g) \in N'H'$だから$G = N'H'$が成り立つ。よって、ある同型$\vp$と群作用$\Phi \colon H' \to \Aut N'$であって、次の図式
  \[
  \xymatrix{
  {} & N' \rtimes_{\Phi} H' \ar[d]^-{\vp} & {} \\
  N' \ar[ru] \ar[r] & G &  H' \ar[l] \ar[lu]
  }
  \]
  を可換にするようなものがある。ここで$i,j$は単射であるので、同型$I \colon H \to H'$と$K \colon N' \to N$が存在して、次の図式
  \[
  \xymatrix{
  N \ar[r] \ar[d]^-{K^{-1}} & N \rtimes_{ {}_K \Phi_I } H  \ar[d]^-{K_*^{-1}} & H \ar[l] \ar@{=}[d] \\
  N' \ar[r] \ar@{=}[d] & N' \rtimes_{\Phi_I } H  \ar[d]^-{I_*} & H \ar[l] \ar[d]^-{I} \\
  N' \ar[r] \ar@{=}[d] & N' \rtimes_{\Phi } H'  \ar[d]^-{\vp} & H' \ar[l] \ar@{=}[d] \\
  N' \ar[r] \ar[d]^-K & G  \ar@{=}[d] & H' \ar[l] \ar[d]^{I^{-1}} \\
  N \ar[r]^-j  & G   & H \ar[l]_-i
   }
  \]
  は可換になる。これで示すべきことがいえた。
\end{proof}



\prop{
(有限巡回群の半直積の表示) \\
群$N$, $H$は有限巡回群であり群作用$\Phi \colon H \to \Aut N$が存在して半直積$N \rtimes_{\Phi} H$を考えているとする。$N$, $H$の生成元$q,h$をそれぞれとって固定し$\Phi_h(q) = q^t$となる$t \in \Z$をとることができる。このとき
\[
N \rtimes_{\Phi} H \cong \setmid{q,h}{q^{\# N} = h^{\# H} = 1, hqh^{-1} = q^t}
\]
が成り立つ。
}
\begin{proof}
  右辺の群を$G$とおく。自由群の普遍性により、自由群$F_2$から$N \rtimes_{\Phi} H$への準同型$\vp$であって$\vp(q) = (q,1)$かつ$\vp(h)=(1,h)$なるものがある。なお、ここで$q \in F_2$と$q \in N$は本来別の記号で書くべきだが、かえって煩雑になるので同じ記号とした。$\vp$は全射である。このとき$q^{\# N}, h^{\# H} \in \Ker \vp$はあきらか。また
  \begin{align*}
    \vp(hqh^{-1}) &= (1,h)(q,1)(1,h^{-1}) \\
    &= (\Phi_h(q),1) \\
    &= (q^t,1) \\
    &= \vp(q)^t
  \end{align*}
  だから$hqh^{-1}q^{-t} \in \Ker \vp$である。したがって、全射$\psi \colon G \to N \rtimes_{\Phi} H$が誘導される。ここで$N \rtimes_{\Phi} H$の位数は$\# (N \tm H)$であるので$\# G \geq \# (N \tm H)$である。
  一方で$\# G \leq \# (N \tm H)$はあきらかなので結局$\# G = \# (N \tm H)$であり、$\psi$は同型でなくてはならない。
\end{proof}

\newpage


\prop{
(半直積とGalois群) \\
有限次Galois拡大$L/K$があり、その中間体$M,N$があって$L = M \cdot N$かつ$K = M \cap N$を満たすとする。
\[
\xymatrix{
{} & L & {} \\
M \ar[ru]^{Gal} & { } & N \ar[lu]_{Gal} \\
{} & K \ar[lu]_{Gal} \ar[ru] & {}
}
\]
さらに$M/K$はGalois拡大であるとする。このとき
\[
\Gal(L/K) \cong \Gal(L/M) \rtimes \Gal(L/N)
\]
が成り立つ。
}
\begin{proof}
  $M/K$はGalois拡大なので$\Gal(L/M) \lhd \Gal(L/K)$である。$L$は$M$と$N$の合成なので$\Gal(L/M) \cap \Gal(L/N) = 1$である。またGalois拡大の推進定理(雪江\cite{雪江2} 定理4.6.1)により$\Gal(L/N) \cong \Gal(M/K)$なのでとくに$[L:N] = [M:K]$であり、したがって$[L:N][L:M] = [L:K]$である。
  ゆえに、$\Gal(L/K) \cong \Gal(L/M) \rtimes \Gal(L/N)$がわかる。
\end{proof}
