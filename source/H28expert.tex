\section{平成28年度 専門科目}

\subsubsection{} %{問1}
\barquo{
有理数係数の既約多項式$f(x) = x^3 + ax + b$を考え、$K \subset \C$を$f(x)$の最小分解体とする。$a > 0$のとき、$K/\Q$のGalois群が3次対称群と同型であることを示せ。
}
\begin{sol}
  $\lim_{x \to + \infty} f(x) = + \infty$, $\lim_{x \to - \infty} f(x) = - \infty$より、$f$は連続なので$f(\beta_1) = 0$なる$\beta_1 \in \R$がある。$f'(x) = 3x^2 + a > 0$より$f$は単調増加なので$f$の実根は$\beta_1$のみである。そこで$f$の残りの根を
  $\beta_2, \beta_3$とすると$\beta_2 = \ol{\beta_3}$である。いま$G = \Gal(K/\Q)$を根への作用により3次対称群$S_3$の部分群とみなす。$G$は複素共役をとる写像を含むので、$\# G$は偶数でなくてはならない。また$f$は既約と仮定したので、$G$は集合$\{ 1,2,3\}$に推移的に作用するはずであり、とくに$\# G$は$3$の倍数である。
  よって$\# G$は$6$の倍数となるが、$\# S_3 = 6$なので$G \cong S_3$となるしかない。
\end{sol}

\newpage

\subsubsection{} %{問2}
\barquo{
$K$を標数が2でない代数的閉体とし、$K$の元$a$に対して、2変数多項式環$K[X,Y]$の剰余環
\[
R_a = K[X,Y] / (X^2 - Y^2 - X - Y -a)
\]
を考える。以下の問に答えよ。
\begin{description}
  \item[(i)] $a=0$のとき$R_a$の各極大イデアル$\frakm$に対して$\dim_K(\frakm / \frakm^2)$を求めよ。また$\frakm$はいつ$R_a$の単項イデアルとなるか?理由をつけて答えよ。
  \item[(ii)] $a \neq 0$のとき$R_a$の各極大イデアル$\frakm$に対して$\dim_K(\frakm / \frakm^2)$を求めよ。また$\frakm$はいつ$R_a$の単項イデアルとなるか?理由をつけて答えよ。
\end{description}
}
\begin{sol}
  式を変形すると$X^2 - Y^2 - X - Y -a = (X+Y)(X-Y-1)-a$である。いま$K$の標数は$2$ではないと仮定したので
  \begin{align*}
    K[S,T] &\to K[X,Y] \\
    S &\mapsto X+Y \\
    T &\mapsto X- Y-1
  \end{align*}
  という$K$準同型を考えると、これは
  \begin{align*}
    K[X,Y] &\to K[S,T] \\
    X &\mapsto (S+T+1)/2 \\
    Y &\mapsto (S-T-1)/2
  \end{align*}
  を逆写像とする同型である。これにより$R_a$は
  \[
U_a := K[S,T]/(ST-a)
  \]
  に対応する。$U_a$の極大イデアルを表すのに、記号の濫用だが$R_a$の極大イデアルと同じ記号$\frakm$を使うことにする。いま$U_a$の極大イデアル$\frakm$は、$E:=K[S,T]$の極大イデアル$\frakn$であって$(ST-a)$を含むものに対応している。Hilbertの零点定理により、$E/\frakn$は$K$の有限次拡大である。$K$は代数閉という仮定があったので、$E/\frakn \cong K$である。この同型による$S,T$の像をそれぞれ$\beta, \grg \in K$とする。すると$(S-\beta,T-\grg) \subset \frakn$であるが
  $(S-\beta,T-\grg)$は極大イデアルなので
  $(S-\beta,T-\grg)=\frakn$である。まとめると、$\frakn$は$\beta \grg = a$なる$\beta,\grg$によって$\frakn = (S-\beta,T-\grg)$と表されることがわかった。このことを踏まえて考察をしていく。

  \begin{description}
    \item[(i)] $a=0$のとき$\beta \grg=0$なので$\beta=0$または$\grg=0$である。どちらでも同じことなので$\beta=0$とする。$\grg =0$である場合には
    \begin{align*}
      \frakm/\frakm^2 &= ((S,T)/ST ) / ( (S^2,ST,T^2)/ST) \\
      &\cong (S,T)/ (S^2,ST,T^2) \\
      &\cong K^2
    \end{align*}
    だから$\dim_K(\frakm /\frakm^2) = 2$である。もし$\frakm$が$U_0$の単項イデアルなら$\frakm/\frakm^2 = \frakm \ts_{U_0} K \cong K$となるはずだから、$\frakm$は単項イデアルではない。

    $\grg \neq 0$のとき。このときには
    \begin{align*}
      (T-\grg)U_0 &= (T-\grg,ST)/ST \\
      &= (S(T-\grg),T-\grg,ST)/ST \\
      &= (S,T-\grg)/ST \\
      &= \frakm
    \end{align*}
    だから$\frakm$は単項イデアルである。とくに$\dim_K(\frakm/\frakm^2)=1$となる。

    以上の結果をまとめると
    \[
    \dim_K(\frakm / \frakm^2) = \begin{cases}
    1 & (\beta,\grg) \neq (0,0) \text{のとき。このとき$\frakm$は単項イデアル} \\
    2 & (\beta,\grg) = (0,0) \text{のとき}
  \end{cases}
    \]
    である。$U_0$から$R_0$に話を戻すと
    \[
    \dim_K(\frakm / \frakm^2) = \begin{cases}
    1 & (b,c) \neq (1/2,1/2) \text{のとき。このとき$\frakm$は単項イデアル} \\
    2 & (b,c) = (1/2,1/2) \text{のとき}
  \end{cases}
    \]
    がわかったことになる。ただし、$b,c$は$\frakm = (X-b,Y-c)/ (X^2 - Y^2 - X - Y)$となるような$b,c \in K$である。
    \item[(ii)] 再び$U_a$に話を持っていく。$a \neq 0$のとき$\beta \neq 0$かつ$\grg \neq 0$である。このとき
    \begin{align*}
      (S-\beta)U_0 &= (S-\beta, ST-a)/ (ST-a) \\
      &= (T(S-\beta), S-\beta, ST-a)/ (ST-a ) \\
      &=  (\beta(T-\grg), S-\beta, ST-a)/ (ST-a ) \\
      &=  (T-\grg, S-\beta)/ (ST-a ) \\
      &= \frakm
    \end{align*}
    だから$\frakm$は単項イデアルであり、とくに$\dim_K(\frakm / \frakm^2) = 1$がわかる。
  \end{description}
\end{sol}

\newpage

\subsubsection{}%問3
\barquo{
位数が奇素数$p$である有限体$\F_p = \Z / p \Z$を考え、
\[
G = GL_2 (\F_p) = \setmid{X \in M_2(\F_p)}{\det X \neq 0}
\]
とおく。ただし、$M_2(\F_p)$は$\F_p$の元を成分とする$2$次正方行列全体の集合を表す。以下の問に答えよ。
\begin{description}
  \item[(i)] $G$の元で対角行列と共役でないものの個数を求めよ。
  \item[(ii)] $G$の$2$つの元$X,Y$の最小多項式$\phi_X(t), \phi_Y(t) \in \F_p[t]$が一致すれば、$X$と$Y$は互いに共役であることを示せ。
  \item[(iii)] 対角行列を含まない$G$の共役類の個数を求めよ。
\end{description}
}
\begin{sol} 以下単位行列を$E$で表すことにする。
  \begin{description}
    \item[(i)] まず、行列の正則性と列ベクトルの一次独立性は同値なので$\# G= (p^2-1)(p^2 - p) = p(p-1)^2(p+1)$である。$G$の対角行列の全体を$\grL$とする。$T \in \grL$は対角成分を入れ替えたものと共役であるので、$\grL$の共役類は定数行列と、対角成分の入れ替えを無視した定数でない対角行列で代表される。$G$の元のうち定数行列全体を$\grL_0$で、定数でない対角行列の全体を$\grL_1$であらわすことにする。$G$の$G$自身への共役による作用を考えると、$G$の元のうち対角行列と共役なものの個数は
    \[
    \# \grL_0 + \f{1}{2} \sum_{T \in \grL_1} \# \Orbit (T)
    \]
    で求まる。$\# \Orbit (T) = \# G / \# \Stab (T)$なので、群$\Stab (T)$を決定すればよい。いま$T \in \grL_1$と$A \in G$をとり
    \[
    T = \pmat{ \beta & 0 \\ 0 & \grg}, \quad A = \pmat{a & b\\ c & d}
    \]
    とおく。このとき計算すると
    \[
 AT - TA = (\beta - \grg) \pmat{0 & -b \\ c & 0}
    \]
    なので$A \in \Stab (T)$であるということは$b=c=0$を意味する。よって$T$に依存せずに$\# \Stab(T) =\# \grL= (p-1)^2$であり$G$の元のうち対角行列と共役なものの個数は
    \[
    (p-1) + \f{1}{2} (p-1)(p-2) p(p+1) = \f{(p-1)^2 (p^2 -2)}{2}
    \]
    である。ゆえに求めるべき、対角行列と共役でない元の個数は
    \[
    \# G - \f{(p-1)^2 (p^2 -2)}{2} = \f{1}{2} (p-1)^2 (p^2 + 2p +2)
    \]
    である。
    \item[(ii)] $X$と$Y$の最小多項式が一致していると仮定し、$\phi = \phi_X = \phi_Y$とおく。$\phi$がどういう式であるかにより場合分けをする。$V = \F_p^2$とおく。

    まず$\phi(t) = t - \grl$と表されるとき。このとき$X$と$Y$は定数行列$\grl E$に一致するので、とくに共役となる。

    次に$\phi(t) = (t- \grl_1)(t-\grl_2)$ $(\grl_1 \neq \grl_2)$と表されるとき。このとき$X$と$Y$は$\F_p$上対角化可能であり、同じ対角行列と共役である。よって互いに共役となる。

    次に$\phi(t) = (t - \grl)^2$という形のとき。このとき$\dim \Ker (X - \grl E) = 1$かつ$\dim \Ker (X - \grl E)^2 = 2$であるので、$v \in \Ker (X - \grl E) \sm \{0\}$と$w \in \Ker (X - \grl E)^2 \sm \Ker (X - \grl E)$をとると$\{v , w \}$は$V$の基底となる。このとき
    \begin{align*}
      Xv &= \grl v \\
      Xw &= cv + dw
    \end{align*}
    なる$c,d \in \F_p$がある。$c \neq 0$なので$w$を$c^{-1}w$で置き換えて$c=1$としてよい。このとき
    \begin{align*}
      (X-dE)w &=  v \\
      (X-dE)v &= (\grl -d )v
    \end{align*}
    より$\det (X-dE)=0$だから$d=\grl$である。したがって
    \begin{align*}
      Xv &= \grl v \\
      Xw &= v + \grl w
    \end{align*}
    だから$X$は$\F_p$係数の正則行列で共役をとることによりJordan標準形に変形できる。これは$Y$についても同様なので$X$と$Y$は共役。

    最後に$\phi$が$\F_p[t]$の$2$次既約多項式であった場合。$X$が定める$R := \F_p[t]$の$V$への作用$R \tm V \to V$は作用$\vp_A \colon R / (\phi) \tm V \to V$を誘導する。$\phi$は既約で$R$はPIDなので、$(\phi) \subset R$は極大イデアルである。したがって$R / (\phi)$は位数$p^2$の有限体
    $F := \F_{p^2}$と同型である。これにより$V$を$F$ベクトル空間と見なす。$\# V = \# F$なので次元は$1$である。よって$v \in V \sm \{0\}$を固定したとき、写像
    \begin{align*}
      F \tm \{ v \} &\xrightarrow{\vp_A} V \\
      (f,v) &\mapsto f \cdot v = f(A)v
      \end{align*}
      は$F$同型である。以上の議論は$Y$についても同様に適用できて、$\vp_B \colon F \tm \{v\} \to V$も$\vp_A$と同様に$F$同型となる。よって次の図式
      \[
      \xymatrix{
      F \tm \{v\} \ar[dr]_-{\vp_B} \ar[r]^-{\vp_A} & V \ar[d]^T \\
{} & V
      }
      \]
      を可換にするような$F$同型$T$が存在する。$F$同型は$\F_p$同型でもあるので、$T$を表現する正則行列$P \in G$がある。このとき
      \[
      \forall f \in F \quad Pf(A)v = f(B)v
      \]
      だから、とくに
      \begin{align*}
        PAv &= Bv \\
        PA^2v &= B^2v \\
        PA^3v &= B^3v \\
      \end{align*}
      である。よって
      \begin{align*}
        B^2 v &= PA^2 v \\
        &= PAP^{-1} PAv \\
        &= PAP^{-1} Bv \\
        B^3 v &= PA^3 v \\
        &= PAP^{-1} PA^2v \\
        &= PAP^{-1} B^2v
      \end{align*}
      だから
      \begin{align*}
        (PAP^{-1} - B) Bv = 0 \\
        (PAP^{-1} - B) B^2v = 0
      \end{align*}
      である。いま$B$の最小多項式は$2$次式なので、$B^2v= a Bv + b v$なる$a \in \F_p$と$b \in \F_p^{\tm}$がある。よって$(PAP^{-1} - B)v = 0$である。$\phi$が既約という仮定より$B$は固有ベクトルを持たないので、$\{ v, Bv\}$は$V$の$\F_p$基底となる。よって$PAP^{-1} = B$を得る。これですべての場合を尽くせたので、示すべきことがいえた。
      \item[(iii)] 共役ならば最小多項式が同じになることはあきらかなので、(ii)により共役であることと最小多項式が一致することの同値性がいえた。$G$の共役類の全体を$C(G)$とおくと、最小多項式を対応させる写像
      \[
      \Phi \colon C(G) \to \setmid{t^2 + at + b}{ b \neq 0  } \cup \setmid{t + c}{ c \neq 0  }
      \]
が単射であることがわかったことになる。逆に
\[
t^2 + at + b = \det \left( t E - \pmat{0 & -b \\ 1 & -a} \right)
\]
であるから、$\Phi$は全射でもある。対角行列を含まない$G$の共役類は、$\Phi$によって$(t- \grl)^2$という形の式か$2$次既約多項式に対応するので、その個数は
\[
(p-1) + \left( p^2 - p - \f{(p-1)(p-2)}{2}  \right) = \f{(p+2)(p-1)}{2}
\]
である。
  \end{description}
\end{sol}
