\section{平成19年度 数学I}

\subsubsection{}%1
\barquo{
$n,k$を自然数とし、$n \geq k$とする。$q$個の元からなる有限体$\F_q$上の$n$次元数ベクトル空間の$k$次元部分ベクトル空間の個数を求めよ。
}
\begin{proof}
  $V$を$\F_q$上の$n$次元ベクトル空間とする。$V$の元で一時独立な$k$個の元の順序付けられた組$(v_1, \cdots , v_k)$の集合を$A$集合の直積と区別するとき
  \[
  \# A = \prod_{i = 0}^{k-1} (q^n - q^i)
  \]
  である。$V$の$k$次元部分ベクトル空間の全体を$B$とする。$A$の元$(v_1, \cdots , v_k)$に対して、それが張るベクトル空間を対応させる$\vp \colon A \to B$は全射であり、各$W \in B$に対して
  \[
  \# \vp^{-1}(W) = \prod_{i=0}^{k-1} (q^k - q^i)
  \]
  だから
  \[
    \# B = \prod_{i=0}^{k-1} (q^n - q^i)(q^k - q^i)^{-1}
  \]
  である。
\end{proof}

\newpage


\subsubsection{}%2
\barquo{
$x$を実変数とする函数項級数
\[
\sum_{n=1}^{\infty} \f{x}{n^2 x^2 + 1}
\]
について次の問に答えよ。
\begin{description}
  \item[(1)] この級数の和を$S(x)$とするとき、$x \neq 0$に対して
  \[
  \abs{S(x)} \geq \f{\pi}{2} - \tan^{-1}\abs{x}
  \]
  が成り立つことを示せ。
  \item[(2)] この級数は、$\R$上では一様収束しないことを示せ。
\end{description}
}
\begin{proof} ${}$
  \begin{description}
    \item[(1)] $x > 0$を固定し、
    \[
    g(y) = \f{x}{y^2 x^2 + 1}
    \]
    とおく。すると$g$は単調減少なので、
    \begin{align*}
      S(x) &= \sum_{n=1}^{\infty} g(n) \\
      &\geq \int_1^{\infty} \f{x}{y^2 x^2 + 1} dy \\
&\geq \int_x^{\infty} \f{dz}{z^2 + 1} \\
&= \f{\pi}{2} - \tan^{-1}(x)
    \end{align*}
    が示せる。$x < 0$のときは
    \begin{align*}
      \abs{S(x)} &= \abs{S(-x)} \\
      &\geq \abs{ \f{\pi}{2} - \tan^{-1}(-x) } \\
      &\geq \f{\pi}{2} - \tan^{-1} \abs{x}
    \end{align*}
    である。
    \item[(2)] ハイリホーで示す。一様収束すると仮定する。このとき、おのおのの部分和
    \[
    S_N(x) = \sum_{n=1}^{N} \f{x}{n^2 x^2 + 1}
    \]
    は連続なので$S$も$\R$上連続なはずである。しかし$S(0) = 0$にも拘わらず$\liminf_{x \to + 0} S(x) \geq \pi /2 $なので$S$は連続ではない。これは矛盾である。
  \end{description}
\end{proof}

\newpage


\subsubsection{}%3
\barquo{
環$\Z[\s{-3}]$のイデアルで、$4$を含むものをすべて求めよ。
}
\begin{rem}
  $A = \Z[\s{-3}]$とおく。$A$は整閉ではないのでDedekind環ではなく、したがって素イデアル分解を使うわけにはいかない。地道きわまる考察が必要である。
\end{rem}
\begin{proof}
$A$のイデアルであって$4$を含むものは、商環$C = A/4A$のイデアルと一対一に対応する。計算すると
\[
C \cong \Z / 4\Z [x] /(x-1)(x+1)
\]
である。$\# C = 16$と比較的少ないので、$C$のすべてのイデアルを地道に数え上げる方針をとる。

まず$C^{\tm}$の元をすべてリストアップすると
\[
\pmat{
{}  & 1  & {} & 3  \\
 x & {} & 2+x & { } \\
{}  & 1+2x & {} & 3 + 2x \\
 3x & { } & 2+3x & {}
}
\]
の$8$個である。次に$C \sm C^{\tm}$の元を$C^{\tm}$の元による積で移り合うものを同一視してリストアップすると
\begin{align*}
  C(0) &= \{ 0 \} \\
  C(2) &= \{ 2 ,2x \} \\
  C(1+x) &= \{ 1+x, 3+3x \} \\
  C(3+x)  &= \{3+x, 1+3x \} \\
  C(2+2x)  &= \{2 + 2x \}
\end{align*}
である。(なお記号$C$は同値類を表す記号で、商環$C$とは無関係である)したがって商環$C$の単項イデアルは、精一杯多く見積もって
\[
0, C , (2) , (1+x), (3+x), (2+2x)
\]
しかない。2元で生成される単項でないイデアルは、調べるとわかるが$(2,1+x)$しかない。3元以上の元でしか生成されないイデアルは、$(2,1+x) \subset C$が極大イデアルなので存在しない。よって、$C$のイデアルは次の高々$7$個しかない。
\[
0, C , (2) , (1+x), (3+x), (2+2x), (2,1+x)
\]
$C$は整域ではないため、これらのイデアルが互いに異なるかどうかはあきらかではない。

対応する$A$のイデアルを考えると
\[
4A , A , 2A , (1+\s{-3}) A , (1 - \s{-3})A , (4, 2 + 2\s{-3})A , (2 , 1 + \s{-3})A
\]
である。これらがすべて異なることを言えばよい。それぞれの商環を考えると
\begin{align*}
  A/4A &= \Z / 4\Z [x] /(x-1)(x+1) \\
  A/2A &= \F_2[x]/(x+1)^2 \\
  A/(1 + \s{-3}) &= \zyu{4} \\
  A/(1 - \s{-3}) &= \zyu{4} \\
  A/(4, 2 + 2\s{-3}) &= \zyu{4} [x] / (x+1)(2,x+1) \\
  A/(2, 1+\s{-3}) &= \F_2
\end{align*}
である。包含関係をまとめると次のようになる。
\[
\xymatrix{
{} & A & { } \\
{} & (2, 1+\s{-3}) \ar[u] & {} \\
(1+\s{-3}) \ar[ur] & (1-\s{-3}) \ar[u] & (2) \ar[lu] \\
{} & (4,2+2\s{-3}) \ar[u] \ar[lu] \ar[ru] & { } \\
{} & (4) \ar[u] \ar[uur] & { }
}
\]
商環の考察で異なることがはっきりしないのは$(1+\s{-3})$と$(1-\s{-3})$だけだが、
\[
(1+\s{-3}) + (1-\s{-3}) = (2, 1 + \s{-3})
\]
より異なることがいえる。

まとめると、$4$を含む$A$のイデアルは
\[
4A , A , 2A , (1+\s{-3}) A , (1 - \s{-3})A , (4, 2 + 2\s{-3})A , (2 , 1 + \s{-3})A
\]
の$7$個である。
\end{proof}

\newpage

\subsubsection{}%4
\barquo{
2次元球面から2次元トーラスへの可微分写像の写像度は$0$であることを示せ。
}
\begin{proof}
  $S^2$は単連結なので、普遍被覆$\pi \colon \R^2 \to T^2$に関するリフト
  \[
  \xymatrix{
  {} & \R^2 \ar[d]^{\pi} \\
  S^2 \ar[ur]^g \ar[r]^f & T^2
  }
  \]
  がある。よって、$\R^2$は可縮なので$f_* \colon H_2(S^2 ) \to H_2(T^2)$はゼロ写像である。
\end{proof}

\newpage

\subsubsection{}%5
\barquo{
$\R$上の常微分方程式
\[
\f{dx}{dt} = (x+1)x(x-1)(x-2) \quad x(0)=0
\]
の解$x(t)$について、$\lim_{t \to \infty} x(t)$が存在すれば、それを求めよ。
}
\begin{proof}
  常微分方程式の解の存在と一意性より、$x$の初期値$a$が$0,-1,1,2$ならば$x$は恒等的に$0,-1,1,2$である。それ以外のとき$x(t)$は決して$0,-1,1,2$のどれにもならない。このとき所与の式を解くと
  \[
  t = \f{1}{6} \log \abs{ \f{(x-2)x^3 }{(x-1)^3 (x+1)} } + const.
  \]
  なので、$\lim_{t\to \infty} x(t)$が存在するのならば、それは$1$か$-1$である。

  初期値$a$に関して場合分けすると次のようになる。
  \begin{description}
    \item[(1)] $a>2$のとき。このとき$x'$はつねに正であり、$x$は単調増加で、極限において$x$は正の無限大に発散する。
    \item[(2)] $a=2$のとき。$x$は恒等的に$2$である。
    \item[(3)] $1<a<2$のとき。$x' < 0$なので、$x$は$1$に収束する。
    \item[(4)] $a=1$のとき。$x$は恒等的に$1$である。
    \item[(5)] $0<a<1$のとき。$x'>0$だから、$x$は$1$に収束する。
    \item[(6)] $a=0$のとき。$x$は恒等的に$0$である。
    \item[(7)] $-1<a<0$のとき。$x' < 0$だから、$x$は$-1$に収束する。
    \item[(8)] $a=-1$のとき。$x$は恒等的に$-1$である。
    \item[(9)] $a<-1$のとき。$x'>0$だから、$-1$に収束する。
  \end{description}
\end{proof}
