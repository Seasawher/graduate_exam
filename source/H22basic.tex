\section{平成22年度 基礎数学}

\subsubsection{}%1
\barquo{
実正方行列$A$, $B$を次のように定める。
\[
A = \pmat{ 3 &2& -1 &-1 \\ 6& 3& 4& 1 \\ 5& 1& 5& 1 \\ 1& -1& 3& 1}, \quad B = \pmat{ 1 &-1 & -1 &1 \\ 2& -1& -1& 3 \\ -1& 2& 3& 3 \\ -1& 2& 1& -2}
\]
$A$の行列式$\abs{A}$と$B$の逆行列$B^{-1}$を求めよ。
}
\begin{sol}
  計算すると$\abs{A}=6$であることがわかる。$B^{-1}$は
  \[
  B^{-1}= \pmat{ -13 &5& -2& -2 \\ 7& -2& 1& 2 \\ -15& 5& -2& -3 \\ 6& -2& 1& 1}
  \]
  と求まる。計算過程など頭脳明晰な読者諸賢には必要ないだろう。
\end{sol}

\newpage

\subsubsection{}%2
\barquo{
$n$行$m$列の実行列$A$の階数が$m$であるとする。このとき、$m$行$l$列の実行列$B$, $C$が
\[
AB = BC
\]
を満たせば
\[
B=C
\]
であることを示せ。
}
\begin{sol}
  $\rank A = m$ということは$n \geq m$かつ$A$は単射でなくてはいけない。一方で$A(B-C)=0$より$\R^l$の標準基底$\{ e_i \}$をとると、すべての$i$について$A(B-C)e_i = 0$である。$A$は単射なので$(B-C)e_i =0$が従う。$i$は任意だったから$B=C$である。
\end{sol}

\newpage

\subsubsection{}%3
\barquo{
次の重積分を求めよ。
\[
\iint_D \f{ dx dy}{ 1 + (x+y)^4 }
\]
ただし$D = \setmid{ (x,y) \in \R^2 }{ x \geq 0, y \geq 0, x+ y \leq 1}$とする。
}
\begin{sol}
  $s=x$, $t=x+y$と変数変換する。$dt ds = dx dy$である。積分領域$D$は
  \[
  E = \setmid{ (s,t ) \in \R^2 }{0 \leq t \leq 1, 0 \leq s \leq t}
  \]
  に対応する。これで計算すると
  \begin{align*}
    \iint_D \f{ dx dy}{ 1 + (x+y)^4 } &= \int_0^1 dt \int_0^t \f{ds}{1+t^4} \\
    &= \int_0^1 \f{t }{1+t^4} \ dt \\
    &= \f{1}{2 } \int_0^1 \f{dt }{1+t^2}  \\
    &= \f{1}{2} \int_0^{\pi/4} \ d\grt  \\
    &= \f{ \pi }{8}
  \end{align*}
がわかる。
\end{sol}


\newpage


\subsubsection{}%4
\barquo{
\begin{description}
  \item[(1)] ある定数$C$が存在し、$-1 \leq x \leq 1$のとき
  \[
  \abs{ e^x - 1 -x } \leq Cx^2
  \]
  となることを示せ。
  \item[(2)] 級数
  \[
  \sum_{n=1}^{\infty} \left( e^{x/n} - 1 - \f{x}{n} \right)
  \]
  は$x$が$[-1,1]$を動くとき一様収束することを示せ。
\end{description}
}
\begin{sol} ${}$
  \begin{description}
    \item[(1)] $e^x$の$x=0$の周りでのTaylor展開を考えると
    \[
    \lim_{x \to 0} \f{ e^x - 1 -x}{x^2 } = \f{1}{2}
    \]
    である。よって$h(x)=  (e^x - 1 -x )/x^2$は$[-1,1]$上の連続関数である。よってとくに有界であり$C = \max_{x \in [-1,1]} \abs{h(x)}$が存在する。この$C$が与えられた条件を満たす。
    \item[(2)] (1)を使うと
    \begin{align*}
      \sum_{n=1}^{\infty} \left( e^{x/n} - 1 - \f{x}{n} \right) &\leq C \sum_{n=1}^{\infty} \left( \f{ \abs{x} }{n} \right)^2 \\
      &\leq   C \sum_{n=1}^{\infty}  \f{ 1}{n^2}
    \end{align*}
    であることが判る。よって一様収束する。
  \end{description}
\end{sol}
