
\bfsection{平成29年度 専門}

\bfsubsection{問1}
\barquo{
$G$を有限群とする。$G$の自己準同形全体のなす群を$\Aut (G)$とおく。また、$G$および$\Aut (G)$の位数をそれぞれ$a= |G|$, $b= |\Aut G|$とおく。以下の問に答えよ。
\begin{description}
  \item[(i)] $b=1$のとき、$G$は自明群であるか、または$\Z / 2 \Z$と同型であることを示せ。
  \item[(ii)] $a$が奇数で$b=2$となるような$G$をすべて求めよ。
\end{description}
}
\begin{sol} この解答では集合の元の個数を$\#$で表記する。
   \begin{description}
     \item[(i)] 群準同形$\phi \colon G \to \Aut G$を$\phi_g(x)=gxg^{-1}$により定める。$\# \Aut G = 1$という仮定より、$G = \Ker \phi = Z(G)$である。したがって$G$はAbel群。よって有限生成Abel群の基本定理により、ある素数の集合$M \subset \Z$と写像$I \colon M \to P(\Z)$が存在して
     \[
     G = \bigoplus_{p \in M} \bigoplus_{i \in I(p)} \Z / p^{e_i} \Z
     \]
     と表すことができる。このとき乗法群$(\prod_{p, i } \Z / p^{e_i} \Z)^{\tm}$は$\Aut G$の部分群とみなせるので
     \[
     \prod_{p, i} p^{e_i - 1}(p -1) = 1
     \]
     である。したがって$M=\{ 2 \}$である。また任意の$i \in I(2)$に対して$e_i = 1$である。よって$G = (\Z / 2 \Z)^n$と表せるが、$\Aut G = 1$という仮定から$n=1$でなくてはならない。($n \geq 2$なら、たとえば順番を入れ替える写像などがある)
     \item[(ii)] 仮定から$G/ Z(G) = G/ \Ker \phi $の位数は2以下である。写像$f \colon G \to G$を$f(x) = x^2$で定義する。$x,y \in G$に対して、もしも$x \in Z(G)$または$y \in Z(G)$ならば$f(xy)=f(x)f(y)$である。また$x , y \in G \sm Z(G)$であれば、$xy \in Z(G)$なので$f(xy)=xy(xy) =f(x)f(y)$である。したがって$f$は群準同形となる。$\# G$は奇数なので$f$は単射であり、位数の考察から同型となる。このことから結局$G= Z(G)$であり、$G$はAbel群である。ゆえに有限生成Abel群の基本定理から
     \[
        G = \bigoplus_{p \in M} \bigoplus_{i \in I(p)} \Z / p^{e_i} \Z
     \]
     と表すことができる。このとき乗法群$(\prod_{p, i } \Z / p^{e_i} \Z)^{\tm}$は$\Aut G$の部分群とみなせるので
     \[
     \prod_{p, i} p^{e_i - 1}(p -1) \leq 2
     \]
     である。$p$としては$3$以上のものしか現れないから、$M = \{ 3 \}$である。また$\# I(3) =1$かつ$i \in I(3)$に対して$e_i = 1$であることもわかる。したがって$G = \Z / 3\Z$である。
   \end{description}
\end{sol}

\newpage

\bfsubsection{問2}
\barquo{
$n$は$2$以上の整数とし、$\zeta = e^{2\pi \I /n}$を$1$の原始$n$乗根とする。$\C[X,Y]$は変数$X,Y$に関する複素数係数の$2$変数多項式環とする。
\[
R = \setmid{f(X,Y) \in \C[X,Y] }{ f(\zeta X,\zeta Y) = f(X,Y)}
\]
とおく。以下の問に答えよ。
\begin{description}
  \item[(i)] $\C$代数として$R$は$n+1$個の元$X^n, X^{n-1}Y , \cdots , XY^{n-1}, Y^n$で生成されることを示せ。
  \item[(ii)] 複素数$a,b,c,d$に対し$m_{a,b} = (X-a,Y-b)$, $m_{c,d} = (X-c,Y-d)$を$\C[X,Y]$のイデアルとする。$m_{a,b} \cap R = m_{c,d} \cap R$が成り立つための$a,b,c,d$に関する必要十分条件を求めよ。
\end{description}
}
\begin{sol} ${}$
  \begin{description}
    \item[(i)] $X^n, X^{n-1}Y , \cdots , XY^{n-1}, Y^n$で$\C$上生成される環を$R'$とおく。$f \in R$が与えられたとする。$f$をゼロでない$d$次斉次元$f_d$の和として$f = \sum_{d \in I} f_d$と書く。このとき仮定から$0 = f(\zeta X,\zeta Y) - f(X,Y) = \sum_{d \in I} (\zeta^d - 1) f_d(X,Y) $である。したがって$I$の元はすべて$n$の倍数である。
    これはつまり$f \in R'$を意味する。よって$R \subset R'$である。逆は明らかだから$R = R'$がいえた。
    \item[(ii)] $1$の$n$乗根全体がなす位数$n$の巡回群を$G$と書くことにする。このとき、$(z a, z b) = (c,d)$なる$z \in G$が存在すれば$m_{a,b} \cap R = m_{c,d} \cap R$が成り立つことはあきらか。逆を示そう。

    $m_{a,b} \cap R = m_{c,d} \cap R$が成り立ったと仮定する。このとき$X^n - a^n$と$Y^n - b^n$はともに$m_{a,b} \cap R = m_{c,d} \cap R$の元である。よって$c^n - a^n = d^n -b^n = 0$であり、ある$z,w \in G$が存在して$c = za$, $d = wb$が成り立つ。ここでさらに$(bX - aY)^n$も$m_{a,b} \cap R = m_{c,d} \cap R$の元だから、$\C$は整域なので$bc - ad=0$である。
    よって$(z - w)ab=0$である。このとき$ab=0$または$z-w=0$であるが、いずれにせよ$(z a, z b) = (c,d)$なる$z \in G$が存在することには違いないので、示すべきことがいえた。
  \end{description}
\end{sol}
