\documentclass[10pt]{jsarticle}%文字サイズが10ptのjsarticle

%%%%%%%%%%%%%%%%%%%%%%%%%%%%%%%%%%%%%%%%%%%%%%%%%%%%%%%
%%  パッケージ                                        %%
%%%%%%%%%%%%%%%%%%%%%%%%%%%%%%%%%%%%%%%%%%%%%%%%%%%%%%%
%使用しないときはコメントアウトしてください
\usepackage{amsthm}%定理環境
\usepackage{framed}%文章を箱で囲う
\usepackage{amsmath,amssymb}%数式全般
\usepackage[dvipdfmx]{graphicx}%図の挿入
%\usepackage{tikz}%描画
\usepackage{titlesec}%見出しの見た目を編集できる
\usepackage[dvipdfmx, usenames]{color}%色をつける
%\usepackage{tikz-cd}%可換図式
%\usepackage{mathtools}%数式関連
\usepackage{amsfonts}%数式のフォント
\usepackage[all]{xy}%可換図式
\usepackage{mathrsfs}%花文字
\usepackage{comment}%コメント環境
\usepackage{picture}%お絵かき
\usepackage{url}%URLを出力
\usepackage[dvipdfmx]{hyperref}
\usepackage{pxjahyper}%日本語しおりの文字化けを防ぐ

\allowdisplaybreaks[1]%数式がページをまたぐことを許す

\begin{comment}
%%%%%%%%%%%%%%%%%%%%%%%%%%%%%%%%%%%%%%%%%%%%%%%%%%%%%%%
%%  表紙                                             %%
%%%%%%%%%%%%%%%%%%%%%%%%%%%%%%%%%%%%%%%%%%%%%%%%%%%%%%%
\makeatletter
\def\thickhrulefill{\leavevmode \leaders \hrule height 1pt\hfill \kern \z@}
\renewcommand{\maketitle}{\begin{titlepage}%
    \let\footnotesize\small
    \let\footnoterule\relax
    \parindent \z@
    \reset@font
    \null\vfil
    \begin{flushleft}
      \huge \@title
    \end{flushleft}
    \par
    \hrule height 4pt
    \par
    \begin{flushright}
      \LARGE \@author \par
    \end{flushright}
    \vskip 60\p@
    \vfil\null
    \begin{flushright}
        {\small \@date}%
    \end{flushright}
  \end{titlepage}%
  \setcounter{footnote}{0}%
}
\makeatother
\end{comment}




%%%%%%%%%%%%%%%%%%%%%%%%%%%%%%%%%%%%%%%%%%%%%%%%%%%%%%%
%%  sectionの修飾                                     %%
%%%%%%%%%%%%%%%%%%%%%%%%%%%%%%%%%%%%%%%%%%%%%%%%%%%%%%%
%四角ではじめて下線付き
\titleformat{\section}[block]
{}{}{0pt}
{
  \colorbox{black}{\begin{picture}(0,10)\end{picture}}
  \hspace{0pt}
  \normalfont \Large\bfseries
  \hspace{-4pt}
}
[
\begin{picture}(100,0)
  \put(3,18){\color{black}\line(1,0){300}}
\end{picture}
\\
\vspace{-30pt}
]

%subsubsectionの編集
\renewcommand{\thesubsubsection}{\textbf{問\arabic{subsubsection}}}%問(さぶさぶせくしょん番号)を出力






%%%%%%%%%%%%%%%%%%%%%%%%%%%%%%%%%%%%%%%%%%%%%%%%%%%%%%%
%%  番号付き定理環境                                  %%
%%%%%%%%%%%%%%%%%%%%%%%%%%%%%%%%%%%%%%%%%%%%%%%%%%%%%%%
%注:defというコマンドはもうある
\theoremstyle{definition}%定理環境のアルファベットを斜体にしない
\renewcommand{\proofname}{\textgt{証明}}%proof環境の修正

%%%%%%%%%%%%%%%%%%%%%%%%%%%%%%%%%%%%%%%%%%%%%%%%%%%%%%%
%%  番号なし定理環境                                  %%
%%%%%%%%%%%%%%%%%%%%%%%%%%%%%%%%%%%%%%%%%%%%%%%%%%%%%%%
%一部箱付き
\newtheorem*{lemma}{補題}
\newtheorem*{proposition}{命題}
\newtheorem*{definition}{定義}
\newcommand{\lem}[1]{\begin{oframed} \begin{lemma} #1 \end{lemma} \end{oframed}}%箱付きほだい
\newcommand{\prop}[1]{\begin{oframed} \begin{proposition} #1 \end{proposition} \end{oframed}}%箱付きめいだい

\newtheorem*{claim}{主張}
\newtheorem*{sol}{解答}
\newtheorem*{prob}{問題}
\newtheorem*{quo}{引用}
\newtheorem*{rem}{注意}



%%%%%%%%%%%%%%%%%%%%%%%%%%%%%%%%%%%%%%%%%%%%%%%%%%%%%%%
%%  左側に線を引く                                  %%
%%%%%%%%%%%%%%%%%%%%%%%%%%%%%%%%%%%%%%%%%%%%%%%%%%%%%%%
%leftbar環境の定義
\makeatletter
\renewenvironment{leftbar}{%
%  \def\FrameCommand{\vrule width 3pt \hspace{10pt}}%  デフォルトの線の太さは3pt
  \renewcommand\FrameCommand{\vrule width 1pt \hspace{10pt}}%
  \MakeFramed {\advance\hsize-\width \FrameRestore}}%
 {\endMakeFramed}
\newcommand{\exbf}[2]{ \begin{leftbar} \textbf{#1} #2 \end{leftbar} }%左線つき太字
%\newcommand{\barquo}[1]{\begin{leftbar} \begin{quo} #1 \end{quo} \end{leftbar}}%左線つき引用%びふぉあ
\newcommand{\barquo}[1]{\begin{leftbar} \noindent #1  \end{leftbar}}%左線つき引用%あふたー
\newcommand{\lbar}[1]{\begin{leftbar} #1 \end{leftbar}}



%%%%%%%%%%%%%%%%%%%%%%%%%%%%%%%%%%%%%%%%%%%%%%%%%%%%%%%
%%  色をつける                                      %%
%%%%%%%%%%%%%%%%%%%%%%%%%%%%%%%%%%%%%%%%%%%%%%%%%%%%%%%
\newcommand{\textblue}[1]{\textcolor{blue}{\textbf{#1}}}

%%%%%%%%%%%%%%%%%%%%%%%%%%%%%%%%%%%%%%%%%%%%%%%%%%%%%%%
%%  よく使う記号の略記                                 %%
%%%%%%%%%%%%%%%%%%%%%%%%%%%%%%%%%%%%%%%%%%%%%%%%%%%%%%%
\newcommand{\setmid}[2]{\left\{ #1 \mathrel{} \middle| \mathrel{} #2 \right\}}%集合の内包記法
\newcommand{\sm}{\setminus}%集合差
\newcommand{\abs}[1]{\left \lvert #1 \right \rvert}%絶対値
\newcommand{\norm}[1]{\left \lVert #1 \right \rVert}%ノルム
\newcommand{\transpose}[1]{\, {\vphantom{#1}}^t\!{#1}}%行列の転置
\newcommand{\pmat}[1]{ \begin{pmatrix} #1 \end{pmatrix} }%まるかっこ行列
\newcommand{\f}[2]{\frac{#1}{#2}}%分数
\newcommand{\kakko}[1]{ \langle #1  \rangle}%鋭角かっこ%\angleはもうある
\newcommand{\I}{\sqrt{-1}}%虚数単位。\iは既にある。
\newcommand{\single}{\{ 0 \}}%0のシングルトン
\newcommand{\clsub}{\subset_{\text{closed}}}%閉部分集合
\newcommand{\opsub}{\subset_{\text{open}}}%開部分集合
\newcommand{\clirr}{\subset_{\text{closed irr}}}%閉既約部分集合
\newcommand{\loc}{\subset_{\text{loc. closed}}}%局所閉部分集合
\newcommand{\wt}[1]{\widetilde{#1}}%わいどちるだあ
\newcommand{\ol}[1]{\overline{#1}}%オーバーライン
\newcommand{\wh}[1]{\widehat{#1}}%ワイドハット
\newcommand{\To}{\Rightarrow}%ならば%自然変換
\newcommand{\xto}[1]{\xrightarrow}%上側文字付き右向き矢印
\newcommand{\st}{\; \; \text{s.t.} \; \;}%空白付きsuch that
\newcommand{\ts}{\otimes}%テンソル積
\newcommand{\tm}{\times}%直積
\newcommand{\vartm}{\times^{\text{Var}}}%多様体の圏における直積。集合の直積と区別するとき用。
\newcommand{\la}{\overleftarrow}%上付き左矢印
\newcommand{\ra}{\overrightarrow}%上付き右矢印
\newcommand{\del}{\partial}%偏微分の記号
\newcommand{\zyu}[1]{ \mathbb{Z} / #1 \mathbb{Z} }%有限巡回群



%%%%%%%%%%%%%%%%%%%%%%%%%%%%%%%%%%%%%%%%%%%%%%%%%%%%%%%
%%       演算子                                       %%
%%%%%%%%%%%%%%%%%%%%%%%%%%%%%%%%%%%%%%%%%%%%%%%%%%%%%%%
%log型
\DeclareMathOperator{\rank}{rank}%行列の階数
\DeclareMathOperator{\rk}{rk}%行列の階数
\DeclareMathOperator{\corank}{corank}%行列の核の次元
\renewcommand{\Re}{\operatorname{Re}}%実部
\DeclareMathOperator{\Res}{Res}%留数
\DeclareMathOperator{\Gal}{Gal}%Galois群
\DeclareMathOperator{\Hom}{Hom}%射の集合
\DeclareMathOperator{\ind}{Ind}
\DeclareMathOperator{\tr}{tr}%トレース
\DeclareMathOperator{\Tr}{Tr}%トレース
\DeclareMathOperator{\Norm}{N}%ノルム
\DeclareMathOperator{\Aut}{Aut}%自己同型群
\DeclareMathOperator{\trdeg}{tr\text{.}deg}%超越次数
\DeclareMathOperator{\Frac}{Frac}%商体をとる操作
\renewcommand{\Im}{\operatorname{Im}}%写像の像。Abel圏の像対象。虚部が出力できなくなった。
\DeclareMathOperator{\Ker}{Ker}%写像の核。Abel圏の核対象。
\DeclareMathOperator{\im}{im}%写像の像
\DeclareMathOperator{\coker}{coker}%余核%対象のほう
\DeclareMathOperator{\Coker}{Coker}%余核%射のほう
\DeclareMathOperator{\Spec}{Spec}%スペクトル
\DeclareMathOperator{\Sing}{Sing}%Singular point.特異点の集合。歌ってるわけではないぞ
\DeclareMathOperator{\Supp}{Supp}%台
\DeclareMathOperator{\ann}{ann}%アナイアレーター
\DeclareMathOperator{\Ass}{Ass}%素因子
\DeclareMathOperator{\ord}{ord}%おーだー
\DeclareMathOperator{\height}{ht}%素イデアルの高度。\htはもうある
\DeclareMathOperator{\coht}{coht}%素イデアルの余高度
\DeclareMathOperator{\Lan}{Lan}%左Kan拡張
\DeclareMathOperator{\Ran}{Ran}%右Kan拡張
\DeclareMathOperator{\Orbit}{Orbit}%作用の軌道
\DeclareMathOperator{\Stab}{Stab}%作用の安定化群


%limit型
\DeclareMathOperator*{\llim}{\varprojlim}%極限。逆極限。射影極限。
\DeclareMathOperator*{\rlim}{\varinjlim}%余極限。順極限。入射極限。

%%%%%%%%%%%%%%%%%%%%%%%%%%%%%%%%%%%%%%%%%%%%%%%%%%%%%%%
%%  黒板太字(blackboard bold)                         %%
%%%%%%%%%%%%%%%%%%%%%%%%%%%%%%%%%%%%%%%%%%%%%%%%%%%%%%%
\newcommand{\bba}{{\mathbb A}}
\newcommand{\bbb}{{\mathbb B}}
\newcommand{\bbc}{{\mathbb C}}
\newcommand{\bbd}{{\mathbb D}}
\newcommand{\bbe}{{\mathbb E}}
\newcommand{\bbf}{{\mathbb F}}
\newcommand{\bbg}{{\mathbb G}}
\newcommand{\bbh}{{\mathbb H}}
\newcommand{\bbi}{{\mathbb I}}
\newcommand{\bbj}{{\mathbb J}}
\newcommand{\bbk}{{\mathbb K}}
\newcommand{\bbl}{{\mathbb L}}
\newcommand{\bbm}{{\mathbb M}}
\newcommand{\bbn}{{\mathbb N}}
\newcommand{\bbo}{{\mathbb O}}
\newcommand{\bbp}{{\mathbb P}}
\newcommand{\bbq}{{\mathbb Q}}
\newcommand{\bbr}{{\mathbb R}}
\newcommand{\bbs}{{\mathbb S}}
\newcommand{\bbt}{{\mathbb T}}
\newcommand{\bbu}{{\mathbb U}}
\newcommand{\bbv}{{\mathbb V}}
\newcommand{\bbw}{{\mathbb W}}
\newcommand{\bbx}{{\mathbb X}}
\newcommand{\bby}{{\mathbb Y}}
\newcommand{\bbz}{{\mathbb Z}}

%%%%%%%%%%%%%%%%%%%%%%%%%%%%%%%%%%%%%%%%%%%%%%%%%%%%%%%
%%  よく使う黒板太字                                  %%
%%%%%%%%%%%%%%%%%%%%%%%%%%%%%%%%%%%%%%%%%%%%%%%%%%%%%%%
\newcommand{\Z}{\bbz}
\newcommand{\A}{\bba}
\newcommand{\Q}{\bbq}
\newcommand{\R}{\bbr}
\newcommand{\C}{\bbc}
\newcommand{\F}{\bbf}
\newcommand{\N}{\bbn}
\renewcommand{\P}{\bbp}%パラグラフ記号が出力できなくなった

%%%%%%%%%%%%%%%%%%%%%%%%%%%%%%%%%%%%%%%%%%%%%%%%%%%%%%%
%%  カリグラフィー                                %%
%%%%%%%%%%%%%%%%%%%%%%%%%%%%%%%%%%%%%%%%%%%%%%%%%%%%%%%
%大文字しかどうせ使わない
\newcommand{\cala}{\mathcal{A}}
\newcommand{\calb}{\mathcal{B}}
\newcommand{\calc}{\mathcal{C}}
\newcommand{\cald}{\mathcal{D}}
\newcommand{\calf}{\mathcal{F}}
\newcommand{\calo}{\mathcal{O}}

%%%%%%%%%%%%%%%%%%%%%%%%%%%%%%%%%%%%%%%%%%%%%%%%%%%%%%%
%%  ギリシャ文字(Greek letters)小文字                 %%
%%%%%%%%%%%%%%%%%%%%%%%%%%%%%%%%%%%%%%%%%%%%%%%%%%%%%%%
%コマンドが5字以上のもの
\newcommand{\gra}{{\alpha}}
\newcommand{\grg}{{\gamma}}
\newcommand{\grd}{{\delta}}
\newcommand{\gre}{{\epsilon}}
\newcommand{\grt}{{\theta}}
\newcommand{\grk}{{\kappa}}
\newcommand{\grl}{{\lambda}}
\newcommand{\grs}{{\sigma}}
\newcommand{\gru}{{\upsilon}}
\newcommand{\gro}{{\omega}}

\newcommand{\ve}{{\varepsilon}}
\newcommand{\vp}{{\varphi}}

%%%%%%%%%%%%%%%%%%%%%%%%%%%%%%%%%%%%%%%%%%%%%%%%%%%%%%%
%%  ギリシャ文字(Greek letters)大文字                 %%
%%%%%%%%%%%%%%%%%%%%%%%%%%%%%%%%%%%%%%%%%%%%%%%%%%%%%%%
%コマンドが5字以上のもの
\newcommand{\grG}{{\Gamma}}
\newcommand{\grD}{{\Delta}}
\newcommand{\grT}{{\Theta}}
\newcommand{\grL}{{\Lambda}}
\newcommand{\grS}{{\Sigma}}
\newcommand{\grU}{{\Upsilon}}
\newcommand{\grO}{{\Omega}}

%%%%%%%%%%%%%%%%%%%%%%%%%%%%%%%%%%%%%%%%%%%%%%%%%%%%%%%
%%  フラクトゥール                                  %%
%%%%%%%%%%%%%%%%%%%%%%%%%%%%%%%%%%%%%%%%%%%%%%%%%%%%%%%
\newcommand{\fraka}{\mathfrak{a}}
\newcommand{\frakb}{\mathfrak{b}}
\newcommand{\frakm}{\mathfrak{m}}
\newcommand{\frakn}{\mathfrak{n}}
\newcommand{\frakp}{\mathfrak{p}}

\newcommand{\frakA}{\mathfrak{A}}
\newcommand{\frakB}{\mathfrak{B}}
\newcommand{\frakS}{\mathfrak{S}}
\newcommand{\frakT}{\mathfrak{T}}

\newcommand{\Top}{\mathfrak{Top}}%開部分集合全体のなす有向集合
\newcommand{\Ab}{\mathfrak{Ab}}%Abel群のなす圏

%%%%%%%%%%%%%%%%%%%%%%%%%%%%%%%%%%%%%%%%%%%%%%%%%%%%%%%
%%  花文字                                          %%
%%%%%%%%%%%%%%%%%%%%%%%%%%%%%%%%%%%%%%%%%%%%%%%%%%%%%%%
%大文字しかどうせ使わない
\newcommand{\scra}{\mathscr{A}}
\newcommand{\scrf}{\mathscr{F}}
\newcommand{\scrg}{\mathscr{G}}
\newcommand{\scrh}{\mathscr{H}}
\newcommand{\scrs}{\mathscr{S}}

%%%%%%%%%%%%%%%%%%%%%%%%%%%%%%%%%%%%%%%%%%%%%%%%%%%%%%%
%%  太字                                            %%
%%%%%%%%%%%%%%%%%%%%%%%%%%%%%%%%%%%%%%%%%%%%%%%%%%%%%%%
\newcommand{\Sh}{\textbf{Sh}}%層の圏
\newcommand{\PSh}{\textbf{PSh}}%前層の圏
\newcommand{\bfzero}{\textbf{0}}%太字のゼロ

%小文字
\newcommand{\bfb}{\textbf{b}}
\newcommand{\bfu}{\textbf{u}}
\newcommand{\bfv}{\textbf{v}}
\newcommand{\bfx}{\textbf{x}}
\newcommand{\bfy}{\textbf{y}}


%大文字
\newcommand{\bfC}{\textbf{C}}
\newcommand{\bfD}{\textbf{D}}
\newcommand{\bfE}{\textbf{E}}



\setcounter{tocdepth}{3}%目次に含めるレベル。1ならsectionまで。2ならsubsectionまで。
\begin{document}



\title{京都大学 数学系 院試}
\author{ \url{https://seasawher.github.io/kitamado/} \\ @seasawher}
\date{\today}
\maketitle




\tableofcontents%目次
\newpage







\bfsection{平成31年度 基礎科目}


\bfsubsection{問1}
\barquo{
$\gra$は$0 < \gra < \f{\pi}{2}$を満たす定数とする。このとき広義積分
\[
\iint_D e^{-(x^2 + 2xy \cos \gra + y^2)} \ dx dy
\]
を計算せよ。ただし、$D=\setmid{(x,y) \in \R^2 }{ x \geq 0, y \geq 0}$とする。
}
\begin{sol}
$x = r \cos \grt$, $y = r \sin \grt$と変数変換する。領域$D$は、$\setmid{(r,\grt)}{r \geq 0, 0 \leq \grt \leq \f{\pi}{2} }$へ移る。すると$dx dy = r dr d\grt$であって
\begin{align*}
  \iint_D e^{-(x^2 + 2xy \cos \gra + y^2)} \ dx dy &= \int_0^{\f{\pi}{2}} \ d\grt \int_{0}^{\infty} e^{-r^2(1 + \sin 2 \grt \cos \gra)} r \ dr \\
  &= \f{1}{2} \int_0^{\f{\pi}{2}} \ d\grt \int_{0}^{\infty} e^{-r(1 + \sin 2 \grt \cos \gra)}  \ dr \\
  &= \f{1}{2} \int_0^{\f{\pi}{2}} \f{d \grt}{1 + \sin 2\grt \cos \gra} \\
  &= \f{1}{4} \int_0^{\pi} \f{d \grt}{1 + \sin \grt \cos \gra}
\end{align*}
と計算できる。さらに$t = \tan \f{\grt}{2}$として変数変換を行う。$d\grt = 2(1+t^2)^{-1} dt$で、$\sin \grt = 2t / (1+t^2)$だから
\begin{align*}
    \iint_D e^{-(x^2 + 2xy \cos \gra + y^2)} \ dx dy &= \f{1}{4} \int_0^{\infty} \f{2(1+t^2)^{-1} dt}{1 + 2t(1+t^2)^{-1}  \cos \gra} \\
    &= \f{1}{2} \int_0^{\infty} \f{dt}{(t+ \cos \gra)^2 + \sin^2 \gra } \\
    &= \f{1}{2} \int_{\cos \gra}^{\infty} \f{dt}{t^2 + \sin^2 \gra } \\
    &= \f{1}{2 \sin \gra} \int_{1 / \tan \gra}^{\infty} \f{dt}{t^2 + 1} \\
    &= \f{1}{2 \sin \gra} \left( \f{\pi}{2} - \arctan \left( \f{1}{\tan \gra} \right) \right)
\end{align*}
である。ここで、$\tan(\f{\pi}{2} - \gra ) = \f{1}{\tan \gra}$であることから、結論として次を得る。
\[
\iint_D e^{-(x^2 + 2xy \cos \gra + y^2)} \ dx dy = \f{\gra}{2 \sin \gra}
\]
\end{sol}


\newpage


\bfsubsection{問2}
\barquo{
複素数$\gra$に対し、3次複素正方行列$A(\gra)$を次のように定める。
\[
A(\gra) = \pmat{ \gra -4 & \gra +4 & -2 \gra +1 \\ -2 & 2 \gra +1 & -2 \gra +2 \\ -1 & \gra & - \gra +2}
\]
\begin{description}
  \item[(1)] $A(\gra)$の行列式を求めよ。
  \item[(2)] $A(\gra)$の階数を求めよ。
\end{description}
}
\begin{sol} ${}$
\begin{description}
  \item[(1)] ある行に別の行の定数倍を足す操作を繰り返し行っていくと
  \begin{align*}
    A(\gra) &\sim \pmat{ \gra -3 & 4 & - \gra -1 \\ 0 & 1  & -2 \\ -1 & \gra & - \gra +2} \\
    &\sim \pmat{ \gra -3 & 0 & - \gra +7 \\ 0 & 1  & -2 \\ -1 & 0 & \gra +2} \\
    &\sim \pmat{ 0 & 0 & (\gra - 1)^2 \\ 0 & 1  & -2 \\ -1 & 0 & \gra +2} \\
  \end{align*}
  と変形できる。よって$\det A(\gra) = (\gra - 1)^2$である。
  \item[(2)] $\gra=1$のときは階数2である。それ以外のときは正則で、階数は3である。
\end{description}

\end{sol}


\newpage



\bfsubsection{問3}
\barquo{
$(x_0, y_0) \in \R^2 \sm \{(0,0)\}$に対して、$\R$上の連立常微分方程式
\[
\begin{cases}
  \f{dx}{dt} = -x^2 y - y^3 \\
    \f{dy}{dt} = x^3 +  xy^2
\end{cases}
\quad
\begin{cases}
x(0) = x_0 \\
y(0) = y_0
\end{cases}
\]
の解$(x(t),y(t))$は周期を持つことを示し、最小の周期を求めよ。ただし正の実数$T$が$(x(t),y(t))$の周期であるとは、任意の$t \in \R$に対して
\[
(x(t + T),y(t + T)) = (x(t),y(t))
\]
が成り立つことである。
}
\begin{sol}
与式より
\begin{align*}
  x \f{dx}{dt} + y \f{dy}{dt} &= 0 \\
  \f{d}{dt}(x^2 + y^2) &= 0
\end{align*}
を得る。したがって$C = x^2 + y^2$は定数であり、$C = x_0^2 + y_0^2$が成り立つ。ゆえに与式は
\[
\begin{cases}
  \f{dx}{dt} = - Cy \\
    \f{dy}{dt} = Cx
\end{cases}
\]
と書き直せる。この連立方程式を一変数にまとめると
\[
\f{d^2 x}{dt^2} = - C^2 x
\]
となるが、この解空間は$\cos (C t)$と$\sin (Ct)$で張られる。したがって、一般解はこの線形結合で書けるのだから
\begin{align*}
  x(t) = x_0 \cos(Ct) - y_0 \sin (Ct) \\
  y(t) = y_0 \cos(Ct) + x_0 \sin (Ct)
\end{align*}
でなくてはならない。常微分方程式の初期値問題の解の一意性より、解はこれだけである。よって求める周期は$2\pi / C$である。
\end{sol}




\bfsubsection{問4}
\barquo{
$f$は$\R$上の実数値$C^1$級関数で任意の$x \in \R$に対して$f(x+1)=f(x)$を満たすとする。このとき以下の2条件は同値であることを示せ。
\begin{description}
  \item[(A)] 広義積分
  \[
  \int_1^{\infty} \f{1}{x^{1+f(x)^2}} \ dx
  \]
  が収束する。
  \item[(B)] $f(x) = 0$となる$x \in \R$が存在しない。
\end{description}
}
\begin{sol}

\end{sol}


\newpage

\bfsubsection{問5}
\barquo{
$n$を2以上の整数、$A$を$n$次複素正方行列とする。$A^{n-1}$は対角化可能でないが、$A^n$が対角化可能であるとき、$A^n=0$となることを示せ。
}
\begin{sol}
$\C$係数なので、Jordan標準形が存在する。$A$ははじめからJordan標準形であるとしてよい。
\[
A = \bigoplus_{i=1}^r J_{\grl_i}(a_i)
\]
とする。$a_1, \cdots , a_r$は(異なるとは限らない)固有値であり、$\grl_i$はそれぞれのジョルダン細胞のサイズである。
\[
A^n = \bigoplus_{i=1}^r J_{\grl_i}(a_i)^n
\]
は対角化可能なので、各$J_{\grl_i}(a_i)^n$も対角化可能。ここで$J_{\grl_i}(a_i)$のJordan分解
\[
S_i = \pmat{a_i &  & &  \\  &  \ddots & &  \\ & & \ddots & \\ &  & & a_i} \quad N_i = \pmat{0 & 1 & & \\ & \ddots & \ddots & \\ & & \ddots & 1 \\ & & & 0 }
\]
を考える。
\[
J_{\grl_i}(a_i)^n = S_i^n + \sum_{k=1}^n \binom{n}{k} S_i^{n-k} N_i^k
\]
であって、$S_i^n$は対角行列で$\sum_{k=1}^n \binom{n}{k} S_i^{n-k} N_i^k$はべき零行列だから、Jordan分解の一意性より
\[
\sum_{k=1}^n \binom{n}{k} S_i^{n-k} N_i^k = 0
\]
を得る。左辺は具体的に書くことができて、次のような$\grl_i$次行列
\[
\pmat{
0 & \binom{n}{1} a_i^{n-1} & \binom{n}{2} a_i^{n-2} & \cdots & \binom{n}{\grl_i - 1} a_i \\
  & 0 & \binom{n}{1} a_i^{n-1} & \cdots & \binom{n}{\grl_i - 2} a_i^2 \\
  &   &  \ddots &  & \vdots \\
  &   &        &  &  0
}
\]
である。$\grl_i = 1$のときにはこの等式から情報を得ることはできない。しかし$\grl_i \geq 2$ならば$a_i = 0$であることがわかる。つまりサイズが$2$以上のJordan細胞はべき零である。そこで仮にサイズが$1$のJordan細胞$J_1(a_i)$が存在したと仮定する。$n \geq 2$という仮定より、このときサイズが$2$以上のJordan細胞のサイズは$n-1$以下でなくてはならない。したがって、サイズが$2$以上のJordan細胞はすべて$n-1$乗するとゼロである。よって$A^{n-1}$は対角化可能となるが、これは矛盾。ゆえにサイズが$1$のJordan細胞は存在しないので、$A$のJordan細胞はただひとつしかなく、$A^n=0$であることが導かれる。
\end{sol}

\newpage

\bfsubsection{問6}
\barquo{
$\R^2$上の実数値連続関数$f$についての次の条件($*$)を考える。

($*$) 任意の正の実数$R$に対して、次の集合は有界である。
\[
\setmid{(x,y) \in \R^2}{ \abs{f(x,y)} \leq R }
\]
以下の問に答えよ。
\begin{description}
  \item[(1)] 条件($*$)をみたす連続関数$f$の例を与え、それが($*$)をみたすことを示せ。
  \item[(2)] 連続関数$f$が条件($*$)を満たすとき、次のいずれかが成り立つことを示せ。

  (a) $f$は最大値を持つが、最小値は持たない。

  (b) $f$は最小値を持つが、最大値は持たない。
\end{description}
}
\begin{sol} ${}$
  \begin{description}
    \item[(1)] たとえば$f(x,y) = x^2 + y^2$とすればよい。これが($*$)を満たすことはあきらか。
    \item[(2)] $f$が条件($*$)を満たすとする。$f$の可能性としては、次の4通りが考えられる。
\begin{description}
  \item[(A1)] $f$は上にも下にも有界
  \item[(A2)] $f$は上に有界だが下に有界でない
  \item[(A3)] $f$は下に有界だが上に有界でない
  \item[(A4)] $f$は上にも下にも有界でない
\end{description}
    それぞれの場合について考えていく。まず(A1)の場合、任意の$x$について$\abs{f(x)} \leq M$なる$M > 0$が存在する。よって仮定より、$\R^2$が有界となって矛盾。つまりそんな関数はない。

    次に(A2)の場合。$\sup f(x) = R$とする。仮定から集合
    \[
    V = \setmid{(x,y) \in \R^2}{ \abs{f(x,y)} \leq R }
    \]
    は有界閉集合である。よって$V$はコンパクト。$f(V)$もコンパクトなので、$f(V)$は最大値$M$を持つ。あきらかに$M \leq R$である。
    任意に$0 \leq \ve \leq R/2$が与えられたとしよう。$\sup f(x) = R$より$R-\ve < f(z)$なる$z$がある。このとき$z \in V$だから$R - M \leq \ve$であり、$0 \leq \ve \leq R/2$は任意だったから$R \leq M$でなくてはならない。よって$R=M$であり、$f$は最大値を持つが、最小値は持たない関数である。(A3)は(A2)と同様で、このとき$f$は最小値を持つが最大値を持たない。

    残る(A4)について考えよう。
    \[
    K = \setmid{(x,y) \in \R^2}{f(x,y)=0 }
    \]
    とすると、仮定から$K$は有界閉集合である。$M$を十分に大きな正の実数として、$K$をすっぽり含むような閉円板
    \[
    B= \setmid{(x,y) \in \R^2}{ x^2 + y^2 \leq M}
    \]
    をとることができる。$\R^2$を全体として補集合をとることにすると、このとき$B^c$は連結開集合である。
    \[
    U = \setmid{(x,y) \in \R^2}{f(x,y) > 0} \quad V = \setmid{(x,y) \in \R^2}{f(x,y) < 0}
    \]
    とおく。このとき$U$と$V$の共通部分は空であり、ともに開集合である。だから、$B^c = (U \cap B^c) \cup (V \cap B^c)$から、$B^c$が連結集合であることに矛盾。よってそのような関数はない。以上により示すべきことがいえた。
  \end{description}
\end{sol}


\newpage

\bfsection{平成31年度 専門科目}

\bfsubsection{問5}
\barquo{
$\C$の部分空間
\[
X = \setmid{1- e^{i\grt} \in \C }{0 \leq \grt < 2\pi} \cup \setmid{-1 + e^{i\grt} \in \C}{0 \leq \grt < 2\pi }
\]
を考える。整数$p,q$に対して、写像$f \colon X \to X$を
\begin{align*}
  f(1- e^{i\grt}) &= -1 + e^{ip\grt} \\
    f(-1 + e^{i\grt}) &= 1 - e^{iq\grt}
\end{align*}
で定め、$X \tm [0,1]$に
\[
(x,0) \sim (f(x),1)
\]
($x \in X$)で生成される同値関係$\sim$を与える。商空間$Y = (X \tm [0,1])/ \sim$の整数係数ホモロジー群を計算せよ。
}
\begin{sol}
\begin{comment}
セル複体を使ってホモロジーを求めよう。空間$Y$を直接書くことは難しいが、次のようなものを想像することはできる。

\begin{center}
\includegraphics[width=5cm]{strange_object.png}
\end{center}

この対になった円筒は、$X \tm I$を表している。円盤が格子でつながっているかのような図だが、本当は側面もある。垂直方向が$I$成分を表しており、上が$t=0$で下が$t=1$であるものとしよう。また右を実軸のプラス方向、奥を虚軸のプラス方向とする。底の緑の部分は原点を表す。この図形の2-セルは、上下にある4枚の円盤を同一視したものと側面の3つである。向きは図のように定めておく。
\end{comment}
\end{sol}


\newpage

\section{平成30年度 基礎科目}

\subsubsection{} %{問1}
\lbar{
広義積分
\[
\iiint_V \f{1}{(1+x^2 + y^2)z^{ \f{3}{2} }} \; dx dy dz
\]
を計算せよ。ただし、$V=\setmid{(x,y,z) \in \R^3}{x^2+y^2 \leq z}$とする。
}

\begin{sol}
  極座標変換$(x,y,z) \mapsto (r,\grt, z)$を考える。このとき$dx dy dz = r dr d\grt dz$であり、
  \begin{align*}
    \iiint_V \f{1}{(1+x^2 + y^2)z^{ \f{3}{2} }} \; dx dy dz &= \int_{0}^{2\pi} \; d\grt \int_{0}^{\infty} \f{r}{1+r^2} \left( \int_{r^2}^{\infty} z^{ -\f{3}{2}  } \; dz \right) \; dr \\
    &= 2\pi \int_{0}^{\infty} \f{r}{1+r^2} \left[ (-2)z^{-\f{1}{2}} \right]^{\infty}_{r^2} \; dr \\
    &= 4\pi \int_{0}^{\infty} \f{r}{1+r^2} \f{1}{r} \; dr \\
    &= 4\pi \int_{0}^{\infty} \f{1}{1+r^2} \; dr \\
    &= 4\pi \cdot \f{\pi}{2} \\
    &= 2\pi^2
  \end{align*}
  と計算できる。
\end{sol}

\newpage

\subsubsection{} %{問2}
\lbar{
$a,b$を実数とする。実行列
\[
A = \begin{pmatrix}
1 & 1 & a & b \\
0 & 1 & 2 & 0 \\
2 & 0 & 1 & 4
\end{pmatrix}
\]
について、以下の問に答えよ。
\begin{description}
\item[(1)] 行列$A$の階数を求めよ。
\item[(2)] 連立1次方程式
\[
A \begin{pmatrix}
x_1 \\
x_2 \\
x_3 \\
x_4
\end{pmatrix} = \begin{pmatrix}
1 \\ 1 \\ 1
\end{pmatrix}
\]
が解を持つような実数$a,b$をすべて求めよ。
\end{description}
}

\begin{sol} ${}$
  \begin{description}
    \item[(1)] ある行に別の行の定数倍を足す操作を繰り返すと
    \begin{align*}
      A &\sim \begin{pmatrix}
      1 & 1 & a & b \\
      0 & 1 & 2 & 0 \\
      0 & -2 & 1-2a & 4-2b
      \end{pmatrix} \\
      &\sim \begin{pmatrix}
      1 & 1 & a & b \\
      0 & 1 & 2 & 0 \\
      0 & 0 & 5-2a & 4-2b
      \end{pmatrix}
    \end{align*}
    と変形できる。したがって$\rank A \geq 2$であり、$(a,b) = (\f{5}{2}, 2)$のときは$\rank A =2$で、$(a,b) \neq  (\f{5}{2}, 2)$のときは$\rank A =3$である。
    \item[(2)] $(a,b) \neq  (\f{5}{2}, 2)$ならば、$A \colon \R^4 \to \R^3$は全射なので、解がある。$(a,b) = (\f{5}{2}, 2)$のとき、拡大係数行列を考えると
    \begin{align*}
      \begin{pmatrix}
      1 & 1 & \f{5}{2} & 2 & 1 \\
      0 & 1 & 2 & 0  & 1 \\
      2 & 0 & 1 & 4 & 1
      \end{pmatrix} \sim
      \begin{pmatrix}
      1 & 1 & \f{5}{2} & 2 & 1 \\
      0 & 1 & 2 & 0  & 1 \\
      0 & -2 & -4 & 0 & -1
      \end{pmatrix}
      \sim \begin{pmatrix}
      1 & 1 & \f{5}{2} & 2 & 1 \\
      0 & 1 & 2 & 0  & 1 \\
      0 & 0 & 0 & 0 &  1
      \end{pmatrix}
    \end{align*}
    となるので、解はない。
  \end{description}
\end{sol}


\newpage

\subsubsection{} %{問 3}
\barquo{
広義積分
\[
\int_{-\infty}^{\infty} \f{\cos(\pi x)}{1 + x^2 + x^4} \; dx
\]
を求めよ。
}
\begin{sol}
$f$, $F$を
    \[
    f(x) = \f{\cos(\pi x)}{1 + x^2 + x^4}, \quad F(z) = \f{e^{\I \pi z}}{1 + z^2 + z^4}
    \]
により定める。$x \in \R$なら$f(x) = \Re F(x)$である。

ここで分母の$1 + z^2 + z^4$を因数分解しておく。
$\zeta = \exp(\I \pi / 3) = (1 + \sqrt{-3})/ 2$とする。$1+z + z^2$の根は$1$の原始$3$乗根であることから
\begin{align*}
  z^4 + z^2 + 1 &= (z^2 - \zeta^2)(z^2 - \zeta^4) \\
  &= (z - \zeta )(z + \zeta) (z - \zeta^2)(z + \zeta^2)
\end{align*}
である。

上反平面に含まれる半径$R$の半円を$C_R$とする。留数定理により、任意の$R > 1$について
\[
2 \pi \I ( \Res_{z=\zeta} F + \Res_{z=\zeta^2} F ) = \int_{-R}^R F(x) \; dx + \int_{C_R} f(z) \; dz
\]
が成り立つ。

ここで、
\begin{align*}
  \abs{\int_{C_R} f(z) \; dz } &\leq \int_0^{\pi} \abs{ \f{R \exp(\I R \pi e^{\I \grt}) }{ 1 + R^2e^{2 \I \grt} +  R^4e^{4 \I \grt}} } \; d\grt \\
  &\leq  \int_0^{\pi} \f{R e^{- R \pi \sin \grt} }{R^4 - R^2 - 1}   \; d\grt \\
  &\leq \f{R  }{R^4 - R^2 - 1}  \int_0^{\pi} \; d\grt \\
  &\leq  \f{R \pi }{R^4 - R^2 - 1}
\end{align*}
だから、$R \to \infty$のとき$\int_{C_R} f(z) \; dz \to 0$である。したがって
\[
 \int_{-\infty}^{\infty} f(x) \; dx =  \Re( 2 \pi \I ( \Res_{z=\zeta} F + \Res_{z=\zeta^2} F ) )
\]
であることがわかる。

実際に留数を計算しよう。詳細は省略するが、堅実な計算により
\begin{align*}
  \Res_{z=\zeta} F &= \f{ \exp(\I \pi \f{1 + \sqrt{-3}}{2} ) }{(2\zeta) (\zeta - \zeta^2) (\zeta + \zeta^2)  } \\
  &= \f{ - \I \exp(- \f{\sqrt{3}\pi}{2} ) }{2(1 - \zeta^2)} \\
  \Res_{z=\zeta^2} F &= \f{ \exp(\I \pi \f{-1 + \sqrt{-3}}{2} ) }{(\zeta^2 - \zeta) (\zeta^2 + \zeta) (\zeta^2 + \zeta^2)  } \\
  &=  \f{ - \I \exp(- \f{\sqrt{3}\pi}{2} ) }{2(1 + \zeta)}
\end{align*}
がわかる。$\gra = \exp(- \f{\sqrt{3}\pi}{2} )$とおこう。すると
\begin{align*}
  2 \pi \I ( \Res_{z=\zeta} F + \Res_{z=\zeta^2} F ) &= \gra \pi \left( \f{1}{1 - \zeta^2} + \f{1}{1+ \zeta} \right) \\
  &= \gra \pi \left( \f{2 - \zeta}{ 1 - \zeta^2} \right) \\
  &= \gra \pi
\end{align*}
である。$\gra \in \R$だから、
\[
 \int_{-\infty}^{\infty} f(x) \; dx = e^{- \f{\sqrt{3}\pi}{2} } \pi
\]
が結論される。
\end{sol}
\newpage

\subsubsection{} %{問 4}
\lbar{
閉区間$[0,1]$上の実数値関数列$\{ f_n \}^{\infty}_{n=1}$について、各$f_n$は広義単調増加であるものとする。つまり、$0 \leq x < y \leq 1$なら、$f_n(x) \leq f_n(y)$である。この関数列$\{ f_n \}^{\infty}_{n=1} $が$n \to \infty$で関数$f$に各点収束したとする。
\begin{description}
  \item[(1)] 任意の$0 \leq x < y \leq 1$に対し、不等式
  \[
  \sup_{x \in [x,y]} \abs{f_n(z) - f(z)} \leq \max\{ \abs{f_n(x)- f(y)}, \abs{f_n(y)-f(x)}  \}
  \]
  を示せ。
  \item[(2)] 関数$f$が連続であるとき、関数列$\{ f_n \}^{\infty}_{n=1}$は$f$に$[0,1]$上で一様収束することを示せ。
\end{description}
}
\begin{sol} ${}$
\begin{description}
  \item[(1)] まず$f$が広義単調増加であることを示す。$0 \leq x < y \leq 1$とする。$\ve > 0$が与えられたとする。$f_n$が$f$に各点収束することにより
  \begin{align*}
    n \geq N(x) \to \abs{f(x) - f_n(x)} < \ve \\
      n \geq N(y) \to \abs{f(y) - f_n(y)} < \ve
  \end{align*}
  なる$N(x), N(y)$の存在がわかる。したがって$n \geq \max\{ N(x), N(y) \}$のとき
  \begin{align*}
    f(y) - f(x) + 2\ve &= (f(y) + \ve ) - f(x) + \ve \\
    &\geq f_n(y) - f(x) + \ve &(-\ve < f(y)-f_n(y) < \ve \text{より}) \\
    &\geq f_n(y) - f_n(x) &(-\ve < f(x)-f_n(x) < \ve \text{より}) \\
    &\geq 0
  \end{align*}
  がわかる。$\ve > 0$は任意だったから、$f(y) \geq f(x)$がわかる。つまり$f$は広義単調増加である。

  したがって任意の$z \in [x,y]$に対して
\begin{align*}
f_n(z) - f(z) \leq f_n(y) - f(x) \\
f(z) - f_n(z) \leq f(y) - f_n(x)
\end{align*}
が成り立つので、
\[
\abs{f_n(z) - f(z)} \leq \max\{ \abs{f_n(x)- f(y)}, \abs{f_n(y)-f(x)}  \}
\]
である。右辺は$z$の取り方によらないので、
\[
\sup_{x \in [x,y]} \abs{f_n(z) - f(z)} \leq \max\{ \abs{f_n(x)- f(y)}, \abs{f_n(y)-f(x)}  \}
\]
がいえた。
\item[(2)] $\ve > 0$が与えられたとする。$I = [0,1]$はコンパクトなので、$f$は一様連続であることまでいえる。そこで
\[
\abs{x -y} < \grd \to \abs{f(x) - f(y)} < \ve
\]
なる$\grd > 0$がある。この$\grd$を固定し、$B(z) = [z - \grd/3, z + \grd/3] \cap I$とする。$\grd > 0$なので、$I = \bigcup_{i=1}^m B(z_i)$なる有限個の$z_i \in I$をとることができる。$B(z_i ) = [x_i, y_i]$と表すことにする。

$f_n$は$f$に各点収束しているので、
\begin{align*}
  n \geq N(x_i) &\to \abs{f(x_i) - f_n(x_i)} < \ve \\
  n \geq N(y_i) &\to \abs{f(y_i) - f_n(y_i)} < \ve
\end{align*}
なる$N(x_i), N(y_i)$がある。そこで
\[
n \geq \max\{ N(x_1), \cdots , N(x_m) , N(y_1) , \cdots , N(y_m)  \}
\]
とする。このとき
\begin{align*}
  \abs{f_n(x_i)- f(y_i) } &\leq \abs{ f_n(x_i) - f(x_i) } + \abs{ f(x_i) - f(y_i) } \\
  &\leq 2\ve \\
\abs{f_n(y_i)- f(x_i) }  &\leq \abs{ f_n(y_i) - f(y_i) } + \abs{ f(y_i) - f(x_i) } \\
&\leq 2\ve
\end{align*}
が成り立つ。したがって(1)により、不等式評価を端点に押しつけることができて
\begin{align*}
\sup_{z \in I } \abs{f_n(z) - f(z)} &\leq \max_{1 \leq i \leq m} \sup_{x \in [x_i,y_i]} \abs{f_n(z) - f(z)} \\
&\leq \max_{1 \leq i \leq m} \max\{ \abs{f_n(x_i)- f(y_i)}, \abs{f_n(y_i)-f(x_i)}  \} \\
&\leq 2\ve
\end{align*}
である。これで一様収束がいえた。
\end{description}
\end{sol}

\newpage

\subsubsection{} %{問 5}
\barquo{
$p$を素数とし、$\F_p = \Z / p \Z$を位数$p$の有限体とする。行列の乗法による群$G$を
\[
G = \setmid{ \pmat{1 & a & b \\ 0 & 1 & c \\ 0 & 0 & 1 } }{a,b,c \in \F_p}
\]
で定める。このとき、$G$から乗法群$\C^{\tm} = \C \sm \{ 0 \}$への準同形写像の個数を求めよ。
}
\begin{sol}
 集合$\Hom(G, \C^{\tm})$と$\Hom(G/[G,G], \C^{\tm})$の間には全単射がある。したがって$G/[G,G]$の構造を決定すればよい。そのためにまず$[G,G]$を決定する。
    \[
    A = \pmat{1 & a & b \\ 0 & 1 & c \\ 0 & 0 & 1},  \quad B = \pmat{1 & \grd & \beta \\ 0 & 1 & \grg \\ 0 & 0 & 1}
    \]
    とおく。($\gra$は$a$と間違えやすいので、$\grd$を使った。)計算すれば、このとき
    \begin{align*}
    ABA^{-1}B^{-1} = \pmat{1 & 0  & a \grg - c \grd \\ 0 & 1 & 0 \\ 0 & 0 & 1 }
  \end{align*}
  であることが判る。$a \grg - c \grd$は$\F_p$全体をわたるので、
  \[
  [G,G] = \setmid{ \pmat{1 & 0 & d \\ 0 & 1 & 0 \\ 0 & 0 & 1} }{d \in \F_p }
  \]
  が結論できる。

  次に$G/[G,G]$の構造を決定したい。
  \[
  E_1 = \pmat{1 & 1 & 0 \\ 0 & 1 & 0 \\ 0 & 0 & 1}, \quad E_2 =  \pmat{1 & 0 & 0 \\ 0 & 1 & 1 \\ 0 & 0 & 1}
  \]
  とし、$E_1, E_2 \in G/ [G,G]$と見なす。
  \[
  E_1^n = \pmat{1 & n & 0 \\ 0 & 1 & 0 \\ 0 & 0 & 1}, \quad E_2^m =  \pmat{1 & 0 & 0 \\ 0 & 1 & m \\ 0 & 0 & 1}
  \]
  なので、$E_1, E_2$は位数がちょうど$p$である。また、$C = E_1^n = E_2^m$とするとき
  \[
  1 = E_1^n E_2^{-m} = \pmat{ 1 & n & -nm \\ 0 & 1 & -m \\ 0 & 0 & 1}
  \]
  だから$n=m=0$が従う。つまり$\kakko{E_1} \cap \kakko{E_2} = 1$である。$G/[G,G]$はAbel群なので積による準同形
  $
\kakko{E_1} \tm \kakko{E_2} \to G/[G,G]
  $
  がある。これは、$\kakko{E_1} \cap \kakko{E_2} = 1$により単射である。位数$p^2$の有限群の間の単射なので、とくに同型である。よって$G/[G,G] \cong \F_p^2$がわかった。

あとは$\# \Hom(\F_p^2, \C^{\tm})$を求めよう。これは$\# \Hom(\F_p,\C^{\tm})$の$2$乗である。$\# \Hom(\F_p,\C^{\tm}) = p$より求める答えは$p^2$である。

\end{sol}

\newpage

\subsubsection{} %{問6}
\barquo{
$\R^4$の部分空間$M$を
\[
M = \setmid{(x,y,z,w) \in \R^4}{x^2 + y^2 + z^2 + w^2 = 1, \; xy + zw = 0}
\]
で定める。
\begin{description}
  \item[(1)] $M$が$2$次元微分可能多様体になることを示せ。
  \item[(2)] $M$上の関数$f$を
  \[
  f(x,y,z,w) = x
  \]
  で定めるとき、$f$の臨界点をすべて求めよ。ただし、$p \in M$が$f$の臨界点であるとは、$p$における$M$の局所座標$(u,v)$に関して
  \[
  \f{\del f}{\del u}(p) =   \f{\del f}{\del v}(p) = 0
  \]
  となることである。
\end{description}
}
\begin{sol} ${}$
  \begin{description}
    \item[(1)] $F \colon \R^4 \to \R^2$を
      \[
      F(x,y,z,w) = \pmat{x^2 + y^2 + z^2 + w^2 - 1 \\ xy+zw}
      \]
      により定める。$M = F^{-1}(O)$である。$p=(x,y,z,w) \in M$としよう。$p$におけるヤコビアンを計算すると
      \[
      JF_p = \pmat{2x & 2y & 2z & 2w \\ y & x & w & z }
      \]
      である。ここで$p \neq O$より$\rank JF_p \geq 1$である。仮に$\rank JF_p = 1$ならば、$JF_p$の2つの行は1次従属である。よって、$p \neq O$により$(y , x, w,z) = c(x,y,z,w)$なる定数$c \in \R$がある。このとき$xy + zw = c(x^2 + z^2) = 0$となり、$p \neq O$に矛盾。よって$\rank JF_p = 2$である。ゆえに$p$は$F$の正則点であり、$M$は$\R^4$の$2$次元部分多様体。
      $F$は$\bfC^{\infty}$級なので、$M$は微分可能になる。
      \item[(2)] $f \colon M \to \R$の$\R^4$への自然な拡張を$\wt{f}$とする。このとき$p \in M$に対して$T_p M \subset \R^4$と見なせば、$T_p M = \Ker JF_p$であるから、
      \begin{align*}
        \text{$p$が$f$の臨界点} &\iff \rank (df_p \colon T_p M \to \R) < 1 \\
        &\iff \dim \Ker df_p = 2 \\
        &\iff \dim \Ker \pmat{JF_p \\ J\wt{f}_p} = 2 \\
        &\iff \rank \pmat{2x & 2y & 2z & 2w \\ y & x & w & z \\ 1  & 0 & 0 & 0} = 2 \\
        &\iff \rank \pmat{0 & 2y & 2z & 2w \\ 0 & x & w & z \\ 1  & 0 & 0 & 0} = 2
      \end{align*}
      である。いま$p=(x,y,z,w) \in M$が臨界点であったと仮定する。このとき$(x,w,z)$と$(y,z,w)$は1次従属である。よって$(y,w,z) = 0$かまたは、ある$c \in \R$が存在して$(x,w,z)=  c(y,z,w )  $である。$(y,w,z) = 0$なら$p =(\pm 1, 0, 0 ,0)$である。$(x,w,z)=  c(y,z,w )  $なら、$c(y^2 + z^2)=0$より$p=(0, \pm 1, 0, 0)$である。

      逆に$p=(\pm 1, 0, 0 ,0), (0, \pm 1, 0, 0)$ならば$p \in M$であり、$f$の臨界点であることはあきらかなので、臨界点はこれですべて求まったことになる。
  \end{description}

\end{sol}


\newpage

\subsubsection{} %{問7}
\barquo{
$A$を実正方行列、$k$を正の整数とし、$\rk (A^{k+1}) = \rk (A^k)$が成り立つとする。このとき、任意の整数$m \geq k$に対し、$\rk (A^m) = \rk (A^k)$であることを証明せよ。ここで行列$X$に対し、$\rk(X)$は$X$の階数を表す。
}
\begin{proof}
  仮定から、$\Ker A^{k+1}$の次元と$\Ker A^k$の次元は等しい。包含関係があって次元が等しいので、$\Ker A^{k+1} = \Ker A^k$である。ここで$m \geq k +2$に対して$x \in \Ker A^m$と仮定する。そうすると$A^m x = A^{m-k-1} A^{k+1} x$だから$A^{m-k-1} x \in \Ker A^{k+1} = \Ker A^k$である。よって$A^k A^{m-k-1} x = A^{m-1} x = 0$であり、$x \in \Ker A^{m-1}$がわかる。これを帰納的に繰り返して、
  $\Ker A^m \subset \Ker A^{m-1} \subset \cdots \subset \Ker A^k$を得る。逆はあきらかなので$\Ker A^m = \Ker A^k$である。よってとくに階数も等しい。
\end{proof}


\newpage

\section{平成30年度 専門科目}

\subsubsection{} %{問1}
\barquo{
$k$を可換体とする。$k[X,Y]$を$k$上の2変数多項式環として、$f \in k[X,Y]$の零点集合$V(f)$を
\[
V(f) = \setmid{(a,b) \in k \tm k}{f(a,b) = 0}
\]
によって定義する。次の2条件は同値であることを示せ。
\begin{description}
  \item[(1)] $k$は代数的閉体ではない。
  \item[(2)] $V(f) = \{(0,0)\}$となる$f \in k[X,Y]$が存在する。
\end{description}
}
\begin{sol}${}$
\begin{description}
  \item[(1)$\To$(2)] $k$は代数的閉体ではないので、ある1次以上の多項式$g \in k[X]$であって、$k$上根を持たないものが存在する。$n = \rm{dim}\ g$とおいて、
  \[
  f(X,Y)= Y^n g \left(\frac{X}{Y} \right)
  \]
  とおく。別の言い方をすれば$g(X)= X^n + a_{n-1}X^{n-1} + \cdots + a_1X + a_0$とするとき, $f(X,Y)= X^n + a_{n-1}X^{n-1}Y + \cdots + a_1XY^{n-1} + a_0Y^n$である。$X \in k$, $Y \in k \sm \{0\}$に対して$Y^n$と$g(X/Y)$は決して0にならないので、$f(X,Y)=0$となるのは$Y=0$のときだけである。$f(X,0)=X^n$なので、
  $V(f) =\{ (0,0)\}$が成り立つ。
  \item[(2)$\To$(1)] 対偶をとり、$k$が代数閉体であってかつ$V(f) =\{ (0,0)\}$となる$f \in k[X,Y]$が存在すると仮定し矛盾を示そう。このとき$k$は無限体($k$が有限体であっても、アイゼンシュタイン多項式は無限個あるため)であることに注意する。またここではそもそも$k$は零環ではないとして考えていることにも注意する。

  さて$a,b \in k^{\tm}$を任意にとると、$f(a,Y) \in k[Y]$、$f(X,b) \in k[X]$は決して0にならないので、定数でなければならない。
  このとき$f(a,Y)=f(a,b)=f(X,b)$であるので、常にこの2つは一致する。割り算を実行して
  \begin{align*}
    f(X,Y) &= (X-a)g(X,Y) + f(a,Y) \\
    f(X,Y) & =(Y-b)h(X,Y)+f(X,b)
  \end{align*}
  なる$g, h \in k[X,Y]$をとってくる。すると辺々引いて
  \[
  0 = (X-a)g(X,Y) - (Y-b)h(X,Y)
  \]
  が成り立つ。この等式は任意の$a,b \in k^{\times}$について成り立つので、$g=h=0 \in k[X,Y]$が判る。ゆえに$f$は定数となるがこれは矛盾。
  \item[別解] (2)$\To$(1)を示す部分についてはHilbertの零点定理を知っていればすこし議論を省略できる。$k$が代数閉体だと仮定し$V(f) =\{ (0,0)\}$となる$f \in k[X,Y]$が存在するとしよう。$k[X,Y]$はUFDなので、$f$は既約であるとしてよい。すると$(f)$は根基イデアルなのでHilbertの零点定理により$(f) = (X,Y)$である。しかし右辺は単項イデアルではないので矛盾。
\end{description}
\end{sol}


\newpage





\subsubsection{} %{問2}
\barquo{
$p$を素数, $k,m$を正の整数で、$k$と$p^2 - p$は互いに素であるとする。位数$kp^m$の有限群$G$が次の性質を満たす部分群$N,H$をもつとする。
\begin{description}
  \item[(1)] $N$は位数$p^m$の巡回群で$G$の正規部分群である。
  \item[(2)] $H$は位数$k$の群である。
\end{description}
このとき、$G$は$N$と$H$の直積であることを示せ。
}
\begin{sol}
  $H \lhd G$を示せば十分である。(付録の「半直積とGaois群」を参照のこと) $N \lhd G$なので、$H$の共役による$N$への作用$\Phi \colon H \rightarrow \Aut N$を$\Phi_h(q)=hqh^{-1}$により定義できる。$H / \Ker \Phi$は$\Aut N$の部分群とみなせる。
  \begin{align*}
    \#(\Aut N) &= \#((\Z/ p^m \Z)^{\tm}) \\
    &= p^m - p^{m-1}
  \end{align*}
  なので、$\#(H / \Ker \Phi)$は$\# H= k$と$ p^m - p^{m-1}$の両方を割り切る。したがって$\#(H / \Ker \Phi) \leq \gcd (k,p^m - p^{m-1})$であるが、右辺は仮定により1だから$\Phi$は自明な作用であって、$H$の元はすべての$N$の元と可換である。

  よって、$G$の元$g=hq \; (h \in H, q \in N)$と$x \in H$に対して$g^{-1}xg = x^g =x^{hq} = (x^h)^q = x^h \; \in H$だから、$H \lhd G$が言えた。
\end{sol}

\newpage


\subsubsection{} %{問3}
\barquo{
多項式$X^7 - 11$の有利数体$\Q$上の最小分解体を$K \subset \C$とする。このとき、次の問に答えよ。
\begin{description}
  \item[(1)] 拡大次数$[K:\Q]$を求めよ。
  \item[(2)] $\Q$と$K$の間の($\Q$でも$K$でもない)真の中間体の個数を求めよ。
  \item[(3)] 上記(2)の中間体のうち、$\Q$上Galois拡大になるものの個数を求めよ。
\end{description}
}
\begin{sol} ${}$
\begin{description}
  \item[(1)] $\gro = \exp (2\pi \sqrt{-1}/7)$とする。あきらかに$K = \Q(\gro, \sqrt[7]{11})$である。状況を図式で表すと次のようになる。
  \[
  \xymatrix{& K &  \\  \Q(\gro) \ar[ur]^{\leq 7} & & \Q(\sqrt[7]{11}) \ar[ul]_{\leq 6} \\  &  \Q \ar[ul]^{6} \ar[ur]_{7} &
  }
  \]
  円分体の一般論から$6= [\Q(\gro):\Q]$である。また$X^7 - 11$はEisenstein多項式なので既約であり$7=[\Q(\sqrt[7]{11}):\Q]$である。$7$と$6$は互いに素なので$\Q(\gro) \cap \Q(\sqrt[7]{11}) = \Q$である。
  $\Q(\gro) / \Q$はGalois拡大なので、Galois拡大の推進定理により$\Gal(K/ \Q(\sqrt[7]{11})) \cong \Gal(\Q(\gro) / \Q)$であり、とくに$[K: \Q(\sqrt[7]{11})] = [\Q(\gro):\Q] = 6$である。したがって、$[K:\Q]=42$である。
  \item[(2)] $G = \Gal(K/ \Q)$とする。付録「半直積とGalois群」により、$G$は半直積
  \[
\Gal(K/ \Q(\gro)) \rtimes \Gal(K/ \Q(\sqrt[7]{11}))
  \]
  と同型である。素数次数なので$\Gal(K/ \Q(\gro)) = \Z / 7 \Z$であり、
  円分体の一般論から
\[
\Gal(K/ \Q(\sqrt[7]{11})) \cong \Gal(\Q(\gro) / \Q) = (\Z / 7 \Z)^{\tm} = \Z / 6 \Z
\]
  である。つまりともに有限巡回群である。$\grs \in \Gal(K/ \Q(\gro))$を$\grs(\sqrt[7]{11}) = \sqrt[7]{11} \gro$により定め、
  $\tau \in \Gal(K/ \Q(\sqrt[7]{11}))$を$\tau(\gro) = \gro^3$により定める。$\grs$, $\tau$はそれぞれ生成元となる。$\tau \grs \tau^{-1}(\sqrt[7]{11}) = \sqrt[7]{11} \gro^3$より
  $\tau \grs \tau^{-1} = \grs^3$である。したがって次の表示
  \[
  G \cong \setmid{\grs,\tau}{\grs^7 = \tau^6 = 1, \tau \grs \tau^{-1} = \grs^3} \cong \Z / 7 \Z \rtimes \Z / 6 \Z
  \]
  を得る。

  Galoisの基本定理により、$G$の自明でない部分群の個数を求めればよい。そこでまずすべての元の位数を決定する。$\kakko{\grs} \rtimes \kakko{\tau} \to \kakko{\tau}$は群準同型なので、$x = \grs^i \tau^j \in G$の共役は$\grs^{*} \tau^j$という形をしている。具体的には
    \begin{align*}
      \grs x \grs^{-1} &= \grs \grs^i \tau^j \grs^{-1} \\
      &= \grs \grs^i (\tau^j \grs^{-1} \tau^{-j} ) \tau^j \\
      &= \grs \grs^i (\grs^{3^j})^{-1} \tau^j \\
      &= \grs^{1 - 3^j} \grs^i \tau^j \\
      &= \grs^{1 - 3^j} x
    \end{align*}
    である。
  そこで共役元を求めることにより次のような位数の表をつくることができる。
  \begin{center}
  \begin{tabular}{ccc}
   \hline
位数 & 元 & 個数 \\
   \hline \hline
   1 & 1 & 1 \\
   2 & $\grs^i \tau^3 \; (0 \leq i \leq 6)$ &  7 \\
  3 & $\grs^i \tau^2 \; (0 \leq i \leq 6)$, $\grs^i \tau^4 \; (0 \leq i \leq 6)$ &  14  \\
  6 & $\grs^i \tau \; (0 \leq i \leq 6)$, $\grs^i \tau^5 \; (0 \leq i \leq 6)$ & 14 \\
  7 & $\grs^i \; ( 1 \leq i \leq 6)$ & 6
   \end{tabular}
  \end{center}
  次に部分群を列挙する作業に移る。$G$の位数は42なので、自明でない部分群の位数としてありえるのは$2,3,6,7,14,21$である。まず位数2の部分群は位数2の元と同じ数だけあるので、7個である。位数3の部分群は、素数位数なのですべて巡回群であり、生成元はひとつの群に対して2つある。よって位数3の部分群は$14/2 = 7$個ある。

  位数6の部分群$M \subset G$が与えられたとする。このとき次のような各行が完全な可換図式がある。
\[
\xymatrix{
1 \ar[r] & \Z / 7 \Z \ar[r]^-j &  \Z / 7 \Z \rtimes \Z / 6 \Z \ar[r]^-p & \Z / 6 \Z \ar[r] & 1 \\
1 \ar[r] & j^{-1}(M) \ar[r]^-j  \ar[u] & M  \ar[u] \ar[r]^-p & p(M) \ar[r] \ar[u] & 1
}
\]
$j^{-1}(M) = 1$でなくてはならないため、$M \cong p(M)$でありしたがって$M$は巡回群である。位数6の巡回群の生成元はひとつの群に対して2つなので、位数6の部分群は$14/2 = 7$個ある。
  位数7の部分群は、Sylow-7部分群なのですべて共役である。ところが$\kakko{\grs}$は正規部分群だったので、ひとつしかない。
  位数14の部分群は、Sylowの定理より位数2の元と位数7の元で生成される。したがって$\kakko{\grs, \tau^3}$しかない。よって1個。位数21の部分群も、Sylowの定理により位数3の元と位数7の元で生成される。したがって$\kakko{\grs, \tau^2}$しかない。よって1個。以上により、次の表のようになる。
  \begin{center}
  \begin{tabular}{ccc}
   \hline
位数 & 部分群 & 個数 \\
   \hline \hline
   2 & $\kakko{\grs^i \tau^3}  \; (0 \leq i \leq 6)$ &  7 \\
  3 & $\kakko{\grs^i \tau^2} \; (0 \leq i \leq 6)$ &  7  \\
  6 & $\kakko{\grs^i \tau} \; (0 \leq i \leq 6)$ & 7 \\
  7 & $\kakko{\grs}$ & 1 \\
  14 & $\kakko{\grs, \tau^3}$ & 1 \\
  21 & $\kakko{\grs, \tau^2}$ & 1
   \end{tabular}
  \end{center}
  したがって非自明な部分群は$7 + 7 + 7 + 1 + 1 + 1 = 24$個ある。
  \item[(3)] Galoisの基本定理により、$G$の自明でない正規部分群の個数を求めればよい。$x = \grs^i \tau^j \in G$の共役$\grs x \grs^{-1}$は$\grs^{1 - 3^j} x$
    であることを思い出そう。これをみると、位数$2,3,6$の群のなかに正規部分群は存在しない。また、位数7,14,21の群はすべて正規部分群である。よって自明でない正規部分群は3個である。
\end{description}
\end{sol}


\newpage

\section{平成29年度 基礎科目}

\subsubsection{} %{問1}
\barquo{
次の重積分を求めよ。
\[
\iint_D e^{- \max \{ x^2,y^2 \} } \ dx dy
\]
ここで$D=\setmid{(x,y) \in \R^2}{0 \leq x \leq 1, 0 \leq y \leq 1}$とする。
}
\begin{sol}
  $E = \setmid{(x,y) \in D}{x \geq y}$とおく。このとき
  \begin{align*}
    \iint_D e^{- \max \{ x^2,y^2 \} } \ dx dy &= 2 \iint_E e^{- x^2 } \ dx dy \\
    &= 2 \int_0^1 \left(  \int_0^x e^{-x^2} \ dy   \right) \ dx \\
    &= 2 \int_0^1 x e^{-x^2} \ dx \\
    &= \int_0^1 e^{-z} \ dz &(z = x^2 \text{とおいた}) \\
    &= 1 - e^{-1}
  \end{align*}
\end{sol}


\newpage


\subsubsection{} %{問2}
\barquo{
実行列
\[
A= \pmat{
1 & -2 & -1 & 1 & 0 \\
-2 & 5 & 3 & -2 & 1 \\
1& 1& 2 &0 &-1 \\
5&  0&  5&  3&  2 \\
}
\]
について、以下の問に答えよ。
\begin{description}
  \item[(i)] 連立一次方程式
  \[
  A \pmat{
  x_1 \\
  x_2 \\
  x_3 \\
  x_4 \\
  x_5
  }
  = \pmat{ 0\\ 0\\ 0\\ 0 }
  \]
  の解をすべて求めよ。
  \item[(ii)]
  連立一次方程式
  \[
  A \pmat{
  x_1 \\
  x_2 \\
  x_3 \\
  x_4 \\
  x_5
  }
  = \pmat{0 \\ -1 \\ 1\\  c}
  \]
  が解を持つような実数$c$をすべて求めよ。
\end{description}
}
\begin{sol} ${}$
  \begin{description}
    \item[(i)] 行列$A$に行基本変形を繰り返し行っていく。
    \begin{align*}
      A &= \pmat{
      1 & -2 & -1 & 1 & 0 \\
      -2 & 5 & 3 & -2 & 1 \\
      1& 1& 2 &0 &-1 \\
      5&  0&  5&  3&  2 \\
      } \\
      &\sim \pmat{
      1 & -2 & -1 & 1 & 0 \\
      0 & 1 & 1 & 0 & 1 \\
      0& 3& 3 &-1 &-1 \\
      0&  10&  10&  -2&  2 \\
      } & \pmat{ R_1 \\ R_2+2R_1 \\ R_3-R_1 \\ R_4-5R_1} \\
     &\sim \pmat{
      1 & 0 & 1 & 1 & 2 \\
      0 & 1 & 1 & 0 & 1 \\
      0& 0& 0 &-1 &-4 \\
      0&  0&  0&  -2&  -8 \\
      } & \pmat{ R_1+2R_2 \\ R_2 \\ R_3-3R_2  \\ R_4 -10R_2} \\
      &\sim \pmat{
        1 & 0 & 1 & 0 & -2 \\
        0 & 1 & 1 & 0 & 1 \\
        0& 0& 0 &1 &4 \\
        0&  0&  0&  0&  0 \\
        } & \pmat{R_1+R_3 \\ R_2 \\ -R_3 \\ R_4-2R_3 }
    \end{align*}
    したがって、$A\bfx=\bfzero$の解空間は$x_3,x_5 \in \R$で貼られる$2$次元実ベクトル空間
    \[
    S = x_3 \pmat{-1 \\-1 \\1 \\0 \\0 } + x_5 \pmat{2\\ -1\\ 0\\ -4\\ 1}
    \]
    である。
    \item[(ii)] 次の事実に気を付ける。
    \prop{
    $k$は体、$A$は$k$係数の$(n,m)$行列であり$\bfx \in k^m, \bfb \in k^n$であるとする。このとき$\bfx$についての一次方程式$A\bfx = \bfb$が解を持つことと、$\rank A = \rank (A \; \bfb)$は同値。
    }
    \begin{proof}
      まず次は同値である。
      \[
        \exists \bfx \; A\bfx = \bfb \iff \exists \bfx \;  \pmat{A & \bfb} \pmat{\bfx \\ -1} = \bfzero
      \]
      ここで、$\Ker A \to \Ker (A \; \bfb) \st \bfx \mapsto {}^t(\bfx \; 0)$によって$\Ker A$は$\Ker (A \; \bfb)$の部分空間$\Ker (A \; \bfb) \cap \setmid{\bfy \in k^{m+1}}{y_{m+1}=0}$だと思えることに気を付けると
      \begin{align*}
      \exists \bfx \; A\bfx = \bfb &\iff \dim \Ker (A \; \bfb) > \dim  \Ker A \\
        &\iff 0 \leq \rank (A \; \bfb) - \rank A < 1 \\
        &\iff \rank A = \rank (A \; \bfb)
      \end{align*}
      であることがわかる。
    \end{proof}
    (ii)の解答に戻る。$\bfb = {}^t(0 \ -1 \;  1 \; c)$とおく。拡大係数行列$(A \; \bfb)$は行基本変形により
    \[
    (A \; \bfb) \sim \pmat{1& 0& 1&  0& -2& 2 \\ 0 &1& 1& 0& 1& -1 \\ 0& 0& 0& 1& 4& -4 \\ 0& 0& 0& 0& 0& c+2}
    \]
    と変形できる。したがって求める$c$の値は$c=-2$である。
  \end{description}
\end{sol}





\newpage





\subsubsection{} %{問3}
\barquo{
$m,n$を正の整数とし、$A$を複素$(n,m)$行列、$B$を複素$(m,n)$行列とする。複素数$\grl \neq 0$について、以下の問に答えよ。
\begin{description}

  \item[(i)] $\grl$が$BA$の固有値ならば、$\grl$は$AB$の固有値でもあることを示せ。
  \item[(ii)] $\C^m$, $\C^n$の部分空間$V,W$をそれぞれ
\begin{align*}
  V &= \setmid{\bfx \in \C^m}{ \text{ある正の整数$k$に対して} (BA - \grl I_m)^k \bfx = \bfzero \text{が成り立つ} } \\
  W &= \setmid{\bfy \in \C^n}{ \text{ある正の整数$l$に対して} (AB - \grl I_n)^l \bfy = \bfzero \text{が成り立つ} }
\end{align*}
で定める。ただし、$I_m,I_n$は単位行列、$\bfzero$は零ベクトルを表す。このとき、$\dim V = \dim W$であることを示せ。
\end{description}
}
\begin{sol} ${}$
  \begin{description}
    \item[(i)] $BA \bfv = \grl \bfv$なる$\bfv \neq \bfzero$があったとする。このとき
    \begin{align*}
      AB(A \bfv) &= A(BA \bfv) \\
      &= A (\grl \bfv) \\
      &= \grl A \bfv
    \end{align*}
    である。もしも$A \bfv = \bfzero$ならば$\grl \bfv = \bfzero$となり矛盾。したがって$A \bfv \in \C^n$は$AB$の固有ベクトルである。
    \item[(ii)] $M = AB, N = BA$とする。$MA = AN$である。いま$\bfx \in V$とする。ある$k$が存在して$(N - \grl I_m)^k \bfx = \bfzero$である。このとき
    \[
(M - \grl I_n)^k (A \bfx) = A  (N - \grl I_m)^k \bfx = \bfzero
    \]
    であるから$A \bfx \in W$である。したがって行列$A$は線形写像$A \colon V \to W$であるとみなせる。このとき$A$は$V$の定義および$\grl \neq 0$により単射だから、$\dim V \leq \dim W$である。同様にして逆が言えるので$\dim V = \dim W$が従う。
  \end{description}
\end{sol}

\newpage


\subsubsection{} %{問4}
\barquo{
$f$を$I=\setmid{x \in \R}{x \geq 0}$上の実数値連続関数とする。正の整数$n$に対し、$I$上の関数$f_n$を
\[
f_n(x)=f(x+n)
\]
で定める。関数列$\{  f_n \}_{n=1}^{\infty}$が$I$上で一様収束するとき、以下の問に答えよ。
\begin{description}
\item[(i)] $I$上の関数$g$を
\[
g(x) = \lim_{n \to \infty} f_n(x)
\]
で定める。このとき$g$は$I$上で一様連続であることを示せ。
\item[(ii)] $f$は$I$上で一様連続であることを示せ。
\end{description}
}
\begin{sol} 以下$I$上の連続関数$h$に対してその一様ノルムを$\norm{h} = \sup_{x \in I} \abs{h(x)}$とかく。
  \begin{description}
    \item[(i)] 連続関数$f_n$の一様極限なので$g$は連続である。さらに定義より$g(x+1)=g(x)$だから、$g$はコンパクト集合$\R / \Z$上の連続関数であるとみなせ、したがって一様連続である。
    \item[(ii)] $\ve > 0$が与えられたとする。$g$の一様連続性から
    \[
    \forall x,y \in I \; \abs{x-y}< \grd_0 \to \abs{g(x) - g(y) } < \ve
    \]
    なる$\grd_0 > 0$がある。$f_n$は$g$に一様収束するので
    \[
    n \geq N \to \norm{f_n - g} < \ve
    \]
    なる$N \in \Z$がある。このとき
    \[
    \forall x,y \in [N,\infty) \; \abs{x - y } < \grd_0 \to \abs{f(x) - f(y)} \leq 3\ve
    \]
    が成り立つ。なぜなら
    \[
    \abs{f(x) - f(y)}  \leq \abs{f_N(x-N) - g(x-N)} + \abs{g(x)-g(y)} + \abs{g(y-N)-f_N(y-N)}
    \]
    であるから。また$f$は連続なので、コンパクト集合$[0,N]$上ではとくに一様連続である。したがって
    \[
    \forall x,y \in [0,N] \; \abs{x - y } < \grd_1 \to \abs{f(x) - f(y)} \leq \ve
    \]
    なる$\grd_1 > 0$がある。したがって$\grd= \min_i{\grd_i}$とすると
    \[
    \forall x,y \in I \; \abs{x - y } < \grd \to \abs{f(x) - f(y)} \leq 4\ve
    \]
    であり、これで$f$が$I$上一様連続であることがいえた。
  \end{description}
\end{sol}






\newpage



\subsubsection{} %{問5}
\barquo{
$p$を正の実数とし、$f(t)$を$\R$上の実数値連続関数で
\[
\int_0^{\infty} \abs{f(t)} \ dt < \infty
\]
を満たすものとする。このとき$\R$上の常微分方程式
\[
\f{dx}{dt} = -px + f(t)
\]
の任意の解$x(t)$に対し$\lim_{t \to \infty} x(t) = 0$が成り立つことを示せ。
}
\begin{sol}
  任意に$\ve > 0$が与えられたとする。仮定により
  \[
  \int_R^{\infty}  \abs{f(t)} \ dt < \ve
  \]
  となるような$R \geq 0$がある。$x = y e^{-pt}$と置いて変数変換をすると
  \[
  \f{dy}{dt} = e^{pt}f(t)
  \]
  となる。よってある定数$C$により
  \[
  y(t) = \int_0^t f(s)e^{ps} \ ds + C
  \]
と表せる。$C$の値は$t \to \infty$での$x$の振る舞いに関与しないので、はじめから$C=0$と仮定してよい。これにより
\[
x(t) =   \int_0^t f(s)e^{p(s-t)} \ ds
\]
であることがわかる。そこで$M =  \int_0^R \abs{f(s)} \ ds $とおき、$ t > \max \{ R, R+ \f{1}{p} \log \f{M}{\ve} \}$とする。このとき
\begin{align*}
  \abs{x(t)} &\leq \int_0^R \abs{f(s)}e^{p(s-t)} \ ds  + \int_R^t \abs{f(s)}e^{p(s-t)} \ ds  \\
  &\leq e^{p(R-t)} M + \int_R^t \abs{f(s)} \ ds  \\
  &\leq \ve  + \int_R^{\infty} \abs{f(s)}  ds \\
  &\leq 2\ve
\end{align*}
である。よって$\lim_{t \to \infty} x(t) = 0$である。
\end{sol}



\newpage





\subsubsection{} %{問6}
\barquo{
$X,Y$を位相空間とし、直積集合$X \tm Y$を積位相によって位相空間とみなす。写像$f \colon X \tm Y \to Y$を$f(x,y)=y$で定める。$X$がコンパクトならば、$X \tm Y$の任意の閉集合$Z$に対し、$f(Z)$は$Y$の閉集合であることを示せ。
}
\begin{rem}
  $X$がコンパクトという仮定は必要である。例えば、$X=Y=\R$かつ$Z=\setmid{(x,y) \in \R^2}{xy=1}$としてみればわかる。
\end{rem}
\begin{sol}
$Y \sm f(Z)$の元$y$が任意に与えられたとする。このとき$f^{-1}(y) \subset X \tm Y \sm Z$である。ここで$Z$が$X\tm Y$の閉集合という仮定から、$X \tm Y \sm Z \opsub X \tm Y$である。したがって積位相の定義により、ある開集合の族$U_i \subset X$と$V_i \subset Y$であって$X \tm Y \sm Z = \bigcup_{i \in I} U_i \tm V_i$なるものがある。
$f^{-1}(y) = X \tm \{ y \} \cong X$はコンパクトであると仮定したので、ある有限集合$J \subset I$が存在して$X \tm \{y\} = f^{-1}(y) \subset \bigcup_{i \in I} U_i \tm V_i$が成り立つ。

ここで$V = \bigcap_{i \in J} V_i$とおく。$J$は有限集合なので$V$は$Y$の開集合であり、かつ$y$を含む。また$X = \bigcup_{i \in J} U_i$であることより$Z \cap f^{-1}(V) = Z \cap (X \tm V) = Z \cap \bigcup_{i \in J} (U_i \tm V) \subset Z \cap (X \tm Y \sm Z) = \emptyset$となる。
これは$V \cap f(Z) = \emptyset$を意味し、$y$は内点であったことがわかった。よって$f(Z)$は$Y$の閉集合。
\end{sol}


\newpage

\subsubsection{} %{問7}
\barquo{
$n$を正の整数とし、$\R^n$の2点$x=(x_1, \cdots , x_n)$, $y= (y_1, \cdots , y_n)$の距離$d(x,y)$を
\[
d(x,y) = \sqrt{(x_1 - y_1)^2 + \cdots +  (x_n - y_n)^2}
\]
と定める。$\R^n$の空でない部分集合$A$に対し、関数$f \colon \R^n \to \R$を
\[
f(x) = \inf_{z \in A} d(x,z)
\]
で定めるとき、$\R^n$の任意の2点$x,y$に対して$\abs{f(x)-f(y)} \leq d(x,y)$が成り立つことを示せ。
}
\begin{sol}
  $d(x,0) = \abs{x}$と書くことにする。任意に$\ve > 0$が与えられたとしよう。このとき$f$の定義から、$f(y) + \ve > \abs{y-w} \geq f(y)$なる$w \in A$が存在する。このとき$f(x) \leq \abs{x-w}$が成り立つので、
  \begin{align*}
    f(x) - f(y) - \ve &\leq f(x) - \abs{y-w} \\
    &\leq \abs{x-w} - \abs{y-w} \\
    &\leq \abs{x-y}
  \end{align*}
  である。$\ve > 0$は任意だったので、$f(x) - f(y) \leq \abs{x-y}$でなくてはならない。同様にして$f(y) - f(x) \leq \abs{x-y}$がいえるので、示すべきことがいえた。
\end{sol}


\newpage


\section{平成29年度 専門科目}

\subsubsection{} %{問1}
\barquo{
$G$を有限群とする。$G$の自己準同形全体のなす群を$\Aut (G)$とおく。また、$G$および$\Aut (G)$の位数をそれぞれ$a= |G|$, $b= |\Aut G|$とおく。以下の問に答えよ。
\begin{description}
  \item[(i)] $b=1$のとき、$G$は自明群であるか、または$\Z / 2 \Z$と同型であることを示せ。
  \item[(ii)] $a$が奇数で$b=2$となるような$G$をすべて求めよ。
\end{description}
}
\begin{sol} この解答では集合の元の個数を$\#$で表記する。
   \begin{description}
     \item[(i)] 群準同形$\phi \colon G \to \Aut G$を$\phi_g(x)=gxg^{-1}$により定める。$\# \Aut G = 1$という仮定より、$G = \Ker \phi = Z(G)$である。したがって$G$はAbel群。よって有限生成Abel群の基本定理により、ある素数の集合$M \subset \Z$と写像$I \colon M \to P(\Z)$が存在して
     \[
     G = \bigoplus_{p \in M} \bigoplus_{i \in I(p)} \Z / p^{e_i} \Z
     \]
     と表すことができる。このとき乗法群$(\prod_{p, i } \Z / p^{e_i} \Z)^{\tm}$は$\Aut G$の部分群とみなせるので
     \[
     \prod_{p, i} p^{e_i - 1}(p -1) = 1
     \]
     である。したがって$M=\{ 2 \}$である。また任意の$i \in I(2)$に対して$e_i = 1$である。よって$G = (\Z / 2 \Z)^n$と表せるが、$\Aut G = 1$という仮定から$n=1$でなくてはならない。($n \geq 2$なら、たとえば順番を入れ替える写像などがある)
     \item[(ii)] 仮定から$G/ Z(G) = G/ \Ker \phi $の位数は2以下である。写像$f \colon G \to G$を$f(x) = x^2$で定義する。$x,y \in G$に対して、もしも$x \in Z(G)$または$y \in Z(G)$ならば$f(xy)=f(x)f(y)$である。また$x , y \in G \sm Z(G)$であれば、$xy \in Z(G)$なので$f(xy)=xy(xy) =f(x)f(y)$である。したがって$f$は群準同形となる。$\# G$は奇数なので$f$は単射であり、位数の考察から同型となる。このことから結局$G= Z(G)$であり、$G$はAbel群である。ゆえに有限生成Abel群の基本定理から
     \[
        G = \bigoplus_{p \in M} \bigoplus_{i \in I(p)} \Z / p^{e_i} \Z
     \]
     と表すことができる。このとき乗法群$(\prod_{p, i } \Z / p^{e_i} \Z)^{\tm}$は$\Aut G$の部分群とみなせるので
     \[
     \prod_{p, i} p^{e_i - 1}(p -1) \leq 2
     \]
     である。$p$としては$3$以上のものしか現れないから、$M = \{ 3 \}$である。また$\# I(3) =1$かつ$i \in I(3)$に対して$e_i = 1$であることもわかる。したがって$G = \Z / 3\Z$である。
   \end{description}
\end{sol}

\newpage

\subsubsection{} %{問2}
\barquo{
$n$は$2$以上の整数とし、$\zeta = e^{2\pi \I /n}$を$1$の原始$n$乗根とする。$\C[X,Y]$は変数$X,Y$に関する複素数係数の$2$変数多項式環とする。
\[
R = \setmid{f(X,Y) \in \C[X,Y] }{ f(\zeta X,\zeta Y) = f(X,Y)}
\]
とおく。以下の問に答えよ。
\begin{description}
  \item[(i)] $\C$代数として$R$は$n+1$個の元$X^n, X^{n-1}Y , \cdots , XY^{n-1}, Y^n$で生成されることを示せ。
  \item[(ii)] 複素数$a,b,c,d$に対し$m_{a,b} = (X-a,Y-b)$, $m_{c,d} = (X-c,Y-d)$を$\C[X,Y]$のイデアルとする。$m_{a,b} \cap R = m_{c,d} \cap R$が成り立つための$a,b,c,d$に関する必要十分条件を求めよ。
\end{description}
}
\begin{sol} ${}$
  \begin{description}
    \item[(i)] $X^n, X^{n-1}Y , \cdots , XY^{n-1}, Y^n$で$\C$上生成される環を$R'$とおく。$f \in R$が与えられたとする。$f$をゼロでない$d$次斉次元$f_d$の和として$f = \sum_{d \in I} f_d$と書く。このとき仮定から$0 = f(\zeta X,\zeta Y) - f(X,Y) = \sum_{d \in I} (\zeta^d - 1) f_d(X,Y) $である。したがって$I$の元はすべて$n$の倍数である。
    これはつまり$f \in R'$を意味する。よって$R \subset R'$である。逆は明らかだから$R = R'$がいえた。
    \item[(ii)] $1$の$n$乗根全体がなす位数$n$の巡回群を$G$と書くことにする。このとき、$(z a, z b) = (c,d)$なる$z \in G$が存在すれば$m_{a,b} \cap R = m_{c,d} \cap R$が成り立つことはあきらか。逆を示そう。

    $m_{a,b} \cap R = m_{c,d} \cap R$が成り立ったと仮定する。このとき$X^n - a^n$と$Y^n - b^n$はともに$m_{a,b} \cap R = m_{c,d} \cap R$の元である。よって$c^n - a^n = d^n -b^n = 0$であり、ある$z,w \in G$が存在して$c = za$, $d = wb$が成り立つ。ここでさらに$(bX - aY)^n$も$m_{a,b} \cap R = m_{c,d} \cap R$の元だから、$\C$は整域なので$bc - ad=0$である。
    よって$(z - w)ab=0$である。このとき$ab=0$または$z-w=0$であるが、いずれにせよ$(z a, z b) = (c,d)$なる$z \in G$が存在することには違いないので、示すべきことがいえた。
  \end{description}
\end{sol}

\newpage


\subsubsection{} %{問3}
\barquo{
以下の問に答えよ。
\begin{description}
  \item[(i)] $S_5$を文字$1,2,3,4,5$に関する対称群とする。$S_3$を文字$1,2,3$に関する対称群とし、$S_3$を$S_5$の部分群とみなす。$\grs = (1 \ 2 \ 3) \in S_5$を長さ$3$の巡回置換とし、$\grs$で生成された$S_5$の部分群を$G = \kakko{ \grs}$とおく。
  $\tau = (4 \ 5) \in S_5$を互換とし、$\tau$で生成された$S_5$の部分群を$H = \kakko{\tau}$とおく。$S_5$の部分群$G$の正規化群を
  \[
  N_{S_5}(G) = \setmid{\eta \in S_5}{ \eta G \eta^{-1} = G}
  \]
  で定める。このとき、$N_{S_5}(G) = S_3 \tm H$であることを示せ。
  \item[(ii)] $f(X)$は$\Q$係数の5次の多項式とする。$K \subset \C$を$f(X)$の$\Q$上の最小分解体とする。$K$は次の条件$(*)$を満たすとする。

  $(*)$ $[K:F]=3$となる$K$の部分体$F$がただ一つ存在する。

  このとき、$f(X)$は$\Q$係数の3次の既約多項式で割り切れることを示せ。
\end{description}
}
\begin{sol} ${}$
  \begin{description}
    \item[(i)] $S_3$の元と$H$の元は互いに可換なので、積をとる写像$S_3 \tm H \to S_5$は準同型となる。$G \lhd S_3$より、$(s,t) \in S_3 \tm H$としたとき
    \[
    (st)\grs (st)^{-1} = s\grs s^{-1} \in G
    \]
    だから積$st$は$N_{S_5}(G)$に含まれる。よって準同型$\vp \colon S_3 \tm H \to N_{S_5}(G)$が構成できたことになる。$S_3 \cap H = 1$より$\vp$は単射である。全射であることを示そう。

    $h \in N_{S_5}(G)$が与えられたとする。このとき定義から$h \grs h^{-1} \in G$である。よって$x \in \{ 4,5\}$への作用を考えると$h \grs h^{-1}(x) = x$, つまり$\grs h^{-1} (x) = h^{-1}(x)$がわかる。$\grs$が固定するのは$\{ 4, 5\}$の元だけなので
    $h^{-1}(x) \in \{ 4, 5\}$である。まとめると、任意の$h \in N_{S_5}(G)$に対して$h = \grs_0 \tau_0$なる$\grs_0 \in S_3$と$\tau_0 \in H$があることが判ったことになり、したがって$\vp$は全射、ゆえに同型である。
\item[(ii)] Galoisの基本定理により、$F \mapsto \Gal(K/F)$で与えられる対応
\[
\{ \text{$K/\Q$の部分体} \} \to \{ \text{ $\grG := \Gal(K/\Q)$の部分群 } \}
\]
は全単射である。ゆえに$\grG$は位数3の部分群をただひとつしかもたない。とくに$\Gal(K/F)$の$\grG$での共役はただひとつなので$\Gal(K/F) \lhd \grG$である。$G = \Gal(K/F)$とおく。$f$の根を$\gra_1, \cdots , \gra_5$とすると$\grG$は$\{ \gra_1, \cdots , \gra_5 \}$への作用により$5$次対称群$S_5$の部分群とみなせる。
$G \cong \Z / 3 \Z$なので必要ならば番号を付けなおすことにより$G$は$\grs = (1 \ 2 \ 3)$で生成されているとしてよい。いま$G \lhd \grG$により$\grG \subset N_{S_5}(G)$である。(i)により$N_{S_5}(G) \cong S_3 \tm H$であるので、$\grG \subset  S_3 \tm H$とみなせる。

$f$は5次多項式であるので、$f$の$\Q[X]$における既約分解の様相には次の可能性がある。
\begin{description}
  \item[(1)] $f$は既約
  \item[(2)] $f = (\text{4次式}) \tm  (\text{1次式})$
  \item[(3)] $f = (\text{3次式}) \tm  (\text{2次式})$
  \item[(4)] $f = (\text{3次式}) \tm  (\text{1次式}) \tm  (\text{1次式})$
  \item[(5)] $f = (\text{2次式}) \tm  (\text{2次式}) \tm  (\text{1次式})$
  \item[(6)] $f = (\text{2次式}) \tm  (\text{1次式}) \tm  (\text{1次式}) \tm  (\text{1次式})$
  \item[(7)] $f = (\text{1次式}) \tm  (\text{1次式}) \tm  (\text{1次式}) \tm  (\text{1次式}) \tm  (\text{1次式})$
\end{description}
ここで(5),(6),(7)は条件$(*)$を満たさないので却下される。(1),(2)の場合$\grG$は集合$\{ 1,2,3,4,5 \}$の位数$4$以上の部分集合に推移的に作用しなければならないが、これは$\grG \subset  S_3 \tm H$に反する。したがって残る可能性は(3),(4)のみであり、示すべきことがいえた。
  \end{description}
\end{sol}


\newpage

\bfsection{平成28年度 基礎科目 I}

\bfsubsection{問1}
\barquo{
線形写像$f \colon \R^4 \to \R^3$を行列
\[
A = \pmat{
2 &1 &1& 0 \\
4 &0 &2 &1 \\
2 &-1& 1 &2
}
\]
を用いて$f(x) = Ax \ (x \in \R^4)$として定める。$V$を3つのベクトル
\[
\pmat{ 1\\ 2\\ -2\\ -4 }, \pmat{0\\ -2\\ 1\\ 3 }, \pmat{ 1\\ 1\\ 0\\ -4 }
\]
が張る$\R^4$の部分空間としたとき、$f$の$V$への制限$g = f|_V \colon V \to \R^3$の階数を求めよ。ただし、$g$の階数とは、$g(V)$の次元のこととする。
}
\begin{sol}
  与えられたベクトルをそれぞれ$v_1, v_2, v_3$とする。このとき$B=(gv_1, gv_2, gv_3)$であるとすると$B$は基本変形で
  \[
  B = \pmat{
2& -1& 3 \\
-4& 5& 0 \\
-10&  8&  -7
  }
  \sim
  \pmat{
  2 &-1 &3 \\
  0 &3 &6 \\
  0 &0& 2
  }
  \]
  と変形できる。したがって$\rank B = 3$であり、$g$の階数は$3$である。
\end{sol}




\newpage


\bfsubsection{問2}
\barquo{
$a$を実数とする。$3$次正方行列
\[
A = \pmat{
a& 1 &2 \\
0 &1 &0 \\
-2 &0& 0
}
\]
について、以下の問に答えよ。
\begin{description}
  \item[(i)] 行列$A$の固有値を求めよ。
  \item[(ii)] 行列$A$が対角化可能となる実数$a$をすべて求めよ。ただし、$A$が対角化可能であるとは、複素正則行列$P$で$P^{-1}AP$が対角行列となるものが存在することをいう。
\end{description}
}
\begin{sol} ${}$
  \begin{description}
\item[(i)] 与えられた$A$を変数$a$を明示して$A(a)$と書くことにしよう。そうして$A(a)$の固有多項式を$\Phi_a(\grl)$で書くことにする。このとき
\[
\Phi_a(\grl) = \det(\grl I - A(a)) = (\grl - 1)(\grl^2 - a \grl + 4)
\]
である。だから固有値は$1, (a \pm \sqrt{a^2 - 16})/2$である。
\item[(ii)] $\grl^2 - a \grl + 4$が$1$を根に持つのは$a=5$のとき。また重根を持つのは$a=\pm 4$のとき。だから$a$が$5, \pm 4$のいずれでもないときには$A(a)$は異なる$3$つの固有値を持ち、したがって対角化可能である。では$a =5, \pm 4 $のときはどうか。

行列$A$が対角化可能であることと、$A$の各固有値についての固有空間の直和が全体と一致することは同値であることに注意する。固有値$\grl$に属する固有空間を$V(\grl)$と表すことにする。

$a=4$のとき。$\Phi_4 (\grl)= (\grl-1)(\grl-2)^2 $である。固有空間$V(2)$の次元は線形写像$2 I - A(4)$の核の次元だから
\[
2 I - A(4) = \pmat{
-2& -1& -2 \\
0 &-3& 0 \\
2 &0& 2
}
\sim
\pmat{
-2 &-1& -2 \\
0 &0& 0 \\
0& -1& 0
}
\]
より$\dim V(2) = 3 - 2 = 1$である。よって$A(4)$は対角化できない。$a=-4, 5$についても同様の考察により$A(a)$が対角化できないことが判るが、詳細は省略する。これですべての$a$について$A(a)$の対角化可能性が求まった。
  \end{description}
\end{sol}

\newpage

\bfsubsection{問3}
\barquo{
次の極限値を求めよ。
\[
\lim_{n \to \infty} \int_0^{\infty} e^{-x}(nx - [nx]) \ dx
\]
ただし、$n$は自然数とし、$[y]$は$y$を超えない最大の整数を表す。
}
\begin{sol}
  $1$以上の$n \in \Z$を固定すると、任意の$x \in \R_{\geq 0}$に対して
  \[
  \f{k}{n} \leq x < \f{k+1}{n}
  \]
  なる$k \in \Z$が一意的にある。このとき$nx - [nx] = nx - k$である。そこで$M_n = \int_0^{\infty} e^{-x}(nx - [nx]) \ dx$とおくと
  \[
  M_n = \sum_{k \geq 0} \int_{k/n}^{(k+1)/n} e^{-x}(nx - k) \ dx
  \]
  である。いったん$n$は固定して$F_k = \int_{k/n}^{(k+1)/n} e^{-x}(nx - k) \ dx$とおこう。部分積分を用いることにより
  \[
  F_k = -(n+1) e^{-\f{k+1}{n}} + n e^{- \f{k}{n}}
  \]
  が示せる。$\zeta = e^{-\f{1}{n}}$とおけば$F_k = -(n+1)\zeta^{k+1} + n \zeta^k$である。したがって
  \begin{align*}
    M_n = (n - (n+1)\zeta) \sum_{k \geq 0} \zeta^k = \f{n - (n+1)\zeta}{1 - \zeta}
  \end{align*}
  を得る。あとは次のように式変形を行えばよい。
  \begin{align*}
\lim_{n \to \infty} M_n &= \lim_{n \to \infty}  \f{n - (n+1)e^{-1/n} }{1 - e^{-1/n}} \\
&= \lim_{n \to \infty}  \f{ne^{1/n} - (n+1) }{e^{1/n} - 1} \\
&= \lim_{h \to 0}  \f{ h^{-1} e^{h} - (h^{-1}+1) }{e^{h} - 1} \\
&= \lim_{h \to 0}  \f{ e^{h} - (1 + h) }{ h(e^{h} - 1) } \\
&= \lim_{h \to 0}  \f{h}{ e^{h} - 1 } \f{ e^h - (1+h) }{ h^2 }
  \end{align*}
そうすると$e^h = 1 + h + h^2/2 + O(h^3)$より$\lim_{n \to \infty} M_n = 1/2$が結論できる。
\end{sol}

\newpage

\bfsubsection{問4}
\barquo{
$\R^2$で定義された関数
\[
f(x,y) = \f{ xy(xy+4) }{ x^2 + y^2  + 1 }
\]
の最大値および最小値のそれぞれについて、存在するなら求め、存在しないならそのことを示せ。
}
\begin{sol}
  $x = r \cos \grt$, $y = r \sin \grt$とおくと
  \[
  f(x,y) = \f{r^4 \sin^2 2\grt  + 8 r^2 \sin 2\grt }{4(r^2 +1)}
  \]
  である。さらに$t = \sin 2 \grt$とおけば次のように変形できる。
  \[
  f(x,y) = \f{ r^4 t^2 + 8 r^2 t}{4(r^2 + 1)}
  \]
  したがって、右辺の関数を$g(r,t)$とおいて$0 \leq r, -1 \leq t \leq 1$における$g$の最大値と最小値を求めればよい。最大値の方はすぐにわかる。$t=1$としてみると$g(r,1)=(r^4 + 8r^2)/4(r^2 +1)$であり、あきらかに$\lim_{r \to \infty} g(r,1)=\infty$なので$g$に最大値はない。

  では最小値はどうか。$g$を$t$について平方完成すると
  \[
  g(r,t) = \f{r^4 }{4(r^2 + 1)} \left( \left(t+ \f{4}{r^2} \right) - \f{16}{r^4} \right)
  \]
  である。$h(t)= (t+ 4/r^2) - 16/r^4$とおく。$h(t)$のグラフは、軸が直線$t = -4 / r^2$であるような下に凸な放物線である。そこで軸が$-1 \leq t \leq 1$に入るかどうかで場合分けをして
  \[
  \min_{-1 \leq t \leq 1} h(t) = \begin{cases}
  h(-4/r^2) = -16/r^4 &(r \geq 2) \\
  h(-1)= - 8/r^2 + 1 &(0 \leq r \leq 2)
\end{cases}
  \]
  を得る。ゆえに
  \[
  \min_{r \geq 0, -1 \leq t \leq 1} g(r,t) = \min \left\{ \min_{r \geq 2} \f{-4}{r^2 +1} ,\; \min_{0 \leq r \leq 2} \f{1}{4} \left( r^2 - 9 + \f{9}{r^2+1} \right) \right\}
  \]
  である。$k(r) =  r^2 - 9 + 9/(r^2+1)$とおいて微分すると$k(r)$の$0 \leq r \leq 2$での最小値は$k(\sqrt{2}) = -4$であることがわかる。あきらかに$\min_{r \geq 2} -4/(r^2 +1) = -4/5$だから、求める最小値は$-1$である。
\end{sol}


\newpage

\bfsection{平成28年度 基礎科目 II}

\bfsubsection{問1}
\barquo{
次の積分が収束するような実数$\gra$の範囲を求めよ。
\[
\iint_D \f{dx \ dy}{(x^2 + y^2)^{\gra} }
\]
ただし、$D=\setmid{(x,y) \in \R^2}{- \infty < x < \infty, 0 < y < 1}$とする。
}
\begin{sol}
与えられた積分を$I(\gra)$と略記する。極座標変換を行うと次の形になる。
\begin{align*}
  I(\gra) &= 2 \iint_{0 < y < 1, 0 \leq x} \f{dx \ dy}{(x^2 + y^2)^{\gra} } \\
  &= 2 \int_0^{\pi /2} \ d\grt \int_0^{1/ \sin \grt} r^{1 - 2 \gra} \ dr
\end{align*}
ここでもし$\gra = 1$ならば
\[
I(1) \geq 2 \int_0^{\pi /2} \ d\grt \int_0^1 \f{dr}{r} = \infty
\]
より発散する。そこで$\gra \neq 1$と仮定して先に進むと、次のようになる。
\[
I(\gra) = \f{1}{1- \gra} \int_0^{\pi /2} (\sin \grt)^{2\gra - 2} - \lim_{\ve \to 0} \ve^{2 - 2\gra} \ d\grt
\]
$\gra > 1$ならこれは発散する。そこで$\gra < 1$と仮定して先へ進むと、次の形に帰着する。
\begin{align*}
  I(\gra) &=  \f{1}{1- \gra} \int_0^{\pi /2}  (\sin \grt)^{2\gra - 2} \ d\grt \\
  &= \f{1}{1- \gra} \int_0^{\pi /2}  \left( \f{\sin \grt}{\grt} \right)^{2\gra - 2} \cdot \grt^{2\gra - 2} \ d\grt \\
\end{align*}
$\sin \grt / \grt$は$[0, \pi/2]$上の連続関数であり、$0$より大きい最小値と最大値を持つ。よって収束には関与しないので、$\gra < 1$のとき
$I(\gra)$が収束することは$\gra > 1/2$と同値であることがわかる。つまり$I(\gra)$は、$\gra \leq 1/2$または$1 \leq \gra$なら無限大に発散、$1/2 < \gra < 1$なら収束するということが結論できたことになる。
\end{sol}

\newpage


\bfsubsection{問2}
\barquo{
$A$と$B$を複素$3$次正方行列とする。$A$の最小多項式は$x^3-1$, $B$の最小多項式は$(x-1)^3$とする。このとき
\[
AB \neq BA
\]
となることを示せ。
}
\begin{sol}
  行列$M \in M(3,\C)$の固有値$\grl$に属する固有空間を$E(\grl, M)$と書くことにする。仮定より、$A$の固有値は$x^2 + x + 1$の根のひとつを$\gro$として、$1,\gro, \gro^2$の3つである。もしも$AB =BA$ならば、$v \in E(\grl,A)$に対して
  \[
AB v = BA v = B(\grl v) = \grl (B v)
  \]
  であるから$Bv \in E(\grl,A)$である。つまり$B$を写像$B \colon E(\grl,A) \to E(\grl,A)$とみなせる。

  ここで$e_i \in E(\gro^i,A) \sm \{ 0 \} \; (0 \leq i \leq 2)$としよう。$e_i$はそれぞ$1$次元ベクトル空間である$E(\gro^i,A)$の基底となる。ゆえに$B e_i = \grl_i e_i$となる$\grl_i$が存在することになる。つまり$e_i$は$B$の固有ベクトルである。$e_i$は$\C^3$全体を張るので、とくに$B$は対角化可能となるが、これは$B$の最小多項式が重根を持つことに矛盾。よって$AB \neq BA$でなくてはならない。
\end{sol}
\newpage


\bfsubsection{問3}
\barquo{
複素関数$f(z)$は$z=0$の近傍で正則な関数で$f(z)e^{f(z)} = z$をみたすとする。
以下の問に答えよ。
\begin{description}
  \item[(i)] 非負整数$n$と十分小さい正数$\ve$に対して次の式が成り立つことを示せ。
  \[
  \f{ f^{(n)}(0) }{ n!} = \f{1}{2\pi i} \int_{C_{\ve}} \f{1+u}{e^{nu} u^n} \ du
  \]
  ここで積分路$C_{\ve}$は円周$C_{\ve} = \setmid{u \in \C}{ \abs{u} = \ve} $を正の向きに一周するものとする。
  \item[(ii)] $f(z)$の$z=0$におけるベキ級数展開を求め、その収束半径を求めよ。
\end{description}
}
\begin{sol} ${}$
  \begin{description}
    \item[(i)] 仮定の式$f(z)e^{f(z)} = z$の両辺を微分して$(1 + f)f' e^f = 1$を得る。とくに$f'$は零点を持たない。したがって逆関数定理により$f$は$0$の十分小さな近傍$U$に制限すれば像への同相写像となる。よって$u = f(z)$と変数変換することができて
    \begin{align*}
      \f{1 + u}{ (e^u u)^n} \ du &= \f{(1 + f(z))f'(z) }{z^n} \ dz \\
      &= \f{ e^{-f(z)} }{z^n} \ dz \\
      &= \f{f(z)}{z^{n+1}} \ dz
    \end{align*}
    であることがわかる。したがってCauchyの積分公式から、十分小さい$\ve$をとればすべての$n$に対して
    \[
\f{ f^{(n)}(0) }{ n!} =  \f{1}{2\pi i} \int_{f^{-1}(C_{\ve})} \f{f(z)}{z^{n+1}} \ dz  = \f{1}{2\pi i} \int_{C_{\ve}} \f{1+u}{e^{nu} u^n} \ du
    \]
    が成り立つ。
    \item[(ii)] ベキ級数展開すると
    \begin{align*}
      (1 + z)e^{-nz} &= (1 + z) \sum_{k=0}^{\infty} \f{ (-n)^k }{k!} z^k \\
      &= 1 + \sum_{k=1}^{\infty} \left( \f{ (-n)^{k-1} }{ (k-1)! } + \f{ (-n)^k }{k!} \right) z^{k}
    \end{align*}
    である。よって$z=0$のまわりでのLaurent展開は
    \[
    \f{ (1 + z)e^{-nz} }{z^n} = \f{1}{z^n} + \sum_{k=1}^{\infty} \left( \f{ (-n)^{k-1} }{ (k-1)! } + \f{ (-n)^k }{k!} \right) z^{n-k}
    \]
    であることがわかる。ゆえに留数定理から
    \[
    \f{1}{2\pi i} \int_{C_{\ve}} \f{1+u}{e^{nu} u^n} \ du = \begin{cases}
    0 &(n=0) \\
    (-n)^{n-1}/ n! &(n \geq 1)
  \end{cases}
    \]
    が従う。よって$f$のベキ級数展開を$f(z) = \sum_{n \geq 1} a_n z^n$とすると$a_n=(-n)^{n-1}/n!$であり、$\lim_{n \to \infty} \abs{a_{n+1} / a_n} = e$だから収束半径は$1/e$である。
  \end{description}
\end{sol}

\newpage

\bfsubsection{問4}
\barquo{
正則な複素2次正方行列のなす群を$GL_2(\C)$とおく。行列
\[
A = \pmat{0 &-1 \\ 1 & 1}, \quad B = \pmat{0& 1\\ 1& 0}
\]
で生成される$GL_2(\C)$の部分群$G$について、以下の問に答えよ。
\begin{description}
  \item[(i)] 群$G$の位数を求めよ。
  \item[(ii)] 群$G$の中心の位数を求めよ。ただし、$G$の中心とは、$G$のすべての元と可換な元全体のなす$G$の部分群のことである。
  \item[(iii)] 群$G$に含まれる位数$2$の元の個数を求めよ。
\end{description}
}
\begin{sol} $I$は単位行列とする。群の特定の部分集合$S$で生成される部分群を$\kakko{S}$で書く。
  \begin{description}
    \item[(i)] 計算により$\kakko{A} \cong \Z / 6 \Z$, $B^2 = I$, $BAB^{-1} = A^{-1}$がわかる。ゆえに$BA = (BAB)B = A^5 B$だから、$G$の元はすべて$A^i B^j \; (0 \leq i \leq 5, 0\leq j \leq 1)$という形をしている。よって$\# G \leq 12$である。

    逆を考察しよう。$BAB^{-1} = A^{-1}$より、$\kakko{A} \lhd G$である。$A^3$は$\kakko{A}$の元で位数が$2$であるような唯一の元なので$BA^3B = A^3$である。つまり$A^3$は$G$の中心$Z(G)$の元である。したがって$\kakko{A^3, B} = \{ I, A^3, B , A^3B\}$であり$G$は位数$4$の部分群を持つ。$G$が位数$3$の部分群を持つことは
    $A^2$の位数が$3$であることからあきらかなので、$\# G \geq 12$を得る。つまり$\# G =12$ということである。
    \item[(ii)] $Z = A^i B^j \; (0 \leq i \leq 5, 0\leq j \leq 1)$が$Z(G)$の元だったとする。このとき
    \begin{align*}
      AZA^{-1}Z^{-1} &= A^{i+1} B^j A^{-1} B^{-j} A^{-i} \\
      &= A^{i+1} (B^j A B^{-j})^{-1} A^{-i} \\
      &= A^{i+1} (A^{1-2j})^{-1} A^{-i} \\
      &= A^{2j}
    \end{align*}
    だから$j=0$でなくてはならない。また
    \begin{align*}
      BZBZ^{-1} &= BA^i B A^{-1} \\
      &= A^{-i } A^{-i} \\
      &= A^{-2i}
    \end{align*}
    より$i=0,3$でなくてはならない。よって$Z(G)= \{ I, A^3 \}$である。これで$\# Z(G)=2$が示せた。
    \item[(iii)] 各々の元の共役類を求めて位数の表を作ると次のようになる。
    \begin{center}
    \begin{tabular}{ccc}
     \hline
  位数 & 元 & 個数 \\
     \hline \hline
     1 & I & 1 \\
     2 & $B,A^2B, A^4B, A^3, AB, A^3B, A^5B$ &  7 \\
    3 & $A^2, A^4$ &  2  \\
    6 & $A, A^5$ & 2
     \end{tabular}
    \end{center}
    よって位数$2$の元の数は$7$個。
  \end{description}
\end{sol}
\newpage


\bfsubsection{問5}
\barquo{
3次元微分多様体$M  = \setmid{(x,y,z,w) \in \R^4}{xy-z^2 = w}$から$\R^3$への写像$f=(f_1,f_2,f_3) \colon M \to \R^3$を
\[
f(x,y,z,w) = (x+y, z, w)
\]
により定める。以下の問に答えよ。
\begin{description}
  \item[(i)] $f$の臨界点の集合$C$を求めよ。ただし$p \in M$が$f$の臨界点であるとは、$p$のまわりの$M$の座標系$(u_1,u_2,u_3)$に関する$f$のヤコビ行列
  \[
  \left( \f{ \del f_i }{\del u_j} \right)_{1 \leq i,j \leq 3}
  \]
  が正則でないことである。
  \item[(ii)] $C$が$M$の部分多様体になることを証明せよ。
\end{description}
}
\begin{sol} ${}$
  \begin{description}
    \item[(i)] $g \colon \R^4 \to \R$を$g(x,y,z,w) = xy - z^2 -w$により定める。このとき$M = g^{-1}(0)$であり、$f$の微分$df_p \colon T_p M \to \R^3$は$f$の$\R^4$への延長のヤコビアン$Jf_p \colon \R^4 \to \R^3$の$\Ker Jg_p$への制限だとみなせる。したがって$p \in M$に対して
    \begin{align*}
      p \in C &\iff \rank df_p \leq 2 \\
      &\iff \dim \Ker df_p \geq 1 \\
      &\iff \dim (\Ker Jf_p \cap \Ker Jg_p) \geq 1 \\
      & \iff \dim \Ker \pmat{ Jg_p \\ JF_p} \geq 1 \\
      &\iff \rank \pmat{ Jg_p \\ JF_p} \leq 3 \\
      &\iff \rank \pmat{ y &x& -2z& -1 \\ 1& 1& 0& 0 \\ 0& 0& 1& 0 \\ 0& 0& 0& 1} \leq 3 \\
      &\iff x=y
    \end{align*}
    だから$C =\setmid{(x,y,z,w) \in \R^4}{xy-z^2 = w, x=y}$である。
    \item[(ii)] $h \colon M \to \R$を$h(x,y,z,w ) = x-y$で定める。任意の点$p \in h^{-1}(0)$が正則点であることをいえばよい。計算すると
    \begin{align*}
      \rank dh_p &= 3 - \dim \Ker dh_p \\
      &= 3- \dim \Ker \pmat{Jg_p \\ Jh_p} \\
      &= \rank \pmat{Jg_p \\ Jh_p} - 1 \\
      %&= \rank \pmat{ y &x &-2z &-1 \\ 1 &-1 &0 &0 } - 1 \\ %レイアウトのために省略
      &= 1
    \end{align*}
    である。よって$0$は$h$の正則値であり、示すべきことがいえた。
  \end{description}
\end{sol}


\newpage

\section{平成28年度 専門科目}

\subsubsection{} %{問1}
\barquo{
有理数係数の既約多項式$f(x) = x^3 + ax + b$を考え、$K \subset \C$を$f(x)$の最小分解体とする。$a > 0$のとき、$K/\Q$のGalois群が3次対称群と同型であることを示せ。
}
\begin{sol}
  $\lim_{x \to + \infty} f(x) = + \infty$, $\lim_{x \to - \infty} f(x) = - \infty$より、$f$は連続なので$f(\beta_1) = 0$なる$\beta_1 \in \R$がある。$f'(x) = 3x^2 + a > 0$より$f$は単調増加なので$f$の実根は$\beta_1$のみである。そこで$f$の残りの根を
  $\beta_2, \beta_3$とすると$\beta_2 = \ol{\beta_3}$である。いま$G = \Gal(K/\Q)$を根への作用により3次対称群$S_3$の部分群とみなす。$G$は複素共役をとる写像を含むので、$\# G$は偶数でなくてはならない。また$f$は既約と仮定したので、$G$は集合$\{ 1,2,3\}$に推移的に作用するはずであり、とくに$\# G$は$3$の倍数である。
  よって$\# G$は$6$の倍数となるが、$\# S_3 = 6$なので$G \cong S_3$となるしかない。
\end{sol}

\newpage

\subsubsection{} %{問2}
\barquo{
$K$を標数が2でない代数的閉体とし、$K$の元$a$に対して、2変数多項式環$K[X,Y]$の剰余環
\[
R_a = K[X,Y] / (X^2 - Y^2 - X - Y -a)
\]
を考える。以下の問に答えよ。
\begin{description}
  \item[(i)] $a=0$のとき$R_a$の各極大イデアル$\frakm$に対して$\dim_K(\frakm / \frakm^2)$を求めよ。また$\frakm$はいつ$R_a$の単項イデアルとなるか?理由をつけて答えよ。
  \item[(ii)] $a \neq 0$のとき$R_a$の各極大イデアル$\frakm$に対して$\dim_K(\frakm / \frakm^2)$を求めよ。また$\frakm$はいつ$R_a$の単項イデアルとなるか?理由をつけて答えよ。
\end{description}
}


\newpage

\bfsection{平成27年度 基礎科目I}

\bfsubsection{問1}
\barquo{
次の広義積分を求めよ。
\[
\iint_D \f{y^2 e^{-xy} }{ x^2 + y^2 } \ dx dy
\]
ここで、$D=\setmid{(x,y) \in \R^2 }{0 < y \leq x}$とする。
}
\begin{sol}
  $x=r \cos \grt$, $y = r\sin \grt$とおくと$dx dy = r dr d\grt$であって
  \begin{align*}
    \iint_D \f{y^2 e^{-xy} }{ x^2 + y^2 } \ dx dy &= \int_0^{\pi/4} \ d\grt \int_0^{\infty} r \sin^2 \grt e^{- r^2 \sin \grt \cos \grt} \ dr \\
    &= \f{1}{2} \int_0^{\pi/4} \sin^2 \grt \left( \int_0^{\infty}   e^{- s \sin \grt \cos \grt} \ ds \right) \ d\grt \\
    &= \f{1}{2} \int_0^{\pi/4} \f{ \sin \grt }{\cos \grt } \ d\grt \\
    &= - \f{1}{2} \int_1^{1/ \sqrt{2} } \f{dt}{t} \\
    &= \f{\log 2}{4}
  \end{align*}
\end{sol}

\newpage

\bfsubsection{問2}
\barquo{
$\R^2$で定義された関数
\[
f(x,y) = \f{4x^2 + (y+2)^2}{x^2 + y^2 +1}
\]
のとりうる値の範囲を求めよ。
}
\begin{sol}
  
\end{sol}


\newpage

\section{平成27年度 基礎科目 II}

\subsubsection{}
\barquo{
$f(x), \phi(x)$は区間$[0,\infty)$上の実数値連続関数とし、さらに$\phi(x)$は
\begin{gather*}
  \phi(x) = \phi(x+1) \quad (x \geq 0) \\
  \int_0^1 \phi(x) \ dx = 1
\end{gather*}
をみたすとする。このとき、任意の実数$a > 0$に対し
\[
\lim_{\grl \to \infty} \int_0^a f(x)\phi(\grl x) \ dx = \int_0^a f(x) \ dx
\]
が成り立つことを示せ。
}
\begin{sol}
  示すべきことは
  \[
  \lim_{\grl \to \infty} \int_0^a f(x)(\phi(\grl x) - 1) \ dx = 0
  \]
  なので$\psi(x) = \phi(x) -1$とおく。$\ve > 0$が任意に与えられたとする。
  \[
  I_{\grl} = \int_0^a f(x)\psi(\grl x) \ dx
  \]
  とおく。すると
  \[
  I_{\grl} = \f{1}{\grl}  \int_0^{a\grl} f( y/ \grl) \psi(y) \ dy
  \]
  である。$a\grl = n + b$なる$n \in \Z$と$0 \leq b < 1$をとると
  \[
  I_{\grl} = \f{1}{\grl}  \sum_{k=0}^{n-1}  \int_k^{k+1} f( y/ \grl) \psi(y) \ dy + \f{1}{\grl}  \int_n^{n+b} f( y/ \grl) \psi(y) \ dy
  \]
  となるが、ここで$M = \int_0^1 \abs{\psi(x)} \ dx$とすると
  \[
  \int_n^{n+b}  \abs{f( y/ \grl) \psi(y) }  \ dy \leq \sup_{0 \leq x \leq a} \abs{f(x)} M
  \]
  だから$\grl \to \infty$のとき第2項は無視してよい。よって
  \[
    I_{\grl} = \f{1}{\grl}  \sum_{k=0}^{n-1}  \int_k^{k+1} f( y/ \grl) \psi(y) \ dy + O(1/\grl)
  \]
  であるが、
  \[
  0 = \sum_{k=0}^{n-1} f(k/\grl) \int_k^{k+1} \psi(y) \ dy
  \]
  であることから
  \[
  \abs{ I_{\grl} } \leq  \f{1}{\grl}  \sum_{k=0}^{n-1}  \int_k^{k+1} \abs{f( y/ \grl)  - f(k/\grl) } \abs{\psi(y)} \ dy + O(1/\grl)
  \]
  である。ここで$0 \leq y \leq n$のとき$0 \leq y/\grl \leq a$であることに注意する。$f$は$[0,a]$上一様連続なので
  \[
  \abs{x-y} < \grd \to \abs{f(x) - f(y)} < \ve
  \]
  なる$\grd > 0$がある。そこで$\grl > 1/\grd$とすると
  \begin{align*}
    \abs{I_{\grl}} &\leq \f{n \ve }{\grl} M + O(1/\grl) \\
    &\leq aM\ve + O(1/\grl)
  \end{align*}
  が成り立つ。$\grl \to \infty$として$\limsup_{\grl \to \infty} \abs{I_{\grl}} \leq a \ve M$を得る。$\ve > 0$は任意だったので$\lim_{\grl \to \infty } I_{\grl} = 0$である。
\end{sol}

\newpage

\setcounter{subsubsection}{3}
\subsubsection{} %{問4}
\barquo{
$1$以上$3500$以下の整数$x$のうち、$x^3 + 3x$が$3500$で割り切れるものの個数を求めよ。
}
\begin{sol}
$3500 = 2^2 \tm 5^3 \tm 7$なので、中国式剰余定理より$\Z / 3500 \Z \cong \Z / 4\Z \tm \Z / 125 \Z \tm \Z / 7\Z$である。
いま$\Z / 4\Z$で$x$に値を代入することにより調べると$x^3 + 3x = 0 \in \Z / 4 \Z \iff x = 0, \pm 1$がわかる。

$\Z / 7 \Z$で考えると$x^3 + 3x = x(x^2 + 3) = x(x^2 - 4)= x(x+2)(x-2)$であり、$\Z / 7 \Z$は整域だから$x^3 + 3x = 0 \in \Z / 7 \Z \iff x = 0, \pm 2$が判る。

$\Z / 125 \Z$で考える。$x^3 + 3x = x(x^2 + 3)$であるが、$x^2 + 3$は決して$5$の倍数にならないので$\Z / 125 \Z$において常に単元である。よって$x^3 + 3x = 0 \in \Z / 125 \Z \iff x = 0$が判る。

以上の議論により求める$x$の個数は$3 \tm 3 \tm 1 = 9$個である。
\end{sol}

\newpage

\setcounter{subsubsection}{6}
\subsubsection{} %{問7}
\barquo{
$A$を実正方行列とする。このとき、ある正の整数$k$が存在して$\tr (A^k) \geq 0$となることを示せ。ただし$\tr$は行列のトレースを表す。
}
\begin{sol}
  行列$A$のサイズを$n$とする。$A$の固有多項式の根を重複を込めて
  \[
  \grl_1, \cdots , \grl_r, \grl_{r+1}, \ol{\grl_{r+1}}, \cdots , \grl_{r+s}, \ol{\grl_{r+s}} \quad (r+2s = n)
  \]
  とおく。するとトレースは固有値の和なので
  \[
  \tr A^k = \sum_{i=1}^r \grl_i^k + 2  \sum_{i=1}^s \Re \grl_{r+i}^k
  \]
  と書ける。ここで$\grl_i$の偏角を考える。$\arg \grl_i = 2\pi \gra_i$ $(0 \leq \gra_i < 1)$とおく。このときDirichletの近似定理から
  \[
  \exists k \in \Z_{\geq 1} \; \forall 1 \leq i \leq s+r \; \exists m_i \in \Z \quad \abs{k\gra_i - m_i} < \f{1}{4}
  \]
  が成り立つ。この$k$について$\forall i \quad \arg \grl_i^k \in [0, \pi/2) \cup (3\pi / 2, 2\pi)$だから$\tr A^k \geq 0$が成り立つ。
\end{sol}


\newpage

\section{平成27年度 専門科目}

\subsubsection{}
\barquo{
$G$は非可換群で次の条件$(*)$を満たすとする。

$(*)$ $N_1,N_2 \subset G$が相異なる自明でない (つまり${1}$とも$G$とも異なる) 正規部分群なら、$N_1 \not\subset N_2$である。

このとき、以下の問に答えよ。
\begin{description}
  \item[(i)] $N_1, N_2$が相異なる$G$の自明でない正規部分群なら、$G = N_1 \tm N_2$であることを証明せよ。
  \item[(ii)] $G$の自明でない正規部分群の数は高々$2$個であることを証明せよ。
\end{description}
}
\begin{sol} ${}$
  \begin{description}
    \item[(i)] 仮定より$N_1 \cap N_2 \subsetneq N_i \subset G$かつ$N_1 \cap N_2 \lhd G$なので$N_1 \cap N_2 = 1$である。また$1 \subsetneq N_i \subsetneq N_1 N_2 $かつ$N_1 N_2 \lhd G$より$N_1 N_2 = G$である。したがって交換子が
    $[N_1,N_2] \subset N_1 \cap N_2 = 1$より自明になるので、$N_1$と$N_2$の元は互いに可換。よって積をとる写像$N_1 \tm N_2 \to G$は準同型でかつ全単射なので同型である。
    \item[(ii)] ハイリホーによる。相異なる$3$つの自明でない正規部分群$N_1, N_2, N_3$が存在したとしよう。相異なるという仮定から、(i)により$N_i N_j = G \; (i \neq j)$であり、$i \neq j$である限り$N_i$と$N_j$の元は互いに可換である。したがって$G = N_1 N_2$と$N_3$の元は可換なので、とくに$N_3$はAbel群である。同様にして各$N_i$がAbel群であることが判る。よってとくに$G$はAbel群であるが、$G$は非可換群であったはずなので矛盾。ゆえに示すべきことがいえた。
\end{description}
\end{sol}


\newpage


\subsubsection{}%問2
\barquo{
$X,Y,T$を変数とし、$A=\Z[X,Y]/(Y^2-6X^2)$, $B = \Z[X,T]/(T^2-6)$とおく。また、$A$における$X,Y$の剰余類を$x,y$, $B$における$X,T$の剰余類を$x',t$とする。$A$のイデアル$P_1,P_2$と$B$のイデアル$Q_1$を
\[
P_1 = xA + yA + 5A, \; P_2 = (x-y)A + 5A, \; Q_1 = x'B + (t+1)B
\]
と定めるとき、以下の問に答えよ。
\begin{description}
  \item[(i)] 単射な環準同型$\phi \colon A \to B$で$\phi(x)=x'$, $\phi(y)=x't$であるものが存在することを証明せよ。
  \item[(ii)] $P_1,P_2$は$A$の素イデアルで$P_2 \subsetneq P_1$であることを証明せよ。
  \item[(iii)] (i)により$A$を$B$の部分環とみなすとき、$Q_1$は$B$の素イデアルで$Q_1 \cap A = P_1$であることを証明せよ。
  \item[(iv)] $B$の素イデアル$Q_2$で$Q_2 \subset Q_1$, $Q_2 \cap A = P_2$となるものは存在しないことを証明せよ。
\end{description}
}
\begin{sol} ${}$
  \begin{description}
\item[(i)] $\wt{\phi} \colon \Z[X,Y] \to B$を$\wt{\phi}(X) = x'$, $\wt{\phi}(Y) = x't$で定める。このときあきらかに$(Y^2 - 6X^2) \subset \Ker \wt{\phi}$である。逆に$f \in \Ker \wt{\phi}$とする。このとき
\[
f(X,Y)=f_0(X) + f_1(X) Y + g(X,Y)(Y^2 - 6X^2)
\]
なる$f_0, f_1 \in \Z[X]$と$g \in \Z[X,T]$がある。よって
\[
f_0(X) + f_1(X,T)XT \in (T^2 - 6)
\]
であるが、$T$についての次数の考察から$f_0 = f_1 = 0$でなくてはならない。$f \in \Ker \wt{\phi}$は任意だったから$\Ker \wt{\phi} = (Y^2 - 6X^2)$である。したがってそのような単射$\phi$は存在する。
\item[(ii)] 商環を計算すると
\begin{align*}
  A/P_1 &\cong \Z[X,Y]/(Y^2 - 6X^2, X, Y , 5) \\
  &\cong \F_5 \\
  A/P_2 &\cong \Z[X,Y]/(Y^2 - 6X^2, X-Y , 5) \\
  &\cong \F_5[X]
\end{align*}
であり、それぞれ整域なので$P_1$と$P_2$は素イデアル。$P_2 \subset P_1$はあきらかであろう。また商環が異なるので$P_1 \neq P_2$である。
\item[(iii)] やはり商環の計算により示す。
\begin{align*}
  B/Q_1 &\cong \Z[X,T]/(T^2 - 6, X,T+1) \\
  &\cong \F_5
\end{align*}
より$\F_5$は整域なので$Q_1$は素イデアルである。また
\begin{align*}
  B/P_1 B &\cong \Z[X,T]/(T^2-6,X,XT,5) \\
  &\cong \Z[X,T]/(T^2-1,X,5) \\
  &\cong \F_5[T]/(T-1)(T+1) \\
  &\cong \F_5 \tm \F_5
\end{align*}
により$P_1B = (x', t+1)(x', t-1)$がわかる。よって$P_1 B \subset Q_1$であり、とくに$P_1 \subset Q_1 \cap A$である。$Q_1 \cap A$は素イデアルで、$P_1$は極大イデアルなので$P_1 = Q_1 \cap A$でなくてはいけない。
\item[(iv)] ハイリホーによる。そのような$Q_2$が存在したとする。
\begin{align*}
  B/P_2B &\cong  \Z[X,T]/(T^2-6,5,X-XT) \\
  &\cong \F_5[X,T]/(T^2-1,5,X(1-T) ) \\
  &\cong \F_5[X,T]/(T-1)(T+1, X) \\
  &\cong \F_5[X] \tm \F_5
\end{align*}
により$P_2B = (t-1)(t+1,x')$である。$Q_2$は素イデアルと仮定したことから、$P_2B \subset Q_2$なので$(t-1) \subset Q_2$あるいは$Q_1 =  (t+1,x') \subset Q_2$でなくてはいけない。$P_1 \neq P_2$なので、$(t-1) \subset Q_2$ということになる。しかしこのとき
\begin{align*}
  Q_1 &= Q_1 + Q_2 \\
  &\supset (t-1, t+1,x') \\
  &\supset B
\end{align*}
となり矛盾。よって示すべきことがいえた。
  \end{description}
\end{sol}

\newpage

\subsubsection{}%問3
\barquo{
$\C(t)$を$\C$上の$1$変数有理関数体とする。$a$を複素数とし、$s=t^3 + 3t^2 + at \in \C(t)$とおく。$\C$上$s$で生成された$\C(t)$の部分体を$\C(s)$とするとき、以下の問に答えよ。
\begin{description}
\item[(i)] 拡大次数$[\C(t):\C(s)]$を求めよ。
\item[(ii)] $\C(t)/\C(s)$がガロア拡大となる複素数$a$をすべて求めよ。
\end{description}
}
\begin{sol} ${}$
  \begin{description}
    \item[(i)] $t$は$\C(s)$係数の多項式
    \[
    F := X^3 + 3X^2 + aX -(t^3+3t^2+at)
    \]
    の根である。よって$[\C(t):\C(s)] \leq 3$である。

    まず$[\C(t):\C(s)] \geq 2$を示そう。ハイリホーによる。仮に$t \in \C(s)$だったとする。$s$は$\C$上超越的なので$\C[s]$はPIDであり、とくに整閉である。よって$t$は$\C[s]$上整なので$t \in \C[s]$である。しかし$s \in \C[t]$は$3$次式なのでこれは矛盾。よって$[\C(t):\C(s)] \geq 2$である。

    次に$[\C(t):\C(s)] \geq 3$を示そう。ハイリホーによる。仮に$[\C(t):\C(s)] = 2$だったとしよう。$t$の$\C(s)$上の共役を$\{t,\ol{t} \}$とする。$2$次拡大は正規拡大なので$\ol{t} \in \C(t)$であるが、
    $\ol{t}$は$\C[t]$上整なので$\ol{t} \in \C[t]$である。仮定より$F$は$\C(s)$係数の多項式として可約な$3$次式なので$1$次式を因数として含む。よって$F$の根$u$であって$u \in \C(s)$なるものがある。むろん$\C[s]$の整閉性により実際には$u \in \C[s]$である。このとき$\C[t]$において
    \[
    ut\ol{t} = t^3 +3t^2 + at
    \]
    だから$u\ol{t} = t^2 +3t + a$であり、右辺の次数が$2$なので$u \in \C$である。よって$\ol{t}$は$2$次式ということになるが、これは$t + \ol{t} \in \C[s]$に矛盾。以上により$[\C(t):\C(s)] = 3$が結論される。
    \item[(ii)] $\C(t)/\C(s)$がGalois拡大であるという命題を(P)であらわすことにする。(P)は次と同値である。
\begin{oframed}
  \textbf{(P1)} \quad $\C(s)$係数の多項式$F$のすべての根は$\C(t)$に含まれる。
\end{oframed}
  多項式の根が$\C(t)$に入ることと、$\C(t)$で因数分解できることは同じなので(P1)は次と同値。
  \begin{oframed}
    \textbf{(P2)} \quad ある$f,g,h \in \C(t)$が存在して$F(X) = (X-f)(X-g)(X-h)$が成り立つ。
  \end{oframed}
  $F$は$\C[t]$係数のモニック多項式であり、かつ$\C[t]$は整閉なので$f,g,h \in \C[t]$としてよい。つまり(P2)は次と同値。
  \begin{oframed}
    \textbf{(P3)} \quad ある$f,g,h \in \C[t]$が存在して
    \begin{align*}
      f+g+h &= -3 \\
      fg + gh + hf &= a \\
      fgh &= t(t-\beta)(t-\grg)
    \end{align*}
    が成り立つ。ただし$\beta,\grg$は$t^2 + 3t + a= (t-\beta)(t-\grg)$なる$\C$の元とする。
  \end{oframed}
  $f+g+h$が定数で$fgh$が$3$次式という条件より、$\deg f = \deg g = \deg h = 1$でなくてはならない。$f,g,h$を適当に並び替えることにより、ある$b,c,d \in \C$が存在して$f(t) =bt$, $g(t) =c(t-\beta)$, $h(t) =d (t-\grg)$と表せるとしてよい。このとき計算すると
  \begin{align*}
    f+g+h + 3&= (b+c+d)t - c\beta -d\grg +3 \\
    fg + gh + hf-a &= (bc+cd+db)t^2 + (-bc\beta +3cd - bd\grg)t +(cd-1)a \\
    fgh &= bcd t(t-\beta)(t-\grg)
  \end{align*}
  である。よって$b,c,d$は次を満たさなくてはならない。
  \begin{align*}
    b+c+d&=0 \\
    bc+cd+db &= 0 \\
    bcd &= 1 \\
    c\beta + d\grg - 3 &= 0 \\
    -bc\beta + 3cd - bd \grg &= 0 \\
    (cd-1)a &= 0
  \end{align*}
  前半の$3$つの条件は、$b,c,d$が$X^3-1$の異なる$3$つの根であることを意味する。残りの条件も使うと
  \[
  3b^2 =  b^2(c\beta +d\grg) = 3bcd = 3
  \]
  より$b=1$が得られる。よって$c,d$は$X^2 + X + 1$の異なる$2$つの根である。よって、与えられた条件は
  \[
  c\beta + d\grg = 3
  \]
  と要約できる。つまり(P3)は次と同値である。
  \begin{oframed}
    \textbf{(P4)} \quad $X^2+X+1$の異なる$2$つの根$c,d$をうまく選べば$c\beta + d\grg = 3$が成り立つ。
  \end{oframed}
  いま(P4)が成り立つと仮定する。$-3 = \beta + \grg$により$(1+c)\beta + (1+d)\grg = 0$であるが、これは$1+d+c=0$により$d\beta + c\grg=0$を意味する。よって
  \begin{align*}
    0 &= (c\beta + d\grg )( d\beta + c\grg) \\
    &= \beta^2 + \grg^2 - a \\
    &= (\beta + \grg)^2 - 3a \\
    &= 9 -3a
  \end{align*}
  である。これは$a=3$ということである。逆に$a=3$だと仮定しよう。このとき$\beta,\grg$が
  \begin{align*}
    \beta &= \f{-3 + \sqrt{-3}}{2} = \sqrt{3} e^{5\pi i/6} \\
        \grg &= \f{-3 - \sqrt{-3}}{2} = \sqrt{3} e^{7 \pi i/6}
  \end{align*}
  と与えられていたとすると、$d = e^{2\pi i/3}$, $c = e^{4\pi i/3}$とおけば
  \[
  c\beta + d\grg = \sqrt{3} ( e^{\pi i/6}  +  e^{11 \pi i/6} ) = 3
  \]
  となり(P4)が成立する。ゆえに求める条件をみたす$a$は$a=3$である。
  \end{description}
\end{sol}


\newpage

\section{平成26年度 基礎科目I}

\subsubsection{}
\barquo{
\begin{description}
  \item[(i)] $\{a_n \}_{n=1}^{\infty}$は実数列で、任意の正整数$k$について
  \[
  \lim_{n \to \infty} (a_{n+k} - a_n) = 0
  \]
  をみたすとする。このとき、この数列$\{a_n \}_{n=1}^{\infty}$は収束するか?理由をつけて答えよ。
  \item[(ii)] 次の広義積分は収束するか?理由をつけて答えよ:
  \[
  \int_0^{\infty} (1 - e^{-1/x}) \ dx.
  \]
\end{description}
}
\begin{sol} ${}$
  \begin{description}
    \item[(i)] 収束するとは限らない。反例はたとえば$a_n = \log n$とすれば得られる。
    \item[(ii)] $y=1/x$とおくと$dx = - 1/y^2 dy$であって
    \[
    \int_0^{\infty} (1 - e^{-1/x}) \ dx = \int_0^{\infty} \f{1 -e^{-y} }{ y^2} \ dy
    \]
    であるはずだから、右辺の収束性を考えればよい。いまTaylor展開を考えると
    \[
    e^{-y} = 1 -y + O(y^2)
    \]
    だから、$g(y) = (1 - e^{-y})/ y^2 $とおくと$g(y) = 1/y + h(y)$なる$[0,\infty)$上の連続関数$h$がある。$0 < y$のとき$g(y) > 0$なので、$0 < \ve < 1$に対して
    \begin{align*}
      \int_{\ve}^{\infty} g(y) \ dy &\geq \int_{\ve}^{1} g(y) \ dy \\
      &\geq \int_{\ve}^{1} \abs{ \f{1}{y} + h(y) } \ dy \\
      &\geq \int_{\ve}^{1} \f{dy}{y} - \int_{\ve}^{1} \abs{ h(y) } \ dy \\
      &\geq \log 1/\ve - \int_{0}^{1} \abs{ h(y) } \ dy
    \end{align*}
    が成り立つ。したがって
    \[
    \liminf_{\ve \to +0} \int_{\ve}^{1} g(y) \ dy = \infty
    \]
    である。よって件の積分は収束しない。
  \end{description}
\end{sol}

\newpage


\subsubsection{}%2
\barquo{
$n$は$2$以上の整数とする。$\R^2$上の関数
\[
f(x,y) = x^{2n} + y^{2n} - nx^2 + 2nxy - ny^2
\]
について次の問に答えよ:
\begin{description}
\item[(i)] $f$の最大値・最小値は存在するか?理由をつけて答えよ。
\item[(ii)] $f$が極大値・極小値をとる点をすべて求めよ。
\end{description}
}
\begin{sol} ${}$
  \begin{description}
    \item[(i)] $f(x,y) = x^{2n} + y^{2n} - n(x-y)^2$と書ける。よって$f(x,x) = 2x^{2n}$なので$f$は最大値を持たない。また$x=r \cos \grt$, $y = r \sin \grt$おいて$g(r,\grt) = f(r \cos \grt, r \sin \grt)$とするとき
    \[
    g(r,\grt)=r^{2n} (\cos^{2n} \grt + \sin^{2n} \grt) - nr^2 (1- \sin 2\grt)
    \]
    であるが、
    \begin{align*}
      \cos^{2n} \grt + \sin^{2n} \grt &\geq \max\{ \cos^{2n} \grt , \sin^{2n} \grt \} \\
      &\geq (1/\sqrt{2})^{2n} \\
      &\geq 1/2^n
    \end{align*}
    であるため$\abs{g(r,\grt)} \geq 2^{-n} r^{2n} - 2nr^2$と評価できる。したがってある$R>0$が存在して$g$は$\setmid{(r,\grt)}{r \geq R}$上で$0$以上となる。ゆえに$f(0,0)=0$より
    \[
    \inf_{(x,y) \in \R^2} f(x,y) = \min_{r \leq R} g(r,\grt)
    \]
    だから$f$は最小値を持つ。
    \item[(ii)] $x,y$についてそれぞれ偏微分すると
    \begin{align*}
      \f{\del f}{\del x} &= 2n (x^{2n-1} - x + y) \\
      \f{\del f}{\del y} &= 2n (y^{2n-1} - y + x)
    \end{align*}
    である。よって点$(x,y)$がもし極値を与えるならば、
    \begin{align*}
      x^{2n-1} - x + y &= 0 \\
      y^{2n-1} - y + x &= 0
    \end{align*}
    である。よって極値を与える点は、$\gra = 2^{1/(2n-2)} > 0$として
    \[
    P = (0,0) , \quad Q_1 = (\gra, - \gra), \quad Q_2 =  (-\gra, \gra)
    \]
    の中にある。これらが実際に極値なのか、そして極値だとすれば極小か極大かを判断するためにHessianを求める。
    \begin{align*}
\f{\del^2 f }{\del x^2} &= 2n ( (2n-1)x^{2n-2} - 1 ) \\
\f{\del^2 f }{\del y^2} &= 2n ( (2n-1)y^{2n-2} - 1 ) \\
\f{\del^2 f }{\del x \del y} &= 2n
    \end{align*}
    であるから、$P$におけるHessianは
    \[
    H_P = 2n \pmat{-1 & 1 \\ 1 & -1}
    \]
    である。$1/2n H_P$の固有値は$0, -2$であり$H_P$は正則でないので、Hessianから$P$が極値であるかどうかを判定することはできない。実際、$f(x,0)=x^2(x^{2n-2}-n)$より直線$y=0$上では$f(P)$は極大値。$f(x,x) = 2x^{2n}$より直線$y=x$上では$f(P)$は極小値。よって$P$は鞍点であり極値を与えない。

$Q_i$におけるHessianは
\[
H_{Q_1} = H_{Q_2} =  2n \pmat{4n-3 & 1 \\ 1 & 4n-3}
\]
である。$1/2n H_{Q_i}$の固有値は$4n-4, 4n-2$でありどちらも正。よってHessianは正定値であるから$f(Q_1)=f(Q_2) = 4\gra^2 (1-n)$は極小値。(最小値でもある)
  \end{description}
\end{sol}

\newpage

\subsubsection{}%3
\barquo{
次の4次正方行列$A,B$は正則か?正則ならば逆行列を求め、正則でないならば階数を求めよ。
\[
A = \pmat{ 2 &0& 1& 3 \\ 0& 8& 2& 4 \\ 2& 0& 1& 4 \\ 0& 4& 0& 1} , \quad  B = \pmat{1 &1& 1& 0 \\ 2& 2 &0 &3 \\ 3 &4 &2 &4 \\ 4 &5& 3& 4 }
\]
}
\begin{sol}
  $E$を単位行列とする。拡大係数行列$(A \; E)$を行基本変形すると
  \[
   (A \; E) \sim \pmat{ 1 &0& 0& 0& 3/2& -1/4& -1& 1/2 \\ 0& 1& 0& 0& -1/4& 0& 1/4& 1/4 \\ 0 &0& 1& 0 & 1& 1/2 &-1& -1 \\ 0& 0& 0& 1& -1& 0& 1& 0}
  \]
  を得る。したがって$A$は正則で、逆行列は
  \[
  \f{1}{4} \pmat{ 6 &-1& -4& 2 \\ -1& 0& 1& 1 \\ 4& 2& -4& -4 \\ -4& 0& 4& 0  }
  \]
  で与えられる。同様に$(B \; E)$を行基本変形していくと$\rank B = 3$であることがわかる。とくに$B$は正則ではない。
\end{sol}

\newpage

\subsubsection{}%4
\barquo{
3次の複素正方行列
\[
A = \pmat{3& 0& -1 \\ -2& 1& 1 \\ 2& 0& 0 }, \quad B = \pmat{1& x& 0 \\ 0& 1& 0 \\ -1& x& 2}
\]
に対して、$A$と$B$が相似になるような複素数$x$をすべて求めよ。ただし、行列$A$と$B$が相似とは、複素正方行列$P$で$A = P^{-1}AP$を満たすものが存在することをいう。
}
\begin{sol}
  $A$の固有多項式は$(t-1)^2(t-2)$なので固有値は$1,2$である。計算すると$\rank (E - A)=1$なので$\Ker (E-A)$は$2$次元空間。したがって$A$のJordan標準形は
  \[
  \pmat{ 1 &0& 0 \\ 0& 1& 0 \\ 0& 0& 2}
  \]
  である。$B$の固有多項式も$(t-1)^2(t-2)$で、固有値は$A$と同じ。しかし
  \[
  \rank (E-B) = \rank \pmat{ 0 &x& 0 \\ 1& 0& -1 \\ 0& 0 &0}
  \]
  なので$B$のJordan標準形は$x \neq 0$のとき対角行列でなく、$x=0$のとき対角行列となる。よって$A$と$B$が相似となる$x$は$x=0$である。
\end{sol}


\newpage

\section{平成26年度 基礎科目II}

\subsubsection{}%1
\barquo{
実数値関数$f(x)$は$[0,\infty)$で連続で$\lim_{x \to \infty} f(x) = 1$とする。このとき
\[
\lim_{n \to \infty} \f{1}{n !} \int_0^{\infty} f(x) e^{-x} x^n \ dx = 1
\]
であることを証明せよ。
}
\begin{sol}
  Gamma関数についてのよく知られた事実として
  \[
  \int_0^{\infty} e^{-x} x^n \ dx = n !
  \]
  が成り立つことを注意しておく。$\ve > 0$が与えられたとする。仮定より
  \[
  x \geq R \to \abs{f(x) - 1} < \ve
  \]
  なる$R > 0$がある。このとき
  \begin{align*}
    \abs{ \f{1}{n !} \int_0^{\infty} f(x) e^{-x} x^n \ dx - 1} &= \f{1}{n !}  \abs{ \int_0^{\infty} f(x) e^{-x} x^n \ dx - \int_0^{\infty} e^{-x} x^n \ dx} \\
    &\leq \f{1}{n !} \int_0^{\infty} \abs{  f(x) - 1} e^{-x} x^n  \ dx \\
    &\leq \ve + \f{1}{n !} \int_0^{R} \abs{  f(x) - 1} e^{-x} x^n  \ dx
  \end{align*}
  である。よって
  \[
  \limsup_{n \to \infty} \abs{ \f{1}{n !} \int_0^{\infty} f(x) e^{-x} x^n \ dx - 1} \leq \ve
  \]
  が従う。$\ve > 0$は任意だったので、示すべきことがいえた。
\end{sol}

\newpage


\subsubsection{}%2
\barquo{
$n,m$を正の整数とする。$x$を変数とする$n$次以下の$\C$係数多項式の全体を$V_n$とし、和・差・スカラー倍により$V_n$を$\C$上のベクトル空間とみなす。$m$個の複素数$\gra_1, \cdots , \gra_m$に対し、線形写像$F \colon V_n \to \C^m$を
\[
F(f) = (f(\gra_1), \cdots , f(\gra_m)))
\]
で定める。このとき
\begin{description}
\item[(i)] $F$が単射になるための必要十分条件を$n,m,\gra_1, \cdots , \gra_m$のみを用いて述べよ。
\item[(ii)] $F$が全射になるための必要十分条件を$n,m,\gra_1, \cdots , \gra_m$のみを用いて述べよ。
\end{description}
}
\begin{sol}
  $\gra_1, \cdots , \gra_m$のうち相異なるものの数を$k$とする。適当に番号を付けなおすことにより$\gra_1, \cdots , \gra_k$が相異なるとしてよい。$V_n$の基底$\{1,x , \cdots , x^n\}$と$\C^m$の標準基底について$F$を行列表示すると
  \[
  \pmat{ 1 & \gra_1 & \cdots &  \gra_1^n \\  1 & \gra_2 & \cdots &  \gra_2^n \\ \vdots & \vdots & & \vdots \\ 1 & \gra_m & \cdots &  \gra_m^n}
  \]
  となる。したがって
  \[
  \rank F = \rank \pmat{ 1 & \gra_1 & \cdots &  \gra_1^n \\  1 & \gra_2 & \cdots &  \gra_2^n \\ \vdots & \vdots & & \vdots \\ 1 & \gra_k & \cdots &  \gra_k^n}
  \]
  である。右辺の行列のサイズが$s := \min\{ k,n+1 \}$の部分正方行列
  \[
  \grD = \pmat{ 1 & \gra_1 & \cdots &  \gra_1^{s-1} \\  1 & \gra_2 & \cdots &  \gra_2^{s-1} \\ \vdots & \vdots & & \vdots \\ 1 & \gra_s & \cdots &  \gra_s^{s-1}}
  \]
  の行列式はVandermondeの行列式であって
  \[
  \abs{\det \grD} = \prod_{i > j} \abs{ \gra_i - \gra_j } \neq 0
  \]
  である。したがって$\rank F = \min\{ k,n+1 \}$である。ここまでの準備をもってすれば問に答えることはやさしい。
  \begin{description}
    \item[(i)] $F$が単射であることは$\rank F = \dim V_n$と同値。つまり$n+1 \leq k$である。
    \item[(ii)] $F$が全射であることは$\rank F = \dim \C^m$と同値。つまり$m = k \leq n+1$である。
  \end{description}

\end{sol}


\newpage


\section{平成26年度 専門科目}

\subsubsection{}%1
\barquo{
$\C[X,Y]$を複素数係数の$2$変数多項式環、$A=\C[X,Y]/(X^2 + Y^3 - 1)$とし、$X,Y \in \C[X,Y]$の$A$での類をそれぞれ$x,y $とおく。
\begin{description}
  \item[(i)] $A$が整域であり、$A$の商体$L$が$\C(y)$の$2$次拡大であることを証明せよ。
  \item[(ii)] $A$が$\C[y]$の$L$における整閉包であることを証明せよ。
  \item[(iii)] $y$が$A$の既約元であることを証明せよ。
  \item[(iv)] $A$がUFD(一意分解整域)であるかどうか理由をつけて決定せよ。
\end{description}
}
\begin{sol} ${}$
  \begin{description}
    \item[(i)] $B:= \C[Y]$, $K:= \C(Y)$とし、$A$は$B$代数として$A= B[X]/(X^2 + Y^3-1)$ととらえなおす。$B$はPIDなので$B[X]$はUFDである。このとき$X^2 + Y^3 - 1$は$\frakp = (Y-1)$に関するEisenstein多項式であるため$K[X]$の元として既約かつ素元である。ゆえに$A$は整域。

    $A$を$B$代数として捉えたのと同様$L$も$K$代数として$L=K[X]/(X^2 + Y^3 -1)$と捉えなおす。
    \[
    \xymatrix{
    K \ar[r] & L \\
    B \ar[u] \ar[r] & A \ar[u]
    }
    \]
    $X \in K[X]$の$L$での像を$x$と書くことにする。$L$は$K$上$x$で生成されていて$x$の$K$上の最小多項式は$T^2 + Y^3 - 1 \in K[T]$なので$[L:K]=2$であることが判る。
    \item[(ii)] $x \in L$は$B$上整なので$A/B$は整拡大。逆に$z \in L$が$B$上整だったと仮定する。$[L:K] =2$なので$z = ax + b$なる$a,b \in K$がある。ここで$L/K$は分離拡大なので、$G := \Hom_K^{\text{al}}(L,\ol{K}) = \{ \grs_1, \grs_2 \}$とするとトレース
    $\Tr_{L/K} \colon L \to K$とノルム$\Norm_{L/K} \colon L \to K$は
    \begin{align*}
      \Tr_{L/K}(s) &= \sum_{\grs \in G} \grs(s) \\
      \Norm_{L/K}(s) &= \prod_{\grs \in G} \grs(s)
    \end{align*}
    と計算できる。$B$はPIDなのでとくに整閉であり、$s \in L$が$B$上整ならば$s$のトレースとノルムも$B$の元となる。したがって
    \begin{align*}
      \Tr_{L/K}(z) &= 2b \in B \\
      \Norm_{L/K}(z-b) &= a^2(Y^3 - 1) \in B
    \end{align*}
    が得られる。$Y^3 - 1$は平方因子を持たないので$a \in B$でなくてはならない。よって$z \in A$である。$z$は任意だったから、これで$A$が$L$における$B$の整閉包であることがいえた。
\item[(iii)] ハイリホーによる。$Y \in A$が可約だったとする。このとき非単元$\gra , \beta \in A$があって$Y = \gra \beta$を満たす。ノルムをとって$Y^2 = \Norm_{L/K}(\gra) \Norm_{L/K}(\beta)$を得る。$B$はUFDなので素元$Y \in B$によるオーダーを考えることができる。(ii)により$A$は$L$における$B$の整閉包であったので、各$\grs \in G$は$\grs(A) \subset A$を満たす。
$\gra, \beta$は$A$の単元ではないので、よって$\Norm_{L/K}(\gra), \Norm_{L/K}(\beta)$も$B$の単元ではない。以上により$\Norm_{L/K}(\gra), \Norm_{L/K}(\beta)$の素元$Y \in B$に関するオーダーは$1$である。よって$\Norm_{L/K}(\gra) = uY$なる単元
$u \in B^{\tm} = \C^{\tm}$がある。$\gra = c x + d \; (c,d \in B)$とおくと
\[
u^{-1} Y = c^2 (Y^3-1) + d^2
\]
が得られる。ここで$\deg(c^2(Y^3-1)) = 2 \deg c + 3$は奇数で$\deg d^2 = 2 \deg d$は偶数なので
\begin{align*}
  1 &= \deg (u^{-1}Y) \\
  &= \deg( c^2 (Y^3-1) + d^2 ) \\
  &= \max\{ 2 \deg c + 3,  2 \deg d \} \\
  &\geq 3
\end{align*}
となって矛盾。よって$Y \in A$は既約元である。
\item[(iv)] 計算すると
\begin{align*}
  A/(Y) &\cong B[X]/(X^2 + Y^3 -1, Y) \\
  &\cong \C[X,Y]/(X^2-1,Y) \\
  &\cong \C[X]/(X-1)(X+1) \\
  &\cong \C^2
\end{align*}
なので$Y \in A$は素元ではない。もしも$A$がUFDなら既約元はすべて素元であるはずなので、$A$はUFDではない。
  \end{description}
\end{sol}

\newpage


\subsubsection{}%2
\barquo{
$K \subset \C$を部分体、$p$を素数とする。$\C$に含まれる任意の有限次拡大$L/K$に対し、$L=K$でなければ$[L:K]$は$p$で割り切れると仮定する。このとき、$\C$に含まれる任意の有限次拡大$L/K$に対し、$[L:K]$は$p$のべき($1$を含む)であることを証明せよ。
}
\begin{sol}
  $K$の有限次拡大$L \subset \C$が与えられたとする。$L$の$K$上のGalois閉包を$\wt{L}$とする。$G:= \Gal(\wt{L}/K)$とおく。$G$のSylow-$p$部分群を$H$とし、$H$の不変体
  \[
  \wt{L}^H = \setmid{x \in \wt{L}}{\forall \grs \in H \quad \grs(x) = x }
  \]
  を考える。Galoisの基本定理により$[\wt{L}:\wt{L}^H] = \# H$だから、$[\wt{L}^H : K] = \# (G / H)$であり$[\wt{L}^H : K]$は$p$で割り切れない。よって仮定により$\wt{L}^H =K$であるから$H=G$であり、とくに$[\wt{L}:K]$は$p$のベキである。$[L:K]$は$[\wt{L}:K]$を割り切るので、
  $[L:K]$も$p$ベキ($1$を含む)である。
\end{sol}


\newpage


\subsubsection{}%3
\barquo{
$\zeta$を$1$の原始$7$乗根$e^{2\pi \I / 7}$とし、$\C$の部分集合
\[
A = \setmid{ a_1 \zeta +  a_2 \zeta^2 +  a_3 \zeta^3 + a_4 \zeta^4 + a_5 \zeta^5  + a_6 \zeta^6 }{a_1, a_2, a_3, a_4, a_5, a_6 \in \{0,1 \}  }
\]
を考える。このとき$\Q(\gra) = \Q(\zeta)$となる$a \in A$となる$a \in A$の個数を求めよ。
}
\begin{sol}
  $\Q(\zeta)$は$\Q$のGalois拡大であり、円分体論により$G:= \Gal(\Q(\zeta)/ \Q)$は$(\zyu{7})^{\tm}$と同型である。ここで次が成り立つ。

\lem{
次は同値。
\begin{description}
  \item[(i)] $\Q(\gra) = \Q(\zeta)$
  \item[(ii)] $\forall \grs \in G \sm \{ 1 \} \quad \grs(\gra) \neq \gra$
\end{description}
}
\begin{proof} ${}$
  \begin{description}
    \item[(i)$\To$(2)] 対偶を示す。ある$\grs \in G \sm \{ 1 \}$に対し$\grs(\gra)= \gra$だとする。このとき$\Q(\gra)$は$\kakko{\grs}$の不変体に含まれ、$\Q(\zeta)$より真に小さい。
    \item[(i)$\To$(ii)] 仮定より単射$G \to \Hom_{\Q}(\Q(\gra), \ol{\Q})$があるので
\[
[\Q(\gra): \Q] = \# \Hom_{\Q}(\Q(\gra), \ol{\Q}) \geq \# G = [\Q(\zeta):\Q]
\]
    が得られる。
    よって$\Q(\gra) = \Q(\zeta)$である。
  \end{description}
\end{proof}
したがって$\# \setmid{ \gra \in A }{ \forall \grs \in G \sm \{ 1 \} \quad \grs(\gra) \neq \gra }$を求めればよい。$(\zyu{7})^{\tm}$は$3$を生成元とする巡回群である。対応する$G$の生成元を$\tau$とする。$I := \{ 0,1\}$とおく。$a_i$の番号を付け替えて
\[
A = \setmid{  a_1 \zeta +  a_2 \zeta^3 +  a_3 \zeta^2 + a_4 \zeta^6 + a_5 \zeta^4  + a_6 \zeta^5 }{ (a_1, \cdots , a_6) \in I^6 }
\]
とみなす。このとき$A$の元に対する$\tau$の作用は$I^6$の元に対する$s := (123456) \in \frakS_6$の作用と解釈できる。ただし
\begin{align*}
\kakko{s} \tm I^6 &\to I^6 \\
(\grs, (a_i)_i ) &\mapsto (a_{\grs(i)})_i
\end{align*}
として作用を定めるものとする。したがって求めるべきものは、集合
\[
 B :=  \setmid{ a \in I^6 }{ \forall \grs \in \kakko{s} \sm \{ 1 \} \quad \grs(a) \neq a   }
\]
の位数である。ここで条件$\forall \grs \in \kakko{s} \sm \{ 1 \} \quad \grs(a) \neq a$は$\# \Orbit (a) = 6$つまり$\# \Stab(a) = 1$を意味する。したがって
\begin{align*}
  \# B &=  \setmid{ a \in I^6 }{ \# \Stab(a) = 1  } \\
  &= 64 - \setmid{ a \in I^6 }{ \# \Stab(a) \geq 2  } \\
  &= 64-  \setmid{ a \in I^6 }{ \Stab(a) = \kakko{s^2}  } - \setmid{ a \in I^6 }{ \Stab(a) = \kakko{s^3}  } - \setmid{ a \in I^6 }{ \Stab(a) = \kakko{s}  }
\end{align*}
である。いま$\Stab(a) = \kakko{s}$となる$a \in I^6$は
\[
(0,0,0,0,0,0) \quad (1,1,1,1,1,1)
\]
の$2$個。$\Stab(a) = \kakko{s^2}$となる$a \in I^6$は
\[
(0,1,0,1,0,1) \quad (1,0,1,0,1,0)
\]
の$2$個。$\Stab(a) = \kakko{s}$となる$a \in I^6$は
\begin{align*}
  &(0,0,1,0,0,1) \quad (0,1,0,0,1,0) \\
&(0,1,1,0,1,1) \quad (1,0,0,1,0,0) \\
&(1,0,1,1,0,1) \quad (1,1,0,1,1,0)
\end{align*}
の$6$個。ゆえに$\# B = 64 - (2+2+6) = 54$が求める答えである。
\end{sol}


\newpage


\section{平成25年度 基礎数学}

\subsubsection{}%1
\barquo{
$\R^4$に標準的な内積を入れる。$V$を
\[
\pmat{1 \\ -1  \\ -1 \\ 1}, \quad \pmat{1 \\ -1  \\ 1 \\ -1}
\]
で生成される$\R^4$の部分ベクトル空間とする。このとき$V$の$\R^4$における直交補空間$W$の基底を$1$組求めよ。
}
\begin{sol}
  計算すると
  \[
  W = \Ker \pmat{ 1 & -1 & -1 & 1 \\ 1 & -1 & 1 & -1  }
  \]
  の基底としてたとえば
  \[
  \pmat{1 \\ 1  \\ 0 \\ 0}, \quad \pmat{0 \\ 0  \\ 1 \\ 1}
  \]
  がとれることがわかる。
\end{sol}

\newpage

\subsubsection{}%2
\barquo{
$3$次の複素正方行列
\[
A = \pmat{ -4 & -1 &-1 \\ 1& -2&  1 \\ 0& 0& -3}, \quad B = \pmat{ -2 & 1 &0 \\ -1& -4&  1 \\ 0& 0& -3}
\]
を考える。行列$A$と$B$は相似かどうか理由を答えよ。ただし、行列$A$と$B$が相似とは、複素正方行列$P$で$A=P^{-1}BP$をみたすものが存在することをいう。
}
\begin{sol}
  固有多項式は$A$も$B$も$(t+3)^3$になるが、
  \begin{align*}
\rank ((-3)E-A) &= 1 \\
\rank ((-3)E-B) &= 2
  \end{align*}
なので固有空間の次元が異なる。よって$A$と$B$は相似ではない。
\end{sol}

\newpage

\subsubsection{}%3
\barquo{
$\R^2$上の関数$f(x,y ) = (3xy+1)e^{-(x^2 + y^2)}$の最大値が存在することを示し、その最大値を求めよ。
}
\begin{sol}
  $x = r \cos \grt$, $y = r \sin \grt$と変数変換して$g(r, \grt ) = f(x,y)$とおく。このとき$\grt$に関係なく一様に
  \[
  \lim_{r \to \infty} \abs{g(r,\grt)} \leq \lim_{r \to \infty} \left( \f{3}{2}r^2 + 1 \right) e^{-r^2} = 0
  \]
  だから、ある$R > 0$があって、$r \geq R$のとき$ \abs{g(r,\grt)} \leq 1 = f(0,0)$である。$g$は連続なので$[0,R] \tm [0,2\pi]$上で最大値を持っており、それが$g$および$f$の最大値となる。よって最大値の存在がいえた。

  $g$の停留点をすべて求めよう。方程式
\begin{align*}
\PD{g}{r} &= (3(1-r^2) \sin 2\grt - 2)re^{-r^2} = 0 \\
\PD{g}{\grt} &= (3r^2 \cos 2\grt) e^{-r^2} = 0
\end{align*}
を考える。これを解いて次の解を得る。
\begin{description}
  \item[(1)] $r=0$
  \item[(2)] $r= 1/ \sqrt{3}$, $\sin 2 \grt = 1$
  \item[(3)] $r = \sqrt{5/3}$, $\sin 2 \grt = -1$
\end{description}
それぞれの場合に$g$の値を求めると(1)のとき$g=1$, (2)のとき$g= \f{3}{2} e^{- 1/3}$で、(3)のとき$g= - \f{3}{2} e^{- 5/3}$である。最大値は停留値のなかにあるので、このうち最大のもの、つまり$\f{3}{2} e^{- 1/3}$が$g$そして$f$の最大値である。
\end{sol}

\newpage

\subsubsection{}%4
\barquo{
$\gra, \beta$を実数とする。広義積分
\[
\int_1^{\infty} \f{x^{\gra} \log x}{(1+x)^{\beta}} \ dx
\]
が収束するような$\gra, \beta$の範囲を求めよ。
}
\begin{sol}
  $F(x) = x^{\gra} (1+x)^{- \beta} \log x $, $G = x^{\gra - \beta } \log x$とおいたとき
  \[
  \lim_{x \to \infty} \f{F}{G} = \left( \f{x}{1+x} \right)^{\beta} = 1
  \]
  なので$\int_1^{\infty} F \ dx$と$\int_1^{\infty} G \ dx$の収束は同値。そこで$\grg = \gra - \beta$とおいて
  \[
  I = \int_1^{\infty} x^{\grg} \log x \ dx
  \]
  の収束を考えればよい。いま$\grg \geq -1$とすると
  \begin{align*}
    I &\geq \int_e^{\infty} x^{\grg} \log x \ dx \\
    &\geq \int_e^{\infty} x^{\grg}  \ dx
  \end{align*}
  より$I$は発散する。逆に$\grg < -1$としよう。このとき
  \begin{align*}
I &= \f{1}{\grg + 1} \int_e^{\infty} (x^{\grg + 1})' \log x \ dx \\
&= - \f{1}{\grg + 1} \int_e^{\infty} x^{\grg + 1} \ dx
  \end{align*}
  より$I$は収束する。まとめると、$\grg = \gra - \beta$としたとき、$\grg \geq -1$なら発散で$\grg < -1$なら収束。
\end{sol}


\newpage

\begin{thebibliography}{1}%参考文献の リスト
\bibitem{雪代1} 雪江明彦『代数学1 群論入門』(日本評論社, 2010)
\end{thebibliography}


\end{document}
