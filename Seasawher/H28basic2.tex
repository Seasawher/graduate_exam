\section{平成28年度 基礎科目 II}

\subsubsection{} %{問1}
\barquo{
次の積分が収束するような実数$\gra$の範囲を求めよ。
\[
\iint_D \f{dx \ dy}{(x^2 + y^2)^{\gra} }
\]
ただし、$D=\setmid{(x,y) \in \R^2}{- \infty < x < \infty, 0 < y < 1}$とする。
}
\begin{sol}
与えられた積分を$I(\gra)$と略記する。極座標変換を行うと次の形になる。
\begin{align*}
  I(\gra) &= 2 \iint_{0 < y < 1, 0 \leq x} \f{dx \ dy}{(x^2 + y^2)^{\gra} } \\
  &= 2 \int_0^{\pi /2} \ d\grt \int_0^{1/ \sin \grt} r^{1 - 2 \gra} \ dr
\end{align*}
ここでもし$\gra = 1$ならば
\[
I(1) \geq 2 \int_0^{\pi /2} \ d\grt \int_0^1 \f{dr}{r} = \infty
\]
より発散する。そこで$\gra \neq 1$と仮定して先に進むと、次のようになる。
\[
I(\gra) = \f{1}{1- \gra} \int_0^{\pi /2} (\sin \grt)^{2\gra - 2} - \lim_{\ve \to 0} \ve^{2 - 2\gra} \ d\grt
\]
$\gra > 1$ならこれは発散する。そこで$\gra < 1$と仮定して先へ進むと、次の形に帰着する。
\begin{align*}
  I(\gra) &=  \f{1}{1- \gra} \int_0^{\pi /2}  (\sin \grt)^{2\gra - 2} \ d\grt \\
  &= \f{1}{1- \gra} \int_0^{\pi /2}  \left( \f{\sin \grt}{\grt} \right)^{2\gra - 2} \cdot \grt^{2\gra - 2} \ d\grt 
\end{align*}
$\sin \grt / \grt$は$[0, \pi/2]$上の連続関数であり、$0$より大きい最小値と最大値を持つ。よって収束には関与しないので、$\gra < 1$のとき
$I(\gra)$が収束することは$\gra > 1/2$と同値であることがわかる。つまり$I(\gra)$は、$\gra \leq 1/2$または$1 \leq \gra$なら無限大に発散、$1/2 < \gra < 1$なら収束するということが結論できたことになる。
\end{sol}

\newpage


\subsubsection{} %{問2}
\barquo{
$A$と$B$を複素$3$次正方行列とする。$A$の最小多項式は$x^3-1$, $B$の最小多項式は$(x-1)^3$とする。このとき
\[
AB \neq BA
\]
となることを示せ。
}
\begin{sol}
  行列$M \in M(3,\C)$の固有値$\grl$に属する固有空間を$E(\grl, M)$と書くことにする。仮定より、$A$の固有値は$x^2 + x + 1$の根のひとつを$\gro$として、$1,\gro, \gro^2$の3つである。もしも$AB =BA$ならば、$v \in E(\grl,A)$に対して
  \[
AB v = BA v = B(\grl v) = \grl (B v)
  \]
  であるから$Bv \in E(\grl,A)$である。つまり$B$を写像$B \colon E(\grl,A) \to E(\grl,A)$とみなせる。

  ここで$e_i \in E(\gro^i,A) \sm \{ 0 \} \; (0 \leq i \leq 2)$としよう。$e_i$はそれぞ$1$次元ベクトル空間である$E(\gro^i,A)$の基底となる。ゆえに$B e_i = \grl_i e_i$となる$\grl_i$が存在することになる。つまり$e_i$は$B$の固有ベクトルである。$e_i$は$\C^3$全体を張るので、とくに$B$は対角化可能となるが、これは$B$の最小多項式が重根を持つことに矛盾。よって$AB \neq BA$でなくてはならない。
\end{sol}
\newpage


\subsubsection{} %{問3}
\barquo{
複素関数$f(z)$は$z=0$の近傍で正則な関数で$f(z)e^{f(z)} = z$をみたすとする。
以下の問に答えよ。
\begin{description}
  \item[(i)] 非負整数$n$と十分小さい正数$\ve$に対して次の式が成り立つことを示せ。
  \[
  \f{ f^{(n)}(0) }{ n!} = \f{1}{2\pi i} \int_{C_{\ve}} \f{1+u}{e^{nu} u^n} \ du
  \]
  ここで積分路$C_{\ve}$は円周$C_{\ve} = \setmid{u \in \C}{ \abs{u} = \ve} $を正の向きに一周するものとする。
  \item[(ii)] $f(z)$の$z=0$におけるベキ級数展開を求め、その収束半径を求めよ。
\end{description}
}
\begin{sol} ${}$
  \begin{description}
    \item[(i)] 仮定の式$f(z)e^{f(z)} = z$の両辺を微分して$(1 + f)f' e^f = 1$を得る。とくに$f'$は零点を持たない。したがって逆関数定理により$f$は$0$の十分小さな近傍$U$に制限すれば像への同相写像となる。よって$u = f(z)$と変数変換することができて
    \begin{align*}
      \f{1 + u}{ (e^u u)^n} \ du &= \f{(1 + f(z))f'(z) }{z^n} \ dz \\
      &= \f{ e^{-f(z)} }{z^n} \ dz \\
      &= \f{f(z)}{z^{n+1}} \ dz
    \end{align*}
    であることがわかる。したがってCauchyの積分公式から、十分小さい$\ve$をとればすべての$n$に対して
    \[
\f{ f^{(n)}(0) }{ n!} =  \f{1}{2\pi i} \int_{f^{-1}(C_{\ve})} \f{f(z)}{z^{n+1}} \ dz  = \f{1}{2\pi i} \int_{C_{\ve}} \f{1+u}{e^{nu} u^n} \ du
    \]
    が成り立つ。
    \item[(ii)] ベキ級数展開すると
    \begin{align*}
      (1 + z)e^{-nz} &= (1 + z) \sum_{k=0}^{\infty} \f{ (-n)^k }{k!} z^k \\
      &= 1 + \sum_{k=1}^{\infty} \left( \f{ (-n)^{k-1} }{ (k-1)! } + \f{ (-n)^k }{k!} \right) z^{k}
    \end{align*}
    である。よって$z=0$のまわりでのLaurent展開は
    \[
    \f{ (1 + z)e^{-nz} }{z^n} = \f{1}{z^n} + \sum_{k=1}^{\infty} \left( \f{ (-n)^{k-1} }{ (k-1)! } + \f{ (-n)^k }{k!} \right) z^{n-k}
    \]
    であることがわかる。ゆえに留数定理から
    \[
    \f{1}{2\pi i} \int_{C_{\ve}} \f{1+u}{e^{nu} u^n} \ du = \begin{cases}
    0 &(n=0) \\
    (-n)^{n-1}/ n! &(n \geq 1)
  \end{cases}
    \]
    が従う。よって$f$のベキ級数展開を$f(z) = \sum_{n \geq 1} a_n z^n$とすると$a_n=(-n)^{n-1}/n!$であり、$\lim_{n \to \infty} \abs{a_{n+1} / a_n} = e$だから収束半径は$1/e$である。
  \end{description}
\end{sol}

\newpage

\subsubsection{} %{問4}
\barquo{
正則な複素2次正方行列のなす群を$GL_2(\C)$とおく。行列
\[
A = \pmat{0 &-1 \\ 1 & 1}, \quad B = \pmat{0& 1\\ 1& 0}
\]
で生成される$GL_2(\C)$の部分群$G$について、以下の問に答えよ。
\begin{description}
  \item[(i)] 群$G$の位数を求めよ。
  \item[(ii)] 群$G$の中心の位数を求めよ。ただし、$G$の中心とは、$G$のすべての元と可換な元全体のなす$G$の部分群のことである。
  \item[(iii)] 群$G$に含まれる位数$2$の元の個数を求めよ。
\end{description}
}
\begin{sol} $I$は単位行列とする。群の特定の部分集合$S$で生成される部分群を$\kakko{S}$で書く。
  \begin{description}
    \item[(i)] 計算により$\kakko{A} \cong \Z / 6 \Z$, $B^2 = I$, $BAB^{-1} = A^{-1}$がわかる。ゆえに$BA = (BAB)B = A^5 B$だから、$G$の元はすべて$A^i B^j \; (0 \leq i \leq 5, 0\leq j \leq 1)$という形をしている。よって$\# G \leq 12$である。

    逆を考察しよう。$BAB^{-1} = A^{-1}$より、$\kakko{A} \lhd G$である。$A^3$は$\kakko{A}$の元で位数が$2$であるような唯一の元なので$BA^3B = A^3$である。つまり$A^3$は$G$の中心$Z(G)$の元である。したがって$\kakko{A^3, B} = \{ I, A^3, B , A^3B\}$であり$G$は位数$4$の部分群を持つ。$G$が位数$3$の部分群を持つことは
    $A^2$の位数が$3$であることからあきらかなので、$\# G \geq 12$を得る。つまり$\# G =12$ということである。
    \item[(ii)] $Z = A^i B^j \; (0 \leq i \leq 5, 0\leq j \leq 1)$が$Z(G)$の元だったとする。このとき
    \begin{align*}
      AZA^{-1}Z^{-1} &= A^{i+1} B^j A^{-1} B^{-j} A^{-i} \\
      &= A^{i+1} (B^j A B^{-j})^{-1} A^{-i} \\
      &= A^{i+1} (A^{1-2j})^{-1} A^{-i} \\
      &= A^{2j}
    \end{align*}
    だから$j=0$でなくてはならない。また
    \begin{align*}
      BZBZ^{-1} &= BA^i B A^{-1} \\
      &= A^{-i } A^{-i} \\
      &= A^{-2i}
    \end{align*}
    より$i=0,3$でなくてはならない。よって$Z(G)= \{ I, A^3 \}$である。これで$\# Z(G)=2$が示せた。
    \item[(iii)] 各々の元の共役類を求めて位数の表を作ると次のようになる。
    \begin{center}
    \begin{tabular}{ccc}
     \hline
  位数 & 元 & 個数 \\
     \hline \hline
     1 & I & 1 \\
     2 & $B,A^2B, A^4B, A^3, AB, A^3B, A^5B$ &  7 \\
    3 & $A^2, A^4$ &  2  \\
    6 & $A, A^5$ & 2
     \end{tabular}
    \end{center}
    よって位数$2$の元の数は$7$個。
  \end{description}
\end{sol}
\newpage


\subsubsection{} %{問5}
\barquo{
3次元微分多様体$M  = \setmid{(x,y,z,w) \in \R^4}{xy-z^2 = w}$から$\R^3$への写像$f=(f_1,f_2,f_3) \colon M \to \R^3$を
\[
f(x,y,z,w) = (x+y, z, w)
\]
により定める。以下の問に答えよ。
\begin{description}
  \item[(i)] $f$の臨界点の集合$C$を求めよ。ただし$p \in M$が$f$の臨界点であるとは、$p$のまわりの$M$の座標系$(u_1,u_2,u_3)$に関する$f$のヤコビ行列
  \[
  \left( \f{ \del f_i }{\del u_j} \right)_{1 \leq i,j \leq 3}
  \]
  が正則でないことである。
  \item[(ii)] $C$が$M$の部分多様体になることを証明せよ。
\end{description}
}
\begin{sol} ${}$
  \begin{description}
    \item[(i)] $g \colon \R^4 \to \R$を$g(x,y,z,w) = xy - z^2 -w$により定める。このとき$M = g^{-1}(0)$であり、$f$の微分$df_p \colon T_p M \to \R^3$は$f$の$\R^4$への延長のヤコビアン$Jf_p \colon \R^4 \to \R^3$の$\Ker Jg_p$への制限だとみなせる。したがって$p \in M$に対して
    \begin{align*}
      p \in C &\iff \rank df_p \leq 2 \\
      &\iff \dim \Ker df_p \geq 1 \\
      &\iff \dim (\Ker Jf_p \cap \Ker Jg_p) \geq 1 \\
      & \iff \dim \Ker \pmat{ Jg_p \\ JF_p} \geq 1 \\
      &\iff \rank \pmat{ Jg_p \\ JF_p} \leq 3 \\
      &\iff \rank \pmat{ y &x& -2z& -1 \\ 1& 1& 0& 0 \\ 0& 0& 1& 0 \\ 0& 0& 0& 1} \leq 3 \\
      &\iff x=y
    \end{align*}
    だから$C =\setmid{(x,y,z,w) \in \R^4}{xy-z^2 = w, x=y}$である。
    \item[(ii)] $h \colon M \to \R$を$h(x,y,z,w ) = x-y$で定める。任意の点$p \in h^{-1}(0)$が正則点であることをいえばよい。計算すると
    \begin{align*}
      \rank dh_p &= 3 - \dim \Ker dh_p \\
      &= 3- \dim \Ker \pmat{Jg_p \\ Jh_p} \\
      &= \rank \pmat{Jg_p \\ Jh_p} - 1 \\
      %&= \rank \pmat{ y &x &-2z &-1 \\ 1 &-1 &0 &0 } - 1 \\ %レイアウトのために省略
      &= 1
    \end{align*}
    である。よって$0$は$h$の正則値であり、示すべきことがいえた。
  \end{description}
\end{sol}


\newpage



\subsubsection{} %{問6}
\barquo{
$A(z) = (a_{ij} (z))_{1 \leq j,k \leq N}$を$N$次正方行列、$D = \setmid{z \in \C}{ \abs{z} < 1}$を単位円板、$m$を正の整数とし、以下の(A),(B)を仮定する。

(A) 各$a_{jk}(z)$は$D$上の正則関数である。

(B) $\det A(z)$は$z=0$に$m$位の零点をもつ。

このとき、十分に小さい正数$\ve$に対して、次式が成り立つことを示せ。
\[
m = \tr \left( \f{1}{2\pi i} \int_{C_{\ve}} A(z)^{-1} \f{d}{dz} A(z) \ dz \right)
\]
ここで積分路$C_{\ve}$は円周$C_{\ve} = \setmid{z \in \C}{ \abs{z} = \ve}$を正の向きに一周するものとし、$\tr (X)$は行列$X$のトレースを表す。
}
\begin{sol}
  偏角の原理により、次を示せば十分である。
  \[
  \tr \left(A^{-1} \f{dA}{dz} \right) = \f{(\det A)'}{\det A}
  \]
  いま$A$の余因子行列を$B$とする。つまり$B$の$(i,j)$成分$b_{ij}$を
  \[
  b_{ij} = (-1)^{i+j} \det A_{ji}
  \]
  により定める。ただし$A_{ji}$とは$A$の$j$行目と$i$列目を飛ばした$(N-1)$次正方行列のことである。このとき$A^{-1}$は$B$を$\det A$で割ったものとなる。そうすると$C := A^{-1} \f{dA}{dz} $の$(i,j)$成分$c_{ij}$は
  \begin{align*}
    c_{ij} &= \f{1}{\det A} \sum_{k=1}^N b_{ik} a'_{kj} \\
    &=  \f{1}{\det A} \sum_{k=1}^N (-1)^{i+k} \det A_{ki} a'_{kj}
  \end{align*}
  である。したがって$a_i$で$A$の$i$列目を表すことにすると
\begin{align*}
  (\det A) \tr C &= \sum_{i,k} (-1)^{i+k} \det A_{ki} a'_{ki} \\
  &= \sum_{k} (-1)^{1+k} \det A_{k1} a'_{k1} + \cdots + \sum_{k} (-1)^{N+k} \det A_{kN} a'_{kN} \\
  &= \det(a_1', a_2, \cdots , a_N) + \cdots + \det(a_1, \cdots a_{N-1} , a'_N) \\
  &= (\det A)'
\end{align*}
である。
\end{sol}
