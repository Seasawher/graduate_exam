\section{令和2年度 専門科目}

\subsubsection{}%1
\barquo{
$p$を素数、$n$を非負整数とする。このとき位数$3p^n$の有限群は可解群であることを示せ。$p$群が可解群であるという事実は用いてもよい。
}
\begin{proof}
  いくつかに場合分けして示す。
  \begin{description}
    \item[(1)] $n=1, p=2$のとき。このとき$\# G=6$である。$M$を$G$のSylow-$3$部分群とする。$M$は指数$2$の部分群なので正規部分群である。よって
    \[
    1 \to M \to G
    \]
    はAbel正規列となる。よって$G$は可解群。
    \item[(2)] $n=1,p =3$のとき。このとき$G$は$p$群なので、可解群である。
    \item[(3)] $n=1, p \geq 5$のとき。$G$のSylow-$p$部分群を$H$とする。$H$の共役群の個数$s$は$\# (G/H) = 3$の約数であり、かつ$s \equiv 1 \mod p$を満たす。$p \geq 5$より、$s = 1$でなくてはならない。よって$H \lhd G$である。$H$は$p$群なので可解群であり、$G/H \cong \zyu{3}$はAbel群なので、$G$はやはり可解群。
    \item[(4)] $n \geq 2$のとき。$n$についての帰納法で示す。集合$X = G/H = \setmid{gH }{g \in G}$への$G$の左作用を考える。$\# X = 3$なので、この作用により群準同型
    \[
    \vp \colon G \to \frakS_3
    \]
    が作れる。$N = \Ker \vp$とおく。$\vp$は自明な作用ではないので$N \neq G$であり、帰納法の仮定から$N$は可解群。$\vp$と$\sgn \colon \frakS_3 \to \zyu{2}$の合成$\psi \colon G \to \zyu{2}$を考える。$M = \Ker \psi$とする。$N \subset M$である。

    いま$\#(G/N)$の可能性は$2,3,6$が残されている。$\# (G/N) = 2,3$なら$G/N$はAbel群なので$G$は可解群だということになる。$\# (G/N) = 6$とする。このとき
    \[
    N \to M \to G
    \]
    は正規列で、$M/N \cong \zyu{3}$と$G/M \cong \zyu{2}$はAbel群なので$G$は可解群。
  \end{description}
\end{proof}

\newpage


\subsubsection{}%2
\barquo{
$p \geq 5$を素数とし、$\s{-p} + \sqrt[3]{p}$を含む$\Q$のGalois拡大体のうち最小のものを$K$とする。このときGalois群$\Gal(K/\Q)$を求めよ。ただし、$\Q$の代数拡大体はすべて$\C$の部分体と考える。
}
\begin{proof}
  $\gra = \s{-p } + \sqrt[3]{p}$とおく。$(\gra - \s{ -p})^3 = p$より
  \[
  \gra^3 - 3 \gra p - p + (p - 3\gra^2) \s{-p} = 0
  \]
  である。ここで$3 \gra^2 - p \not\in \R$はゼロではないので
  \[
  \s{-p} = \f{\gra^3 - 3\gra - p}{3\gra^2 -p}
  \]
  である。よって$\s{-p} \in \Q(\gra)$である。またこれにより$\sqrt[3]{p} \in \Q(\gra)$もわかる。$\sqrt[3]{p}$の$\Q$上の最小多項式は$x^3-p$であるから、共役元は
  \[
  \sqrt[3]{p}, \; \gro \sqrt[3]{p} , \; \gro^2 \sqrt[3]{p}
  \]
  である。ただし$\gro$は1の原始三乗根で、$\gro = \exp(2 \pi \I/3)$とする。よって$K/\Q$がGalois拡大であることにより$K$は$\Q(\s{-p})$と$\Q(\sqrt[3]{p}, \s{-3})$を含んでいる。
  よって$M$を$N_1 := \Q(\s{-p})$と$N_2 = \Q(\sqrt[3]{p}, \s{-3})$の合成体とすれば$M \subset K$ということになる。$N_1$と$N_2$はともに$\Q$のGalois拡大であることに注意する。

  一方で$N_1 \cap N_2 = \Q$であることを示そう。ハイリホーによる。仮に$N_1 \cap N_2 \neq \Q$だとする。$[N_1 : \Q] = 2$なので、$N_1 = N_1 \cap N_2$ということになる。つまり
  \[
  \s{-p} = x + y \s{-3}
  \]
  なる$x, y \in \Q(\sqrt[3]{p})$がある。このとき$xy = x^2 - 3y^2 + p=0$である。とくに$x=0$または$y=0$。$x=0$なら、$\s{p/3} \in \Q(\sqrt[3]{p})$ということになる。$p \geq 5$より$[\Q(\s{p/3}): \Q ] = 2$であるが、
  これは$[\Q(\sqrt[3]{p}) : \Q] = 3$に矛盾。$y=0$なら$\s{-p} \in \Q(\sqrt[3]{p}) \subset \R$となってやはり矛盾。よって$N_1 \cap N_2 = \Q$である。

  したがってGalois拡大の推進定理により$M$は$\Q$上のGalois拡大であり、とくに$K = M$である。よって
  \[
  \Gal(K/\Q) \cong \Gal(N_1 / \Q) \tm \Gal(N_2/\Q)
  \]
  である。$\Gal(N_1/\Q) \cong \zyu{2}$はあきらか。また$N_2$は$x^3 - p$の$\Q$上の最小分解体なので$\Gal(N_2 / \Q) \subset \frakS_3$だが、$[N_2: \Q]=6$なので$\Gal(N_2 / \Q) \cong \frakS_3$である。よって
  \[
    \Gal(K/\Q) \cong \zyu{2} \tm \frakS_3
  \]
  である。
\end{proof}

\newpage

\subsubsection{}%3
\barquo{
整域$A$に対する次の性質$(*)$を考える。

$(*)$ $A$の$\{0\}$でない素イデアルのうち極小なものは単項イデアルである

$A$をNoether整域、$x \in A$を$0$でない$A$の素元とする。このとき$A[1/x]$が性質$(*)$を持てば、$A$も性質$(*)$を持つことを示せ。
}
\begin{proof}
  $A$は整域なので$S = \setmid{x^n}{n \geq 0}$は積閉集合である。$A$の素イデアルであって極小なもの$\frakp \subset A$が与えられたとする。$\frakp = xA$ならば示すことはないので$\frakp \neq xA$とする。このとき$\frakp$の極小性から$x \not\in \frakp$である。よって$\frakp \cap S = \emptyset$である。そこで$\frakq = \frakp A_S$とおくと$\frakq$は$A_S $のゼロでない素イデアルであって極小なものになっている。
  よって$(*)$より$\frakq \subset A_S$は単項イデアル。

  ここで補題を示す。
  \lem{
  $a \in A_S \sm \{0\}$に対して$a = cx^m$なる$c \in A \sm (x)$と$m \in \Z$が存在する。
  }
  \begin{proof}
    $a = dx^{l}$なる$d \in A \sm \{0\}$と$l \in \Z$がある。仮に任意の$k \geq 1$に対して$d \in (x^k)A$だとすると$d x^{-k} \in A$なので、$A$におけるイデアルの増大列
    \[
    (d ) \subset (dx^{-1}) \subset (dx^{-2}) \subset \cdots
    \]
    が存在することになる。$A$のNoether性より$(dx^{-m}) = (dx^{-m-1})$なる$m \geq 0$があるが、$x$は単元ではないのでこれは矛盾。よってある$k \geq 0$が存在して、$d \in (x^k) \sm (x^{k+1})$を満たす。よって$d = x^k c$なる$c \in A \sm (x)$がとれる。ゆえに$a = cx^{k+l}$と表せる。
  \end{proof}

  もともとの問題の証明に戻る。$\frakq$は単項イデアルだったので、この補題により$\frakq$の生成元$\pi \in A \sm (x)$がとれる。いま$\frakp = \frakp A_S \cap A = \frakq \cap A$によりあきらかに$\pi A \subset \frakp$である。逆を示そう。$z \in \frakp \sm \{0\}$とする。$\frakp = \frakq \cap A$より
  $z = \pi y$なる$y \in A_S \sm \{ 0 \}$がある。補題より$y = cx^m$なる$c \in A \sm (x)$と$m \in \Z$がとれる。このとき$z = \pi c x^m \in A$だから、言い換えれば$\pi c \in x^{-m} A$である。$\pi \in A \sm (x)$かつ$c \in A \sm (x)$なので、
  $x$が素元であることにより$m \geq 0$でなくてはならない。よって$z = \pi c x^m \in \pi A$である。$z$は任意だったから$\pi A = \frakp$が結論される。とくに$\frakp$は単項イデアルであり、$A$が$(*)$を満たしていることがいえた。
\end{proof}
