\section{平成23年度 基礎数学}

\subsubsection{}%1
\barquo{
$x$を複素数とする。$4$次複素正方行列
\[
\pmat{
x & 1& -1& 1 \\ 1& x &1 &-1 \\ -1& 1& x& 1 \\ 1& -1& 1& x
}
\]
の階数を求めよ。
}
\begin{sol}
  与えられた行列を$A$とする。基本変形を繰り返して
  \[
  \rank A = \begin{cases}
  1 &(x= -1) \\
  3 &(x=3) \\
  4 &(x \neq 3, -1)
\end{cases}
  \]
  を得る。
\end{sol}

\newpage


\subsubsection{}%2
\barquo{
$3$次複素正方行列
\[
A = \pmat{
6 & -3 &-2 \\ 4& -1& -2 \\ 3& -2& 0
}, \quad
B = \pmat{
1 & -1 &-1 \\ 1& 3& 1 \\ -1& -1& 1
}
\]
を考える。
\begin{description}
  \item[(1)] $A,B$の固有値$2$に属する固有空間の基底を一組ずつ求めよ。
  \item[(2)] $A$と$B$は相似かどうか理由をつけて答えよ。ただし、$2$つの$3$次複素正方行列$A,B$が相似とは、$A=P^{-1}BP$をみたす正則な$3$次複素正方行列$P$が存在することをいう。
\end{description}
}
\begin{sol} 以下単位行列を$E$で表すことにする。
  \begin{description}
    \item[(1)] 計算すると
    \begin{align*}
      \Ker (2E - A) &= \setmid{ z \pmat{2 \\ 2 \\ 1} \in \C^3 }{ z \in \C} \\
        \Ker (2E - B) &= \setmid{ y\pmat{-1\\ 1 \\0}+  z \pmat{-1 \\ 0 \\ 1} \in \C^3 }{y, z \in \C}
    \end{align*}
    である。
    \item[(2)] 固有値$2$に属する固有空間の次元が異なるので、$A$と$B$は相似ではない。
  \end{description}
\end{sol}

\newpage

\subsubsection{}%3
\barquo{
函数項級数
\[
\sum_{n=1}^{\infty} \f{x}{(1+x)^n}
\]
が区間$[0,1]$上で一様収束するかどうかを、理由を付けて答えよ。
}
\begin{sol}
  与えられた級数を$f(x)$で表す。計算すると$f(0)=0$かつ$f(x)=1 \; (x > 0)$であることがわかる。よって$f$は連続ではないが、もし一様収束するならば連続であるはずなので、一様収束しない。
\end{sol}

\newpage

\subsubsection{}%4
\barquo{
$a < b$とし区間$[a,b]$上の実数値連続函数$f(x), \vp(x)$は共に狭義単調増加とする。
\[
\int_a^b f(x) \ dx = 0
\]
ならば、$\int_a^b \vp(x) f(x) \ dx > 0$となることを証明せよ。
}
\begin{sol}
  仮定より$(b-a)f(a) < 0 < (b-a)f(b)$である。よって$f(a)<0$かつ$f(b)>0$であることがわかる。$f$は連続であるから、$f(c)=0$なる$a < c < b$が存在することになる。このとき
  \begin{align*}
    \int_a^b \vp(x) f(x) \ dx  &= \int_a^b \vp(x) f(x) \ dx  - \vp(c) \int_a^b f(x) \ dx \\
    &= \int_a^b (\vp(x) - \vp(c)) f(x) \ dx \\
    &= \int_a^c (\vp(x) - \vp(c)) f(x) \ dx + \int_c^b (\vp(x) - \vp(c)) f(x) \ dx \\
    &> 0
   \end{align*}
   という評価ができる。
\end{sol}
