\section{平成28年度 基礎科目 I}

\subsubsection{} %{問1}
\barquo{
線形写像$f \colon \R^4 \to \R^3$を行列
\[
A = \pmat{
2 &1 &1& 0 \\
4 &0 &2 &1 \\
2 &-1& 1 &2
}
\]
を用いて$f(x) = Ax \ (x \in \R^4)$として定める。$V$を3つのベクトル
\[
\pmat{ 1\\ 2\\ -2\\ -4 }, \pmat{0\\ -2\\ 1\\ 3 }, \pmat{ 1\\ 1\\ 0\\ -4 }
\]
が張る$\R^4$の部分空間としたとき、$f$の$V$への制限$g = f|_V \colon V \to \R^3$の階数を求めよ。ただし、$g$の階数とは、$g(V)$の次元のこととする。
}
\begin{sol}
  与えられたベクトルをそれぞれ$v_1, v_2, v_3$とする。このとき$B=(gv_1, gv_2, gv_3)$であるとすると$B$は基本変形で
  \[
  B = \pmat{
2& -1& 3 \\
-4& 5& 0 \\
-10&  8&  -7
  }
  \sim
  \pmat{
  2 &-1 &3 \\
  0 &3 &6 \\
  0 &0& 2
  }
  \]
  と変形できる。したがって$\rank B = 3$であり、$g$の階数は$3$である。
\end{sol}




\newpage


\subsubsection{} %{問2}
\barquo{
$a$を実数とする。$3$次正方行列
\[
A = \pmat{
a& 1 &2 \\
0 &1 &0 \\
-2 &0& 0
}
\]
について、以下の問に答えよ。
\begin{description}
  \item[(i)] 行列$A$の固有値を求めよ。
  \item[(ii)] 行列$A$が対角化可能となる実数$a$をすべて求めよ。ただし、$A$が対角化可能であるとは、複素正則行列$P$で$P^{-1}AP$が対角行列となるものが存在することをいう。
\end{description}
}
\begin{sol} ${}$
  \begin{description}
\item[(i)] 与えられた$A$を変数$a$を明示して$A(a)$と書くことにしよう。そうして$A(a)$の固有多項式を$\Phi_a(\grl)$で書くことにする。このとき
\[
\Phi_a(\grl) = \det(\grl I - A(a)) = (\grl - 1)(\grl^2 - a \grl + 4)
\]
である。だから固有値は$1, (a \pm \sqrt{a^2 - 16})/2$である。
\item[(ii)] $\grl^2 - a \grl + 4$が$1$を根に持つのは$a=5$のとき。また重根を持つのは$a=\pm 4$のとき。だから$a$が$5, \pm 4$のいずれでもないときには$A(a)$は異なる$3$つの固有値を持ち、したがって対角化可能である。では$a =5, \pm 4 $のときはどうか。

行列$A$が対角化可能であることと、$A$の各固有値についての固有空間の直和が全体と一致することは同値であることに注意する。固有値$\grl$に属する固有空間を$V(\grl)$と表すことにする。

$a=4$のとき。$\Phi_4 (\grl)= (\grl-1)(\grl-2)^2 $である。固有空間$V(2)$の次元は線形写像$2 I - A(4)$の核の次元だから
\[
2 I - A(4) = \pmat{
-2& -1& -2 \\
0 &-3& 0 \\
2 &0& 2
}
\sim
\pmat{
-2 &-1& -2 \\
0 &0& 0 \\
0& -1& 0
}
\]
より$\dim V(2) = 3 - 2 = 1$である。よって$A(4)$は対角化できない。$a=-4, 5$についても同様の考察により$A(a)$が対角化できないことが判るが、詳細は省略する。これですべての$a$について$A(a)$の対角化可能性が求まった。
  \end{description}
\end{sol}

\newpage

\subsubsection{} %{問3}
\barquo{
次の極限値を求めよ。
\[
\lim_{n \to \infty} \int_0^{\infty} e^{-x}(nx - [nx]) \ dx
\]
ただし、$n$は自然数とし、$[y]$は$y$を超えない最大の整数を表す。
}
\begin{sol}
  $1$以上の$n \in \Z$を固定すると、任意の$x \in \R_{\geq 0}$に対して
  \[
  \f{k}{n} \leq x < \f{k+1}{n}
  \]
  なる$k \in \Z$が一意的にある。このとき$nx - [nx] = nx - k$である。そこで$M_n = \int_0^{\infty} e^{-x}(nx - [nx]) \ dx$とおくと
  \[
  M_n = \sum_{k \geq 0} \int_{k/n}^{(k+1)/n} e^{-x}(nx - k) \ dx
  \]
  である。いったん$n$は固定して$F_k = \int_{k/n}^{(k+1)/n} e^{-x}(nx - k) \ dx$とおこう。部分積分を用いることにより
  \[
  F_k = -(n+1) e^{-\f{k+1}{n}} + n e^{- \f{k}{n}}
  \]
  が示せる。$\zeta = e^{-\f{1}{n}}$とおけば$F_k = -(n+1)\zeta^{k+1} + n \zeta^k$である。したがって
  \begin{align*}
    M_n = (n - (n+1)\zeta) \sum_{k \geq 0} \zeta^k = \f{n - (n+1)\zeta}{1 - \zeta}
  \end{align*}
  を得る。あとは次のように式変形を行えばよい。
  \begin{align*}
\lim_{n \to \infty} M_n &= \lim_{n \to \infty}  \f{n - (n+1)e^{-1/n} }{1 - e^{-1/n}} \\
&= \lim_{n \to \infty}  \f{ne^{1/n} - (n+1) }{e^{1/n} - 1} \\
&= \lim_{h \to 0}  \f{ h^{-1} e^{h} - (h^{-1}+1) }{e^{h} - 1} \\
&= \lim_{h \to 0}  \f{ e^{h} - (1 + h) }{ h(e^{h} - 1) } \\
&= \lim_{h \to 0}  \f{h}{ e^{h} - 1 } \f{ e^h - (1+h) }{ h^2 }
  \end{align*}
そうすると$e^h = 1 + h + h^2/2 + O(h^3)$より$\lim_{n \to \infty} M_n = 1/2$が結論できる。
\end{sol}

\newpage

\subsubsection{} %{問4}
\barquo{
$\R^2$で定義された関数
\[
f(x,y) = \f{ xy(xy+4) }{ x^2 + y^2  + 1 }
\]
の最大値および最小値のそれぞれについて、存在するなら求め、存在しないならそのことを示せ。
}
\begin{sol}
  $x = r \cos \grt$, $y = r \sin \grt$とおくと
  \[
  f(x,y) = \f{r^4 \sin^2 2\grt  + 8 r^2 \sin 2\grt }{4(r^2 +1)}
  \]
  である。さらに$t = \sin 2 \grt$とおけば次のように変形できる。
  \[
  f(x,y) = \f{ r^4 t^2 + 8 r^2 t}{4(r^2 + 1)}
  \]
  したがって、右辺の関数を$g(r,t)$とおいて$0 \leq r, -1 \leq t \leq 1$における$g$の最大値と最小値を求めればよい。最大値の方はすぐにわかる。$t=1$としてみると$g(r,1)=(r^4 + 8r^2)/4(r^2 +1)$であり、あきらかに$\lim_{r \to \infty} g(r,1)=\infty$なので$g$に最大値はない。

  では最小値はどうか。$g$を$t$について平方完成すると
  \[
  g(r,t) = \f{r^4 }{4(r^2 + 1)} \left( \left(t+ \f{4}{r^2} \right) - \f{16}{r^4} \right)
  \]
  である。$h(t)= (t+ 4/r^2) - 16/r^4$とおく。$h(t)$のグラフは、軸が直線$t = -4 / r^2$であるような下に凸な放物線である。そこで軸が$-1 \leq t \leq 1$に入るかどうかで場合分けをして
  \[
  \min_{-1 \leq t \leq 1} h(t) = \begin{cases}
  h(-4/r^2) = -16/r^4 &(r \geq 2) \\
  h(-1)= - 8/r^2 + 1 &(0 \leq r \leq 2)
\end{cases}
  \]
  を得る。ゆえに
  \[
  \min_{r \geq 0, -1 \leq t \leq 1} g(r,t) = \min \left\{ \min_{r \geq 2} \f{-4}{r^2 +1} ,\; \min_{0 \leq r \leq 2} \f{1}{4} \left( r^2 - 9 + \f{9}{r^2+1} \right) \right\}
  \]
  である。$k(r) =  r^2 - 9 + 9/(r^2+1)$とおいて微分すると$k(r)$の$0 \leq r \leq 2$での最小値は$k(\sqrt{2}) = -4$であることがわかる。あきらかに$\min_{r \geq 2} -4/(r^2 +1) = -4/5$だから、求める最小値は$-1$である。
\end{sol}
