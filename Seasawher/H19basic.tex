\section{平成19年度 基礎数学}

\subsubsection{}%1
\barquo{
次の行列は正則かどうか判定せよ。正則行列ならば、その逆行列を求めよ。
\[
\pmat{ 1 &2& 3 \\ 2& 0& 2 \\ 3& 2& 1}, \quad \pmat{1 &2& 1 \\ 2& 1& 3 \\ 1& 5& 0}
\]
}
\begin{proof}
  \[
  A = \pmat{ 1 &2& 3 \\ 2& 0& 2 \\ 3& 2& 1}, \quad B = \pmat{1 &2& 1 \\ 2& 1& 3 \\ 1& 5& 0}
  \]
  とおく。結果だけ書くと、$A$は正則であって
  \[
  A^{-1} = \f{1}{4} \pmat{-1& 1& 1 \\ 1& -2& 1 \\ 1& 1 &-1 }
  \]
  である。一方で$\rank B = 2$なので$B$は正則ではない。
\end{proof}

\newpage

\subsubsection{}%2
\barquo{
$\R$を係数に持つ2次以下の多項式のなすベクトル空間を$V$で表す。$V$の元$f(x)$に対して、$x f''(x) - 2 f'(x)$を対応させる$V$の一次変換を$F$とする。$V$の基底$1,x,x^2$に関する$F$の行列表示を与えよ。また$F$の階数を求めよ。
}
\begin{proof}
  結果だけを書く。行列表示は
  \[
  F = \pmat{0 &-2& 0 \\ 0& 0& -2 \\ 0& 0& 0 }
  \]
  であり、$\rank F = 2$である。
\end{proof}

\newpage


\subsubsection{}%3
\barquo{
次の積分を計算せよ。
\[
\iint_D (x^2 - y^2) dx dy
\]
ここで領域$D$は、$D = \setmid{(x,y) \in \R^2}{ x-y \geq 0, x+y \leq 2, y \geq 0 }$で与えられるものとする。
}
\begin{proof}
  見方を変えれば$D = \setmid{(x,y) \in \R^2}{0 \leq y \leq 1, y \leq x \leq 2 -y}$とみなせる。したがって
  \begin{align*}
    \iint_D (x^2 - y^2) dx dy &= \int_0^1 dy \int_y^{2-y} (x^2 - y^2) dx = 1
  \end{align*}
\end{proof}

\newpage

\subsubsection{}%4
\barquo{
区間$(0,\infty)$で定義された次の函数は一様連続か、理由をつけて答えよ。
\begin{description}
  \item[(a)] $\f{1}{x}$
  \item[(b)] $\sin x$
\end{description}
}
\begin{proof} ${}$
  \begin{description}
    \item[(a)] $\grd > 0$をどんなに小さくとっても、$0 < x < \min\{1, \grd/2 \}$とすると
    \begin{align*}
      \f{1}{x} - \f{1}{x + \grd} &= \f{ \grd }{ x (x + \grd) } \\
      &\geq \f{ \grd }{ x + \grd } \\
      &\geq \f{2}{3}
    \end{align*}
    となるから$1/x$は一様連続ではない。
    \item[(b)] 平均値の定理により
    \[
    \abs{ \sin x - \sin y } \leq \abs{ x -y }
    \]
    なので$\sin x$は$(0,\infty)$上Lipschitz連続であり、とくに一様連続。
  \end{description}
\end{proof}
