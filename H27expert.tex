\section{平成27年度 専門科目}

\subsubsection{}
\barquo{
$G$は非可換群で次の条件$(*)$を満たすとする。

$(*)$ $N_1,N_2 \subset G$が相異なる自明でない (つまり${1}$とも$G$とも異なる) 正規部分群なら、$N_1 \not\subset N_2$である。

このとき、以下の問に答えよ。
\begin{description}
  \item[(i)] $N_1, N_2$が相異なる$G$の自明でない正規部分群なら、$G = N_1 \tm N_2$であることを証明せよ。
  \item[(ii)] $G$の自明でない正規部分群の数は高々$2$個であることを証明せよ。
\end{description}
}
\begin{sol} ${}$
  \begin{description}
    \item[(i)] 仮定より$N_1 \cap N_2 \subsetneq N_i \subset G$かつ$N_1 \cap N_2 \lhd G$なので$N_1 \cap N_2 = 1$である。また$1 \subsetneq N_i \subsetneq N_1 N_2 $かつ$N_1 N_2 \lhd G$より$N_1 N_2 = G$である。したがって交換子が
    $[N_1,N_2] \subset N_1 \cap N_2 = 1$より自明になるので、$N_1$と$N_2$の元は互いに可換。よって積をとる写像$N_1 \tm N_2 \to G$は準同型でかつ全単射なので同型である。
    \item[(ii)] ハイリホーによる。相異なる$3$つの自明でない正規部分群$N_1, N_2, N_3$が存在したとしよう。相異なるという仮定から、(i)により$N_i N_j = G \; (i \neq j)$であり、$i \neq j$である限り$N_i$と$N_j$の元は互いに可換である。したがって$G = N_1 N_2$と$N_3$の元は可換なので、とくに$N_3$はAbel群である。同様にして各$N_i$がAbel群であることが判る。よってとくに$G$はAbel群であるが、これは$G$が非可換群であるとした仮定に反する。





  \end{description}
\end{sol}
