\section{平成28年度 専門科目}

\subsubsection{} %{問1}
\barquo{
有理数係数の既約多項式$f(x) = x^3 + ax + b$を考え、$K \subset \C$を$f(x)$の最小分解体とする。$a > 0$のとき、$K/\Q$のGalois群が3次対称群と同型であることを示せ。
}
\begin{sol}
  $\lim_{x \to + \infty} f(x) = + \infty$, $\lim_{x \to - \infty} f(x) = - \infty$より、$f$は連続なので$f(\beta_1) = 0$なる$\beta_1 \in \R$がある。$f'(x) = 3x^2 + a > 0$より$f$は単調増加なので$f$の実根は$\beta_1$のみである。そこで$f$の残りの根を
  $\beta_2, \beta_3$とすると$\beta_2 = \ol{\beta_3}$である。いま$G = \Gal(K/\Q)$を根への作用により3次対称群$S_3$の部分群とみなす。$G$は複素共役をとる写像を含むので、$\# G$は偶数でなくてはならない。また$f$は既約と仮定したので、$G$は集合$\{ 1,2,3\}$に推移的に作用するはずであり、とくに$\# G$は$3$の倍数である。
  よって$\# G$は$6$の倍数となるが、$\# S_3 = 6$なので$G \cong S_3$となるしかない。
\end{sol}

\newpage

\subsubsection{} %{問2}
\barquo{
$K$を標数が2でない代数的閉体とし、$K$の元$a$に対して、2変数多項式環$K[X,Y]$の剰余環
\[
R_a = K[X,Y] / (X^2 - Y^2 - X - Y -a)
\]
を考える。以下の問に答えよ。
\begin{description}
  \item[(i)] $a=0$のとき$R_a$の各極大イデアル$\frakm$に対して$\dim_K(\frakm / \frakm^2)$を求めよ。また$\frakm$はいつ$R_a$の単項イデアルとなるか?理由をつけて答えよ。
  \item[(ii)] $a \neq 0$のとき$R_a$の各極大イデアル$\frakm$に対して$\dim_K(\frakm / \frakm^2)$を求めよ。また$\frakm$はいつ$R_a$の単項イデアルとなるか?理由をつけて答えよ。
\end{description}
}
\begin{sol}
  式を変形すると$X^2 - Y^2 - X - Y -a = (X+Y)(X-Y-1)-a$である。いま$K$の標数は$2$ではないと仮定したので
  \begin{align*}
    K[S,T] &\to K[X,Y] \\
    S &\mapsto X+Y \\
    T &\mapsto X- Y-1
  \end{align*}
  という$K$準同型を考えると、これは
  \begin{align*}
    K[X,Y] &\to K[S,T] \\
    X &\mapsto (S+T+1)/2 \\
    Y &\mapsto (S-T-1)/2
  \end{align*}
  を逆写像とする同型である。これにより$R_a$は
  \[
U_a := K[S,T]/(ST-a)
  \]
  に対応する。$U_a$の極大イデアルを表すのに、記号の濫用だが$R_a$の極大イデアルと同じ記号$\frakm$を使うことにする。いま$U_a$の極大イデアル$\frakm$は、$E:=K[S,T]$の極大イデアル$\frakn$であって$(ST-a)$を含むものに対応している。Hilbertの零点定理により、$E/\frakn$は$K$の有限次拡大である。$K$は代数閉という仮定があったので、$E/\frakn \cong K$である。この同型による$S,T$の像をそれぞれ$\beta, \grg \in K$とする。すると$(S-\beta,T-\grg) \subset \frakn$であるが
  $(S-\beta,T-\grg)$は極大イデアルなので
  $(S-\beta,T-\grg)=\frakn$である。まとめると、$\frakn$は$\beta \grg = a$なる$\beta,\grg$によって$\frakn = (S-\beta,T-\grg)$と表されることがわかった。このことを踏まえて考察をしていく。

  \begin{description}
    \item[(i)] $a=0$のとき$\beta \grg=0$なので$\beta=0$または$\grg=0$である。どちらでも同じことなので$\beta=0$とする。$\grg =0$である場合には
    \begin{align*}
      \frakm/\frakm^2 &= ((S,T)/ST ) / ( (S^2,ST,T^2)/ST) \\
      &\cong (S,T)/ (S^2,ST,T^2) \\
      &\cong K^2
    \end{align*}
    だから$\dim_K(\frakm /\frakm^2) = 2$である。もし$\frakm$が$U_0$の単項イデアルなら$\frakm/\frakm^2 = \frakm \ts_{U_0} K \cong K$となるはずだから、$\frakm$は単項イデアルではない。

    $\grg \neq 0$のとき。このときには
    \begin{align*}
      (T-\grg)U_0 &= (T-\grg,ST)/ST \\
      &= (S(T-\grg),T-\grg,ST)/ST \\
      &= (S,T-\grg)/ST \\
      &= \frakm
    \end{align*}
    だから$\frakm$は単項イデアルである。とくに$\dim_K(\frakm/\frakm^2)=1$となる。

    以上の結果をまとめると
    \[
    \dim_K(\frakm / \frakm^2) = \begin{cases}
    1 & (\beta,\grg) \neq (0,0) \text{のとき。このとき$\frakm$は単項イデアル} \\
    2 & (\beta,\grg) = (0,0) \text{のとき}
  \end{cases}
    \]
    である。$U_0$から$R_0$に話を戻すと
    \[
    \dim_K(\frakm / \frakm^2) = \begin{cases}
    1 & (b,c) \neq (1/2,1/2) \text{のとき。このとき$\frakm$は単項イデアル} \\
    2 & (b,c) = (1/2,1/2) \text{のとき}
  \end{cases}
    \]
    がわかったことになる。ただし、$b,c$は$\frakm = (X-b,Y-c)/ (X^2 - Y^2 - X - Y)$となるような$b,c \in K$である。
    \item[(ii)] 再び$U_a$に話を持っていく。$a \neq 0$のとき$\beta \neq 0$かつ$\grg \neq 0$である。このとき
    \begin{align*}
      (S-\beta)U_0 &= (S-\beta, ST-a)/ (ST-a) \\
      &= (T(S-\beta), S-\beta, ST-a)/ (ST-a ) \\
      &=  (\beta(T-\grg), S-\beta, ST-a)/ (ST-a ) \\
      &=  (T-\grg, S-\beta)/ (ST-a ) \\
      &= \frakm
    \end{align*}
    だから$\frakm$は単項イデアルであり、とくに$\dim_K(\frakm / \frakm^2) = 1$がわかる。
  \end{description}
\end{sol}
