\bfsection{平成28年度 専門科目}

\bfsubsection{問1}
\barquo{
有理数係数の既約多項式$f(x) = x^3 + ax + b$を考え、$K \subset \C$を$f(x)$の最小分解体とする。$a > 0$のとき、$K/\Q$のGalois群が3次対称群と同型であることを示せ。
}
\begin{sol}
  $\lim_{x \to + \infty} f(x) = + \infty$, $\lim_{x \to - \infty} f(x) = - \infty$より、$f$は連続なので$f(\beta_1) = 0$なる$\beta_1 \in \R$がある。$f'(x) = 3x^2 + a > 0$より$f$は単調増加なので$f$の実根は$\beta_1$のみである。そこで$f$の残りの根を
  $\beta_2, \beta_3$とすると$\beta_2 = \ol{\beta_3}$である。いま$G = \Gal(K/\Q)$を根への作用により3次対称群$S_3$の部分群とみなす。$G$は複素共役をとる写像を含むので、$\# G$は偶数でなくてはならない。また$f$は既約と仮定したので、$G$は集合$\{ 1,2,3\}$に推移的に作用するはずであり、とくに$\# G$は$3$の倍数である。
  よって$\# G$は$6$の倍数となるが、$\# S_3 = 6$なので$G \cong S_3$となるしかない。
\end{sol}

\newpage

\bfsubsection{問2}
\barquo{
$K$を標数が2でない代数的閉体とし、$K$の元$a$に対して、2変数多項式環$K[X,Y]$の剰余環
\[
R_a = K[X,Y] / (X^2 - Y^2 - X - Y -a)
\]
を考える。以下の問に答えよ。
\begin{description}
  \item[(i)] $a=0$のとき$R_a$の各極大イデアル$\frakm$に対して$\dim_K(\frakm / \frakm^2)$を求めよ。また$\frakm$はいつ$R_a$の単項イデアルとなるか?理由をつけて答えよ。
  \item[(ii)] $a \neq 0$のとき$R_a$の各極大イデアル$\frakm$に対して$\dim_K(\frakm / \frakm^2)$を求めよ。また$\frakm$はいつ$R_a$の単項イデアルとなるか?理由をつけて答えよ。
\end{description}
}
