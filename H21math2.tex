\section{平成21年度 数学II}

\subsubsection{}%1
\barquo{
$K$は虚$2$次体とする。すなわち、$K = \Q(\s{-d})$で、$d$は正の有理数であるとする。このとき、$\Q$の$4$次の巡回拡大体$L$で$K$を含むものは存在しないことを示せ。ただし、$L/\Q$が巡回拡大であるとは、$L/\Q$がGalois拡大であって、Galois群$\Gal(L/\Q)$が巡回群となっていることをいう。
}
\begin{sol}
ハイリホーによる。$K$を含む$4$次拡大$L$が存在したとする。
\[
\xymatrix{
{} & \C & {} \\
\R \ar[ru] & {} & \L \ar[lu] \\
{} & \Q \ar[lu] \ar[ru] & {}
}
\]
$\R$と$L$の合成は$\C$に一致する。$\R \cap L$は$L / \Q$の中間体なのでGaloisの基本定理により$\Q, \Q(\s{-d}), L$のいずれかであるが$\R$に含まれているので$\R \cap L = \Q$である。したがって$L/\Q$が有限次Galois拡大であることにより、Galois拡大の推進定理より
\[
\zyu{4} \cong \Gal(L/\Q) \cong \Gal(\C/\R) \cong \zyu{2}
\]
となり矛盾。よって示すべきことがいえた。
\end{sol}


\newpage


\subsubsection{}%2
\barquo{
$p$を奇素数とする。$G$を位数が$p^3$の有限群で、単位元以外の各元の位数が$p$であるようなものとする。このとき、$G$は$\C$上の一般線形群$GL_2(\C)$の部分群と同型ではないことを示せ。
}
\begin{sol}
ハイリホーによる。ある部分群$G \subset GL_2(\C)$が存在して、$\# G = p^3$かつ単位元でないすべての元$A \in G$の位数が$p$であったとする。

このとき$A \in G$とすると$A$の最小多項式は$X^p - 1$を割り切る。よってとくに重根を持たないので$A$は対角化可能である。また$A^p = E$より$A$の固有値はすべて$1$の$p$乗根である。とくに、$G$のスカラー行列は$\gro = \exp(2 \pi i / p)$として
\[
E, \gro E, \cdots , \gro^{p-1} E
\]
の$p$個のうちのどれかだということが判る。

$G$は$p$群なのでとくに中心が自明でない。よって$Z \in Z(G) \sm \{ E \}$が存在する。$Z$の位数は$p$であってかつ$\kakko{Z} \lhd G$なので$G / \kakko{Z}$は位数$p^2$の群である。位数がある素数の$2$乗となるような群はAbel群であることが知られている。よって$G / \kakko{Z}$もAbel群である。ゆえに$[G,G] \subset \kakko{Z} \subset Z(G)$である。

以下$Z(G)$にスカラーでない行列があるかどうかで場合分けを行い、いずれにせよ矛盾が導かれることを示す。

$Z(G)$の元がすべてスカラー行列であったと仮定しよう。$\# G = p^3$より$Z(G)$の位数は$1, p , p^2, p^3$のどれかだが、$G$にはスカラー行列は$p$個しかないので$\# Z(G) =1$または$p$である。さらに$Z(G)$は自明ではないから$\# Z(G) = p$であって、
\[
Z(G) = \{ E, \gro E, \cdots , \gro^{p-1} E \}
\]
であることが従う。このとき$G$がAbel群になることが示せてしまう。なぜならば!$B \in G$と$C \in G \sm Z(G)$が与えられたとする。$G$の元は対角化可能なので、$B$だけでも対角化しておく。つまり
\begin{align*}
  PBP^{-1} &= \pmat{ a & 0 \\ 0  & d} \\
  PCP^{-1} &= \pmat{ e & f \\ g  & h}
\end{align*}
となるような$P \in GL_2(\C)$をとっておく。このとき$BCB^{-1}C^{-1} \in [G,G] \subset Z(G)$により$BCB^{-1}C^{-1} = \gro^i E$なる$0 \leq i \leq p-1$がある。つまり$BC - \gro^i CB = 0$ということだが、これは
\begin{align*}
  0 &= P(BC - \gro^i CB)P^{-1} \\
  &= \pmat{ae (1 - \gro^i) & f(a-\gro^i d) \\ g (d - \gro^i a) & dh(1 - \gro^i) }
\end{align*}
を意味する。$B \in GL_2(\C)$より$a \neq 0$かつ$d \neq 0$なので$e = h = 0$または$1 - \gro^i = 0$である。仮に$e=h=0$だとすると$(PCP^{-1})^2 = fg E$より$C^2 = fg E$である。$C$はスカラーではないと仮定したので、これは$C \in G$の位数が偶数であることを示唆するが、$p$は奇素数なのでこれはありえない。だから$1 - \gro^i = 0$であり$i=0$でなくてはならないが、これは$BC = CB$を意味する。つまり$G$はAbel群であったということになる。ゆえにかくのごとし。
したがって$G = Z(G)$ということになるが、これは$\# G = p^3$に矛盾する。


また$Z(G)$の元であってスカラーでない元$A$が存在したとしよう。$G$の元はすべて対角化可能なので$A$も対角化しておく。つまり
\[
PAP^{-1} = \pmat{ \beta & 0 \\ 0 & \grg}
\]
なる$P \in GL_2(\C)$をとる。このとき、$GL_2(\C)$の部分集合
\[
M = \setmid{ \pmat{ \gro^i & 0 \\ 0 & \gro^j } }{ 0 \leq i,j \leq p-1}
\]
を考えると、単射$G \to M$が構成できてしまう。なぜならば!$B \in G$が与えられたとする。
\[
PBP^{-1} = \pmat{ a & b \\ c & d}
\]
とおく。このとき
\[
0 = P(AB - BA)P^{-1} = \pmat{ 0 & b(\beta - \grg) \\ c(\grg - \beta) & 0}
\]
である。$A$はスカラーでないと仮定していたので$\beta \neq \grg$であり、$b = c=0$であることがわかる。つまり$A$を対角化する行列としてとった$P$は、実は$G$の任意の元を対角化する。したがって$B \mapsto PBP^{-1}$という対応により単射$G \to M$が定まる。ゆえにかくのごとし。とくに$\# G \leq \# M = p^2$となっているはずだが、これは矛盾。

いずれにせよ矛盾が得られたので、これで示すべきことがいえた。
\end{sol}
